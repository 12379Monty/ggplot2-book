
\section{Converting between grammar syntaxes}

The biggest difference between Lee's grammar and mine is how the grammar is written; the syntax of the grammars are quite different.  The Grammar of Graphics uses two specifications.  A concise format is used to caption figures, and a more detailed xml format stored on disk.  The following example of the concise format is adapted from \citet[][Figure 1.5, page 13]{wilkinson:2006}.

\begin{verbatim}
DATA: source("demographics")
DATA: longitude, latitude = map(source("World"))
TRANS: bd = max(birth - death, 0)
COORD: project.mercator()
ELEMENT: point(position(lon * lat), size(bf), color(color.red))
ELEMENT: polygon(position(longitude * latitude))
\end{verbatim}

This is relatively simple to adapt to the syntax of ggplot:

\begin{itemize}
	
	\item {\tt ggplot()} is used to specify the default data and default aesthetic mappings.  {\tt aes} is short for aesthetic mapping and specifies which variables in the data should be mapped to which aesthetic attributes.
	
	\item Data is provided as standard R data.frames existing in the global environment; it does not need to be explicitly loaded.  We also use a slightly different world data set, with columns lat and long.  This lets us use the same aesthetic mappings for both datasets. Layers can override the default data and aesthetic mappings provided by the plot. 
	
	\item We replace {\sf TRANS} with an explicit transformation by R code.

	\item {\sf ELEMENT}s are replaced with layers, which explicitly specify where the data comes from.  Each geom has a default statistic which is used to transform the data prior to plotting.  For the geoms in this example, the default statistic is the identity function.  Fixed aesthetics (the colour red in this example) are supplied as additional arguments to the layer, rather than as special constants.

	\item The {\sf SCALE} component has been omitted from this example (so that the defaults are used)In both the ggplot and GoG examples, scales are defined by default.  In ggplot you can override the defaults by adding a scale object, e.g. {\tt scale\_colour} or {\tt scale\_size}

	\item {\sf COORD} uses a slightly different format.  In general, most of the components specifications in ggplot are slightly different to those in GoG, in order to be more familiar to R users.

	\item Each component is added together with $+$ to create the final plot

\end{itemize}

This gives us:

\begin{verbatim}
demographics <- transform(demographics, bd = max(birth - death, 0))

ggplot(data = demographic, mapping = aes(x = lon, y = lat)) + 
layer(geom = "point", mapping = aes(size = bd), colour="red") +
layer(geom = "polygon", data = world) +
coord_map(projection = "mercator")
\end{verbatim}
