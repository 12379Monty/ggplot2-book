\providecommand{\setflag}{\newif \ifwhole \wholefalse}
\setflag
\ifwhole\else

% Typography and geometry ----------------------------------------------------
\documentclass[letterpaper]{scrbook}
\usepackage[inner=3cm,top=2.5cm,outer=3.5cm]{geometry}

\renewcommand\familydefault{bch}
\usepackage[utf8]{inputenc}
\usepackage{microtype}
\usepackage[small]{caption}
\usepackage[small]{titlesec}
\raggedbottom

% Graphics -------------------------------------------------------------------
\usepackage[pdftex]{graphicx}
\graphicspath{{_include/}}
\DeclareGraphicsExtensions{.png,.pdf}

% Code formatting ------------------------------------------------------------
\usepackage{fancyvrb}
\usepackage{courier}
\usepackage{listings}
\usepackage{color}
\usepackage{alltt}


\definecolor{comment}{rgb}{0.60, 0.60, 0.53}
\definecolor{background}{rgb}{0.97, 0.97, 1.00}
\definecolor{string}{rgb}{0.863, 0.066, 0.266}
\definecolor{number}{rgb}{0.0, 0.6, 0.6}
\definecolor{variable}{rgb}{0.00, 0.52, 0.70}
\lstset{
  basicstyle=\ttfamily,
  keywordstyle=\bfseries, 
  identifierstyle=,  
  commentstyle=\color{comment} \emph,
  stringstyle=\color{string},
  showstringspaces=false,
  columns = fullflexible,
  backgroundcolor=\color{background},
  mathescape = true,
  escapeinside=&&,
  fancyvrb
}
\newcommand{\code}[1]{\lstinline!#1!}
\newcommand{\f}[1]{\lstinline!#1()!}



% Links ----------------------------------------------------------------------

\usepackage{hyperref}
\definecolor{slateblue}{rgb}{0.07,0.07,0.488}
\hypersetup{colorlinks=true,linkcolor=slateblue,anchorcolor=slateblue,citecolor=slateblue,filecolor=slateblue,urlcolor=slateblue,bookmarksnumbered=true,pdfview=FitB}
\usepackage{url}

% Tables ---------------------------------------------------------------------
\usepackage{longtable}
\usepackage{booktabs}

% Miscellaneous --------------------------------------------------------------
\usepackage{pdfsync}
\usepackage{appendix}

\usepackage[round,sort&compress,sectionbib]{natbib}
\bibliographystyle{plainnat}


\title{ggplot2}
\author{Hadley Wickham}

\begin{document}
\fi


% SET_DEFAULTS
%   GG-WIDTH: 4  GG-HEIGHT: 4
%   TEX-WIDTH: 0.5\linewidth
%   CACHE: TRUE
%   INLINE: FALSE
% 
% options(digits = 2)

% END

\chapter{Reducing duplication}
\label{cha:duplication}

\section{Introduction}

A major requirement of a good data analysis is flexibility. If the data changes, or you discover something that makes you rethink your basic assumptions, you need to be able to easily change many plots at once. The main inhibitor of flexibility is duplication. If you have the same plotting statement repeated over and over again, you have to make the same change in many different places. Often just the thought of making all those changes is exhausting!

This chapter describes three ways of reducing duplication. In Section~\ref{sec:iteration}, you will learn how to iteratively modify the previous plot, allowing you to build on top of your previous work without having to retype a lot of code. Section~\ref{sec:templates} will show you how to produce plot ``templates'' that encapsulate repeated components that are defined once and used in many different places. Finally, \ref{sec:functions} talks about how to create functions that create or modify plots. 

\section{Iteration}
\label{sec:iteration}

Whenever you create or modify a plot, \ggplot saves a copy of the result so you can refer to it in later expressions. You can access this plot with \f{last_plot}. This is useful in interactive work as you can start with a basic plot and then iteratively add layers and tweak the scales until you get to the final result. The following code demonstrates iteratively zooming in on a plot to find a region of interest, and then adding a layer which highlights something interesting that we have found: very few diamonds have equal x and y dimensions. The plots are shown in Figure~\ref{fig:iterate-limits}.

% FIGLISTING
%   TEX-WIDTH: 0.33\linewidth COL: 3 FILETYPE: PNG
%   LABEL: iterate-limits
%   CAPTION: When ``zooming'' in on the plot, it's useful to
%   use \f{last_plot} iteratively to quickly find the best view.  The final
%   plot adds a line with slope 1 and intercept 0, confirming it is the square
%   diamonds that are missing.
% 
% qplot(x, y, data = diamonds, na.rm = TRUE)
% last_plot() + xlim(3, 11) + ylim(3, 11)
% last_plot() + xlim(4, 10) + ylim(4, 10)
% last_plot() + xlim(4, 5) + ylim(4, 5)
% last_plot() + xlim(4, 4.5) + ylim(4, 4.5)
% last_plot() + geom_abline(colour = "red")
\input{_include/3104147103fb5054b627d22fb0cd4798.tex}
% END

Once you have tweaked the plot to your liking, it's a good idea to go back and create a single expression that generates your final plot. This is important as when you come back to the plot, you'll be able to re-create the plot quickly, without having to step through your original process. You many want to add a comment to your code to indicate exactly why you chose that final plot. This is good practice in general for R code: after experimenting interactively, you always want to create a source file that re-creates your analysis.  The following code shows the final plot after our interactive modifications above.

% LISTING
% 
% qplot(x, y, data = diamonds, na.rm = T) + 
%   geom_abline(colour = "red") +
%   xlim(4, 4.5) + ylim(4, 4.5)
\input{_include/a47cb4a2a7c5845dba7ea8caccdb373b.tex}
% END

\section{Plot templates}
\label{sec:templates}

Each component of a \ggplot plot is its own object and can be created, stored and applied independently to a plot. This makes it possible to create reusable components that can automate common tasks and helps to offset the cost of typing the long function names. The following example creates some colour scales and then applies them to plots. The results are shown in Figure~\ref{fig:gradient-rb}

% FIGLISTING
%   LABEL: gradient-rb
%   CAPTION: Saving a scale to a variable makes it easy to apply exactly the
%   same scale to multiple plots.  You can do the same thing with layers and 
%   facets too.
% 
% gradient_rb <- scale_colour_gradient(low = "red", high = "blue")
% qplot(cty, hwy, data = mpg, colour = displ) + gradient_rb
% qplot(bodywt, brainwt, data = msleep, colour = awake, log="xy") +
%   gradient_rb
\input{_include/2cf0400ac5341033bf8fbb072e6ace40.tex}
% END

As well as saving single objects, you can also save vectors of \ggplot components. Adding a vector of components to a plot is equivalent to adding each component of the vector in turn. The following example creates two continuous scales that can be used to turn off the display of axis labels and ticks. You only need to create these objects once and you can apply them to many different plots, as shown in the code below and Figure~\ref{fig:quiet}.

% FIGLISTING
%   LABEL: quiet
%   CAPTION: Using ``quiet'' x and y scales removes the labels and hides
%   ticks and gridlines.
% 
% xquiet <- scale_x_continuous("", breaks = NA)
% yquiet <- scale_y_continuous("", breaks = NA)
% quiet <- c(xquiet, yquiet)
% 
% qplot(mpg, wt, data = mtcars) + quiet
% qplot(displ, cty, data = mpg) + quiet
\input{_include/4da48bb7ba0f4671db810f6729776142.tex}
% END

Similarly, it's easy to write simple functions that change the defaults of a layer. For example, if you wanted to create a function that added linear models to a plot, you could create a function like the one below. The results are shown in Figure~\ref{fig:geom-lm}.

% FIGLISTING
%   LABEL: geom-lm
%   CAPTION: Creating a custom geom function saves typing when creating
%   plots with similar (but not the same) components.
% 
% geom_lm <- function(formula = y ~ x) {
%   geom_smooth(formula = formula, se = FALSE, method = "lm")
% }
% qplot(mpg, wt, data = mtcars) + geom_lm()
% library(splines)
% qplot(mpg, wt, data = mtcars) + geom_lm(y ~ ns(x, 3))
\input{_include/5e75dfcdae152dd41b431f89a71ab1bc.tex}
% END

Depending on how complicated your function is, it might even return multiple components in a vector. You can build up arbitrarily complex plots this way, reducing duplication wherever you find it.  If you want to create a plot that combines together many different components in a pre-specified way, you might need to write a function that produces the entire plot. This is described in the next section.

\section{Plot functions}
\label{sec:functions}

If you are using the same basic plot again and again with different datasets or different parameters, it may be worthwhile to wrap up all the different options into a single function. Maybe you need to perform some data restructuring or transformation, or need to combine the data with a predefined model. In that case you will need to write a function that produces \ggplot plots. It's hard to give advice on how to go about this because there are so many different possible scenarions, but this section aims to point out some important things to think about.

\begin{itemize}
  \item Since you're creating the plot within the environment of a function, you need to be extra careful about supplying the data to \f{ggplot} as a data frame, and you need to double check that you haven't accidentally referred to any function local variables in your aesthetic mappings.

  \item If you want to allow the user to provide their own variables for aesthetic mappings, I'd suggest using \f{aes_string}.  This function works just like \f{aes}, but uses strings rather than unevaluated expressions. \code{aes_string("cty", colour = "hwy")} is equivalent to \code{aes(cty, colour = hwy)}.  Strings are much easier to work with than expressions.

  \item As mentioned in Chapter~\ref{cha:data}, you want to separate your plotting code into a function that does any data transformations and manipulations and a function that creates the plot. Generally, your plotting function should do no data manipulation, just create a plot. The following example shows one way to create parallel coordinate plot function, wrapping up the code used in Section~\ref{sub:molten_data}.

  % INTERWEAVE
  % 
  % pcp_data <- function(df) {
  %   numeric <- laply(df, is.numeric)
  %   # Rescale numeric columns
  %   df[numeric] <- colwise(range01)(df[numeric])
  %   # Add row identified
  %   df$.row <- rownames(df)
  %   # Melt, using non-numeric variables as id vars
  %   dfm <- melt(df, id = c(".row", names(df)[!numeric]))
  %   # Add pcp to class of the data frame
  %   class(dfm) <- c("pcp", class(dfm))
  %   dfm
  % }
  % pcp <- function(df, ...) {
  %   df <- pcp_data(df)
  %   ggplot(df, aes(variable, value)) + geom_line(aes(group = .row))
  % }
  % pcp(mpg)
  % pcp(mpg) + aes(colour = drv)
  \input{_include/c7d3f61c6ab6e8a643c9083972b3809e.tex}  
  % END
  
\end{itemize}

The best example of this technique is \f{qplot}, and if you're interesting in writing your own functions I strongly recommend you have a look at the source code for this function and step through it line by line to see how it works.  If you've made your way this far through the book you should have a pretty good grasp of all the \ggplot related code: most of the complexity is R tricks to correctly interpret all of the possible plot types.



\ifwhole
\else
  \nobibliography{/Users/hadley/documents/phd/references}
  \end{document}
\fi
