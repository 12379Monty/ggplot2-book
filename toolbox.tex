\providecommand{\setflag}{\newif \ifwhole \wholefalse}
\setflag
\ifwhole\else

% Typography and geometry ----------------------------------------------------
\documentclass[letterpaper]{scrbook}
\usepackage[inner=3cm,top=2.5cm,outer=3.5cm]{geometry}

\renewcommand\familydefault{bch}
\usepackage[utf8]{inputenc}
\usepackage{microtype}
\usepackage[small]{caption}
\usepackage[small]{titlesec}
\raggedbottom

% Graphics -------------------------------------------------------------------
\usepackage[pdftex]{graphicx}
\graphicspath{{_include/}}
\DeclareGraphicsExtensions{.png,.pdf}

% Code formatting ------------------------------------------------------------
\usepackage{fancyvrb}
\usepackage{courier}
\usepackage{listings}
\usepackage{color}
\usepackage{alltt}


\definecolor{comment}{rgb}{0.60, 0.60, 0.53}
\definecolor{background}{rgb}{0.97, 0.97, 1.00}
\definecolor{string}{rgb}{0.863, 0.066, 0.266}
\definecolor{number}{rgb}{0.0, 0.6, 0.6}
\definecolor{variable}{rgb}{0.00, 0.52, 0.70}
\lstset{
  basicstyle=\ttfamily,
  keywordstyle=\bfseries, 
  identifierstyle=,  
  commentstyle=\color{comment} \emph,
  stringstyle=\color{string},
  showstringspaces=false,
  columns = fullflexible,
  backgroundcolor=\color{background},
  mathescape = true,
  escapeinside=&&,
  fancyvrb
}
\newcommand{\code}[1]{\lstinline!#1!}
\newcommand{\f}[1]{\lstinline!#1()!}



% Links ----------------------------------------------------------------------

\usepackage{hyperref}
\definecolor{slateblue}{rgb}{0.07,0.07,0.488}
\hypersetup{colorlinks=true,linkcolor=slateblue,anchorcolor=slateblue,citecolor=slateblue,filecolor=slateblue,urlcolor=slateblue,bookmarksnumbered=true,pdfview=FitB}
\usepackage{url}

% Tables ---------------------------------------------------------------------
\usepackage{longtable}
\usepackage{booktabs}

% Miscellaneous --------------------------------------------------------------
\usepackage{pdfsync}
\usepackage{appendix}

\usepackage[round,sort&compress,sectionbib]{natbib}
\bibliographystyle{plainnat}


\title{ggplot2}
\author{Hadley Wickham}

\begin{document}
\fi


\chapter{Toolbox}
\label{cha:toolbox}

\section{Introduction}\label{sec:introduction}

Graphical objects, or geoms for short, are a key feature of {\tt geom\_plot}.  Geoms create visual objects on the plot that allow you to see you data.  By choosing various types of geoms, you can recreate common plots.  For example, the {\tt geom\_point} geom will make a scatterplot, and the {\tt geom\_line} geom will create a line plot.  The advantage of the geom based system is that you can easily combine different geoms to create almost any plot that you can think of.  This section describes geom functions in detail, including what geoms are currently available in {\tt geom\_plot} and how you can go about creating your own.

Use a mixture of {\tt ggplot()} and {\tt qplot()} calls.  If you need a reminder on how to translate between the two, see Section~\ref{sec:qplot-ggplot}.

\section{Basics}\label{sub:basics}

These geoms are the basic geoms used to build up almost all of the other geoms.  These are useful for creating basic graphics, and when building your own geom function (see section \ref{sec:writing_your_own}).

\begin{itemize}
  \item {\tt geom\_point}: points
  \item {\tt geom\_line}, {\tt geom\_path}: paths and lines.  Lines are paths that have their x-axis values ordered in increasing value.
  \item {\tt geom\_polygon}: polygons
  \item {\tt geom\_bar}: bars
  \item {\tt geom\_text}: text
  \item {\tt geom\_tile}: tiles, rectangles which form a regular tessellation of the plane
\end{itemize}

\section{Displaying distributions}\label{sec:distributions}

There are quite a few geoms associated with displaying distributions:

\begin{itemize}
	\item {\tt geom\_boxplot}: box and whisker plot, for a continuous variable possibly conditioned by a categorical variable
	\item {\tt geom\_jitter}: a crude way of investigating densities
	\item {\tt geom\_quantile}: quantiles, for a continuous variable conditional on another continuous variable.
	\item {\tt geom\_density}: 
	\item {\tt geom\_histogram}: 
	\item {\tt geom\_2ddensity}: for displaying the density of points on the plot surface.
\end{itemize}

(Ask Heike about this)

We have a number of plots available to investigate distributions, depending on the number of variables, and whether we are interested in the conditional or joint distribution.

\begin{Verbatim}
  
Single variable
+ cont: histogram, density plot, boxplot
+ cat:  barchart

Two variables: conditional
+ cat  | cat:  mosaic plot
+ cat  | cont: ?
+ cont | cont: quantiles
+ cat  | cont: boxplot

Two variables: joint
+ cat  * cat:  fluctuation diagram
+ cat  * cont: boxplots?
+ cont * cont: bagplot, geom\_2ddensity
\end{Verbatim}

Jittered points can be used for any joint distribution (or conditional if one or both variables are categorical)

\section{Dealing with overplotting}\label{sec:overplotting}

The simplest way to deal with overplotting is to bin the plot into small squares and count the number of points that lies in each square, much like a 2D histogram.  This count can then be visualised as the third variable on a plot.  However, breaking the plot into many small squares produces distracting visual artefacts.  Carr (reference) suggests using hexagons instead, and this is implemented with {\tt geom\_hexagon}, using the capabilities of the {\tt hexbin package}.

A continuous analogue of this is to compute a 2D density function and then visualise this as coloured tiles or contour lines.  This can be done with {\tt geom\_2ddensity}.

Another approach to dealing with overplotting is to add supplemental information to help guide the eye to the true shape of or pattern within the data:

\begin{itemize}
	\item {\tt geom\_smooth} add a smooth line showing the mean.
	\item {\tt geom\_quantile} add a smooth line showing any quantile you are interested in.
\end{itemize}

\section{``3d'' plots}

{\tt geom\_plot} currently does not support true 3D plots.  However, it does offer two tools for producing pseudo-3d plots, the imageplot and the contour plot.

\begin{itemize}
	\item {\tt geom\_tile}: map z variable to fill colour
	\item {\tt geom\_contour}: useful for smoother surfaces
	\item {\tt geom\_point}: can map abs(z) variable to size and sign(z) to colour
\end{itemize}

These are often better than ``true'' 3d for static plots anyway, because many perceptual cues necessary for accurate depth perception (eg. occlusion, parallax) are not present in static plots.  You can also manually project data points in a higher dimensional space by multiplying by a projection matrix.  However, correctly representing occlusion or generating correct perspective effects will require considerably more effort.  You may want to look at RGL (\url{http://www.rgl.com}) and rggobi (\url{http://www.ggobi.org/ggobi}) for other solutions.

\section{Revealing uncertainty}\label{sub:displaying_uncertainty}

If you have information about the uncertainty present in your data, possibly from a model or distributional assumptions, it is often useful to visualise.  There are two geoms that allow you to do this depending on whether you have point or functional confidence intervals:

\begin{itemize}
	\item {\tt geom\_errorbar}: for pointwise confidence intervals
	\item {\tt geom\_ribbon}: ribbons of variable width, useful for displaying confidence intervals around functions
\end{itemize}

There are two ways to display standard errors with {\tt ggplot}.  For point standard errors, you can use the {\tt errorbar} geom.  For continuous or functional standard errors, you can use the {\tt ribbon} grob.  We've have already seen an example of this: the {\tt ribbon} grob is used inside {\tt smooth} to display the standard errors of the smooth.  Because there are so many different ways to calculate standard errors, the calculation is up to you.  {\tt ggplot} only provides facilities for displaying them once you have them.

For both {\tt ribbon} and {\tt errobar} you can specify confidence internals using {\tt min} and {\tt lower} which specify the upper and lower edges of the confidence band.

Calculating with {\tt stat\_sum}, {\tt stat\_smooth} etc.

\section{Annotating a plot}\label{sub:annotating_a_plot}

Any of the basic geoms can be used to annotate a plot with additional output, for example, adding text with {\tt geom\_text}, or a point illustrating the mean with {\tt geom\_point}.  Additionally, there are several geoms whose use is almost entirely for annotation.  These are:

\begin{itemize}
	\item {\tt geom\_vline}, {\tt geom\_hline}: add vertical or horizontal lines to a plot
	\item {\tt geom\_abline}: add lines with arbitrary slope and intercept to a plot
\end{itemize}

See also \secref{sec:adding_annotation} for ways to add more general types of annotation using grid graphics.

\section{Plots for weighted data}\label{sec:weighted_data}

When you have aggregated data where each row in the dataset represents multiple observations, you need some way to take into account the weighting variable.  Since there are no variables appropriate for weighting in the diamonds data, we will use some data collected on Midwest states in the 2000 US census.  The data consists mainly of percentages (eg. percent white, percent below poverty line, percentage with college degree) and some information for each county (area, total population, population density).

There are few different things we might want to weight by: 

\begin{itemize}
	\item nothing, to look at county numbers
	\item total population, to work with absolute numbers
	\item area, to investigate geographic effects
\end{itemize}

\noindent The choice of a weighting variable profoundly effects what we are looking at in the plot and the conclusions that we will draw.  There are two aesthetic attributes that can be used to adjust for weights.  Firstly, for simple geoms like lines and points, you can make the size of the grob proportional to the number of points, using the {\tt size} aesthetic, as follows:

% decumar<<< 
% interweave({
% midwest <- read.csv("~/Documents/graphics/weighted/midwest.csv")
% qplot(percwhite, percbelowpoverty, data=midwest)
% qplot(percwhite, percbelowpoverty, data=midwest, size=poptotal)
% qplot(percwhite, percbelowpoverty, data=midwest, size=area)
% })
% |||
\begin{alltt}
> midwest <- read.csv("~/Documents/graphics/weighted/midwest.csv")
> qplot(percwhite, percbelowpoverty, data = midwest)
\includegraphics[scale=0.5]{4b8c600efdc1f31f77c7c3368cac11d2}

> qplot(percwhite, percbelowpoverty, data = midwest, size = poptotal)
\includegraphics[scale=0.5]{f3976daae1f4a7ef912ea259ca3dd8f5}

> qplot(percwhite, percbelowpoverty, data = midwest, size = area)
\includegraphics[scale=0.5]{440b95f469e5b96a08e077d4a177e603}

\end{alltt}
% >>>

For more complicated grobs which involve some statistical transformation, we specify weights with the {\tt weight} aesthetic.  These weights will be passed on to the statistical summary function.  Weights are supported for every case where it makes sense: smoothers, quantile regressions, box plots, histograms, and density plots.  You can't see this weighting variable directly, and it doesn't produce a legend, but it will change the results of the statistical summary.

The following example shows how weighting by population density effects the relationship between percent white and percent below the poverty line.

% decumar<<< 
% interweave({
% qplot(percwhite, percbelowpoverty, data=midwest, geom=c("point","smooth"), method=lm)
% qplot(percwhite, percbelowpoverty, data=midwest, size=popdensity, weight=popdensity,geom=c("point","smooth"), method=lm)
% })
% |||
\begin{alltt}
> qplot(percwhite, percbelowpoverty, data = midwest, geom = c("point", 
+     "smooth"), method = lm)
\includegraphics[scale=0.5]{54d5b5a95badb06ff2339bfd51826af0}

> qplot(percwhite, percbelowpoverty, data = midwest, size = popdensity, 
+     weight = popdensity, geom = c("point", "smooth"), method = lm)
\includegraphics[scale=0.5]{3353081c3ffa3fd62b69fb2c0aef9142}

\end{alltt}
% >>>

When we weight a histogram or density plot by total population, we change from looking at the distribution of the number of counties, to the distribution of the number of people.  This example shows the difference this makes for a histogram and density plot of the percentage below the poverty line.

% decumar<<< 
% interweave({
% qplot(percbelowpoverty, data=midwest, geom="histogram", binwidth=1)
% qplot(percbelowpoverty, data=midwest, geom="histogram", weight=poptotal, binwidth=1)
% })
% |||
\begin{alltt}
> qplot(percbelowpoverty, data = midwest, geom = "histogram", binwidth = 1)
\includegraphics[scale=0.5]{d8cef4ee59175bda59eaa2647a64b930}

> qplot(percbelowpoverty, data = midwest, geom = "histogram", weight = poptotal, 
+     binwidth = 1)
\includegraphics[scale=0.5]{affb33e87c76bc94bd3ea9eb2df0190c}

\end{alltt}
% >>>

\ifwhole
\else
  \nobibliography{/Users/hadley/documents/phd/references}
  \end{document}
\fi
