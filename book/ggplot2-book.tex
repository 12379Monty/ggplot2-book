\documentclass{krantz}
\usepackage[T1]{fontenc}
\usepackage{lmodern}
\usepackage{amssymb,amsmath}
\usepackage{upquote}

% artifacts left from the 1st edition
\usepackage{subfigure}
\usepackage{appendix}

\usepackage{fontspec}
\usepackage{xltxtra,xunicode}
\defaultfontfeatures{Mapping=tex-text,Scale=MatchLowercase}
\setmonofont[Mapping=tex-ansi]{Inconsolata}

\usepackage{index}
% index functions separately
\newindex{code}{adx}{and}{R code index}
\newcommand{\indexf}[1]{\index[code]{#1@\f{#1}}}
\newcommand{\indexc}[1]{\index[code]{#1@\code{#1}}}
% do we want to adopt the adv-r approach to indexing (below)?
%\newcommand{\indexc}[1]{\index{#1@\texttt{#1}}}

% Taken from pandoc x.md -o test.tex --standalone
\usepackage{color}
\usepackage{fancyvrb}
\newcommand{\VerbBar}{|}
\newcommand{\VERB}{\Verb[commandchars=\\\{\}]}
\DefineVerbatimEnvironment{Highlighting}{Verbatim}{commandchars=\\\{\}}
\newenvironment{Shaded}{}{}
\newcommand{\KeywordTok} [1]{\textcolor[rgb]{0.00,0.44,0.13}{{#1}}}
\newcommand{\DataTypeTok}[1]{\textcolor[rgb]{0.56,0.13,0.00}{{#1}}}
\newcommand{\DecValTok}  [1]{\textcolor[rgb]{0.25,0.63,0.44}{{#1}}}
\newcommand{\BaseNTok}   [1]{\textcolor[rgb]{0.25,0.63,0.44}{{#1}}}
\newcommand{\FloatTok}   [1]{\textcolor[rgb]{0.25,0.63,0.44}{{#1}}}
\newcommand{\CharTok}    [1]{\textcolor[rgb]{0.25,0.44,0.63}{{#1}}}
\newcommand{\StringTok}  [1]{\textcolor[rgb]{0.25,0.44,0.63}{{#1}}}
\newcommand{\CommentTok} [1]{\textcolor[rgb]{0.38,0.63,0.69}{{#1}}}
\newcommand{\OtherTok}   [1]{\textcolor[rgb]{0.00,0.44,0.13}{{#1}}}
\newcommand{\AlertTok}   [1]{\textcolor[rgb]{1.00,0.00,0.00}{{#1}}}
\newcommand{\FunctionTok}[1]{\textcolor[rgb]{0.02,0.16,0.49}{{#1}}}
\newcommand{\ErrorTok}   [1]{\textcolor[rgb]{1.00,0.00,0.00}{{#1}}}
\newcommand{\NormalTok}  [1]{{#1}}

\usepackage{longtable}
\usepackage{booktabs}
\usepackage{graphicx}
\usepackage{emptypage}
\raggedbottom

\usepackage[hyphens]{url}

% See comment http://tex.stackexchange.com/questions/133302/package-keyval-error-hyperref-undefined-l-4319-processkeyvaloptionshyp
\usepackage[setpagesize=false, % page size defined by xetex
            unicode=false, % unicode breaks when used with xetex
% hyperref package doesn't allow for setting hidelinks key?
% Strangely, this isn't a problem for my mac, but it is for travis
% https://travis-ci.org/hadley/ggplot2-book/builds/49660668#L3137
%            hidelinks,
            xetex]{hyperref}

% Place links as footnotes
\renewcommand{\href}[2]{#2 (\url{#1})}
% Use ref for internal links
\renewcommand{\hyperref}[2][???]{\autoref{#1}}
\def\chapterautorefname{Chapter}
\def\sectionautorefname{Section}
\def\subsectionautorefname{Section}
\def\subsubsectionautorefname{Section}

\DeclareGraphicsExtensions{.png,.pdf}

\setlength{\parindent}{0pt}
\setlength{\parskip}{6pt plus 2pt minus 1pt}
\setlength{\emergencystretch}{3em}  % prevent overfull lines
\vbadness=10000 % suppress underfull \vbox
\hbadness=10000 % suppress underfull \vbox
\hfuzz=1pt

\makeindex
\title{ggplot2}
\author{Hadley Wickham}

\begin{document}

\frontmatter
\maketitle

\setcounter{page}{7}
\cleardoublepage
\vspace*{\stretch{1}}
\hfill
\begin{minipage}[t]{0.66\textwidth}
\raggedleft
\thispagestyle{empty}
\textit{To Jeff, who makes me happy, and who made sure I had a life outside this book.}
\end{minipage}
\vspace*{\stretch{3}}
\clearpage

% \listoffigures
% \listoftables
\tableofcontents

\mainmatter

\include{Introduction}
\providecommand{\setflag}{\newif \ifwhole \wholefalse}
\setflag
\ifwhole\else

% Typography and geometry ----------------------------------------------------
\documentclass[letterpaper]{scrbook}
\usepackage[inner=3cm,top=2.5cm,outer=3.5cm]{geometry}

\renewcommand\familydefault{bch}
\usepackage[utf8]{inputenc}
\usepackage{microtype}
\usepackage[small]{caption}
\usepackage[small]{titlesec}
\raggedbottom

% Graphics -------------------------------------------------------------------
\usepackage[pdftex]{graphicx}
\graphicspath{{_include/}}
\DeclareGraphicsExtensions{.png,.pdf}

% Code formatting ------------------------------------------------------------
\usepackage{fancyvrb}
\usepackage{courier}
\usepackage{listings}
\usepackage{color}
\usepackage{alltt}


\definecolor{comment}{rgb}{0.60, 0.60, 0.53}
\definecolor{background}{rgb}{0.97, 0.97, 1.00}
\definecolor{string}{rgb}{0.863, 0.066, 0.266}
\definecolor{number}{rgb}{0.0, 0.6, 0.6}
\definecolor{variable}{rgb}{0.00, 0.52, 0.70}
\lstset{
  basicstyle=\ttfamily,
  keywordstyle=\bfseries, 
  identifierstyle=,  
  commentstyle=\color{comment} \emph,
  stringstyle=\color{string},
  showstringspaces=false,
  columns = fullflexible,
  backgroundcolor=\color{background},
  mathescape = true,
  escapeinside=&&,
  fancyvrb
}
\newcommand{\code}[1]{\lstinline!#1!}
\newcommand{\f}[1]{\lstinline!#1()!}



% Links ----------------------------------------------------------------------

\usepackage{hyperref}
\definecolor{slateblue}{rgb}{0.07,0.07,0.488}
\hypersetup{colorlinks=true,linkcolor=slateblue,anchorcolor=slateblue,citecolor=slateblue,filecolor=slateblue,urlcolor=slateblue,bookmarksnumbered=true,pdfview=FitB}
\usepackage{url}

% Tables ---------------------------------------------------------------------
\usepackage{longtable}
\usepackage{booktabs}

% Miscellaneous --------------------------------------------------------------
\usepackage{pdfsync}
\usepackage{appendix}

\usepackage[round,sort&compress,sectionbib]{natbib}
\bibliographystyle{plainnat}


\title{ggplot2}
\author{Hadley Wickham}

\begin{document}
\fi



% qplot chapter suggestions:
% 
% * exercises - see plot and then try and create on your own
% * wall chart of different plot types
% * more error bar and ribbon examples
% * table of options

% decumar<<< 
% source("~/documents/ggplot/ggplot/load.r")
% library(xtable)
% doptions(width=8, height=4.8, lscale=0.5)
% ggopt(axis.color = "black")
% >>>

%\setchapterpreamble[u]{% 
%\dictum[Friedrich Nietzsche]{He who fights with monsters might take care lest he thereby become a monster. And if you gaze for long into an abyss, the abyss gazes also into you.}} 

\chapter{Getting started with ggplot: {\tt qplot}}

\section{Introduction} 

In this chapter, you will learn to make a wide variety of plots with your first ggplot function, {\tt qplot}, short for {\bf q}uick plot. qplot makes it easy to produce complex plots, normally requiring several lines of code in R, in one line. qplot can do this because it's based on the grammar of graphics, which allows you to create a simple, yet expressive, description of the plot.  In later chapters you'll learn to use all of the expressive power of the grammar, but here we'll start simple so you can work your way up.

{\tt qplot} has been designed be very similar to {\tt plot}, which should make it easy if you're already familiar with plotting in R.  Remember, during an R session you can get a summary of all the arguments to {\tt qplot} with R help, {\tt ?qplot}.

In this chapter you'll learn:

\begin{itemize}
	\item The basic use of {\tt qplot}---If you're already familiar with {\tt plot}, this will be particularly easy. Page \pageref{sec:basic_use}.
	\item How to map variables to aesthetic attributes, like colour, size and shape. Page \pageref{sec:aesthetic_attributes}.
	\item How to create many different types of plots by specifying different geoms, and how to combine multiple types in a single plot. Page \pageref{sec:plot_geoms}.
	\item The use of facetting, also known as trellising or conditioning, to break apart subsets of your data. Page \pageref{sec:facetting}.
	\item How to tune the appearance of the plot by specifying some basic options. Page \pageref{sec:other_options}.
	\item A few important differences between {\tt plot} and {\tt plot}.  Page \pageref{sec:plot_diffs}
\end{itemize}

\section{Data sets}\label{sec:data_sets}

In this chapter we'll just use one data source, so you can get familiar with the plotting details rather than having to familiarise yourself with different datasets.  The {\tt diamonds} dataset consists of prices and quality information about 54,000 diamonds, and is included in the ggplot package.  The dataset has not been well cleaned, so as well as demonstrating interesting relationships about diamonds, it also demonstrates some data quality problems.  We'll also use another dataset, {\tt dsmall}, which is a random sample of 1000 diamonds.  We'll use this for plots which are more appropriate for smaller datasets.

The first few rows of the data are shown in \ref{tab:diamonds}.

% decumar<<< 
% set.seed(1410)
% dsmall <- diamonds[sample(nrow(diamonds), 1000),]
% xtable(head(diamonds), digits=c(0,1,2,0,1,1,1,0,2,2,2), align="l|rrrrrrrrrr", label="tab:diamonds", caption="{\\tt diamonds} dataset")
% |||
% latex table generated in R 2.5.1 by xtable 1.5-1 package
% Fri Aug 31 13:09:07 2007
\begin{table}[ht]
\begin{center}
\begin{tabular}{l|rrrrrrrrrr}
  \hline
 & carat & cut & color & clarity & depth & table & price & x & y & z \\
  \hline
1 & 0.2 & Ideal & E & SI2 & 61.5 & 55.0 & 326 & 3.95 & 3.98 & 2.43 \\
  2 & 0.2 & Premium & E & SI1 & 59.8 & 61.0 & 326 & 3.89 & 3.84 & 2.31 \\
  3 & 0.2 & Good & E & VS1 & 56.9 & 65.0 & 327 & 4.05 & 4.07 & 2.31 \\
  4 & 0.3 & Premium & I & VS2 & 62.4 & 58.0 & 334 & 4.20 & 4.23 & 2.63 \\
  5 & 0.3 & Good & J & SI2 & 63.3 & 58.0 & 335 & 4.34 & 4.35 & 2.75 \\
  6 & 0.2 & Very Good & J & VVS2 & 62.8 & 57.0 & 336 & 3.94 & 3.96 & 2.48 \\
   \hline
\end{tabular}
\caption{{\tt diamonds} dataset}
\label{tab:diamonds}
\end{center}
\end{table}
% >>>

\section{Basic use}\label{sec:basic_use}

Just like {\tt plot}, the first two arguments to {\tt qplot} are {\tt x} and {\tt y}, giving the x- and y-coordinates for the objects on the plot. There is also an optional {\tt data} argument.  If this is specified, {\tt qplot} will look inside that data frame before looking for objects in your workspace.  I recommend that you keep all the data for one plot in a data frame, instead of scatter across multiple vectors.

Here is a simple example of using {\tt qplot}, investigating the relationship between price and carats (weight) of a diamond.  

% decumar<<< 
% interweave({
% qplot(carat, price, data=diamonds)
% })
% |||
\begin{alltt}
> qplot(carat, price, data = diamonds)
\includegraphics[scale=0.5]{67565e33f3be02998261183455d5fbdf}

\end{alltt}
% >>>

You are not limited to specifying names of existing vectors: you can use functions of them as well.  Here we look at the relationship of log(price) to log(weight), and the relationship between weight and volume ($x * y * z$).

% decumar<<< 
% interweave({
% qplot(log(carat), log(price), data=diamonds)
% qplot(carat, x * y * z, data=diamonds)
% })
% |||
\begin{alltt}
> qplot(log(carat), log(price), data = diamonds)
\includegraphics[scale=0.5]{67258e4066f2cf8f7c5d41e6509d0c36}

> qplot(carat, x * y * z, data = diamonds)
\includegraphics[scale=0.5]{8d1dcad578d3f467a416826a23c73887}

\end{alltt}
% >>>

These plots illustrate some unusual features of this dataset.  There seems to be a narrow band of prices for which there are no diamonds, and there are some diamonds with very unusual volumes for their weight.  We will investigate these relationships more as we continue.

\section{Aesthetic attributes}\label{sec:aesthetic_attributes}

The first big difference with using {\tt qplot} compared to {\tt plot} comes when you want to assign colours---or sizes or shapes---to the points on your plot.  With {\tt plot}, it's your responsibility to change your data (eg. ``apples'', ``bananas'', ``pears'') into something that {\tt plot} knows how to use (eg. ``red'', ``yellow'', ``green'').  {\tt qplot} will do this for you automatically, and will automatically provide a legend to make it easier to look up the actual data values.  This makes it easy to include additional data on the plot.  

In the next example, we augment the plot of carat and price with information about colour, clarity and cut of the diamonds.

% decumar<<< 
% interweave({
% dsmall <- diamonds[sample(nrow(diamonds), 1000),]
% qplot(carat, price, data=dsmall, colour=color)
% qplot(carat, price, data=dsmall, size=clarity)
% qplot(carat, price, data=dsmall, shape=cut)
% })
% |||
\begin{alltt}
> dsmall <- diamonds[sample(nrow(diamonds), 1000), ]
> qplot(carat, price, data = dsmall, colour = color)
\includegraphics[scale=0.5]{419d737524b746a1421166a5b8c292a6}

> qplot(carat, price, data = dsmall, size = clarity)
\includegraphics[scale=0.5]{c2822a1a032f9f94e53e05329821b6fe}

> qplot(carat, price, data = dsmall, shape = cut)
\includegraphics[scale=0.5]{c5ecac79df2292776951236c805681ef}

\end{alltt}
% >>>

Colour, size and shape are all examples of aesthetic attributes.  An aesthetic attribute is some property that affects how the observations are displayed.  For every aesthetic attribute, there is some function, called a scale, which maps data values to valid values for that aesthetic.  It is this scale that controls how the points appear.  For example, in the above plots, the colour scale maps J to purple and F to green.  We will learn how to configure this in Chapter X.  Note that while I use British spelling through this book, color will also work.

Different types of aesthetic attributes work better with different types of variables.  For example, colour and shape work well with categorical variables, while size works better with continuous variables.  You can always convert a continuous variable to a categorical one using the {\tt chop} function. The amount of data also makes a difference:  size works poorly when you have a lot of data, because you can't distinguish the individual points.  These issues are discussed more in chapter X.

\section{Plot geoms}\label{sec:plot_geoms}

{\tt qplot} is not limited to just producing scatterplots.  In fact, it can produce almost any type of plots, by varying the {\bf geom} used. Geom, short for geometric object, describes the type of object that is used to display the data.  Some geoms have an associated statistical transformation, for example, a histogram is a binning statistic plus a bar geom.  These different components are described in the next chapter.  Here we'll introduce you to the most common and useful geoms, categorised by whether they are useful for 1D or 2D data.

The following geoms enable you to investigate 2D relationships:

\begin{itemize}
	\item {\tt geom="point"} draws points to produce a scatterplot (the default), as described above.
	\item {\tt geom="smooth"} fits a smoother to the data and displays the smooth and its standard error.
	\item {\tt geom="quantiles"} displays conditional density estimates.  You can think of this as a extension of boxplots to deal with the case of a continuous conditioning variable.
	\item {\tt geom="density2d"} adds contours of a 2d density estimate.  This is very useful when you have a lot of overplotting.
	\item {\tt geom="path"} and {\tt geom="line"} draw lines between the data points.  Traditionally these are used to explore relationships between time and another variable, but lines may to be use to join observations connected in some other way.  A line plot is constrained to produce lines that travel from left to right, while paths can go in any direction.
	\item {\tt geom="boxplot"} produces a box and whisker plot to summarise the distribution of a set of points.
  % \item {\tt geom="errorbar"} adds error bars to help indicate uncertainty associated with measurements.
\end{itemize}

These geoms are useful for exploring 1d distributions:

\begin{itemize}
	\item {\tt geom="histogram"} draws a histogram of the $x$ variable (continuous or categorical)
	\item {\tt geom="density"} creates a density plot for the $x$ variable (continuous only)
  % \item {\tt geom="bar"} makes a barchart.
\end{itemize}

Other types of plots for dealing with categorical data (bar charts, mosaic plots, spineplots, spinograms) are dealt with in chapter X, using the special categorical plotting function {\tt catplot}.

\subsection{Adding a smoother to a plot}\label{sub:smooth}

If you have a scatterplot with many data points, it can be hard to see exactly what trend is shown by the data.  In this case you may want to add a smoothed line to the plot.  This is easily done using the {\tt smooth} geom:

% decumar<<< 
% interweave({
% qplot(carat, price, data=dsmall, geom=c("smooth", "point"))
% })
% |||
\begin{alltt}
> qplot(carat, price, data = dsmall, geom = c("smooth", "point"))
\includegraphics[scale=0.5]{d2a845e76d14b06b6370b0061bdc6ae1}

\end{alltt}
% >>>

Notice that we have combined multiple geoms by supplying a vector of geom names to {\tt qplot}.  The geoms will be overlaid in the order that you specified them.

There are many different smoothers you can choose using the {\tt method} argument:

\begin{itemize}
	\item {\tt method="loess"}, the default, uses a smooth local regression.  More details about the algorithm used can be found in {\tt ?loess}.  You can modify the wiggliness of the line by varying the span between 0, exceeding wiggly, and 1, not so wiggly.  Loess does not work well for large datasets (it's $O(n^2)$ in memory), so you'll need to use one of the other methods listed below.

  % decumar<<< 
  % interweave({
  % qplot(carat, price, data=dsmall, geom=c("smooth", "point"), span=0.1)
  % qplot(carat, price, data=dsmall, geom=c("smooth", "point"), span=1)
  % })
  % |||
\begin{alltt}
> qplot(carat, price, data = dsmall, geom = c("smooth", "point"))
\end{alltt}
\begin{alltt}

> qplot(carat, price, data = dsmall, geom = c("smooth", "point"), 
+     span = 0.2)
\end{alltt}
\begin{alltt}

> qplot(carat, price, data = dsmall, geom = c("smooth", "point"), 
+     span = 1)
\end{alltt}
\begin{alltt}

\end{alltt}
  % >>>

	\item {\tt method="lm"} fits a linear model.  The default will fit a straight line to your data, or you can specify {\tt formula = y $\sim$ poly(x, 2)} to specify a degree 2 polynomial, or better, load the {\tt splines} library and use a natural spline: {\tt formula = y $\sim$ ns(x, 2)}. The second parameter is the degrees of freedom: a higher number will create a wigglier curve. You are free to specify any formula involving $x$ and $y$.  

  % decumar<<< 
  % interweave({
  % qplot(carat, price, data=dsmall, geom=c("smooth", "point"), method="lm")
  % library(splines)
  % qplot(carat, price, data=dsmall, geom=c("smooth", "point"), method="lm", formula=y ~ ns(x,3))
  % })
  % |||
\begin{alltt}
> qplot(carat, price, data = dsmall, geom = c("smooth", "point"), 
+     method = "lm")
\end{alltt}
\begin{alltt}

> library(splines)
> qplot(carat, price, data = dsmall, geom = c("smooth", "point"), 
+     method = "lm", formula = y ~ ns(x, 3))
\end{alltt}
\begin{alltt}

\end{alltt}
  % >>>

	\item {\tt method="rlm"} works the same as {\tt lm}, but uses a robust fitting algorithm so that outliers don't affect the fit as much.  It's part of the {\tt MASS} package, so remember to load that first.

	\item You could also load the {\tt mgcv} library and use {\tt method="gam", formula = y $\sim$ s(x)} to fit a generalised additive model.  This is similar 
to using a spline with {\tt lm}, but the degree of smoothness is estimated from the data.  For large data, you should use the formula {\tt y $\sim$ s(x, bs="cr")}
 
  % decumar<<< 
  % interweave({
  % library(mgcv)
  % qplot(carat, price, data=dsmall, geom=c("smooth", "point"), method="gam", formula= y ~ s(x))
  % qplot(carat, price, data=diamonds, geom=c("smooth", "point"), method="gam", formula= y ~ s(x, bs="cr"))
  % })
  % |||
\begin{alltt}
> library(mgcv)
> qplot(carat, price, data = dsmall, geom = c("smooth", "point"), 
+     method = "gam", formula = y ~ s(x))
\end{alltt}
\begin{alltt}

> qplot(carat, price, data = diamonds, geom = c("smooth", "point"), 
+     method = "gam", formula = y ~ s(x, bs = "cr"))
\end{alltt}
\begin{alltt}

\end{alltt}
  % >>>

\end{itemize}

By default, the standard errors are shown with a grey band around the smoother.  If you want to turn them off, use {\tt se=FALSE}.

\subsection{Quantiles}\label{sub:quantiles}

A smoother displays a smoothed conditional mean.  It's often useful to see a smooth estimate of other summaries of the distribution, for example the upper and lower quartiles to show the spread of the data.  We can do this with quantile regression \citep{koenker:2005}, which is basically a process to estimate smoothed conditional quantiles.  The types of smoothers we can use is much more limited compared to {\tt geom="smooth"}, but we can learn more about the conditional distribution.

For this example, we're going to zoom into a small range of diamond sizes so we can look at the distribution more closely.

% decumar<<< 
% interweave({
% dlittle <- subset(diamonds, carat < 2)
% qplot(carat, price, data=dlittle, geom=c("point", "quantile"))
% })
% |||
\begin{alltt}
> dlittle <- subset(diamonds, carat < 2)
> qplot(carat, price, data = dlittle, geom = c("point", "quantile"))
\includegraphics[scale=0.5]{bc9207316bc946500aa0332f21fc0c0f}

\end{alltt}
% >>>

By default, the relationship between x and y is assumed to be linear, but we can adjust this using the {\tt formula} argument, in the same way that we could for smoothers.  In this example we'll try using a natural spline, and then manually adding some change points where we can see an obvious jump on the graph.

% decumar<<< 
% interweave({
% qplot(carat, price, data=dlittle, geom=c("point", "quantile"), formula=y~ns(x, 3))
% qplot(carat, price, data=dlittle, geom=c("point", "quantile"), formula=y~ns(x, 3) + I(x > 1) + I(x > 1.5))
% })
% |||
\begin{alltt}
> qplot(carat, price, data = dlittle, geom = c("point", "quantile"), 
+     formula = y ~ ns(x, 3))
\includegraphics[scale=0.5]{c93ebaa7d8baa1143f55e3f8dfb8912a}

> qplot(carat, price, data = dlittle, geom = c("point", "quantile"), 
+     formula = y ~ ns(x, 3) + I(x > 1) + I(x > 1.5))
\includegraphics[scale=0.5]{e0c544942e136b2dfa4e22e131767df8}

\end{alltt}
% >>>

You can also adjust the quantiles displayed with the {\tt quantiles} argument:

% decumar<<< 
% interweave({
% qplot(carat, price, data=dlittle, geom=c("point", "quantile"), formula=y~ns(x, 3) + I(x > 1) + I(x > 1.5), quantiles=seq(0.05,0.95, 0.1))
% })
% |||
\begin{alltt}
> qplot(carat, price, data = dlittle, geom = c("point", "quantile"), 
+     formula = y ~ ns(x, 3) + I(x > 1) + I(x > 1.5), quantiles = seq(0.05, 
+         0.95, 0.1))
\includegraphics[scale=0.5]{819810e4230c08be87049d5a4d39b377}

\end{alltt}
% >>>

\subsection{2d density contours}

When there is a lot of over-plotting on a plot, it is very hard to judge the relative density of points.  One way to get around this is to supplement the plot with contour lines from a 2d density estimate.  This is shown below.

% decumar<<< 
% interweave({
% qplot(log(carat), log(price), data=diamonds, geom=c("point","density2d"))
% })
% |||
\begin{alltt}
> qplot(log(carat), log(price), data = diamonds, geom = c("point", 
+     "density2d"))
\includegraphics[scale=0.5]{26a4f04e6e8fb8f1a2fc7c6537bb42e3}

\end{alltt}
% >>>

The density estimation is described in more detail in {\tt ?kde2d}.

\subsection{Time series with line and path plots}\label{sub:line_plot}

Line and path plots are typically used for time series data.  Line plots always join the points from left to right, while path plots join them in the order that they appear in the data set (a line plot is just a path plot of the data sorted by x value).  Line plots usually have time on the x-axis, showing how a single variable has changed over time.  Path plots show how two variables have simultaneously changed over time, with time encoded in the way that the points are joined together.

Because there isn't a time variable in the diamonds data, we will use another dataset, {\tt economics}, which contains some economic data on the US measured over the last 40 years.

Let's start with a plot of unemployment over time, first as percent of people unemployed, and second as the median number of weeks unemployed:

% decumar<<< 
% interweave({
% qplot(date, unemploy/pop, data=economics, geom="line")
% qplot(date, uempmed, data=economics, geom="line")
% })
% |||
\begin{alltt}
> qplot(date, unemploy/pop, data = economics, geom = "line")
\includegraphics[scale=0.5]{85d2f6a1805f6fae33aa5982aee5d4d9}

> qplot(date, uempmed, data = economics, geom = "line")
\includegraphics[scale=0.5]{aafe21785c331b694bcecd1effc44782}

\end{alltt}
% >>>

These plots show the behaviour of percent unemployed, and length of unemployment individually, but it is not so easy to see the joint pattern.  Each time point occupies a point on the 2d grid of length and number, and if we joint up each point to its neighbours we get a 2d trajectory.

This can be illustrated with a path plot.  Below we plot number of unemployed vs length of unemployment and then join the individual observations with a path to show the pattern over time.  To make it more obvious in which direction time flows, we can use the {\tt size} aesthetic, as in the second plot.  We can see that number of unemployed and length of unemployment seems highly correlated, although in recent years the length of unemployment has been increasing relative to the total number of unemployed.

% decumar<<< 
% interweave({
% year <- function(x) as.POSIXlt(x)$year + 1900
% qplot(unemploy/pop, uempmed, data=economics, geom="path")
% year <- function(x) as.POSIXlt(x)$year + 1900
% qplot(unemploy/pop, uempmed, data=economics, geom="path", size=year(date))
% })
% |||
\begin{alltt}
> year <- function(x) as.POSIXlt(x)$year + 1900
> qplot(unemploy/pop, uempmed, data = economics, geom = "path")
\includegraphics[scale=0.5]{59dbe0b1a85f9706d42440adace33bed}

> year <- function(x) as.POSIXlt(x)$year + 1900
> qplot(unemploy/pop, uempmed, data = economics, geom = "path", 
+     size = year(date))
\includegraphics[scale=0.5]{71fb04fc975f436840af76b6f52734a3}

\end{alltt}
% >>>

\subsection{Boxplots and jittered points}\label{sub:boxplot}

If you have one categorical variable, and one or more continuous variables, you will probably be interested to know how the values of the continuous variables vary with the categorical.  Box plots and jittered points offer to ways to do this.  

This example looks at how the price per carat varies with colour of the diamond.

% decumar<<< 
% interweave({
% qplot(color, price/carat, data=diamonds, geom="jitter")
% qplot(color, price/carat, data=diamonds, geom="boxplot")
% })
% |||
\begin{alltt}
> qplot(color, price/carat, data = diamonds, geom = "jitter")
\includegraphics[scale=0.5]{9f0a578f1aab3347c42ffde9dd3ce49b}

> qplot(color, price/carat, data = diamonds, geom = "boxplot")
\includegraphics[scale=0.5]{8db4c38efb75de4185c91fadaaa83a92}

\end{alltt}
% >>>

% Both boxplots and jittered points will try to guess which orientation they should lie, and while most of the time they will get it right, sometimes you will need to say exactly what you want.  For the boxplot, use {\tt orientation="horizontal"} or {\tt orientation="vertical"}, and for the jittered points, use {\tt xjitter} and {\tt yjitter} to specify the amount of jittering to use. 
% 
For jittered points, you have the same control over aesthetics as you do with a normal scatterplot: {\tt size}, {\tt colour}, {\tt shape}.  The options for boxplots are more limited (and it is hard to imagine when they would be useful), you can control only the outline colour ({\tt colour}) and the internal fill {\tt fill}.

Another way to look at conditional distributions is to plot a separate histogram or density plot for each value of the categorical variable.

\subsection{Histogram and density plots}\label{sub:density}

Histogram and density plots show the distribution of a single variable.  They provide more information about the distribution of a single group than boxplots do, but it is harder to compare many groups (although we will look at one way to do so).  The next example shows the distribution of carats with a histogram and a density plot.

% decumar<<< 
% interweave({
% qplot(carat, data=diamonds, geom="histogram")
% qplot(carat, data=diamonds, geom="density")
% })
% |||
\begin{alltt}
> qplot(carat, data = diamonds, geom = "histogram")
\includegraphics[scale=0.5]{e068762d4226b413f0b71e1738686a59}

> qplot(carat, data = diamonds, geom = "density")
\includegraphics[scale=0.5]{01e17160ad313fe26a70b3fef6994c34}

\end{alltt}
% >>>

You can control the amount of smoothing using the {\tt binwidth} argument for the histogram, which specifies the bin size to use.  You can also specify the break points explicitly, using the {\tt breaks} argument.  For the density plot, you can use {\tt adjust} argument, which adjusts the bandwidth of the smoother (high values of {\tt adjust} produce smoother plots).  It is {\bf very important} to experiment with the level of smoothing.  With a histogram you should try many bin widths before you decide on the one (or two, or three) which best describe the data.

For this example, we need to use a rather small binwidth before we can see the full story.

% decumar<<< 
% interweave({
% qplot(carat, data=diamonds, geom="histogram", binwidth=1)
% qplot(carat, data=diamonds, geom="histogram", binwidth=0.1)
% qplot(carat, data=diamonds, geom="histogram", binwidth=0.01)
% })
% |||
\begin{alltt}
> qplot(carat, data = diamonds, geom = "histogram", binwidth = 1)
\includegraphics[scale=0.5]{b02975da39d590a5bc395634bf8ef9f3}

> qplot(carat, data = diamonds, geom = "histogram", binwidth = 0.1)
\includegraphics[scale=0.5]{937172b313266d920991a83d53e8f578}

> qplot(carat, data = diamonds, geom = "histogram", binwidth = 0.01)
\includegraphics[scale=0.5]{4c744daba465dca205fe47480cc813a7}

\end{alltt}
% >>>

Density plots are useful in that they are easier to overlay, but are generally less flexible than histograms, and it is more difficult to understand exactly what a density plot is showing.  (And we are {\bf not} trying to estimate a density: we're trying to see what's going on with our data)  If you want to compare the distributions of different subgroups, all you need to do is add an aesthetic mapping that differentiates the different groups, as follows:

% decumar<<< 
% interweave({
% qplot(carat, data=diamonds, geom="density", colour=color)
% })
% |||
\begin{alltt}
> qplot(carat, data = diamonds, geom = "density", colour = color)
\includegraphics[scale=0.5]{b7a522d2c46281ca40c25ac307e2f0db}

\end{alltt}
% >>>

% \subsection{Displaying uncertainty with error bars and ribbons}\label{sub:error_bars}
% 
% + Use with statistics
% 
% There are two ways to display standard errors with {\tt ggplot}.  For point standard errors, you can use the {\tt errorbar} geom.  For continuous or functional standard errors, you can use the {\tt ribbon} grob.  We've have already seen an example of this: the {\tt ribbon} grob is used inside {\tt smooth} to display the standard errors of the smooth.  Because there are so many different ways to calculate standard errors, the calculation is up to you.  {\tt ggplot} only provides facilities for displaying them once you have them.
% 
% For both {\tt ribbon} and {\tt errobar} you can specify confidence internals in two ways:
% 
% \begin{itemize}
%   \item using {\tt upper} and {\tt lower} which specify the upper and lower edges of the confidence band
% 
%   \item using {\tt y}, {\tt plus} and {\tt minus} which specify the estimate and positive and negative displacements (if you only specify one of plus and minus, the other will default to the negative of the one that is supplied)
% \end{itemize}
% 
% [Need an example here]

\section{Plots for weighted data}\label{sec:weighted_data}

When you have aggregated data where each row in the dataset represents multiple observations, you need some way to take into account the weighting variable.  Since there are no variables appropriate for weighting in the diamonds data, we will use some data collected on Midwest states in the 2000 US census.  The data consists mainly of percentages (eg. percent white, percent below poverty line, percentage with college degree) and some information for each county (area, total population, population density).

There are few different things we might want to weight by: 

\begin{itemize}
	\item nothing, to look at county numbers
	\item total population, to work with absolute numbers
	\item area, to investigate geographic effects
\end{itemize}

\noindent The choice of a weighting variable profoundly effects what we are looking at in the plot and the conclusions that we will draw.  There are two aesthetic attributes that can be used to adjust for weights.  Firstly, for simple geoms like lines and points, you can make the size of the grob proportional to the number of points, using the {\tt size} aesthetic, as follows:

% decumar<<< 
% interweave({
% midwest <- read.csv("~/Documents/graphics/weighted/midwest.csv")
% qplot(percwhite, percbelowpoverty, data=midwest)
% qplot(percwhite, percbelowpoverty, data=midwest, size=poptotal)
% qplot(percwhite, percbelowpoverty, data=midwest, size=area)
% })
% |||
\begin{alltt}
> midwest <- read.csv("~/Documents/graphics/weighted/midwest.csv")
> qplot(percwhite, percbelowpoverty, data = midwest)
\includegraphics[scale=0.5]{4b8c600efdc1f31f77c7c3368cac11d2}

> qplot(percwhite, percbelowpoverty, data = midwest, size = poptotal)
\includegraphics[scale=0.5]{f3976daae1f4a7ef912ea259ca3dd8f5}

> qplot(percwhite, percbelowpoverty, data = midwest, size = area)
\includegraphics[scale=0.5]{440b95f469e5b96a08e077d4a177e603}

\end{alltt}
% >>>

For more complicated grobs which involve some statistical transformation, we specify weights with the {\tt weight} aesthetic.  These weights will be passed on to the statistical summary function.  Weights are supported for every case where it makes sense: smoothers, quantile regressions, box plots, histograms, and density plots.  You can't see this weighting variable directly, and it doesn't produce a legend, but it will change the results of the statistical summary.

The following example shows how weighting by population density effects the relationship between percent white and percent below the poverty line.

% decumar<<< 
% interweave({
% qplot(percwhite, percbelowpoverty, data=midwest, geom=c("point","smooth"), method=lm)
% qplot(percwhite, percbelowpoverty, data=midwest, size=popdensity, weight=popdensity,geom=c("point","smooth"), method=lm)
% })
% |||
\begin{alltt}
> qplot(percwhite, percbelowpoverty, data = midwest, geom = c("point", 
+     "smooth"), method = lm)
\includegraphics[scale=0.5]{54d5b5a95badb06ff2339bfd51826af0}

> qplot(percwhite, percbelowpoverty, data = midwest, size = popdensity, 
+     weight = popdensity, geom = c("point", "smooth"), method = lm)
\includegraphics[scale=0.5]{3353081c3ffa3fd62b69fb2c0aef9142}

\end{alltt}
% >>>

When we weight a histogram or density plot by total population, we change from looking at the distribution of the number of counties, to the distribution of the number of people.  This example shows the difference this makes for a histogram and density plot of the percentage below the poverty line.

% decumar<<< 
% interweave({
% qplot(percbelowpoverty, data=midwest, geom="histogram", binwidth=1)
% qplot(percbelowpoverty, data=midwest, geom="histogram", weight=poptotal, binwidth=1)
% })
% |||
\begin{alltt}
> qplot(percbelowpoverty, data = midwest, geom = "histogram", binwidth = 1)
\includegraphics[scale=0.5]{d8cef4ee59175bda59eaa2647a64b930}

> qplot(percbelowpoverty, data = midwest, geom = "histogram", weight = poptotal, 
+     binwidth = 1)
\includegraphics[scale=0.5]{affb33e87c76bc94bd3ea9eb2df0190c}

\end{alltt}
% >>>

\section{Facetting}\label{sec:facetting}

{\bf rewrite!}

Facetting, also known as trellising or conditioning, allows you to display small multiples of subsets of your data.  This example displays a histogram of price for each value of cut:

% decumar<<< 
% interweave({
% qplot(price, data=diamonds, facets= cut~ ., geom="histogram")
% })
% |||
\begin{alltt}
> qplot(price, data = diamonds, facets = cut ~ ., geom = "histogram")
\includegraphics[scale=0.5]{c0126e55453f6c8b2335f12d9fc3a755}

\end{alltt}
% >>>

Each small multiple is called a facet, and contains the same plot for a different subset of the data.  The grid is specified with a facetting formula which looks like $row\_var \sim col\_var $.  You can specify as many row and column variables as you like, but in most cases more than one or two variables will produce a plot so large that it is difficult to see on screen.  If you want to facet on columns, or rows, not both, you can use {\tt .} as a place holder.  For example, $row\_var \sim .$ will facet by rows with a single variable.  

% decumar<<< 
% interweave({
% qplot(price, data=diamonds, facets= cut~ clarity, geom="histogram")
% })
% |||
\begin{alltt}
> qplot(price, data = diamonds, facets = cut ~ clarity, geom = "histogram")
\includegraphics[scale=0.5]{987289d85e3df1750d63394cd3c917d1}

\end{alltt}
% >>>

Facetting is used to investigate conditional relationships, e.g. conditional on sex, what is the relationship between amount of smoking and lung cancer.  Facetting can also be useful for creating tables of graphics.  For some examples of this, and more ways to use facetting, see chapter XXX.

\subsection{Margins}\label{sub:margins}

Facetting a plot is like creating a contingency table.  In contingency tables it is often useful to display marginal totals (totals over a row or column) as well as the individual cells.  It is also useful to be able to do this with graphics.  We can produce graphical margins using the the {\tt margins} argument.  This allows you to compare the conditional patterns with the marginal patterns.

You can either specify that all margins should be displayed, using {\tt margins = TRUE}, or by listing the names of the variables that you want margins for, {\tt margins = c("sex","age")}.  You can also use \verb|"grand_row"| or \verb|"grand_col"| to produce grand row and grand column margins respectively.

This example shows how the margins appear.  In the first plot, there are no margins, and we only see conditional plots.  In the second example, we see margins over columns, but not rows, and in the final example we see all possible margins.  The facet in the lower right corner displays all data points.

% decumar<<< 
% interweave({
% qplot(price, data=diamonds, facets= cut~ ., geom="histogram")
% })
% |||
\begin{alltt}
> qplot(price, data = diamonds, facets = cut ~ ., geom = "histogram")
\includegraphics[scale=0.5]{1a07c6d5903b310295255cfcfa587a9a}

\end{alltt}
% >>>

Plots with many facets and margins may be more appropriate for printing, rather than on screen display, as the higher resolution allows you to compare many more subsets.

\section{Other options}\label{sec:other_options}

There are a few other options that {\tt qplot} provides to control the output of your graphic.  These all have the same effect as their {\tt plot} equivalents:

\begin{itemize}
	\item {\tt xlim}, {\tt ylim}: set limits for the x- and y-axes, each a numeric vector of length two, e.g. {\tt xlim=c(0, 20)} or {\tt ylim=c(-0.9, -0.5)}.
	\item {\tt log}: a character vector indicating which (if any) axes should be logged.  For example, {\tt log="x"} will log the x-axis, {\tt log="xy"} will log both.
	\item {\tt main}: main title for the plot, displayed in large text at the top-centre of the plot.  This can be a string (eg. {\tt main="plot title"}) or an expression (eg. {\tt main = expression(beta[1] == 1)}).  See {\tt ?plotmath} for more examples of using mathematical formulae.
	\item {\tt xlab}, {\tt ylab}: labels for the x- and y-axes.  As with the plot title, these can be character strings or mathematical expressions.
\end{itemize}

The following examples show the options in action.

% decumar<<< 
% interweave({
% qplot(carat, price, data=diamonds, xlab="Price ($)", ylab="Weight (carats)",  main="Price-weight relationship")
% qplot(carat, price/carat, data=diamonds, ylab=expression(frac(price,carat)), xlab="Weight (carats)",  main="Small diamonds", xlim=c(.2,1))
% qplot(carat, price, data=diamonds, log="xy")
% })
% |||
\begin{alltt}
> qplot(carat, price, data = diamonds, xlab = "Price ($)", ylab = "Weight (carats)", 
+     main = "Price-weight relationship")
\includegraphics[scale=0.5]{6011083a65e420e845f68d6cef1d6635}

> qplot(carat, price/carat, data = diamonds, ylab = expression(frac(price, 
+     carat)), xlab = "Weight (carats)", main = "Small diamonds", 
+     xlim = c(0.2, 1))
\includegraphics[scale=0.5]{5995d7f1798efade38d712ca267a3162}

> qplot(carat, price, data = diamonds, log = "xy")
\includegraphics[scale=0.5]{503c1c92c142101d06b567cd7b9fc545}

\end{alltt}
% >>>

\section{Differences from plot}
\label{sec:plot_diffs}

There are a few important differences between {\tt plot} and {\tt qplot}:

\begin{itemize}
  \item Because {\tt qplot} can produce both 1d and 2d plots, you must always specify both x and y for 2d plots.  This is different to the behaviour of plot, which uses {\tt seq_along(y)} for x if it is not explicitly specified.
  
  \item {\tt qplot} is not generic: you can not pass any type of R object to qplot and expect to get some kind of default plot.  Note that, however, the {\tt ggplot()} is generic, and may provide a starting point for producing visualisations of arbitrary R objects.
  
  \item ggplot2 doesn't have subtitles, so there is no {\tt sub} parameter.    
  
  \item While you can continue to use the base R aesthetic names ({\tt col},  {\tt pch}, {\tt cex} etc), it's a good idea to switch to the more descriptive ggplot2 aesthetic names ({\tt colour} , {\tt shape} and {\tt size}).

  \item If you want to add extra points or lines or text to an existing plot from base graphics you can use the {\tt points()}, {\tt lines()} and {\tt text()} functions to draw on top of the current plot.  With {\tt ggplot2} you need to add additional {\bf layers} to the existing plot, described in the next chapter.
  
\end{itemize}

\ifwhole
\else
	\bibliography{bibliography}
  \end{document}
\fi

\providecommand{\setflag}{\newif \ifwhole \wholefalse}
\setflag
\ifwhole\else

% Typography and geometry ----------------------------------------------------
\documentclass[letterpaper]{scrbook}
\usepackage[inner=3cm,top=2.5cm,outer=3.5cm]{geometry}

\renewcommand\familydefault{bch}
\usepackage[utf8]{inputenc}
\usepackage{microtype}
\usepackage[small]{caption}
\usepackage[small]{titlesec}
\raggedbottom

% Graphics -------------------------------------------------------------------
\usepackage[pdftex]{graphicx}
\graphicspath{{_include/}}
\DeclareGraphicsExtensions{.png,.pdf}

% Code formatting ------------------------------------------------------------
\usepackage{fancyvrb}
\usepackage{courier}
\usepackage{listings}
\usepackage{color}
\usepackage{alltt}


\definecolor{comment}{rgb}{0.60, 0.60, 0.53}
\definecolor{background}{rgb}{0.97, 0.97, 1.00}
\definecolor{string}{rgb}{0.863, 0.066, 0.266}
\definecolor{number}{rgb}{0.0, 0.6, 0.6}
\definecolor{variable}{rgb}{0.00, 0.52, 0.70}
\lstset{
  basicstyle=\ttfamily,
  keywordstyle=\bfseries, 
  identifierstyle=,  
  commentstyle=\color{comment} \emph,
  stringstyle=\color{string},
  showstringspaces=false,
  columns = fullflexible,
  backgroundcolor=\color{background},
  mathescape = true,
  escapeinside=&&,
  fancyvrb
}
\newcommand{\code}[1]{\lstinline!#1!}
\newcommand{\f}[1]{\lstinline!#1()!}



% Links ----------------------------------------------------------------------

\usepackage{hyperref}
\definecolor{slateblue}{rgb}{0.07,0.07,0.488}
\hypersetup{colorlinks=true,linkcolor=slateblue,anchorcolor=slateblue,citecolor=slateblue,filecolor=slateblue,urlcolor=slateblue,bookmarksnumbered=true,pdfview=FitB}
\usepackage{url}

% Tables ---------------------------------------------------------------------
\usepackage{longtable}
\usepackage{booktabs}

% Miscellaneous --------------------------------------------------------------
\usepackage{pdfsync}
\usepackage{appendix}

\usepackage[round,sort&compress,sectionbib]{natbib}
\bibliographystyle{plainnat}


\title{ggplot2}
\author{Hadley Wickham}

\begin{document}
\fi


\chapter{Mastering the grammar}
\label{cha:mastery}

% Introduction to the components of the grammar
% Introduction to the data structure
% Roadmap for next few chapters

\section{Introduction}\label{sec:introduction}

You can choose to use just {\tt qplot}, without any understanding of the underlying grammar, but you will not be able to use the full power of ggplot.  By learning more about the grammar, and the components that make it up, you will be able to create a wider range of plots, as well as being able to combine multiple sources of data, and customise to your heart's content.

This chapter describes the theoretical basis of ggplot2: the layered grammar of graphics, a based based on Wilkinson's grammar of graphics \citep{wilkinson:2006}.  The next chapters then describe parts of the grammar in more details: layers (geoms and stats), scales, and positioning (coordinate systems, faceting and position adjustments),

Pull more from layered grammar paper, but needs to be rather different.

\section{Building a plot}
\label{sec:building_a_plot}

Consider the mammal's sleep dataset illustrated in Table~\ref{tbl:sleep}

\begin{alltt}
  qplot(bodywt, brainwt, data=msleep)
  qplot(bodywt, brainwt, data=msleep, log="xy")
  qplot(bodywt, brainwt, data=msleep, log="xy", facets = . ~ sleepyhead)
\end{alltt}

Pick out 5 animals to use throughout in tables.

\section{What is a plot?}
\label{sec:what_is_a_plot}

One way to think about the grammar of graphics is as a question: what is a plot?  The grammar answers this by describing a plot as a collection of independent components, each describing an independent part of the plot.  There are three basic things we need for a plot: one or more layers, scales to map variables from data space to visual space, and a coordinate system.  These are described below.

\begin{itemize}
  \item One or more layers.  A layer is composed of data and a description of which data variables should be mapped to which aesthetic properties, a geometric object, and a statistical transformation:
  
  \begin{itemize}
  	\item Data is obviously the most important part, and it is what you provide.  This is what you are displaying visually to aid communication or analysis.  You also need to describe how variables in the dataset are mapped to visual properties.  For example, in Figure 2.X we mapped diamond price to y position, carat to y position and colour to colour.  Because the data and aesthetic mapping set is usually the same in most layers, these can also be set as defaults at the plot level.
  	
  	\item {\bf Geoms}, short for geometric objects, control the type of plot that you create.  For example, using a point geom will create a scatterplot, while using a line geom will create a line plot.

  	\item {\bf Stats}, or statistical transformations, reduce or augment the data in a statistical manner.  For example, a useful stat is the smoother, which shows the mean of y, conditional on x.  Another common stat is the binner, which bins data in to bins.   Every geom has a default statistic, and every statistic a default geom.  For example, the bin statistic has defaults to using the bar geom to produce a histogram.

  	\item {\bf Position adjustment}
  \end{itemize}

  \item A scale for all the aesthetic properties.  {\bf Scales} control the mapping from data attributes to aesthetic attributes.  They also provide an inverse mapping in the form of a guide, an axis or legend, which facilitates reading the final graphic.  Aesthetic attributes are things like position, size, colour---anything that you can perceive.  The function that maps data to aesthetic attributes is a scale. It takes values in data space (continuous or categorical) and maps them into an aesthetic space (eg. colour, size, shape).  A scale also provides guides to convert back from the aesthetic attribute to the original data.  Guides are either axes (for position) or legends (for everything else)

  \item A coordinate system.  A {\bf coord}, or coordinate systems, maps the position of objects on to the plane of the plot.  Typically we will use the cartesian coordinate system, but sometimes others are useful.
\end{itemize}

There is also another thing that turns out to be sufficiently useful that we should include it in our general framework: faceting (also known as conditioned or trellis plots). This allows us to easily create small multiples of different subsets of an entire dataset. This is a powerful tool when investigating whether patterns hold across all conditions.


\section{Data structures}
\label{sec:data_structures}

These principles are encoded as data structures in a fairly straightforward way.

One thing to note is that all ggplot2 objects (with the exception of the main plot object) are proto objects.  Proto is a package which implements the prototype-style of object-oriented programming.  There are some major differences between this and the typical S3 or S4 style of OO in R, but the good news is that you only need to worry about them if you want to develop your own extensions to ggplot2.  For everyday use, the proto objects are hidden behind a facade which makes them act like normal R objects.

{\tt str} to see full structure (it can be large!)

{\tt summary} briefly describes the structure of the plot

Data stored inside the plot - if you change the data outside of the plot, and then redraw a saved plot, it will not be updated.  Consequence of R copying semantics.

\ifwhole
\else
  \nobibliography{/Users/hadley/documents/phd/references}
  \end{document}
\fi
\chapter{Build a plot layer by layer}\label{cha:layers}

\section{Introduction}

One of the key ideas behind ggplot2 is that it allows you to easily
iterate, building up a complex plot a layer at a time. Each layer can
come from a different dataset and have a different aesthetic mapping,
making it possible to create sophisticated plots that display data from
multiple sources.

You've already created layers with functions like \texttt{geom\_point()}
and \texttt{geom\_histogram()}. In this chapter, you'll dive into the
details of a layer, and how you can control all five components: data,
the aesthetic mappings, the geom, stat, and position adjustments. The
goal here is to give you the tools to build sophisticated plots tailored
to the problem at hand.

\section{Building a plot}

So far, whenever we've created a plot with \texttt{ggplot()}, we've
immediately added on a layer with a geom function. But it's important to
realise that there really are two distinct steps. First we create a plot
with default dataset and aesthetic mappings:

\begin{Shaded}
\begin{Highlighting}[]
\NormalTok{p <-}\StringTok{ }\KeywordTok{ggplot}\NormalTok{(mpg, }\KeywordTok{aes}\NormalTok{(displ, hwy))}
\NormalTok{p}
\end{Highlighting}
\end{Shaded}

\begin{figure}[H]
  \centering
  \includegraphics[width=0.65\linewidth]{_figures/layers/layer1-1}
\end{figure}

There's nothing to see yet, so we need to add a layer:

\begin{Shaded}
\begin{Highlighting}[]
\NormalTok{p +}\StringTok{ }\KeywordTok{geom_point}\NormalTok{()}
\end{Highlighting}
\end{Shaded}

\begin{figure}[H]
  \centering
  \includegraphics[width=0.65\linewidth]{_figures/layers/unnamed-chunk-1-1}
\end{figure}

\texttt{geom\_point()} is a shortcut. Behind the scenes it calls the
\texttt{layer()} function to create a new layer: \indexf{layer}

\begin{Shaded}
\begin{Highlighting}[]
\NormalTok{p +}\StringTok{ }\KeywordTok{layer}\NormalTok{(}
  \DataTypeTok{mapping =} \OtherTok{NULL}\NormalTok{, }
  \DataTypeTok{data =} \OtherTok{NULL}\NormalTok{,}
  \DataTypeTok{geom =} \StringTok{"point"}\NormalTok{, }\DataTypeTok{geom_params =} \KeywordTok{list}\NormalTok{(),}
  \DataTypeTok{stat =} \StringTok{"identity"}\NormalTok{, }\DataTypeTok{stat_params =} \KeywordTok{list}\NormalTok{(),}
  \DataTypeTok{position =} \StringTok{"identity"}
\NormalTok{)}
\end{Highlighting}
\end{Shaded}

This call fully specifies the five components to the layer:
\index{Layers!components}

\begin{itemize}
\item
  \textbf{mapping}: A set of aesthetic mappings, specified using the
  \texttt{aes()} function and combined with the plot defaults as
  described in \hyperref[sec:aes]{aesthetic mappings}. If \texttt{NULL},
  uses the default mapping set in \texttt{ggplot()}.
\item
  \textbf{data}: A dataset which overrides the default plot dataset. It
  is usually omitted (set to \texttt{NULL}), in which case the layer
  will use the default data specified in \texttt{ggplot()}. The
  requirements for data are explained in more detail in
  \hyperref[sec:data]{data}.
\item
  \textbf{geom}: The name of the geometric object to use to draw each
  observation. Geoms are discussed in more detail in
  \hyperref[sec:data]{geom}, and \hyperref[cha:toolbox]{the toolbox}
  explores their use in more depth.

  Geoms can have additional arguments. All geoms take aesthetics as
  parameters. If you supply an aesthetic (e.g.~colour) as a parameter,
  it will not be scaled, allowing you to control the appearance of the
  plot, as described in \hyperref[sub:setting-mapping]{setting
  vs.~mapping}. You can pass params in \texttt{...} (in which case stat
  and geom parameters are automatically teased apart), or in a list
  passed to \texttt{geom\_params}.
\item
  \textbf{stat}: The name of the statistical tranformation to use. A
  statistical transformation performs some useful statistical summary,
  and is key to histograms and smoothers. To keep the data as is, use
  the ``identity'' stat. Learn more in \hyperref[sec:stat]{statistical
  transformations}.

  You only need to set one of stat and geom: every geom has a default
  stat, and every stat a default geom.

  Most stats take additional parameters to specify the details of
  statistical transformation. You can supply params either in
  \texttt{...} (in which case stat and geom parameters are automatically
  teased apart), or in a list called \texttt{stat\_params}.
\item
  \textbf{position}: The method used to adjust overlapping objects, like
  jittering, stacking or dodging. More details in
  \hyperref[sec:position]{position}.
\end{itemize}

It's useful to understand the \texttt{layer()} function so you have a
better mental model of the layer object. But you'll rarely use the full
\texttt{layer()} call because it's so verbose. Instead, you'll use the
shortcut \texttt{geom\_} functions:
\texttt{geom\_point(mapping,\ data,\ ...)} is exactly equivalent to
\texttt{layer(mapping,\ data,\ geom\ =\ "point",\ ...)}.

\hyperdef{}{sec:data}{\section{Data}\label{sec:data}}

Every layer must have some data associated with it, and that data must
be in a tidy data frame. You'll learn about tidy data in
\hyperref[cha:data]{tidy data}, but for now, all you need to know is
that a tidy data frame has variables in the columns and observations in
the rows. This is a strong restriction, but there are good reasons for
it: \index{Data} \indexf{data.frame}

\begin{itemize}
\item
  Your data is very important, so it's best to be explicit about it.
\item
  A single data frame is also easier to save than a multitude of
  vectors, which means it's easier to reproduce your results or send
  your data to someone else.
\item
  It enforces a clean separation of concerns: ggplot2 turns data frames
  into visualisations. Other packages can make data frames in the right
  format (learn more about that in \hyperref[sub:modelvis]{model
  visualisation}).
\end{itemize}

The data on each layer doesn't need to be the same, and it's often
useful to combine multiple datasets in a single plot. To illustrate that
idea I'm going to generate two new datasets related to the mpg dataset.
First I'll fit a loess model and generate predictions from it. (This is
what \texttt{geom\_smooth()} does behind the scenes)

\begin{Shaded}
\begin{Highlighting}[]
\NormalTok{mod <-}\StringTok{ }\KeywordTok{loess}\NormalTok{(hwy ~}\StringTok{ }\NormalTok{displ, }\DataTypeTok{data =} \NormalTok{mpg)}
\NormalTok{grid <-}\StringTok{ }\KeywordTok{data_frame}\NormalTok{(}\DataTypeTok{displ =} \KeywordTok{seq}\NormalTok{(}\KeywordTok{min}\NormalTok{(mpg$displ), }\KeywordTok{max}\NormalTok{(mpg$displ), }\DataTypeTok{length =} \DecValTok{50}\NormalTok{))}
\NormalTok{grid$hwy <-}\StringTok{ }\KeywordTok{predict}\NormalTok{(mod, }\DataTypeTok{newdata =} \NormalTok{grid)}

\NormalTok{grid}
\CommentTok{#> Source: local data frame [50 x 2]}
\CommentTok{#> }
\CommentTok{#>    displ   hwy}
\CommentTok{#>    (dbl) (dbl)}
\CommentTok{#> 1   1.60  33.1}
\CommentTok{#> 2   1.71  32.2}
\CommentTok{#> 3   1.82  31.3}
\CommentTok{#> 4   1.93  30.4}
\CommentTok{#> 5   2.04  29.6}
\CommentTok{#> 6   2.15  28.8}
\CommentTok{#> ..   ...   ...}
\end{Highlighting}
\end{Shaded}

Next, I'll isolate observations that are particularly far away from
their predicted values:

\begin{Shaded}
\begin{Highlighting}[]
\NormalTok{std_resid <-}\StringTok{ }\KeywordTok{resid}\NormalTok{(mod) /}\StringTok{ }\NormalTok{mod$s}
\NormalTok{outlier <-}\StringTok{ }\KeywordTok{filter}\NormalTok{(mpg, }\KeywordTok{abs}\NormalTok{(std_resid) >}\StringTok{ }\DecValTok{2}\NormalTok{)}
\NormalTok{outlier}
\CommentTok{#> Source: local data frame [6 x 11]}
\CommentTok{#> }
\CommentTok{#>   manufacturer      model displ  year   cyl      trans   drv   cty}
\CommentTok{#>          (chr)      (chr) (dbl) (int) (int)      (chr) (chr) (int)}
\CommentTok{#> 1    chevrolet   corvette   5.7  1999     8 manual(m6)     r    16}
\CommentTok{#> 2      pontiac grand prix   3.8  2008     6   auto(l4)     f    18}
\CommentTok{#> 3      pontiac grand prix   5.3  2008     8   auto(s4)     f    16}
\CommentTok{#> 4   volkswagen      jetta   1.9  1999     4 manual(m5)     f    33}
\CommentTok{#> 5   volkswagen new beetle   1.9  1999     4 manual(m5)     f    35}
\CommentTok{#> 6   volkswagen new beetle   1.9  1999     4   auto(l4)     f    29}
\CommentTok{#> Variables not shown: hwy (int), fl (chr), class (chr)}
\end{Highlighting}
\end{Shaded}

I've generated these datasets because it's common to enhance the display
of raw data with a statistical summary and some annotations. With these
new datasets, I can improve our initial scatterplot by overlaying a
smoothed line, and labelling the outlying points:

\begin{Shaded}
\begin{Highlighting}[]
\KeywordTok{ggplot}\NormalTok{(mpg, }\KeywordTok{aes}\NormalTok{(displ, hwy)) +}\StringTok{ }
\StringTok{  }\KeywordTok{geom_point}\NormalTok{() +}\StringTok{ }
\StringTok{  }\KeywordTok{geom_line}\NormalTok{(}\DataTypeTok{data =} \NormalTok{grid, }\DataTypeTok{colour =} \StringTok{"blue"}\NormalTok{, }\DataTypeTok{size =} \FloatTok{1.5}\NormalTok{) +}\StringTok{ }
\StringTok{  }\KeywordTok{geom_text}\NormalTok{(}\DataTypeTok{data =} \NormalTok{outlier, }\KeywordTok{aes}\NormalTok{(}\DataTypeTok{label =} \NormalTok{model))}
\end{Highlighting}
\end{Shaded}

\begin{figure}[H]
  \centering
  \includegraphics[width=0.65\linewidth]{_figures/layers/unnamed-chunk-2-1}
\end{figure}

(The labels aren't particularly easy to read, but you can fix that with
some manual tweaking.)

Note that you need the explicit \texttt{data\ =} in the layers, but not
in the call to \texttt{ggplot()}. That's because the argument order is
different. This is a little inconsistent, but it reduces typing for the
common case where you specify the data once in \texttt{ggplot()} and
modify aesthetics in each layer.

In this example, every layer uses a different dataset. We could define
the same plot in another way, omitting the default dataset, and
specifying a dataset for each layer:

\begin{Shaded}
\begin{Highlighting}[]
\KeywordTok{ggplot}\NormalTok{(}\DataTypeTok{mapping =} \KeywordTok{aes}\NormalTok{(displ, hwy)) +}\StringTok{ }
\StringTok{  }\KeywordTok{geom_point}\NormalTok{(}\DataTypeTok{data =} \NormalTok{mpg) +}\StringTok{ }
\StringTok{  }\KeywordTok{geom_line}\NormalTok{(}\DataTypeTok{data =} \NormalTok{grid) +}\StringTok{ }
\StringTok{  }\KeywordTok{geom_text}\NormalTok{(}\DataTypeTok{data =} \NormalTok{outlier, }\KeywordTok{aes}\NormalTok{(}\DataTypeTok{label =} \NormalTok{model))}
\end{Highlighting}
\end{Shaded}

I don't particularly like this style in this example because it makes it
less clear what the primary dataset is (and because of the way that the
arguments to \texttt{ggplot()} are ordered, it actually requires more
keystrokes). However, you may prefer it in cases where there isn't a
clear primary dataset, or where the aesthetics also vary from layer to
layer.

\subsection{Exercises}

\begin{enumerate}
\def\labelenumi{\arabic{enumi}.}
\item
  The first two arguments to ggplot are \texttt{data} and
  \texttt{mapping}. The first two arguments to all layer functions are
  \texttt{mapping} and \texttt{data}. Why does the order of the
  arguments differ? (Hint: think about what you set most commonly.)
\item
  The following code uses dplyr to generate some summary statistics
  about each class of car (you'll learn how it works in
  \hyperref[cha:dplyr]{data transformation}).

\begin{Shaded}
\begin{Highlighting}[]
\KeywordTok{library}\NormalTok{(dplyr)}
\NormalTok{class <-}\StringTok{ }\NormalTok{mpg %>%}\StringTok{ }
\StringTok{  }\KeywordTok{group_by}\NormalTok{(class) %>%}\StringTok{ }
\StringTok{  }\KeywordTok{summarise}\NormalTok{(}\DataTypeTok{n =} \KeywordTok{n}\NormalTok{(), }\DataTypeTok{hwy =} \KeywordTok{mean}\NormalTok{(hwy))}
\end{Highlighting}
\end{Shaded}

  Use the data to recreate this plot:

  \begin{figure}[H]
    \centering
    \includegraphics[width=0.65\linewidth]{_figures/layers/unnamed-chunk-5-1}
  \end{figure}
\end{enumerate}

\hyperdef{}{sec:aes}{\section{Aesthetic mappings}\label{sec:aes}}

The aesthetic mappings, defined with \texttt{aes()}, describe how
variables are mapped to visual properties or \textbf{aesthetics}.
\texttt{aes()} takes a sequence of aesthetic-variable pairs like this:
\index{Aesthetics!mapping} \indexf{aes}

\begin{Shaded}
\begin{Highlighting}[]
\KeywordTok{aes}\NormalTok{(}\DataTypeTok{x =} \NormalTok{displ, }\DataTypeTok{y =} \NormalTok{hwy, }\DataTypeTok{colour =} \NormalTok{class)}
\end{Highlighting}
\end{Shaded}

(If you're American, you can use \emph{color}, and behind the scenes
ggplot2 will correct your spelling ;)

Here we map x-position to \texttt{displ}, y-position to \texttt{hwy},
and colour to \texttt{class}. The names for the first two arguments can
be omitted, in which case they correspond to the x and y variables. That
makes this specification equivalent to the one above:

\begin{Shaded}
\begin{Highlighting}[]
\KeywordTok{aes}\NormalTok{(displ, hwy, }\DataTypeTok{colour =} \NormalTok{class)}
\end{Highlighting}
\end{Shaded}

While you can do data manipulation in \texttt{aes()}, e.g.
\texttt{aes(log(carat),\ log(price))}, it's best to only do simple
calculations. It's better to move complex transformations out of the
\texttt{aes()} call and into an explicit \texttt{dplyr::mutate()} call,
as you'll learn about in \hyperref[mutate]{mutate}. This makes it easier
to check your work and it's often faster because you need only do the
transformation once, not every time the plot is drawn.

Never refer to a variable with \texttt{\$} (e.g.,
\texttt{diamonds\$carat}) in \texttt{aes()}. This breaks containment, so
that the plot no longer contains everything it needs, and causes
problems if ggplot2 changes the order of the rows, as it does when
facetting. \indexc{\$}

\subsection{Specifying the aesthetics in the plot vs.~in the
layers}\label{sub:plots-and-layers}

Aesthetic mappings can be supplied in the initial \texttt{ggplot()}
call, in individual layers, or in some combination of both. All of these
calls create the same plot specification:
\index{Aesthetics!plot vs. layer}

\begin{Shaded}
\begin{Highlighting}[]
\KeywordTok{ggplot}\NormalTok{(mpg, }\KeywordTok{aes}\NormalTok{(displ, hwy, }\DataTypeTok{colour =} \NormalTok{class)) +}\StringTok{ }
\StringTok{  }\KeywordTok{geom_point}\NormalTok{()}
\KeywordTok{ggplot}\NormalTok{(mpg, }\KeywordTok{aes}\NormalTok{(displ, hwy)) +}\StringTok{ }
\StringTok{  }\KeywordTok{geom_point}\NormalTok{(}\KeywordTok{aes}\NormalTok{(}\DataTypeTok{colour =} \NormalTok{class))}
\KeywordTok{ggplot}\NormalTok{(mpg, }\KeywordTok{aes}\NormalTok{(displ)) +}\StringTok{ }
\StringTok{  }\KeywordTok{geom_point}\NormalTok{(}\KeywordTok{aes}\NormalTok{(}\DataTypeTok{y =} \NormalTok{hwy, }\DataTypeTok{colour =} \NormalTok{class))}
\KeywordTok{ggplot}\NormalTok{(mpg) +}\StringTok{ }
\StringTok{  }\KeywordTok{geom_point}\NormalTok{(}\KeywordTok{aes}\NormalTok{(displ, hwy, }\DataTypeTok{colour =} \NormalTok{class))}
\end{Highlighting}
\end{Shaded}

Within each layer, you can add, override, or remove mappings:

\begin{longtable}[c]{@{}lll@{}}
\toprule
Operation & Layer aesthetics & Result\tabularnewline
\midrule
\endhead
Add & \texttt{aes(colour\ =\ cyl)} &
\texttt{aes(mpg,\ wt,\ colour\ =\ cyl)}\tabularnewline
Override & \texttt{aes(y\ =\ disp)} &
\texttt{aes(mpg,\ disp)}\tabularnewline
Remove & \texttt{aes(y\ =\ NULL)} & \texttt{aes(mpg)}\tabularnewline
\bottomrule
\end{longtable}

If you only have one layer in the plot, the way you specify aesthetics
doesn't make any difference. However, the distinction is important when
you start adding additional layers. These two plots are both valid and
interesting, but focus on quite different aspects of the data:

\begin{Shaded}
\begin{Highlighting}[]
\KeywordTok{ggplot}\NormalTok{(mpg, }\KeywordTok{aes}\NormalTok{(displ, hwy, }\DataTypeTok{colour =} \NormalTok{class)) +}\StringTok{ }
\StringTok{  }\KeywordTok{geom_point}\NormalTok{() +}\StringTok{ }
\StringTok{  }\KeywordTok{geom_smooth}\NormalTok{(}\DataTypeTok{method =} \StringTok{"lm"}\NormalTok{, }\DataTypeTok{se =} \OtherTok{FALSE}\NormalTok{) +}
\StringTok{  }\KeywordTok{theme}\NormalTok{(}\DataTypeTok{legend.position =} \StringTok{"none"}\NormalTok{)}

\KeywordTok{ggplot}\NormalTok{(mpg, }\KeywordTok{aes}\NormalTok{(displ, hwy)) +}\StringTok{ }
\StringTok{  }\KeywordTok{geom_point}\NormalTok{(}\KeywordTok{aes}\NormalTok{(}\DataTypeTok{colour =} \NormalTok{class)) +}\StringTok{ }
\StringTok{  }\KeywordTok{geom_smooth}\NormalTok{(}\DataTypeTok{method =} \StringTok{"lm"}\NormalTok{, }\DataTypeTok{se =} \OtherTok{FALSE}\NormalTok{) +}\StringTok{ }
\StringTok{  }\KeywordTok{theme}\NormalTok{(}\DataTypeTok{legend.position =} \StringTok{"none"}\NormalTok{)}
\end{Highlighting}
\end{Shaded}

\begin{figure}[H]
  \includegraphics[width=0.5\linewidth]{_figures/layers/unnamed-chunk-7-1}%
  \includegraphics[width=0.5\linewidth]{_figures/layers/unnamed-chunk-7-2}
\end{figure}

Generally, you want to set up the mappings to illuminate the structure
underlying the graphic and minimise typing. It may take some time before
the best approach is immediately obvious, so if you've iterated your way
to a complex graphic, it may be worthwhile to rewrite it to make the
structure more clear.

\hyperdef{}{sub:setting-mapping}{\subsection{Setting
vs.~mapping}\label{sub:setting-mapping}}

Instead of mapping an aesthetic property to a variable, you can set it
to a \emph{single} value by specifying it in the layer parameters. We
\textbf{map} an aesthetic to a variable (e.g.,
\texttt{aes(colour\ =\ cut)}) or \textbf{set} it to a constant (e.g.,
\texttt{colour\ =\ "red"}). If you want appearance to be governed by a
variable, put the specification inside \texttt{aes()}; if you want
override the default size or colour, put the value outside of
\texttt{aes()}. \index{Aesthetics!setting}

The following plots are created with similar code, but have rather
different outputs. The second plot \textbf{maps} (not sets) the colour
to the value `darkblue'. This effectively creates a new variable
containing only the value `darkblue' and then scales it with a colour
scale. Because this value is discrete, the default colour scale uses
evenly spaced colours on the colour wheel, and since there is only one
value this colour is pinkish.

\begin{Shaded}
\begin{Highlighting}[]
\KeywordTok{ggplot}\NormalTok{(mpg, }\KeywordTok{aes}\NormalTok{(cty, hwy)) +}\StringTok{ }
\StringTok{  }\KeywordTok{geom_point}\NormalTok{(}\DataTypeTok{colour =} \StringTok{"darkblue"}\NormalTok{) }

\KeywordTok{ggplot}\NormalTok{(mpg, }\KeywordTok{aes}\NormalTok{(cty, hwy)) +}\StringTok{ }
\StringTok{  }\KeywordTok{geom_point}\NormalTok{(}\KeywordTok{aes}\NormalTok{(}\DataTypeTok{colour =} \StringTok{"darkblue"}\NormalTok{))}
\end{Highlighting}
\end{Shaded}

\begin{figure}[H]
  \includegraphics[width=0.5\linewidth]{_figures/layers/layer15-1}%
  \includegraphics[width=0.5\linewidth]{_figures/layers/layer15-2}
\end{figure}

A third approach is to map the value, but override the default scale:

\begin{Shaded}
\begin{Highlighting}[]
\KeywordTok{ggplot}\NormalTok{(mpg, }\KeywordTok{aes}\NormalTok{(cty, hwy)) +}\StringTok{ }
\StringTok{  }\KeywordTok{geom_point}\NormalTok{(}\KeywordTok{aes}\NormalTok{(}\DataTypeTok{colour =} \StringTok{"darkblue"}\NormalTok{)) +}\StringTok{ }
\StringTok{  }\KeywordTok{scale_colour_identity}\NormalTok{()}
\end{Highlighting}
\end{Shaded}

\begin{figure}[H]
  \centering
  \includegraphics[width=0.5\linewidth]{_figures/layers/unnamed-chunk-8-1}
\end{figure}

This is most useful if you always have a column that already contains
colours. You'll learn more about that in
\hyperref[sub:scale-identity]{the identity scale}.

It's sometimes useful to map aesthetics to constants. For example, if
you want to display multiple layers with varying parameters, you can
``name'' each layer:

\begin{Shaded}
\begin{Highlighting}[]
\KeywordTok{ggplot}\NormalTok{(mpg, }\KeywordTok{aes}\NormalTok{(displ, hwy)) +}\StringTok{ }
\StringTok{  }\KeywordTok{geom_point}\NormalTok{() +}
\StringTok{  }\KeywordTok{geom_smooth}\NormalTok{(}\KeywordTok{aes}\NormalTok{(}\DataTypeTok{colour =} \StringTok{"loess"}\NormalTok{), }\DataTypeTok{method =} \StringTok{"loess"}\NormalTok{, }\DataTypeTok{se =} \OtherTok{FALSE}\NormalTok{) +}\StringTok{ }
\StringTok{  }\KeywordTok{geom_smooth}\NormalTok{(}\KeywordTok{aes}\NormalTok{(}\DataTypeTok{colour =} \StringTok{"lm"}\NormalTok{), }\DataTypeTok{method =} \StringTok{"lm"}\NormalTok{, }\DataTypeTok{se =} \OtherTok{FALSE}\NormalTok{) +}
\StringTok{  }\KeywordTok{labs}\NormalTok{(}\DataTypeTok{colour =} \StringTok{"Method"}\NormalTok{)}
\end{Highlighting}
\end{Shaded}

\begin{figure}[H]
  \centering
  \includegraphics[width=0.65\linewidth]{_figures/layers/unnamed-chunk-9-1}
\end{figure}

\subsection{Exercises}

\begin{enumerate}
\def\labelenumi{\arabic{enumi}.}
\item
  Simplify the following plot specifications:

\begin{Shaded}
\begin{Highlighting}[]
\KeywordTok{ggplot}\NormalTok{(mpg) +}\StringTok{ }
\StringTok{  }\KeywordTok{geom_point}\NormalTok{(}\KeywordTok{aes}\NormalTok{(mpg$disp, mpg$hwy))}

\KeywordTok{ggplot}\NormalTok{() +}\StringTok{ }
\StringTok{ }\KeywordTok{geom_point}\NormalTok{(}\DataTypeTok{mapping =} \KeywordTok{aes}\NormalTok{(}\DataTypeTok{y =} \NormalTok{hwy, }\DataTypeTok{x =} \NormalTok{cty), }\DataTypeTok{data =} \NormalTok{mpg) +}
\StringTok{ }\KeywordTok{geom_smooth}\NormalTok{(}\DataTypeTok{data =} \NormalTok{mpg, }\DataTypeTok{mapping =} \KeywordTok{aes}\NormalTok{(cty, hwy))}

\KeywordTok{ggplot}\NormalTok{(diamonds, }\KeywordTok{aes}\NormalTok{(carat, price)) +}\StringTok{ }
\StringTok{  }\KeywordTok{geom_point}\NormalTok{(}\KeywordTok{aes}\NormalTok{(}\KeywordTok{log}\NormalTok{(brainwt), }\KeywordTok{log}\NormalTok{(bodywt)), }\DataTypeTok{data =} \NormalTok{msleep)}
\end{Highlighting}
\end{Shaded}
\item
  What does the following code do? Does it work? Does it make sense?
  Why/why not?

\begin{Shaded}
\begin{Highlighting}[]
\KeywordTok{ggplot}\NormalTok{(mpg) +}
\StringTok{  }\KeywordTok{geom_point}\NormalTok{(}\KeywordTok{aes}\NormalTok{(class, cty)) +}\StringTok{ }
\StringTok{  }\KeywordTok{geom_boxplot}\NormalTok{(}\KeywordTok{aes}\NormalTok{(trans, hwy))}
\end{Highlighting}
\end{Shaded}
\item
  What happens if you try to use a continuous variable on the x axis in
  one layer, and a categorical variable in another layer? What happens
  if you do it in the opposite order?
\end{enumerate}

\section{Geoms}\label{sec:geom}

Geometric objects, or \textbf{geoms} for short, perform the actual
rendering of the layer, controlling the type of plot that you create.
For example, using a point geom will create a scatterplot, while using a
line geom will create a line plot.

\begin{itemize}
\tightlist
\item
  Graphical primitives:

  \begin{itemize}
  \tightlist
  \item
    \texttt{geom\_blank()}: display nothing. Most useful for adjusting
    axes limits using data.
  \item
    \texttt{geom\_point()}: points.
  \item
    \texttt{geom\_path()}: paths.
  \item
    \texttt{geom\_ribbon()}: ribbons, a path with vertical thickness.
  \item
    \texttt{geom\_segment()}: a line segment, specified by start and end
    position.
  \item
    \texttt{geom\_rect()}: rectangles.
  \item
    \texttt{geom\_polyon()}: filled polygons.
  \item
    \texttt{geom\_text()}: text.
  \end{itemize}
\item
  One variable:

  \begin{itemize}
  \tightlist
  \item
    Discrete:

    \begin{itemize}
    \tightlist
    \item
      \texttt{geom\_bar()}: display distribution of discrete variable.
    \end{itemize}
  \item
    Continuous

    \begin{itemize}
    \tightlist
    \item
      \texttt{geom\_histogram()}: bin and count continuous variable,
      display with bars.
    \item
      \texttt{geom\_density()}: smoothed density estimate.
    \item
      \texttt{geom\_dotplot()}: stack individual points into a dot plot.
    \item
      \texttt{geom\_freqpoly()}: bin and count continuous variable,
      display with lines.
    \end{itemize}
  \end{itemize}
\item
  Two variables:

  \begin{itemize}
  \tightlist
  \item
    Both continuous:

    \begin{itemize}
    \tightlist
    \item
      \texttt{geom\_point()}: scatterplot.
    \item
      \texttt{geom\_quantile()}: smoothed quantile regression.
    \item
      \texttt{geom\_rug()}: marginal rug plots.
    \item
      \texttt{geom\_smooth()}: smoothed line of best fit.
    \item
      \texttt{geom\_text()}: text labels.
    \end{itemize}
  \item
    Show distribution:

    \begin{itemize}
    \tightlist
    \item
      \texttt{geom\_bin2d()}: bin into rectangles and count.
    \item
      \texttt{geom\_density2d()}: smoothed 2d density estimate.
    \item
      \texttt{geom\_hex()}: bin into hexagons and count.
    \end{itemize}
  \item
    At least one discrete:

    \begin{itemize}
    \tightlist
    \item
      \texttt{geom\_count()}: count number of point at distinct
      locations
    \item
      \texttt{geom\_jitter()}: randomly jitter overlapping points.
    \end{itemize}
  \item
    One continuous, one discrete:

    \begin{itemize}
    \tightlist
    \item
      \texttt{geom\_bar(stat\ =\ "identity")}: a bar chart of
      precomputed summaries.
    \item
      \texttt{geom\_boxplot()}: boxplots.
    \item
      \texttt{geom\_violin()}: show density of values in each group.
    \end{itemize}
  \item
    One time, one continuous

    \begin{itemize}
    \tightlist
    \item
      \texttt{geom\_area()}: area plot.
    \item
      \texttt{geom\_line()}: line plot.
    \item
      \texttt{geom\_step()}: step plot.
    \end{itemize}
  \item
    Display uncertainty:

    \begin{itemize}
    \tightlist
    \item
      \texttt{geom\_crossbar()}: vertical bar with center.
    \item
      \texttt{geom\_errorbar()}: error bars.
    \item
      \texttt{geom\_linerange()}: vertical line.
    \item
      \texttt{geom\_pointrange()}: vertical line with center.
    \end{itemize}
  \item
    Spatial

    \begin{itemize}
    \tightlist
    \item
      \texttt{geom\_map()}: fast version of \texttt{geom\_polygon()} for
      map data.
    \end{itemize}
  \end{itemize}
\item
  Three variables:

  \begin{itemize}
  \tightlist
  \item
    \texttt{geom\_contour()}: contours.
  \item
    \texttt{geom\_tile()}: tile the plane with rectangles.
  \item
    \texttt{geom\_raster()}: fast version of \texttt{geom\_tile()} for
    equal sized tiles.
  \end{itemize}
\end{itemize}

Each geom has a set of aesthetics that it understands, some of which
\emph{must} be provided. For example, the point geoms requires x and y
position, and understands colour, size and shape aesthetics. A bar
requires height (\texttt{ymax}), and understands width, border colour
and fill colour. Each geom lists its aesthetics in the documentation.

Some geoms differ primarily in the way that they are parameterised. For
example, you can draw a square in three ways:
\index{Geoms!parameterisation}

\begin{itemize}
\item
  By giving \texttt{geom\_tile()} the location (\texttt{x} and
  \texttt{y}) and dimensions (\texttt{width} and \texttt{height}).
  \indexf{geom\_tile}
\item
  By giving \texttt{geom\_rect()} top (\texttt{ymax}), bottom
  (\texttt{ymin}), left (\texttt{xmin}) and right (\texttt{xmax})
  positions. \indexf{geom\_rect}
\item
  By giving \texttt{geom\_polygon()} a four row data frame with the
  \texttt{x} and \texttt{y} positions of each corner.
\end{itemize}

Other related geoms are:

\begin{itemize}
\tightlist
\item
  \texttt{geom\_segment()} and \texttt{geom\_line()}
\item
  \texttt{geom\_area()} and \texttt{geom\_ribbon()}.
\end{itemize}

If alternative parameterisations are available, picking the right one
for your data will usually make it much easier to draw the plot you
want.

\subsection{Exercises}

\begin{enumerate}
\def\labelenumi{\arabic{enumi}.}
\item
  Download and print out the ggplot2 cheatsheet from
  \url{http://www.rstudio.com/resources/cheatsheets/} so you have a
  handy visual reference for all the geoms.
\item
  Look at the documentation for the graphical primitive geoms. Which
  aesthetics do they use? How can you summarise them in a compact form?
\item
  What's the best way to master an unfamiliar geom? List three resources
  to help you get started.
\item
  For each of the plots below, identify the geom used to draw it.

  \begin{figure}[H]
    \includegraphics[width=0.5\linewidth]{_figures/layers/unnamed-chunk-12-1}%
    \includegraphics[width=0.5\linewidth]{_figures/layers/unnamed-chunk-12-2}
  \end{figure}

  \begin{figure}[H]
    \includegraphics[width=0.5\linewidth]{_figures/layers/unnamed-chunk-13-1}%
    \includegraphics[width=0.5\linewidth]{_figures/layers/unnamed-chunk-13-2}
  \end{figure}

  \begin{figure}[H]
    \includegraphics[width=0.5\linewidth]{_figures/layers/unnamed-chunk-14-1}%
    \includegraphics[width=0.5\linewidth]{_figures/layers/unnamed-chunk-14-2}
  \end{figure}
\item
  For each of the following problems, suggest a useful geom:

  \begin{itemize}
  \tightlist
  \item
    Display how a variable has changed over time.
  \item
    Show the detailed distribution of a single variable.
  \item
    Focus attention on the overall trend in a large dataset.
  \item
    Draw a map.
  \item
    Label outlying points.
  \end{itemize}
\end{enumerate}

\hyperdef{}{sec:stat}{\section{Stats}\label{sec:stat}}

A statistical transformation, or \textbf{stat}, transforms the data,
typically by summarising it in some manner. For example, a useful stat
is the smoother, which calculates the smoothed mean of y, conditional on
x. You've already used many of ggplot2's stats because they're used
behind the scenes to generate many important geoms:

\begin{itemize}
\tightlist
\item
  \texttt{stat\_bin()}: \texttt{geom\_bar()}, \texttt{geom\_freqpoly()},
  \texttt{geom\_histogram()}
\item
  \texttt{stat\_bin2d()}: \texttt{geom\_bin2d()}
\item
  \texttt{stat\_bindot()}: \texttt{geom\_dotplot()}
\item
  \texttt{stat\_binhex()}: \texttt{geom\_hex()}
\item
  \texttt{stat\_boxplot()}: \texttt{geom\_boxplot()}
\item
  \texttt{stat\_contour()}: \texttt{geom\_contour()}
\item
  \texttt{stat\_quantile()}: \texttt{geom\_quantile()}
\item
  \texttt{stat\_smooth()}: \texttt{geom\_smooth()}
\item
  \texttt{stat\_sum()}: \texttt{geom\_count()}
\end{itemize}

You'll rarely call these functions directly, but they are useful to know
about because their documentation often provides more detail about the
corresponding statistical transformation.

Other stats can't be created with a \texttt{geom\_} function:

\begin{itemize}
\tightlist
\item
  \texttt{stat\_ecdf()}: compute a empirical cumulative distribution
  plot.
\item
  \texttt{stat\_function()}: compute y values from a function of x
  values.
\item
  \texttt{stat\_summary()}: summarise y values at distinct x values.
\item
  \texttt{stat\_summary2d()}, \texttt{stat\_summary\_hex()}: summarise
  binned values.
\item
  \texttt{stat\_qq()}: perform calculations for a quantile-quantile
  plot.
\item
  \texttt{stat\_spoke()}: convert angle and radius to position.
\item
  \texttt{stat\_unique()}: remove duplicated rows.
\end{itemize}

There are two ways to use these functions. You can either add a
\texttt{stat\_()} function and override the default geom, or add a
\texttt{geom\_()} function and override the default stat:

\begin{Shaded}
\begin{Highlighting}[]
\KeywordTok{ggplot}\NormalTok{(mpg, }\KeywordTok{aes}\NormalTok{(trans, cty)) +}\StringTok{ }
\StringTok{  }\KeywordTok{geom_point}\NormalTok{() +}\StringTok{ }
\StringTok{  }\KeywordTok{stat_summary}\NormalTok{(}\DataTypeTok{geom =} \StringTok{"point"}\NormalTok{, }\DataTypeTok{fun.y =} \StringTok{"mean"}\NormalTok{, }\DataTypeTok{colour =} \StringTok{"red"}\NormalTok{, }\DataTypeTok{size =} \DecValTok{4}\NormalTok{)}

\KeywordTok{ggplot}\NormalTok{(mpg, }\KeywordTok{aes}\NormalTok{(trans, cty)) +}\StringTok{ }
\StringTok{  }\KeywordTok{geom_point}\NormalTok{() +}\StringTok{ }
\StringTok{  }\KeywordTok{geom_point}\NormalTok{(}\DataTypeTok{stat =} \StringTok{"summary"}\NormalTok{, }\DataTypeTok{fun.y =} \StringTok{"mean"}\NormalTok{, }\DataTypeTok{colour =} \StringTok{"red"}\NormalTok{, }\DataTypeTok{size =} \DecValTok{4}\NormalTok{)}
\end{Highlighting}
\end{Shaded}

\begin{figure}[H]
  \centering
  \includegraphics[width=0.75\linewidth]{_figures/layers/unnamed-chunk-15-1}
\end{figure}

I think it's best to use the second form because it makes it more clear
that you're displaying a summary, not the raw data.

\subsection{Generated variables}

Internally, a stat takes a data frame as input and returns a data frame
as output, and so a stat can add new variables to the original dataset.
It is possible to map aesthetics to these new variables. For example,
\texttt{stat\_bin}, the statistic used to make histograms, produces the
following variables: \index{Stats!creating new variables}
\indexf{stat\_bin}

\begin{itemize}
\tightlist
\item
  \texttt{count}, the number of observations in each bin
\item
  \texttt{density}, the density of observations in each bin (percentage
  of total / bar width)
\item
  \texttt{x}, the centre of the bin
\end{itemize}

These generated variables can be used instead of the variables present
in the original dataset. For example, the default histogram geom assigns
the height of the bars to the number of observations (\texttt{count}),
but if you'd prefer a more traditional histogram, you can use the
density (\texttt{density}). To refer to a generated variable like
density, ``\texttt{..}'' must surround the name. This prevents confusion
in case the original dataset includes a variable with the same name as a
generated variable, and it makes it clear to any later reader of the
code that this variable was generated by a stat. Each statistic lists
the variables that it creates in its documentation. \indexc{..} Compare
the y-axes on these two plots:

\begin{Shaded}
\begin{Highlighting}[]
\KeywordTok{ggplot}\NormalTok{(diamonds, }\KeywordTok{aes}\NormalTok{(price)) +}\StringTok{ }
\StringTok{  }\KeywordTok{geom_histogram}\NormalTok{(}\DataTypeTok{binwidth =} \DecValTok{500}\NormalTok{)}
\KeywordTok{ggplot}\NormalTok{(diamonds, }\KeywordTok{aes}\NormalTok{(price)) +}\StringTok{ }
\StringTok{  }\KeywordTok{geom_histogram}\NormalTok{(}\KeywordTok{aes}\NormalTok{(}\DataTypeTok{y =} \NormalTok{..density..), }\DataTypeTok{binwidth =} \DecValTok{500}\NormalTok{)}
\end{Highlighting}
\end{Shaded}

\begin{figure}[H]
  \includegraphics[width=0.5\linewidth]{_figures/layers/hist-1}%
  \includegraphics[width=0.5\linewidth]{_figures/layers/hist-2}
\end{figure}

This technique is particularly useful when you want to compare the
distribution of multiple groups that have very different sizes. For
example, it's hard to compare the distribution of \texttt{price} within
\texttt{cut} because some groups are quite small. It's easier to compare
if we standardise each group to take up the same area:

\begin{Shaded}
\begin{Highlighting}[]
\KeywordTok{ggplot}\NormalTok{(diamonds, }\KeywordTok{aes}\NormalTok{(price, }\DataTypeTok{colour =} \NormalTok{cut)) +}\StringTok{ }
\StringTok{  }\KeywordTok{geom_freqpoly}\NormalTok{(}\DataTypeTok{binwidth =} \DecValTok{500}\NormalTok{) +}
\StringTok{  }\KeywordTok{theme}\NormalTok{(}\DataTypeTok{legend.position =} \StringTok{"none"}\NormalTok{)}

\KeywordTok{ggplot}\NormalTok{(diamonds, }\KeywordTok{aes}\NormalTok{(price, }\DataTypeTok{colour =} \NormalTok{cut)) +}\StringTok{ }
\StringTok{  }\KeywordTok{geom_freqpoly}\NormalTok{(}\KeywordTok{aes}\NormalTok{(}\DataTypeTok{y =} \NormalTok{..density..), }\DataTypeTok{binwidth =} \DecValTok{500}\NormalTok{) +}\StringTok{ }
\StringTok{  }\KeywordTok{theme}\NormalTok{(}\DataTypeTok{legend.position =} \StringTok{"none"}\NormalTok{)}
\end{Highlighting}
\end{Shaded}

\begin{figure}[H]
  \includegraphics[width=0.5\linewidth]{_figures/layers/freqpoly-1}%
  \includegraphics[width=0.5\linewidth]{_figures/layers/freqpoly-2}
\end{figure}

The result of this plot is rather surprising: low quality diamonds seem
to be more expensive on average. We'll come back to this result in
\hyperref[sub:trend]{removing trend}.

\subsection{Exercises}

\begin{enumerate}
\def\labelenumi{\arabic{enumi}.}
\item
  The code below creates a similar dataset to \texttt{stat\_smooth()}.
  Use the appropriate geoms to mimic the default \texttt{geom\_smooth()}
  display.

\begin{Shaded}
\begin{Highlighting}[]
\NormalTok{mod <-}\StringTok{ }\KeywordTok{loess}\NormalTok{(hwy ~}\StringTok{ }\NormalTok{displ, }\DataTypeTok{data =} \NormalTok{mpg)}
\NormalTok{smoothed <-}\StringTok{ }\KeywordTok{data.frame}\NormalTok{(}\DataTypeTok{displ =} \KeywordTok{seq}\NormalTok{(}\FloatTok{1.6}\NormalTok{, }\DecValTok{7}\NormalTok{, }\DataTypeTok{length =} \DecValTok{50}\NormalTok{))}
\NormalTok{pred <-}\StringTok{ }\KeywordTok{predict}\NormalTok{(mod, }\DataTypeTok{newdata =} \NormalTok{smoothed, }\DataTypeTok{se =} \OtherTok{TRUE}\NormalTok{) }
\NormalTok{smoothed$hwy <-}\StringTok{ }\NormalTok{pred$fit}
\NormalTok{smoothed$hwy_lwr <-}\StringTok{ }\NormalTok{pred$fit -}\StringTok{ }\FloatTok{1.96} \NormalTok{*}\StringTok{ }\NormalTok{pred$se.fit}
\NormalTok{smoothed$hwy_upr <-}\StringTok{ }\NormalTok{pred$fit +}\StringTok{ }\FloatTok{1.96} \NormalTok{*}\StringTok{ }\NormalTok{pred$se.fit}
\end{Highlighting}
\end{Shaded}
\item
  What stats were used to create the following plots?

  \begin{figure}[H]
    \includegraphics[width=0.333\linewidth]{_figures/layers/unnamed-chunk-17-1}%
    \includegraphics[width=0.333\linewidth]{_figures/layers/unnamed-chunk-17-2}%
    \includegraphics[width=0.333\linewidth]{_figures/layers/unnamed-chunk-17-3}
  \end{figure}
\item
  Read the help for \texttt{stat\_sum()} then use \texttt{geom\_count()}
  to create a plot that shows the proportion of cars that have each
  combination of \texttt{drv} and \texttt{trans}.
\end{enumerate}

\hyperdef{}{sec:position}{\section{Position
adjustments}\label{sec:position}}

\index{Position adjustments}

Position adjustments apply minor tweaks to the position of elements
within a layer. Three adjustments apply primarily to bars:

\index{Dodging} \index{Side-by-side|see{Dodging}}
\indexf{position\_dodge} \index{Stacking} \indexf{position\_stack}
\indexf{position\_fill}

\begin{itemize}
\tightlist
\item
  \texttt{position\_stack()}: stack overlapping bars (or areas) on top
  of each other.
\item
  \texttt{position\_fill()}: stack overlapping bars, scaling so the top
  is always at 1.
\item
  \texttt{position\_dodge()}: place overlapping bars (or boxplots)
  side-by-side.
\end{itemize}

\begin{Shaded}
\begin{Highlighting}[]
\NormalTok{dplot <-}\StringTok{ }\KeywordTok{ggplot}\NormalTok{(diamonds, }\KeywordTok{aes}\NormalTok{(color, }\DataTypeTok{fill =} \NormalTok{cut)) +}\StringTok{ }
\StringTok{  }\KeywordTok{xlab}\NormalTok{(}\OtherTok{NULL}\NormalTok{) +}\StringTok{ }\KeywordTok{ylab}\NormalTok{(}\OtherTok{NULL}\NormalTok{) +}\StringTok{ }\KeywordTok{theme}\NormalTok{(}\DataTypeTok{legend.position =} \StringTok{"none"}\NormalTok{)}
\CommentTok{# position stack is the default for bars, so `geom_bar()` }
\CommentTok{# is equivalent to `geom_bar(position = "stack")`.}
\NormalTok{dplot +}\StringTok{ }\KeywordTok{geom_bar}\NormalTok{()}
\NormalTok{dplot +}\StringTok{ }\KeywordTok{geom_bar}\NormalTok{(}\DataTypeTok{position =} \StringTok{"fill"}\NormalTok{)}
\NormalTok{dplot +}\StringTok{ }\KeywordTok{geom_bar}\NormalTok{(}\DataTypeTok{position =} \StringTok{"dodge"}\NormalTok{)}
\end{Highlighting}
\end{Shaded}

\begin{figure}[H]
  \includegraphics[width=0.333\linewidth]{_figures/layers/position-bar-1}%
  \includegraphics[width=0.333\linewidth]{_figures/layers/position-bar-2}%
  \includegraphics[width=0.333\linewidth]{_figures/layers/position-bar-3}
\end{figure}

There's also a position adjustment that does nothing:
\texttt{position\_identity()}. The identity position adjustment is not
useful for bars, because each bar obscures the bars behind, but there
are many geoms that don't need adjusting, like the frequency polygon:

\begin{Shaded}
\begin{Highlighting}[]
\NormalTok{dplot +}\StringTok{ }\KeywordTok{geom_bar}\NormalTok{(}\DataTypeTok{position =} \StringTok{"identity"}\NormalTok{, }\DataTypeTok{alpha =} \DecValTok{1} \NormalTok{/}\StringTok{ }\DecValTok{2}\NormalTok{, }\DataTypeTok{colour =} \StringTok{"grey50"}\NormalTok{)}

\KeywordTok{ggplot}\NormalTok{(diamonds, }\KeywordTok{aes}\NormalTok{(color, }\DataTypeTok{colour =} \NormalTok{cut)) +}\StringTok{ }
\StringTok{  }\KeywordTok{geom_freqpoly}\NormalTok{(}\KeywordTok{aes}\NormalTok{(}\DataTypeTok{group =} \NormalTok{cut), }\DataTypeTok{stat =} \StringTok{"count"}\NormalTok{) +}\StringTok{ }
\StringTok{  }\KeywordTok{xlab}\NormalTok{(}\OtherTok{NULL}\NormalTok{) +}\StringTok{ }\KeywordTok{ylab}\NormalTok{(}\OtherTok{NULL}\NormalTok{) +}\StringTok{ }
\StringTok{  }\KeywordTok{theme}\NormalTok{(}\DataTypeTok{legend.position =} \StringTok{"none"}\NormalTok{)}
\end{Highlighting}
\end{Shaded}

\begin{figure}[H]
  \includegraphics[width=0.5\linewidth]{_figures/layers/position-identity-1}%
  \includegraphics[width=0.5\linewidth]{_figures/layers/position-identity-2}
\end{figure}

There are three position adjustments that are primarily useful for
points:

\begin{itemize}
\tightlist
\item
  \texttt{position\_nudge()}: move points by a fixed offset.
\item
  \texttt{position\_jitter()}: add a little random noise to every
  position.
\item
  \texttt{position\_jitterdodge()}: dodge points within groups, then add
  a little random noise.
\end{itemize}

\indexf{position\_nudge} \indexf{position\_jitter}
\indexf{position\_jitterdodge}

Note that the way you pass parameters to position adjustments differs to
stats and geoms. Instead of including additional arguments in
\texttt{...}, you construct a position adjustment object, supplying
additional arguments in the call:

\begin{Shaded}
\begin{Highlighting}[]
\KeywordTok{ggplot}\NormalTok{(mpg, }\KeywordTok{aes}\NormalTok{(displ, hwy)) +}\StringTok{ }
\StringTok{  }\KeywordTok{geom_point}\NormalTok{(}\DataTypeTok{position =} \StringTok{"jitter"}\NormalTok{)}
\KeywordTok{ggplot}\NormalTok{(mpg, }\KeywordTok{aes}\NormalTok{(displ, hwy)) +}\StringTok{ }
\StringTok{  }\KeywordTok{geom_point}\NormalTok{(}\DataTypeTok{position =} \KeywordTok{position_jitter}\NormalTok{(}\DataTypeTok{width =} \FloatTok{0.05}\NormalTok{, }\DataTypeTok{height =} \FloatTok{0.5}\NormalTok{))}
\end{Highlighting}
\end{Shaded}

\begin{figure}[H]
  \includegraphics[width=0.5\linewidth]{_figures/layers/position-point-1}%
  \includegraphics[width=0.5\linewidth]{_figures/layers/position-point-2}
\end{figure}

This is rather verbose, so \texttt{geom\_jitter()} provides a convenient
shortcut:

\begin{Shaded}
\begin{Highlighting}[]
\KeywordTok{ggplot}\NormalTok{(mpg, }\KeywordTok{aes}\NormalTok{(displ, hwy)) +}\StringTok{ }
\StringTok{  }\KeywordTok{geom_jitter}\NormalTok{(}\DataTypeTok{width =} \FloatTok{0.05}\NormalTok{, }\DataTypeTok{height =} \FloatTok{0.5}\NormalTok{)}
\end{Highlighting}
\end{Shaded}

Continuous data typically doesn't overlap exactly, and when it does
(because of high data density) minor adjustments, like jittering, are
often insufficient to fix the problem. For this reason, position
adjustments are generally most useful for discrete data.

\subsection{Exercises}

\begin{enumerate}
\def\labelenumi{\arabic{enumi}.}
\item
  When might you use \texttt{position\_nudge()}? Read the documentation.
\item
  Many position adjustments can only be used with a few geoms. For
  example, you can't stack boxplots or errors bars. Why not? What
  properties must a geom possess in order to be stackable? What
  properties must it possess to be dodgeable?
\item
  Why might you use \texttt{geom\_jitter()} instead of
  \texttt{geom\_count()}? What are the advantages and disadvantages of
  each technique?
\item
  When might you use a stacked area plot? What are the advantages and
  disadvantages compared to a line plot?
\end{enumerate}

\providecommand{\setflag}{\newif \ifwhole \wholefalse}
\setflag
\ifwhole\else

% Typography and geometry ----------------------------------------------------
\documentclass[letterpaper]{scrbook}
\usepackage[inner=3cm,top=2.5cm,outer=3.5cm]{geometry}

\renewcommand\familydefault{bch}
\usepackage[utf8]{inputenc}
\usepackage{microtype}
\usepackage[small]{caption}
\usepackage[small]{titlesec}
\raggedbottom

% Graphics -------------------------------------------------------------------
\usepackage[pdftex]{graphicx}
\graphicspath{{_include/}}
\DeclareGraphicsExtensions{.png,.pdf}

% Code formatting ------------------------------------------------------------
\usepackage{fancyvrb}
\usepackage{courier}
\usepackage{listings}
\usepackage{color}
\usepackage{alltt}


\definecolor{comment}{rgb}{0.60, 0.60, 0.53}
\definecolor{background}{rgb}{0.97, 0.97, 1.00}
\definecolor{string}{rgb}{0.863, 0.066, 0.266}
\definecolor{number}{rgb}{0.0, 0.6, 0.6}
\definecolor{variable}{rgb}{0.00, 0.52, 0.70}
\lstset{
  basicstyle=\ttfamily,
  keywordstyle=\bfseries, 
  identifierstyle=,  
  commentstyle=\color{comment} \emph,
  stringstyle=\color{string},
  showstringspaces=false,
  columns = fullflexible,
  backgroundcolor=\color{background},
  mathescape = true,
  escapeinside=&&,
  fancyvrb
}
\newcommand{\code}[1]{\lstinline!#1!}
\newcommand{\f}[1]{\lstinline!#1()!}



% Links ----------------------------------------------------------------------

\usepackage{hyperref}
\definecolor{slateblue}{rgb}{0.07,0.07,0.488}
\hypersetup{colorlinks=true,linkcolor=slateblue,anchorcolor=slateblue,citecolor=slateblue,filecolor=slateblue,urlcolor=slateblue,bookmarksnumbered=true,pdfview=FitB}
\usepackage{url}

% Tables ---------------------------------------------------------------------
\usepackage{longtable}
\usepackage{booktabs}

% Miscellaneous --------------------------------------------------------------
\usepackage{pdfsync}
\usepackage{appendix}

\usepackage[round,sort&compress,sectionbib]{natbib}
\bibliographystyle{plainnat}


\title{ggplot2}
\author{Hadley Wickham}

\begin{document}
\fi


\chapter{Toolbox}
\label{cha:toolbox}

\section{Introduction}\label{sec:introduction}

Graphical objects, or geoms for short, are a key feature of {\tt geom\_plot}.  Geoms create visual objects on the plot that allow you to see you data.  By choosing various types of geoms, you can recreate common plots.  For example, the {\tt geom\_point} geom will make a scatterplot, and the {\tt geom\_line} geom will create a line plot.  The advantage of the geom based system is that you can easily combine different geoms to create almost any plot that you can think of.  This section describes geom functions in detail, including what geoms are currently available in {\tt geom\_plot} and how you can go about creating your own.

Use a mixture of {\tt ggplot()} and {\tt qplot()} calls.  If you need a reminder on how to translate between the two, see Section~\ref{sec:qplot-ggplot}.

\section{Basics}\label{sub:basics}

These geoms are the basic geoms used to build up almost all of the other geoms.  These are useful for creating basic graphics, and when building your own geom function (see section \ref{sec:writing_your_own}).

\begin{itemize}
  \item {\tt geom\_point}: points
  \item {\tt geom\_line}, {\tt geom\_path}: paths and lines.  Lines are paths that have their x-axis values ordered in increasing value.
  \item {\tt geom\_polygon}: polygons
  \item {\tt geom\_bar}: bars
  \item {\tt geom\_text}: text
  \item {\tt geom\_tile}: tiles, rectangles which form a regular tessellation of the plane
\end{itemize}

\section{Displaying distributions}\label{sec:distributions}

There are quite a few geoms associated with displaying distributions:

\begin{itemize}
	\item {\tt geom\_boxplot}: box and whisker plot, for a continuous variable possibly conditioned by a categorical variable
	\item {\tt geom\_jitter}: a crude way of investigating densities
	\item {\tt geom\_quantile}: quantiles, for a continuous variable conditional on another continuous variable.
	\item {\tt geom\_density}: 
	\item {\tt geom\_histogram}: 
	\item {\tt geom\_2ddensity}: for displaying the density of points on the plot surface.
\end{itemize}

(Ask Heike about this)

We have a number of plots available to investigate distributions, depending on the number of variables, and whether we are interested in the conditional or joint distribution.

Single variable
+ cont: histogram, density plot, boxplot
+ cat:  barchart

Two variables: conditional
+ cat  | cat:  mosaic plot
+ cat  | cont: ?
+ cont | cont: quantiles
+ cat  | cont: boxplot

Two variables: joint
+ cat  * cat:  fluctuation diagram
+ cat  * cont: boxplots?
+ cont * cont: bagplot, geom\_2ddensity

Jittered points can be used for any joint distribution (or conditional if one or both variables are categorical)

\section{Dealing with overplotting}\label{sec:overplotting}

The simplest way to deal with overplotting is to bin the plot into small squares and count the number of points that lies in each square, much like a 2D histogram.  This count can then be visualised as the third variable on a plot.  However, breaking the plot into many small squares produces distracting visual artefacts.  Carr (reference) suggests using hexagons instead, and this is implemented with {\tt geom\_hexagon}, using the capabilities of the {\tt hexbin package}.

A continuous analogue of this is to compute a 2D density function and then visualise this as coloured tiles or contour lines.  This can be done with {\tt geom\_2ddensity}.

Another approach to dealing with overplotting is to add supplemental information to help guide the eye to the true shape of or pattern within the data:

\begin{itemize}
	\item {\tt geom\_smooth} add a smooth line showing the mean.
	\item {\tt geom\_quantile} add a smooth line showing any quantile you are interested in.
\end{itemize}

\section{``3d'' plots}

{\tt geom\_plot} currently does not support true 3D plots.  However, it does offer two tools for producing pseudo-3d plots, the imageplot and the contour plot.

\begin{itemize}
	\item {\tt geom\_tile}: map z variable to fill colour
	\item {\tt geom\_contour}: useful for smoother surfaces
	\item {\tt geom\_point}: can map abs(z) variable to size and sign(z) to colour
\end{itemize}

These are often better than ``true'' 3d for static plots anyway, because many perceptual cues necessary for accurate depth perception (eg. occlusion, parallax) are not present in static plots.  You can also manually project data points in a higher dimensional space by multiplying by a projection matrix.  However, correctly representing occlusion or generating correct perspective effects will require considerably more effort.  You may want to look at RGL (\url{http://www.rgl.com}) and rggobi (\url{http://www.ggobi.org/ggobi}) for other solutions.

\section{Revealing uncertainty}\label{sub:displaying_uncertainty}

If you have information about the uncertainty present in your data, possibly from a model or distributional assumptions, it is often useful to visualise.  There are two geoms that allow you to do this depending on whether you have point or functional confidence intervals:

\begin{itemize}
	\item {\tt geom\_errorbar}: for pointwise confidence intervals
	\item {\tt geom\_ribbon}: ribbons of variable width, useful for displaying confidence intervals around functions
\end{itemize}

There are two ways to display standard errors with {\tt ggplot}.  For point standard errors, you can use the {\tt errorbar} geom.  For continuous or functional standard errors, you can use the {\tt ribbon} grob.  We've have already seen an example of this: the {\tt ribbon} grob is used inside {\tt smooth} to display the standard errors of the smooth.  Because there are so many different ways to calculate standard errors, the calculation is up to you.  {\tt ggplot} only provides facilities for displaying them once you have them.

For both {\tt ribbon} and {\tt errobar} you can specify confidence internals using {\tt min} and {\tt lower} which specify the upper and lower edges of the confidence band.

Calculating with {\tt stat\_sum}, {\tt stat\_smooth} etc.

\section{Annotating a plot}\label{sub:annotating_a_plot}

Any of the basic geoms can be used to annotate a plot with additional output, for example, adding text with {\tt geom\_text}, or a point illustrating the mean with {\tt geom\_point}.  Additionally, there are several geoms whose use is almost entirely for annotation.  These are:

\begin{itemize}
	\item {\tt geom\_vline}, {\tt geom\_hline}: add vertical or horizontal lines to a plot
	\item {\tt geom\_abline}: add lines with arbitrary slope and intercept to a plot
\end{itemize}

See also \secref{sec:adding_annotation} for ways to add more general types of annotation using grid graphics.

\section{Plots for weighted data}\label{sec:weighted_data}

When you have aggregated data where each row in the dataset represents multiple observations, you need some way to take into account the weighting variable.  Since there are no variables appropriate for weighting in the diamonds data, we will use some data collected on Midwest states in the 2000 US census.  The data consists mainly of percentages (eg. percent white, percent below poverty line, percentage with college degree) and some information for each county (area, total population, population density).

There are few different things we might want to weight by: 

\begin{itemize}
	\item nothing, to look at county numbers
	\item total population, to work with absolute numbers
	\item area, to investigate geographic effects
\end{itemize}

\noindent The choice of a weighting variable profoundly effects what we are looking at in the plot and the conclusions that we will draw.  There are two aesthetic attributes that can be used to adjust for weights.  Firstly, for simple geoms like lines and points, you can make the size of the grob proportional to the number of points, using the {\tt size} aesthetic, as follows:

% decumar<<< 
% interweave({
% midwest <- read.csv("~/Documents/graphics/weighted/midwest.csv")
% qplot(percwhite, percbelowpoverty, data=midwest)
% qplot(percwhite, percbelowpoverty, data=midwest, size=poptotal)
% qplot(percwhite, percbelowpoverty, data=midwest, size=area)
% })
% |||
\begin{alltt}
> midwest <- read.csv("~/Documents/graphics/weighted/midwest.csv")
> qplot(percwhite, percbelowpoverty, data = midwest)
\includegraphics[scale=0.5]{4b8c600efdc1f31f77c7c3368cac11d2}

> qplot(percwhite, percbelowpoverty, data = midwest, size = poptotal)
\includegraphics[scale=0.5]{f3976daae1f4a7ef912ea259ca3dd8f5}

> qplot(percwhite, percbelowpoverty, data = midwest, size = area)
\includegraphics[scale=0.5]{440b95f469e5b96a08e077d4a177e603}

\end{alltt}
% >>>

For more complicated grobs which involve some statistical transformation, we specify weights with the {\tt weight} aesthetic.  These weights will be passed on to the statistical summary function.  Weights are supported for every case where it makes sense: smoothers, quantile regressions, box plots, histograms, and density plots.  You can't see this weighting variable directly, and it doesn't produce a legend, but it will change the results of the statistical summary.

The following example shows how weighting by population density effects the relationship between percent white and percent below the poverty line.

% decumar<<< 
% interweave({
% qplot(percwhite, percbelowpoverty, data=midwest, geom=c("point","smooth"), method=lm)
% qplot(percwhite, percbelowpoverty, data=midwest, size=popdensity, weight=popdensity,geom=c("point","smooth"), method=lm)
% })
% |||
\begin{alltt}
> qplot(percwhite, percbelowpoverty, data = midwest, geom = c("point", 
+     "smooth"), method = lm)
\includegraphics[scale=0.5]{54d5b5a95badb06ff2339bfd51826af0}

> qplot(percwhite, percbelowpoverty, data = midwest, size = popdensity, 
+     weight = popdensity, geom = c("point", "smooth"), method = lm)
\includegraphics[scale=0.5]{3353081c3ffa3fd62b69fb2c0aef9142}

\end{alltt}
% >>>

When we weight a histogram or density plot by total population, we change from looking at the distribution of the number of counties, to the distribution of the number of people.  This example shows the difference this makes for a histogram and density plot of the percentage below the poverty line.

% decumar<<< 
% interweave({
% qplot(percbelowpoverty, data=midwest, geom="histogram", binwidth=1)
% qplot(percbelowpoverty, data=midwest, geom="histogram", weight=poptotal, binwidth=1)
% })
% |||
\begin{alltt}
> qplot(percbelowpoverty, data = midwest, geom = "histogram", binwidth = 1)
\includegraphics[scale=0.5]{d8cef4ee59175bda59eaa2647a64b930}

> qplot(percbelowpoverty, data = midwest, geom = "histogram", weight = poptotal, 
+     binwidth = 1)
\includegraphics[scale=0.5]{affb33e87c76bc94bd3ea9eb2df0190c}

\end{alltt}
% >>>

\ifwhole
\else
  \nobibliography{/Users/hadley/documents/phd/references}
  \end{document}
\fi

\providecommand{\setflag}{\newif \ifwhole \wholefalse}
\setflag
\ifwhole\else
\documentclass[oneside,letterpaper]{scrbook}
\usepackage{fullpage}
\usepackage[utf8]{inputenc}
\usepackage[pdftex]{graphicx}
\usepackage{hyperref}
\usepackage{minitoc}
\usepackage{pdfsync}
\usepackage{alltt}
\usepackage[round,sort&compress,sectionbib]{natbib}
\bibliographystyle{plainnat}

%\setcounter{secnumdepth}{-1}

\title{ggplot}
\author{Hadley Wickham}

\renewcommand{\topfraction}{0.9}	% max fraction of floats at top
\renewcommand{\bottomfraction}{0.8}	% max fraction of floats at bottom
%   Parameters for TEXT pages (not float pages):
\setcounter{topnumber}{2}
\setcounter{bottomnumber}{2}
\renewcommand{\dbltopfraction}{0.9}	% fit big float above 2-col. text
\renewcommand{\textfraction}{0.07}	% allow minimal text w. figs
%   Parameters for FLOAT pages (not text pages):
\renewcommand{\floatpagefraction}{0.7}	% require fuller float pages
% N.B.: floatpagefraction MUST be less than topfraction !!
\renewcommand{\dblfloatpagefraction}{0.7}	% require fuller float pages

\newcommand{\grobref}[1]{{\tt #1} (page \pageref{sub:#1})}
\newcommand{\secref}[1]{\ref{#1} (\pageref{#1})}


\raggedbottom

\begin{document}
\fi

\setchapterpreamble[u]{% 
\dictum[Anonymous]{Forecasting is the art of saying 
what is going to happen and then to explain 
why it didn’t.}} 

\chapter{Scales}

\section{Introduction}\label{sec:introduction}

Scales control the mapping between data space and aesthetic space.  They convert your data values into aesthetic attributes that can be a perceived: colour, shape, size, etc.  Each type of aesthetic attribute has a default scale, and may have other scales that provide different types of mappings.  For example, the default colour scale uses equally space hues, but other scales allow you to generate a gradient between two or three different colours.  This section describes the basic operation of a scale, the details of all the different scales and instructions on how to make your own.

\section{Legend and axis labels}\label{sec:legend_and_axis_labels}

Each scale function has a ``name'' argument, usually the first argument after the plot name.  This determines the label which will appear on the axis or legend. You can supply text strings (using ``\\n'' for line breaks) or mathematical expressions (as described in \verb|plotmath|):

% decumar<<< 
% interweave({
% qplot(tip, total_bill, data=tips, colour=tip/total_bil)
% sccolour(p, "Tip rate")
% sccolour(p, "This is a very long label\nsplit over two lines")
% sccolour(p, expression(beta * x^2))
% })
% |||
% >>>

\verb|qplot| provides some additional shortcuts for axis labels, see \secref{sec:other_options} for details.

\section{Aesthetic attributes}\label{sec:aesthetic_attributes}

The following aesthetic attributes are available in {\tt ggplot}.  This list is not exhaustive, other authors may add grob functions that use other mappings.  

Any aesthetic that a grob does not understand will be silently ignored.  If a scale is not doing what you expect, make sure to check the you don't have any spelling mistakes, as these will not raise an error message.

Position scales
Other scales
 * categorical
 * continuous

\begin{itemize}
	\item colour
	\item fill
	\item group
	\item id
	\item linetype
	\item shape
	\item size
	\item dimension: height, width
	\item position: x, y, z
\end{itemize}

There are also a couple of aesthetic attributes that can't be perceived directly.  The most important of these is the {\tt id} aesthetic, which divides the the data set into discrete components.   This is used in line and path plots to separate the data for different lines, and in the groups grob to divide the different groups. 

Scales may take multiple inputs and return multiple outputs.

\section{Using scales}\label{sec:using_scales}

Scales are added by default whenever you use {\tt qplot} or {\tt ggplot}.  If you want to modify them, you need to manually add a scale to the plot.  This will automatically override any defaults.

% decumar<<< 
% interweave({
% quakes$depthc <- chop(quakes$depth)
% p <- qplot(long, lat, . ~ depthc, data=quakes)
% str(p$scales)
% p <- pscontinuous(p, "x", breaks=c(5,10))
% str(p$scales)
% })
% |||
% >>>

All position scales start with {\tt sc}, all others with {\tt sc}.

\section{How to write your own scales}\label{sec:how_to_write_your_own_scales}


\ifwhole
\else
	\bibliography{bibliography}
  \end{document}
\fi

\providecommand{\setflag}{\newif \ifwhole \wholefalse}
\setflag
\ifwhole\else

% Typography and geometry ----------------------------------------------------
\documentclass[letterpaper]{scrbook}
\usepackage[inner=3cm,top=2.5cm,outer=3.5cm]{geometry}

\renewcommand\familydefault{bch}
\usepackage[utf8]{inputenc}
\usepackage{microtype}
\usepackage[small]{caption}
\usepackage[small]{titlesec}
\raggedbottom

% Graphics -------------------------------------------------------------------
\usepackage[pdftex]{graphicx}
\graphicspath{{_include/}}
\DeclareGraphicsExtensions{.png,.pdf}

% Code formatting ------------------------------------------------------------
\usepackage{fancyvrb}
\usepackage{courier}
\usepackage{listings}
\usepackage{color}
\usepackage{alltt}


\definecolor{comment}{rgb}{0.60, 0.60, 0.53}
\definecolor{background}{rgb}{0.97, 0.97, 1.00}
\definecolor{string}{rgb}{0.863, 0.066, 0.266}
\definecolor{number}{rgb}{0.0, 0.6, 0.6}
\definecolor{variable}{rgb}{0.00, 0.52, 0.70}
\lstset{
  basicstyle=\ttfamily,
  keywordstyle=\bfseries, 
  identifierstyle=,  
  commentstyle=\color{comment} \emph,
  stringstyle=\color{string},
  showstringspaces=false,
  columns = fullflexible,
  backgroundcolor=\color{background},
  mathescape = true,
  escapeinside=&&,
  fancyvrb
}
\newcommand{\code}[1]{\lstinline!#1!}
\newcommand{\f}[1]{\lstinline!#1()!}



% Links ----------------------------------------------------------------------

\usepackage{hyperref}
\definecolor{slateblue}{rgb}{0.07,0.07,0.488}
\hypersetup{colorlinks=true,linkcolor=slateblue,anchorcolor=slateblue,citecolor=slateblue,filecolor=slateblue,urlcolor=slateblue,bookmarksnumbered=true,pdfview=FitB}
\usepackage{url}

% Tables ---------------------------------------------------------------------
\usepackage{longtable}
\usepackage{booktabs}

% Miscellaneous --------------------------------------------------------------
\usepackage{pdfsync}
\usepackage{appendix}

\usepackage[round,sort&compress,sectionbib]{natbib}
\bibliographystyle{plainnat}


\title{ggplot2}
\author{Hadley Wickham}

\begin{document}
\fi

\chapter{Positioning}
\label{cha:position}

\section{Introduction}

This chapter introduces ways of adjusting how plots are laid out on page, and how to manipulate the coordinate system within a plot.

Faceting is a mechanism for automatically laying out multiple plots on a plot.  It splits the data into subsets, and then plots each subset on a different place on the page.  This is only one small type of plot layout that you might want to do.  Section~\ref{sec:grid_layout} shows how to use grid to layout any number of plots in any way that you want.


\section{Position adjustments}
\label{sec:position}

Position adjustments apply minor tweaks to the position of elements within a layer.  Figure~\ref{fig:position} lists all of the position adjustments available with in ggplot.  

They are normally used with discrete data - continuous data typically doesn't overlap exactly, and when it does (because of high data density) minor adjustments are sufficient to fix the problem.

\begin{table}
  \begin{center}
  \begin{tabular}{ll}
    \toprule
    Name      & Description  \\
    \midrule
    dodge    & Adjust position by dodging overlaps to the side \\
    fill     & Stack overlapping objects on top of one another, and standardise have equal height\\
    identity & Don't adjust position \\
    jitter   & Jitter points to avoid overplotting \\
    stack    & Stack overlapping objects on top of one another \\
    \bottomrule
  \end{tabular}
  \end{center}
  \caption{caption}
  \label{fig:position}
\end{table}

Best illustrated with a bar chart.  Figure~\ref{fig:position-bar} shows stacking, filling and dodging.  Stacking puts bars on the same x on top of one-another; filling does the same, but normalises height to 1; and dodging places the bars side-by-side.  The identity adjustment (i.e.\ do nothing) doesn't make much sense for bars, but is shown in Figure~\ref{fig:position-identity} along with a line plot of the same data for reference.

Filling makes a conditional display - removes the marginal effect of column total. 

All geoms have a default position adjustment associated with them.  You can see by printing the geom.

\begin{alltt}
  > geom_point()
  ...
  position_identity: ()
  ...
  > geom_bar()
  ...
  position_stack: ()
  ...
\end{alltt}

\begin{figure}[htbp]
  \centering
    \includegraphics[width = 0.5 \linewidth]{position-stack}%
    \includegraphics[width = 0.5 \linewidth]{position-fill}
  \caption{Position adjustments for a bar chart}
  \label{fig:position-bar}
\end{figure}

\begin{figure}[htbp]
  \centering
    \includegraphics[width = 0.5 \linewidth]{position-dodge}
  \caption{Position adjustments for a bar chart}
  \label{fig:position-dodge}
\end{figure}

\begin{figure}[htbp]
  \centering
    \includegraphics[width = 0.5 \linewidth]{position-identity}%
    \includegraphics[width = 0.5 \linewidth]{position-identity2}
  \caption{The identity position adjustment}
  \label{fig:position-identity}
\end{figure}


\section{Coordinate systems}
\label{sec:coord}

99\% of the time we use a Cartesian coordinate system, where the 2d position of an element is made from the independent combination of the two.

Changing the coordinate system changes the appearance of the grob - a straight line maybe no longer be straight.  Illustrate with polar coordinates and coord\_trans.  

Similar to a scale, but inputs and outputs a tuple.

The only thing that is invariant under transformation is a point.  Assumption that coordinate transformations are smooth in some sense, so that very short lines will still be lines in the new coordinate system.

All coordinate systems are two dimensional.  Maybe 3d graphics one day.

Coordinate systems are the most complicated aspect of the grammar to get absolutely correct and many are still are work in progress.  However, do try them and out and report problems that you experience.


Coordinate systems are responsible for drawing the guides within the plot (i.e.\ grid lines) and 

\begin{table}
  \begin{center}
  \begin{tabular}{ll}
    \toprule
    Name      & Description  \\
    \midrule
    cartesian & Cartesian coordinates                  \\
    equal     & Equal scale cartesian coordinates      \\
    flip      & Flipped cartesian coordinates          \\
    map       & Map projections                        \\
    polar     & Polar coordinates                      \\
    trans     & Transformed cartesian coordinate system\\
    \bottomrule
    
  \end{tabular}
  \end{center}
  \caption{Coordinate systems available in ggplot}
  \label{tbl:coord}
\end{table}

As with the other components in ggplot2, you generate the R name by joining {\tt coord\_} and the name of the coordinate system.  For example, the default Cartesian coordinate system is {\tt coord\_cartesian}.

Writing your own coordinate system is described in Section~\ref{sec:my_coord}.


\section{Faceting}
\label{sec:faceting}

Faceting can be thought of a special type of coordinate system, which is hierarchical.  At the top level we have a coordinate system created by the categorical variables that we are faceting by, and then within each of these regions another coordinate system generated by the x and y position of each graphic.

Faceting is discussed previous in XXX and XXX.  Here I will go into more detail, and provide more examples.  When specifying a faceting formula, you specify a grid of row and column variables.  Variables appearing on a row or column together, is like nesting, only combinations that appear in the data will appear in the plot.  If variables that are specified on rows and columns are crossed: all combinations will be shown, including those that didn't appear in the original data set.

Used to generate small multiples of the same plot for different subsets of the data.  Small multiples are a powerful tool for ...  

\subsection{Facet grid}

You can also specify margins to display, by giving the names of the variables you want margins for.  

% decumar<<< 
% interweave({
% str(smiths)
% })
% |||
%>>>

Currently, {\tt ggplot} doesn't make any distinction between structural and non-structural missings in the facets - you can't tell whether a combination did not appear in the original dataset, or if it had no data.  Dealing with this properly may require some extensions to R's missing value system.  


\subsubsection{Margins}\label{sub:margins}

Faceting a plot is like creating a contingency table.  In contingency tables it is often useful to display marginal totals (totals over a row or column) as well as the individual cells.  It is also useful to be able to do this with graphics.  We can produce graphical margins using the the {\tt margins} argument.  This allows you to compare the conditional patterns with the marginal patterns.

You can either specify that all margins should be displayed, using {\tt margins = TRUE}, or by listing the names of the variables that you want margins for, {\tt margins = c("sex","age")}.  You can also use \verb|"grand_row"| or \verb|"grand_col"| to produce grand row and grand column margins respectively.

This example shows how the margins appear.  In the first plot, there are no margins, and we only see conditional plots.  In the second example, we see margins over columns, but not rows, and in the final example we see all possible margins.  The facet in the lower right corner displays all data points.

% decumar<<< 
% interweave({
% qplot(price, data=diamonds, facets= cut~ ., geom="histogram")
% })
% |||
\begin{alltt}
> qplot(price, data = diamonds, facets = cut ~ ., geom = "histogram")
\includegraphics[scale=0.5]{1a07c6d5903b310295255cfcfa587a9a}

\end{alltt}
% >>>

Plots with many facets and margins may be more appropriate for printing, rather than on screen display, as the higher resolution allows you to compare many more subsets.

\subsection{Facet wrap}
\label{sub:facet_wrap}

An alternative is facet wrap.  Instead of having a 2d grid generate by the combination of two (or more) variables, facet\_wrap essentially makes a long ribbon of panels and wraps it onto the screen.  

This is useful if you have a single variable that with many levels and want to arrange the plots in a more space efficient manner.

This is very similar to how trellising in lattice works.

\subsection{Controlling scales}
\label{sub:controlling_scales}

For both types of faceting you have some control over how the position scales work across plots.  There are two extremes.  One extreme is the default where all plots can share the same scale.  The other extreme is where each plot has it's own set of position scales.  

There also some intermediate levels of sharing:

\begin{itemize}
  \item shared
  \item individual
  \item rows-columns: each row has its own horizontal scale, and each column has its own vertical scale (facet\_grid only)
  \item variable: each plot has its own scale, but they are sizes so that the mapping between the plot and the data is the same in every panel (facet\_wrap only.)
\end{itemize}



\subsection{Missing faceting columns}\label{sub:missing_faceting_columsn}

If you facet on the original data set and add grobs with a new dataset, {\tt ggplot} follows some simple heuristics to try and do what you want.  

If the new dataset contains all of the faceting columns then each facet will have a specialised grob. If the new dataset contains none of the faceting columns, then each facet will use the same grob.  For anything in between, i.e. the new dataset contains some of the existing faceting columns, then the grobs will be duplicated over the faceting columns that are not present.  This sounds complicated, but most of the time it should do what you want.  If you need precise control, just make sure that every new dataset contains all the faceting columns used in the original dataset.

% decumar<<< 
% interweave({
% auckland <- data.frame(lat=-20 , lat=-25, long=175, label="Auck" )
% p <- ggpoint(ggplot(quakes, formula=. ~ depth, aes=list(x=lat, y=long)))
% ggtext(p, data=auckland, aes=list(label=label))
% })
% |||
% >>>  


\subsection{Grouping vs. faceting}

Faceting is an alternative to using aesthetics (like colour, shape or size) to distinguish different groups.  Each has strengths and weaknesses, centred around the distance between the groups.

With faceting, each group is quite far apart, in its own panel - there is no overlap between the groups.  This is good if the groups overlap a lot, but it does make small differences harder to see.  The groups are in the same panel when using aesthetics, so may overlap, but otherwise small differences are easier to see.  Illustrated in Figure~\ref{fig:facet-vs-groups}

Faceting will also work with much larger number of groups, and because you can split in two dimensions, you can compare two variables simultaneously more easily than using two different aesthetics (conjunctions are not preattentive).

\begin{figure}[htbp]
  \centering
    \includegraphics[width=0.19\textwidth]{position-fvg-1}
    \includegraphics[width=0.79\textwidth]{position-fvg-2}

    \includegraphics[width=0.19\textwidth]{position-fvg-3}
    \includegraphics[width=0.79\textwidth]{position-fvg-4}
  \caption{caption}
  \label{fig:facet-vs-groups}
\end{figure}


% \subsection{Continuous variables}\label{sub:continuous_variables}
% 
% To use continuous variables as faceting variables, you will first need to convert them to categorical.  Also, as the faceting formula currently does not support calculated variables, so you will need to save them in the data frame first.
% 
% {\tt ggplot} provides a convenient function for converting a continuous variable to a categorical variable: {\tt chop}.  Chop takes the following arguments:
% 
% \begin{itemize}
%   \item {\tt x}: vector of numbers to chop up into categorical variable
%   \item {\tt n}: number of bins to cut the variable into
%   \item {\tt method}: method to use, equal ranges, equal numbers in each range or pretty ranges
% \end{itemize}
% 
% The following examples shows the use of chop to explore the earthquakes data set.
% 
% % decumar<<< 
% % ggopt(aspect.ratio = 1)
% % interweave({
% % quakes$depthc <- chop(quakes$depth)
% % qplot(long, lat, . ~ depthc, data=quakes)
% % quakes$magc <- chop(quakes$mag, n=6, method="pretty")
% % qplot(long, lat, . ~ magc, data=quakes)
% % })
% % |||
% % >>>
% 
% Breaking by quantiles, so that each facet has approximately the same number of points, is usually the best choice as it is easier to compare the facets because you know the same number of points are in each, even in the presences of overplotting.

\ifwhole
\else
  \nobibliography{/Users/hadley/documents/phd/references}
  \end{document}
\fi
\chapter{Polishing your plots for publication}\label{cha:polishing}

In this chapter you will learn how to prepare polished plots for
publication. Most of this chapter focusses on the theming capability of
\texttt{ggplot} which allows you to control many non-data aspects of
plot appearance, but you will also learn how to adjust geom, stat and
scale defaults, and the best way to save plots for inclusion into other
software packages. Together with the next chapter, manipulating plot
rendering with \textbf{grid}, you will learn how to control every visual
aspect of the plot to get exactly the appearance that you want.
\index{Publication!polishing plots for}

The visual appearance of the plot is determined by both data and
non-data related components. \hyperref[sec:themes]{Themes} introduces
the theme system which controls all aspects of non-data display. By now
you should be familiar with the many ways that you can alter the
data-related components of the plot---layers and scales---to visualise
your data and change the appearance of the plot. In
\hyperref[sec:theme-scale-geom]{customising scales and geoms} you will
learn how you can change the defaults for these, so that you do not need
to repeat the same parameters again and again.

\hyperref[sec:saving]{Saving your output} discusses the chapter with a
discussion about how to get your graphics out of R and into LaTeX, Word
or other presentation or word-processing software.
\hyperref[sec:grid-layout]{Multiple plots on the same page} concludes
with a discussion of how to lay out multiple plots on a single page.

\hyperdef{}{sec:themes}{\section{Themes}\label{sec:themes}}

The appearance of non-data elements of the plot is controlled by the
theme system. The theme system does not affect how the data is rendered
by geoms, or how it is transformed by scales. Themes don't change the
perceptual properties of the plot, but they do help you make the plot
aesthetically pleasing or match existing style guides. Themes give you
control over things like the fonts in all parts of the plot: the title,
axis labels, axis tick labels, strips, legend labels and legend key
labels; and the colour of ticks, grid lines and backgrounds (panel,
plot, strip and legend). \index{Themes} \index{Publication!themes}

This separation of control into data and non-data parts is quite
different than base and lattice graphics. In base and lattice graphics,
most functions take a large number of arguments that specify both data
and non-data appearance, which makes the functions complicated and hard
to learn. \texttt{ggplot} takes a different approach: when creating the
plot you determine how the data is displayed, then \emph{after} it has
been created you can edit every detail of the rendering, using the
theming system. Some of the effects of changing the theme of a plot are
shown in Figure\textasciitilde{}\ref{fig:themes}. The two plots show the
two themes included by default in \texttt{ggplot}.

\includegraphics{figures/themes-1.pdf}
\includegraphics{figures/themes-2.pdf}

Like many other areas of \texttt{ggplot}, themes can be controlled on
multiple levels from the coarse to fine. You can:

\begin{itemize}
\itemsep1pt\parskip0pt\parsep0pt
\item
  Use a \hyperref[sec:built-in]{built-in theme}. This affects every
  element of the plot in a visually consistent manner. The default theme
  uses a grey panel background with white gridlines, while the
  alternative theme uses a white background with grey gridlines.
\item
  \hyperref[sec:theme-elements]{Modify a single element of a built-in
  theme}. Each theme is made up of multiple elements. The theme system
  comes with a number of built-in element rendering functions with a
  limited set of parameters. By adjusting these parameters you can
  control things like text size and colour, background and grid line
  colours and text orientation. By combining multiple elements you can
  create your own theme. 
\end{itemize}

\noindent Generally each of these theme settings can be applied
globally, to all plots, or locally to a single plot. How to do this is
described in each section.

\hyperdef{}{sec:built-in}{\subsection{Built-in
themes}\label{sec:built-in}}

There are two built-in themes. \index{Themes!built-in} The default,
\texttt{theme\_gray()}, uses a very light grey background with white
gridlines. This follows from the advice of {[}Tufte (2006); tufte:1990;
tufte:2001; tufte:1997{]} and {[}Brewer (1994); carr:2002; carr:1994;
carr:1999{]}. We can still see the gridlines to aid in the judgement of
position (Cleveland 1993), but they have little visual impact and we can
easily `tune' them out. The grey background gives the plot a similar
colour (in a typographical sense) to the remainder of the text, ensuring
that the graphics fit in with the flow of a text without jumping out
with a bright white background. Finally, the grey background creates a
continuous field of colour which ensures that the plot is perceived as a
single visual entity. \indexf{theme_grey}

The other built-in theme, \texttt{theme\_bw()}, has a more traditional
white background with dark grey gridlines.
Figure\textasciitilde{}\ref{fig:themes} shows some of the difference
between these themes. \index{White background}
\index{Themes!white background} \indexf{theme_bw}

Both themes have a single parameter, \texttt{base\_size}, which controls
the base font size. The base font size is the size that the axis titles
use: the plot title is 20\% bigger, and the tick and strip labels are
20\% smaller. If you want to control these sizes separately, you'll need
to modify the individual elements as described in the following section.

You can apply themes in two ways:

\begin{itemize}
\itemsep1pt\parskip0pt\parsep0pt
\item
  Globally, affecting all plots when they are drawn:
  \texttt{theme\_set(theme\_grey())} or
  \texttt{theme\_set(theme\_bw())}. \texttt{theme\_set()} returns the
  previous theme so that you can restore it later if you want.
  \indexf{theme_set}
\item
  Locally, for an individual plot: \texttt{qplot(...) + theme\_grey()}.
  A locally applied theme will override the global default.
\end{itemize}

\noindent The following example shows a few of these combinations:

\begin{Shaded}
\begin{Highlighting}[]
\NormalTok{>}\StringTok{ }\NormalTok{hgram <-}\StringTok{ }\KeywordTok{qplot}\NormalTok{(rating, }\DataTypeTok{data =} \NormalTok{movies, }\DataTypeTok{binwidth =} \DecValTok{1}\NormalTok{)}
\NormalTok{>}\StringTok{ }
\ErrorTok{>}\StringTok{ }\CommentTok{# Themes affect the plot when they are drawn, }
\ErrorTok{>}\StringTok{ }\CommentTok{# not when they are created}
\ErrorTok{>}\StringTok{ }\NormalTok{hgram}
\end{Highlighting}
\end{Shaded}

\begin{flushleft}\includegraphics{figures/hgram-1} \end{flushleft}

\begin{Shaded}
\begin{Highlighting}[]
\NormalTok{>}\StringTok{ }\NormalTok{previous_theme <-}\StringTok{ }\KeywordTok{theme_set}\NormalTok{(}\KeywordTok{theme_bw}\NormalTok{())}
\NormalTok{>}\StringTok{ }\NormalTok{hgram}
\end{Highlighting}
\end{Shaded}

\begin{flushleft}\includegraphics{figures/hgram-2} \end{flushleft}

\begin{Shaded}
\begin{Highlighting}[]
\NormalTok{>}\StringTok{ }
\ErrorTok{>}\StringTok{ }\CommentTok{# You can override the theme for a single plot by adding }
\ErrorTok{>}\StringTok{ }\CommentTok{# the theme to the plot. Here we apply the original theme}
\ErrorTok{>}\StringTok{ }\NormalTok{hgram +}\StringTok{ }\NormalTok{previous_theme}
\end{Highlighting}
\end{Shaded}

\begin{flushleft}\includegraphics{figures/hgram-3} \end{flushleft}

\begin{Shaded}
\begin{Highlighting}[]
\NormalTok{>}\StringTok{ }
\ErrorTok{>}\StringTok{ }\CommentTok{# Permanently restore the original theme}
\ErrorTok{>}\StringTok{ }\KeywordTok{theme_set}\NormalTok{(previous_theme)}
\end{Highlighting}
\end{Shaded}

\hyperdef{}{sec:theme-elements}{\subsection{Theme elements and element
functions}\label{sec:theme-elements}}

A theme is made up of multiple \emph{elements} which control the
appearance of a single item on the plot, as listed in
Table\textasciitilde{}\ref{tbl:elements}. There are three elements that
have individual \texttt{x} and \texttt{y} settings: \texttt{axis.text},
\texttt{axis.title} and \texttt{strip.text}. Having a different setting
for the horizontal and vertical elements allows you to control how text
should appear in different orientations. The appearance of each element
is controlled by an \emph{element function}. \index{Themes!elements}

\begin{table}
  \begin{center}
  \begin{tabular}{lll}\\
    \toprule
    Theme element              & Type     & Description  \\
    \midrule                              
    \texttt{axis.line}         & segment  & line along axis  \\
    \texttt{axis.text.x}       & text     & x axis label  \\
    \texttt{axis.text.y}       & text     & y axis label  \\
    \texttt{axis.ticks}        & segment  & axis tick marks  \\
    \texttt{axis.title.x}      & text     & horizontal tick labels  \\
    \texttt{axis.title.y}      & text     & vertical tick labels  \\[0.5em]

    \texttt{legend.background} & rect     & background of legend  \\
    \texttt{legend.key}        & rect     & background underneath legend keys \\
    \texttt{legend.text}       & text     & legend labels  \\
    \texttt{legend.title}      & text     & legend name  \\[0.5em]

    \texttt{panel.background}  & rect     & background of panel  \\
    \texttt{panel.border}      & rect     & border around panel  \\
    \texttt{panel.grid.major}  & line     & major grid lines \\
    \texttt{panel.grid.minor}  & line     & minor grid lines \\
    \texttt{plot.background}   & rect     & background of the entire plot \\
    \texttt{plot.title}        & text     & plot title   \\[0.5em]

    \texttt{strip.background}  & rect     & background of facet labels   \\
    \texttt{strip.text.x}      & text     & text for horizontal strips  \\
    \texttt{strip.text.y}      & text     & text for vertical strips  \\
    \bottomrule                           
  
  \end{tabular}
  \end{center}
  \caption{Theme elements}
  \label{tbl:elements}
\end{table}

There are four basic types of built-in element functions: text, lines
and segments, rectangles and blank. Each element function has a set of
parameters that control the appearance as described below:

\begin{itemize}
\itemsep1pt\parskip0pt\parsep0pt
\item
  \texttt{element\_text()} draws labels and headings. You can control
  the font \texttt{family}, \texttt{face}, \texttt{colour},
  \texttt{size}, \texttt{hjust}, \texttt{vjust}, \texttt{angle} and
  \texttt{lineheight}. \index{Themes!labels} \indexf{element_text}
\end{itemize}

The following code shows the effect of changing these parameters on the
plot title. The results are shown in
Figure\textasciitilde{}\ref{fig:theme-text}. Changing the angle is
probably more useful for tick labels. When changing the angle you will
probably also need to change `hjust\} to 0 or 1.

\begin{Shaded}
\begin{Highlighting}[]
\NormalTok{hgramt <-}\StringTok{ }\NormalTok{hgram +}\StringTok{ }\KeywordTok{labs}\NormalTok{(}\DataTypeTok{title =} \StringTok{"This is a histogram"}\NormalTok{)}
\NormalTok{hgramt}
\end{Highlighting}
\end{Shaded}

\begin{figure}
\includegraphics[width=0.32\linewidth]{figures/theme-text-1} \caption{Changing the appearance of the plot title.\label{fig:theme-text1}}
\end{figure}

\begin{Shaded}
\begin{Highlighting}[]
\NormalTok{hgramt +}\StringTok{ }\KeywordTok{theme}\NormalTok{(}\DataTypeTok{plot.title =} \KeywordTok{element_text}\NormalTok{(}\DataTypeTok{size =} \DecValTok{20}\NormalTok{))}
\end{Highlighting}
\end{Shaded}

\begin{figure}
\includegraphics[width=0.32\linewidth]{figures/theme-text-2} \caption{Changing the appearance of the plot title.\label{fig:theme-text2}}
\end{figure}

\begin{Shaded}
\begin{Highlighting}[]
\NormalTok{hgramt +}\StringTok{ }\KeywordTok{theme}\NormalTok{(}\DataTypeTok{plot.title =} 
              \KeywordTok{element_text}\NormalTok{(}\DataTypeTok{size =} \DecValTok{20}\NormalTok{, }\DataTypeTok{colour =} \StringTok{"red"}\NormalTok{))}
\end{Highlighting}
\end{Shaded}

\begin{figure}
\includegraphics[width=0.32\linewidth]{figures/theme-text-3} \caption{Changing the appearance of the plot title.\label{fig:theme-text3}}
\end{figure}

\begin{Shaded}
\begin{Highlighting}[]
\NormalTok{hgramt +}\StringTok{ }\KeywordTok{theme}\NormalTok{(}\DataTypeTok{plot.title =} 
              \KeywordTok{element_text}\NormalTok{(}\DataTypeTok{size =} \DecValTok{20}\NormalTok{, }\DataTypeTok{hjust =} \DecValTok{0}\NormalTok{))}
\end{Highlighting}
\end{Shaded}

\begin{figure}
\includegraphics[width=0.32\linewidth]{figures/theme-text-4} \caption{Changing the appearance of the plot title.\label{fig:theme-text4}}
\end{figure}

\begin{Shaded}
\begin{Highlighting}[]
\NormalTok{hgramt +}\StringTok{ }\KeywordTok{theme}\NormalTok{(}\DataTypeTok{plot.title =} 
              \KeywordTok{element_text}\NormalTok{(}\DataTypeTok{size =} \DecValTok{20}\NormalTok{, }\DataTypeTok{face =} \StringTok{"bold"}\NormalTok{))}
\end{Highlighting}
\end{Shaded}

\begin{figure}
\includegraphics[width=0.32\linewidth]{figures/theme-text-5} \caption{Changing the appearance of the plot title.\label{fig:theme-text5}}
\end{figure}

\begin{Shaded}
\begin{Highlighting}[]
\NormalTok{hgramt +}\StringTok{ }\KeywordTok{theme}\NormalTok{(}\DataTypeTok{plot.title =} 
              \KeywordTok{element_text}\NormalTok{(}\DataTypeTok{size =} \DecValTok{20}\NormalTok{, }\DataTypeTok{angle =} \DecValTok{180}\NormalTok{))}
\end{Highlighting}
\end{Shaded}

\begin{figure}
\includegraphics[width=0.32\linewidth]{figures/theme-text-6} \caption{Changing the appearance of the plot title.\label{fig:theme-text6}}
\end{figure}

\begin{itemize}
\itemsep1pt\parskip0pt\parsep0pt
\item
  \texttt{element\_line()} draws lines with the same options but in a
  slightly different way. Make sure you match the appropriate type or
  you will get strange grid errors. For these element functions you can
  control the \texttt{colour}, \texttt{size} and \texttt{linetype}.
  These options are illustrated with the code and the results are shown
  in Figure\textasciitilde{}\ref{fig:theme-line}. \indexf{element_line}
\end{itemize}

\begin{Shaded}
\begin{Highlighting}[]
\NormalTok{hgram +}\StringTok{ }\KeywordTok{theme}\NormalTok{(}\DataTypeTok{panel.grid.major =} \KeywordTok{element_line}\NormalTok{(}\DataTypeTok{colour =} \StringTok{"red"}\NormalTok{))}
\end{Highlighting}
\end{Shaded}

\begin{figure}
\includegraphics[width=0.32\linewidth]{figures/theme-line-1} \caption{Changing the appearance of lines and segments in the plot.\label{fig:theme-line1}}
\end{figure}

\begin{Shaded}
\begin{Highlighting}[]
\NormalTok{hgram +}\StringTok{ }\KeywordTok{theme}\NormalTok{(}\DataTypeTok{panel.grid.major =} \KeywordTok{element_line}\NormalTok{(}\DataTypeTok{size =} \DecValTok{2}\NormalTok{))}
\end{Highlighting}
\end{Shaded}

\begin{figure}
\includegraphics[width=0.32\linewidth]{figures/theme-line-2} \caption{Changing the appearance of lines and segments in the plot.\label{fig:theme-line2}}
\end{figure}

\begin{Shaded}
\begin{Highlighting}[]
\NormalTok{hgram +}\StringTok{ }\KeywordTok{theme}\NormalTok{(}\DataTypeTok{panel.grid.major =} \KeywordTok{element_line}\NormalTok{(}\DataTypeTok{linetype =} \StringTok{"dotted"}\NormalTok{))}
\end{Highlighting}
\end{Shaded}

\begin{figure}
\includegraphics[width=0.32\linewidth]{figures/theme-line-3} \caption{Changing the appearance of lines and segments in the plot.\label{fig:theme-line3}}
\end{figure}

\begin{Shaded}
\begin{Highlighting}[]
\NormalTok{hgram +}\StringTok{ }\KeywordTok{theme}\NormalTok{(}\DataTypeTok{axis.line =} \KeywordTok{element_line}\NormalTok{())}
\end{Highlighting}
\end{Shaded}

\begin{figure}
\includegraphics[width=0.32\linewidth]{figures/theme-line-4} \caption{Changing the appearance of lines and segments in the plot.\label{fig:theme-line4}}
\end{figure}

\begin{Shaded}
\begin{Highlighting}[]
\NormalTok{hgram +}\StringTok{ }\KeywordTok{theme}\NormalTok{(}\DataTypeTok{axis.line =} \KeywordTok{element_line}\NormalTok{(}\DataTypeTok{colour =} \StringTok{"red"}\NormalTok{))}
\end{Highlighting}
\end{Shaded}

\begin{figure}
\includegraphics[width=0.32\linewidth]{figures/theme-line-5} \caption{Changing the appearance of lines and segments in the plot.\label{fig:theme-line5}}
\end{figure}

\begin{Shaded}
\begin{Highlighting}[]
\NormalTok{hgram +}\StringTok{ }\KeywordTok{theme}\NormalTok{(}\DataTypeTok{axis.line =} \KeywordTok{element_line}\NormalTok{(}\DataTypeTok{size =} \FloatTok{0.5}\NormalTok{, }\DataTypeTok{linetype =} \StringTok{"dashed"}\NormalTok{))}
\end{Highlighting}
\end{Shaded}

\begin{figure}
\includegraphics[width=0.32\linewidth]{figures/theme-line-6} \caption{Changing the appearance of lines and segments in the plot.\label{fig:theme-line6}}
\end{figure}

\begin{itemize}
\itemsep1pt\parskip0pt\parsep0pt
\item
  \texttt{theme\_rect()} draws rectangles, mostly used for backgrounds,
  you can control the \texttt{fill} colour and border \texttt{colour},
  \texttt{size} and \texttt{linetype}. Examples shown in
  Figure\textasciitilde{}\ref{fig:theme-background} are created with the
  code below: \index{Background} \index{Themes!background}
  \indexf{theme_rect}
\end{itemize}

\begin{Shaded}
\begin{Highlighting}[]
\NormalTok{hgram +}\StringTok{ }\KeywordTok{theme}\NormalTok{(}\DataTypeTok{plot.background =} \KeywordTok{element_rect}\NormalTok{(}\DataTypeTok{fill =} \StringTok{"grey80"}\NormalTok{, }\DataTypeTok{colour =} \OtherTok{NA}\NormalTok{))}
\end{Highlighting}
\end{Shaded}

\begin{figure}
\includegraphics[width=0.32\linewidth]{figures/theme-background-1} \caption{Changing the appearance of the plot and panel background\label{fig:theme-background1}}
\end{figure}

\begin{Shaded}
\begin{Highlighting}[]
\NormalTok{hgram +}\StringTok{ }\KeywordTok{theme}\NormalTok{(}\DataTypeTok{plot.background =} \KeywordTok{element_rect}\NormalTok{(}\DataTypeTok{size =} \DecValTok{2}\NormalTok{))}
\end{Highlighting}
\end{Shaded}

\begin{figure}
\includegraphics[width=0.32\linewidth]{figures/theme-background-2} \caption{Changing the appearance of the plot and panel background\label{fig:theme-background2}}
\end{figure}

\begin{Shaded}
\begin{Highlighting}[]
\NormalTok{hgram +}\StringTok{ }\KeywordTok{theme}\NormalTok{(}\DataTypeTok{plot.background =} \KeywordTok{element_rect}\NormalTok{(}\DataTypeTok{colour =} \StringTok{"red"}\NormalTok{))}
\end{Highlighting}
\end{Shaded}

\begin{figure}
\includegraphics[width=0.32\linewidth]{figures/theme-background-3} \caption{Changing the appearance of the plot and panel background\label{fig:theme-background3}}
\end{figure}

\begin{Shaded}
\begin{Highlighting}[]
\NormalTok{hgram +}\StringTok{ }\KeywordTok{theme}\NormalTok{(}\DataTypeTok{panel.background =} \KeywordTok{element_rect}\NormalTok{())}
\end{Highlighting}
\end{Shaded}

\begin{figure}
\includegraphics[width=0.32\linewidth]{figures/theme-background-4} \caption{Changing the appearance of the plot and panel background\label{fig:theme-background4}}
\end{figure}

\begin{Shaded}
\begin{Highlighting}[]
\NormalTok{hgram +}\StringTok{ }\KeywordTok{theme}\NormalTok{(}\DataTypeTok{panel.background =} \KeywordTok{element_rect}\NormalTok{(}\DataTypeTok{colour =} \OtherTok{NA}\NormalTok{))}
\end{Highlighting}
\end{Shaded}

\begin{figure}
\includegraphics[width=0.32\linewidth]{figures/theme-background-5} \caption{Changing the appearance of the plot and panel background\label{fig:theme-background5}}
\end{figure}

\begin{Shaded}
\begin{Highlighting}[]
\NormalTok{hgram +}\StringTok{ }\KeywordTok{theme}\NormalTok{(}\DataTypeTok{panel.background =} \KeywordTok{element_rect}\NormalTok{(}\DataTypeTok{linetype =} \StringTok{"dotted"}\NormalTok{))}
\end{Highlighting}
\end{Shaded}

\begin{figure}
\includegraphics[width=0.32\linewidth]{figures/theme-background-6} \caption{Changing the appearance of the plot and panel background\label{fig:theme-background6}}
\end{figure}

\begin{itemize}
\itemsep1pt\parskip0pt\parsep0pt
\item
  \texttt{element\_blank()} draws nothing. Use this element type if you
  don't want anything drawn, and no space allocated for that element.
  The following example uses \texttt{element\_blank()} to progressively
  suppress the appearance of elements we're not interested in. The
  results are shown in Figure\textasciitilde{}\ref{fig:theme-blank}.
  Notice how the plot automatically reclaims the space previously used
  by these elements: if you don't want this to happen (perhaps because
  they need to line up with other plots on the page), use
  \texttt{colour = NA, fill = NA} as parameter to create invisible
  elements that still take up space. \indexf{element_blank}
\end{itemize}

\begin{Shaded}
\begin{Highlighting}[]
\NormalTok{hgramt}
\end{Highlighting}
\end{Shaded}

\begin{figure}
\includegraphics[width=0.32\linewidth]{figures/theme-blank-1} \caption{Progressively removing non-data elements from a plot with \texttt{element\_blank}.\label{fig:theme-blank1}}
\end{figure}

\begin{Shaded}
\begin{Highlighting}[]
\KeywordTok{last_plot}\NormalTok{() +}\StringTok{ }\KeywordTok{theme}\NormalTok{(}\DataTypeTok{panel.grid.minor =} \KeywordTok{element_blank}\NormalTok{())}
\end{Highlighting}
\end{Shaded}

\begin{figure}
\includegraphics[width=0.32\linewidth]{figures/theme-blank-2} \caption{Progressively removing non-data elements from a plot with \texttt{element\_blank}.\label{fig:theme-blank2}}
\end{figure}

\begin{Shaded}
\begin{Highlighting}[]
\KeywordTok{last_plot}\NormalTok{() +}\StringTok{ }\KeywordTok{theme}\NormalTok{(}\DataTypeTok{panel.grid.major =} \KeywordTok{element_blank}\NormalTok{())}
\end{Highlighting}
\end{Shaded}

\begin{figure}
\includegraphics[width=0.32\linewidth]{figures/theme-blank-3} \caption{Progressively removing non-data elements from a plot with \texttt{element\_blank}.\label{fig:theme-blank3}}
\end{figure}

\begin{Shaded}
\begin{Highlighting}[]
\KeywordTok{last_plot}\NormalTok{() +}\StringTok{ }\KeywordTok{theme}\NormalTok{(}\DataTypeTok{panel.background =} \KeywordTok{element_blank}\NormalTok{())}
\end{Highlighting}
\end{Shaded}

\begin{figure}
\includegraphics[width=0.32\linewidth]{figures/theme-blank-4} \caption{Progressively removing non-data elements from a plot with \texttt{element\_blank}.\label{fig:theme-blank4}}
\end{figure}

\begin{Shaded}
\begin{Highlighting}[]
\KeywordTok{last_plot}\NormalTok{() +}\StringTok{ }
\StringTok{  }\KeywordTok{theme}\NormalTok{(}\DataTypeTok{axis.title.x =} \KeywordTok{element_blank}\NormalTok{(), }
        \DataTypeTok{axis.title.y =} \KeywordTok{element_blank}\NormalTok{())}
\end{Highlighting}
\end{Shaded}

\begin{figure}
\includegraphics[width=0.32\linewidth]{figures/theme-blank-5} \caption{Progressively removing non-data elements from a plot with \texttt{element\_blank}.\label{fig:theme-blank5}}
\end{figure}

\begin{Shaded}
\begin{Highlighting}[]
\KeywordTok{last_plot}\NormalTok{() +}\StringTok{ }\KeywordTok{theme}\NormalTok{(}\DataTypeTok{axis.line =} \KeywordTok{element_line}\NormalTok{())}
\end{Highlighting}
\end{Shaded}

\begin{figure}
\includegraphics[width=0.32\linewidth]{figures/theme-blank-6} \caption{Progressively removing non-data elements from a plot with \texttt{element\_blank}.\label{fig:theme-blank6}}
\end{figure}

You can see the settings for the current theme with
\texttt{theme\_get()}. The output isn't included here because it takes
up several pages. You can modify the elements locally for a single plot
with \texttt{theme()} (as seen above), or globally for all future plots
with \texttt{theme\_update}.
Figure\textasciitilde{}\ref{fig:theme-update} shows the results of
pulling together multiple theme settings with the following code.
\index{Themes!updating} \indexf{theme_get} \indexf{theme}

\begin{Shaded}
\begin{Highlighting}[]
\NormalTok{old_theme <-}\StringTok{ }\KeywordTok{theme_update}\NormalTok{(}
  \DataTypeTok{plot.background =} \KeywordTok{element_rect}\NormalTok{(}\DataTypeTok{fill =} \StringTok{"#3366FF"}\NormalTok{),}
  \DataTypeTok{panel.background =} \KeywordTok{element_rect}\NormalTok{(}\DataTypeTok{fill =} \StringTok{"#003DF5"}\NormalTok{),}
  \DataTypeTok{axis.text.x =} \KeywordTok{element_text}\NormalTok{(}\DataTypeTok{colour =} \StringTok{"#CCFF33"}\NormalTok{),}
  \DataTypeTok{axis.text.y =} \KeywordTok{element_text}\NormalTok{(}\DataTypeTok{colour =} \StringTok{"#CCFF33"}\NormalTok{, }\DataTypeTok{hjust =} \DecValTok{1}\NormalTok{),}
  \DataTypeTok{axis.title.x =} \KeywordTok{element_text}\NormalTok{(}\DataTypeTok{colour =} \StringTok{"#CCFF33"}\NormalTok{, }\DataTypeTok{face =} \StringTok{"bold"}\NormalTok{),}
  \DataTypeTok{axis.title.y =} \KeywordTok{element_text}\NormalTok{(}\DataTypeTok{colour =} \StringTok{"#CCFF33"}\NormalTok{, }\DataTypeTok{face =} \StringTok{"bold"}\NormalTok{, }
   \DataTypeTok{angle =} \DecValTok{90}\NormalTok{)}
\NormalTok{)}
\KeywordTok{qplot}\NormalTok{(cut, }\DataTypeTok{data =} \NormalTok{diamonds, }\DataTypeTok{geom=}\StringTok{"bar"}\NormalTok{)}
\end{Highlighting}
\end{Shaded}

\includegraphics{figures/theme-update-1.pdf}

\begin{Shaded}
\begin{Highlighting}[]
\KeywordTok{qplot}\NormalTok{(cty, hwy, }\DataTypeTok{data =} \NormalTok{mpg)}
\end{Highlighting}
\end{Shaded}

\includegraphics{figures/theme-update-2.pdf}

\begin{Shaded}
\begin{Highlighting}[]
\KeywordTok{theme_set}\NormalTok{(old_theme)}
\end{Highlighting}
\end{Shaded}

There is some duplication in this example because we have to specify the
x and y elements separately. This is a necessary evil so that you can
have total control over the appearance of the elements. If you are
writing your own theme, you would probably want to write a function to
minimise this repetition.

\hyperdef{}{sec:theme-scale-geom}{\section{Customising scales and
geoms}\label{sec:theme-scale-geom}}

When producing a consistent theme, you may also want to tune some of the
scale and geom defaults. Rather than having to manually specify the
changes every time you add the scale or geom, you can use the following
functions to alter the default settings for scales and geoms.

\subsection{Scales}\label{sub:customise-scales}

DEPRECATED!

To change the default scale associated with an aesthetic, use
\texttt{set\_default\_scale()}. (See
Table\textasciitilde{}\ref{tbl:default-scales} for the defaults.) This
function takes three arguments: the name of the aesthetic, the type of
variable (discrete or continuous) and the name of the scale to use as
the default. Further arguments override the default parameters of the
scale. The following example sets up colour and fill scales for
black-and-white printing: \index{Scales!customising defaults}
\indexf{set_default_scale}

\subsection{Geoms and stats}\label{sub:geoms-and-stats}

You can customise geoms and stats in a similar way with
\texttt{update\_geom\_defaults()} and \texttt{update\_stat\_defaults()}.
Unlike the other theme settings these will only affect plots
\emph{created} after the setting has been changed, not all plots drawn
after the setting has been changed. The following example demonstrates
changing the default point colour and changing the default histogram to
a density (`true') histogram. \index{Geoms!customising defaults}
\index{Scales!customising defaults} \indexf{update_geom_defaults}

\begin{Shaded}
\begin{Highlighting}[]
\KeywordTok{update_geom_defaults}\NormalTok{(}\StringTok{"point"}\NormalTok{, }\KeywordTok{aes}\NormalTok{(}\DataTypeTok{colour =} \StringTok{"darkblue"}\NormalTok{))}
\KeywordTok{qplot}\NormalTok{(mpg, wt, }\DataTypeTok{data =} \NormalTok{mtcars)}
\KeywordTok{update_stat_defaults}\NormalTok{(}\StringTok{"bin"}\NormalTok{, }\KeywordTok{aes}\NormalTok{(}\DataTypeTok{y =} \NormalTok{..density..))}
\KeywordTok{qplot}\NormalTok{(rating, }\DataTypeTok{data =} \NormalTok{movies, }\DataTypeTok{geom =} \StringTok{"histogram"}\NormalTok{, }\DataTypeTok{binwidth =} \DecValTok{1}\NormalTok{)}
\end{Highlighting}
\end{Shaded}

Table\textasciitilde{}\ref{tbl:geom-defaults} lists all of the common
aesthetic defaults. If you change the defaults for one geom, it's a good
idea to change all the defaults for all the other geoms that you
commonly use so that your plots look consistent. If you are unsure on
what makes for a valid colour, line type, shape or size,
\hyperref[cha:specifications]{Specifications} gives the details.
\index{Aesthetics!defaults}

\begin{verbatim}
#> 
#> Attaching package: 'rvest'
#> 
#> The following object is masked from 'package:utils':
#> 
#>     history
\end{verbatim}

\begin{longtable}[c]{@{}lll@{}}
\caption{Default aesthetic values for geoms. See
\hyperref[cha:specifications]{specifications} for how the values are
interpreted by R.}\tabularnewline
\toprule
Aesthetic & Default & Cap\tabularnewline
\midrule
\endfirsthead
\toprule
Aesthetic & Default & Cap\tabularnewline
\midrule
\endhead
\texttt{alpha} & \texttt{0.4} & \texttt{smooth}\tabularnewline
\texttt{angle} & \texttt{0} & \texttt{text}\tabularnewline
\texttt{colour} & \texttt{\#3366FF} & \texttt{contour},
\texttt{density2d}, \texttt{quantile}, \texttt{smooth}\tabularnewline
\texttt{colour} & \texttt{black} & \texttt{abline}, \texttt{crossbar},
\texttt{density}, \texttt{dotplot}, \texttt{errorbar},
\texttt{errorbarh}, \texttt{freqpoly}, \texttt{hline},
\texttt{linerange}, \texttt{path}, \texttt{point}, \texttt{pointrange},
\texttt{rug}, \texttt{segment}, \texttt{step}, \texttt{text},
\texttt{vline}\tabularnewline
\texttt{colour} & \texttt{grey20} & \texttt{boxplot},
\texttt{violin}\tabularnewline
\texttt{colour} & \texttt{NA} & \texttt{polygon}\tabularnewline
\texttt{family} & `` & \texttt{text}\tabularnewline
\texttt{fill} & \texttt{black} & \texttt{dotplot}\tabularnewline
\texttt{fill} & \texttt{grey20} & \texttt{area}, \texttt{bar},
\texttt{polygon}, \texttt{rect}, \texttt{ribbon},
\texttt{tile}\tabularnewline
\texttt{fill} & \texttt{grey50} & \texttt{hex}\tabularnewline
\texttt{fill} & \texttt{grey60} & \texttt{bin2d},
\texttt{smooth}\tabularnewline
\texttt{fill} & \texttt{white} & \texttt{boxplot},
\texttt{violin}\tabularnewline
\texttt{fontface} & \texttt{1} & \texttt{text}\tabularnewline
\texttt{height} & \texttt{0.5} & \texttt{errorbarh}\tabularnewline
\texttt{hjust} & \texttt{0.5} & \texttt{text}\tabularnewline
\texttt{lineheight} & \texttt{1.2} & \texttt{text}\tabularnewline
\texttt{linetype} & \texttt{1} & \texttt{abline}, \texttt{area},
\texttt{bar}, \texttt{bin2d}, \texttt{contour}, \texttt{crossbar},
\texttt{density}, \texttt{density2d}, \texttt{errorbar},
\texttt{errorbarh}, \texttt{freqpoly}, \texttt{hline},
\texttt{linerange}, \texttt{path}, \texttt{pointrange},
\texttt{polygon}, \texttt{quantile}, \texttt{rect}, \texttt{ribbon},
\texttt{rug}, \texttt{segment}, \texttt{smooth}, \texttt{step},
\texttt{tile}, \texttt{vline}\tabularnewline
\texttt{linetype} & \texttt{solid} & \texttt{boxplot},
\texttt{violin}\tabularnewline
\texttt{shape} & \texttt{16} & \texttt{boxplot}, \texttt{point},
\texttt{pointrange}\tabularnewline
\texttt{size} & \texttt{0.1} & \texttt{tile}\tabularnewline
\texttt{size} & \texttt{0.5} & \texttt{abline}, \texttt{area},
\texttt{bar}, \texttt{bin2d}, \texttt{boxplot}, \texttt{contour},
\texttt{crossbar}, \texttt{density}, \texttt{density2d},
\texttt{errorbar}, \texttt{errorbarh}, \texttt{freqpoly}, \texttt{hex},
\texttt{hline}, \texttt{linerange}, \texttt{path}, \texttt{pointrange},
\texttt{polygon}, \texttt{quantile}, \texttt{rect}, \texttt{ribbon},
\texttt{rug}, \texttt{segment}, \texttt{smooth}, \texttt{step},
\texttt{violin}, \texttt{vline}\tabularnewline
\texttt{size} & \texttt{2} & \texttt{point}\tabularnewline
\texttt{size} & \texttt{5} & \texttt{text}\tabularnewline
\texttt{vjust} & \texttt{0.5} & \texttt{text}\tabularnewline
\texttt{weight} & \texttt{1} & \texttt{bar}, \texttt{bin2d},
\texttt{boxplot}, \texttt{contour}, \texttt{density}, \texttt{quantile},
\texttt{smooth}, \texttt{violin}\tabularnewline
\texttt{width} & \texttt{0.5} & \texttt{errorbar}\tabularnewline
\texttt{y} & \texttt{..count..} & \texttt{dotplot}\tabularnewline
\bottomrule
\end{longtable}

\hyperdef{}{sec:saving}{\section{Saving your output}\label{sec:saving}}

You have two basic choices of output: raster or vector. Vector graphics
are procedural. This means that they are essentially `infinitely'
zoomable; there is no loss of detail. Raster graphics are stored as an
array of pixels and have a fixed optimal viewing size.
Figure\textasciitilde{}\ref{fig:vector-raster} illustrates the basic
differences for a basic circle. A good description is available at
\url{http://tinyurl.com/rstrvctr}. \index{Saving} \index{Exporting}
\index{Publication!saving output}

Generally, vector output is more desirable, but for complex graphics
containing thousands of graphical objects it can be slow to render. In
this case, it may be better to switch to raster output. For printed use,
a high-resolution (e.g., 600 dpi) graphic may be an acceptable
compromise, but may be large.

\begin{figure}[htbp]
  \centering
    \includegraphics[width= 0.5\linewidth]{diagrams/vector-raster}
  \caption{The schematic difference between raster (left) and vector (right) graphics. }
  \label{fig:vector-raster}
\end{figure}

To save your output, you can use the typical R way with disk-based
graphics devices, which works for all packages, or a special function
from \texttt{ggplot} that saves the current plot: \texttt{ggsave()}.
\texttt{ggsave()} is optimised for interactive use and has the following
important arguments: \indexf{ggsave}

\begin{itemize}
\itemsep1pt\parskip0pt\parsep0pt
\item
  The \texttt{path} specifies the path where the image should be saved.
  The file extension will be used to automatically select the correct
  graphics device.
\item
  Three arguments control output size. If left blank, the size of the
  current on-screen graphics device will be used. \texttt{width} and
  \texttt{height} can be used to specify the absolute size, or
  \texttt{scale} to specify the size of the plot relative to the
  on-screen display. When creating the final versions of graphics it's a
  good idea to set \texttt{width} and \texttt{height} so you know
  exactly what size output you're going to get.
\item
  For raster graphics, the \texttt{dpi} argument controls the resolution
  of the plot. It defaults to 300, which is appropriate for most
  printers, but you may want to use 600 for particularly high-resolution
  output, or 72 for on-screen (e.g., web) display.
\end{itemize}

The following code shows these two methods. If you want to save multiple
plots to a single file, you will need to explicitly open a disk-based
graphics device (like \texttt{png()} or \texttt{pdf()}), print the plots
and then close it with \texttt{dev.off()}.

\begin{Shaded}
\begin{Highlighting}[]
\KeywordTok{qplot}\NormalTok{(mpg, wt, }\DataTypeTok{data =} \NormalTok{mtcars)}
\KeywordTok{ggsave}\NormalTok{(}\DataTypeTok{file =} \StringTok{"output.pdf"}\NormalTok{)}

\KeywordTok{pdf}\NormalTok{(}\DataTypeTok{file =} \StringTok{"output.pdf"}\NormalTok{, }\DataTypeTok{width =} \DecValTok{6}\NormalTok{, }\DataTypeTok{height =} \DecValTok{6}\NormalTok{)}
\CommentTok{# If inside a script, you will need to explicitly print() plots}
\KeywordTok{qplot}\NormalTok{(mpg, wt, }\DataTypeTok{data =} \NormalTok{mtcars)}
\KeywordTok{qplot}\NormalTok{(wt, mpg, }\DataTypeTok{data =} \NormalTok{mtcars)}
\KeywordTok{dev.off}\NormalTok{()}
\end{Highlighting}
\end{Shaded}

Table\textasciitilde{}\ref{tbl:graphic-recommendation} lists recommended
graphic formats for various tasks. R output generally works best as part
of a linux development tool chain: using png or pdf output in LaTeX
documents. With Microsoft Office it is easiest to use a high-resolution
(\texttt{dpi = 600}) png file. You can use vector output, but neither
Windows meta files nor postscript supports transparency, and while
postscript prints fine, it is only shown on screen if you add a preview
in another software package. Transparency is used to show confidence
intervals with the points showing through. If you copy and paste a graph
into Word, and see that the confidence interval bands have vanished,
that is the cause. The same advice holds for OpenOffice.
\index{Exporting!to Word} \index{Exporting!to Powerpoint}

If you are using LaTeX, I recommend including
\texttt{\textbackslash{}DeclareGraphicsExtensions\{.png,.pdf\}} in the
preamble. Then you don't need to specify the file extension in
\texttt{\textbackslash{}includegraphics\{\}} commands, but LaTeX will
pick png files in preference to pdf. \index{Exporting!to Latex} I choose
this order because you can produce all your files in pdf, and then go
back and re-render any big ones as png. Another useful command is
\texttt{\textbackslash{}graphicspath\{\}} which specifies a path in
which to look for graphics, allowing you to keep graphics in a separate
directory to the text.

\begin{table}
  \begin{center}
  \begin{tabular}{lll}
    \toprule
    Software & Recommended graphics device \\
    \midrule
    Illustrator & svg \\
    latex & ps \\
    MS Office & png (600 dpi) \\
    Open Office & png (600 dpi) \\
    pdflatex & pdf, png (600 dpi) \\
    web & png (72 dpi) \\
    \bottomrule 
  \end{tabular}
  \end{center}
  \caption{Recommended graphic output for different purposes.}
  \label{tbl:graphic-recommendation}
\end{table}

\hyperdef{}{sec:grid-layout}{\section{Multiple plots on the same
page}\label{sec:grid-layout}}

If you want to arrange multiple plots on a single page, you'll need to
learn a little bit of grid, the underlying graphics system used by
\texttt{ggplot}. The key concept you'll need to learn about is a
viewport: a rectangular subregion of the display. The default viewport
takes up the entire plotting region, and by customising the viewport you
can arrange a set of plots in just about any way you can imagine.
\index{Layout} \index{Publication!multiple plots on the same page}

To begin, let's create three plots that we can experiment with. When
arranging multiple plots on a page, it will usually be easiest to create
them, assign them to variables and then plot them. This makes it easier
to experiment with plot placement independent of content. The plots
created by the code below are shown in
Figure\textasciitilde{}\ref{fig:layout}.

\begin{Shaded}
\begin{Highlighting}[]
\NormalTok{(a <-}\StringTok{ }\KeywordTok{qplot}\NormalTok{(date, unemploy, }\DataTypeTok{data =} \NormalTok{economics, }\DataTypeTok{geom =} \StringTok{"line"}\NormalTok{))}
\end{Highlighting}
\end{Shaded}

\includegraphics{figures/layout-1.pdf}

\begin{Shaded}
\begin{Highlighting}[]
\NormalTok{(b <-}\StringTok{ }\KeywordTok{qplot}\NormalTok{(uempmed, unemploy, }\DataTypeTok{data =} \NormalTok{economics) +}\StringTok{ }
\StringTok{  }\KeywordTok{geom_smooth}\NormalTok{(}\DataTypeTok{se =} \NormalTok{F))}
\CommentTok{#> geom_smooth: method="auto" and size of largest group is <1000, so using loess. Use 'method = x' to change the smoothing method.}
\end{Highlighting}
\end{Shaded}

\includegraphics{figures/layout-2.pdf}

\begin{Shaded}
\begin{Highlighting}[]
\NormalTok{(c <-}\StringTok{ }\KeywordTok{qplot}\NormalTok{(uempmed, unemploy, }\DataTypeTok{data =} \NormalTok{economics, }\DataTypeTok{geom=}\StringTok{"path"}\NormalTok{))}
\end{Highlighting}
\end{Shaded}

\includegraphics{figures/layout-3.pdf}

\subsection{Subplots}

One common layout is to have a small subplot embedded drawn on top of
the main plot. To achieve this effect, we first plot the main plot, and
then draw the subplot in a smaller viewport. Viewports are created with
(surprise!) the \texttt{viewport()} function, with parameters
\texttt{x}, \texttt{y}, \texttt{width} and \texttt{height} to control
the size and position of the viewport. By default, the measurements are
given in `npc' units, which range from 0 to 1. The location (0, 0) is
the bottom left, (1, 1) the top right and (0.5, 0.5) the centre of
viewport. If these relative units don't work for your needs, you can
also use absolute units, like \texttt{unit(2, "cm")} or
\texttt{unit(1, "inch")}. \index{Sub-figures} \index{Subplots}

\begin{Shaded}
\begin{Highlighting}[]
\CommentTok{# A viewport that takes up the entire plot device}
\NormalTok{vp1 <-}\StringTok{ }\KeywordTok{viewport}\NormalTok{(}\DataTypeTok{width =} \DecValTok{1}\NormalTok{, }\DataTypeTok{height =} \DecValTok{1}\NormalTok{, }\DataTypeTok{x =} \FloatTok{0.5}\NormalTok{, }\DataTypeTok{y =} \FloatTok{0.5}\NormalTok{)}
\NormalTok{vp1 <-}\StringTok{ }\KeywordTok{viewport}\NormalTok{()}

\CommentTok{# A viewport that takes up half the width and half the height, }
\CommentTok{# located in the middle of the plot.}
\NormalTok{vp2 <-}\StringTok{ }\KeywordTok{viewport}\NormalTok{(}\DataTypeTok{width =} \FloatTok{0.5}\NormalTok{, }\DataTypeTok{height =} \FloatTok{0.5}\NormalTok{, }\DataTypeTok{x =} \FloatTok{0.5}\NormalTok{, }\DataTypeTok{y =} \FloatTok{0.5}\NormalTok{)}
\NormalTok{vp2 <-}\StringTok{ }\KeywordTok{viewport}\NormalTok{(}\DataTypeTok{width =} \FloatTok{0.5}\NormalTok{, }\DataTypeTok{height =} \FloatTok{0.5}\NormalTok{)}

\CommentTok{# A viewport that is 2cm x 3cm located in the center}
\NormalTok{vp3 <-}\StringTok{ }\KeywordTok{viewport}\NormalTok{(}\DataTypeTok{width =} \KeywordTok{unit}\NormalTok{(}\DecValTok{2}\NormalTok{, }\StringTok{"cm"}\NormalTok{), }\DataTypeTok{height =} \KeywordTok{unit}\NormalTok{(}\DecValTok{3}\NormalTok{, }\StringTok{"cm"}\NormalTok{))}
\end{Highlighting}
\end{Shaded}

By default, the x and y parameters control the location of the centre of
the viewport. When positioning the plot in other locations, you may need
to use the \texttt{just} parameter to control which corner of the plot
you are positioning. The following code gives some examples.

\begin{Shaded}
\begin{Highlighting}[]
\CommentTok{# A viewport in the top right}
\NormalTok{vp4 <-}\StringTok{ }\KeywordTok{viewport}\NormalTok{(}\DataTypeTok{x =} \DecValTok{1}\NormalTok{, }\DataTypeTok{y =} \DecValTok{1}\NormalTok{, }\DataTypeTok{just =} \KeywordTok{c}\NormalTok{(}\StringTok{"right"}\NormalTok{, }\StringTok{"top"}\NormalTok{))}
\CommentTok{# Bottom left}
\NormalTok{vp5 <-}\StringTok{ }\KeywordTok{viewport}\NormalTok{(}\DataTypeTok{x =} \DecValTok{0}\NormalTok{, }\DataTypeTok{y =} \DecValTok{0}\NormalTok{, }\DataTypeTok{just =} \KeywordTok{c}\NormalTok{(}\StringTok{"right"}\NormalTok{, }\StringTok{"bottom"}\NormalTok{))}
\end{Highlighting}
\end{Shaded}

To draw the plot in our new viewport, we use the \texttt{vp} argument of
the \texttt{ggplot.print()} method. This method is normally called
automatically whenever you evaluate something on the command line, but
because we want to customise the viewport, we need to call it ourselves.
The result of this is shown in
Figure\textasciitilde{}\ref{fig:subplot-1}.

\begin{Shaded}
\begin{Highlighting}[]
\KeywordTok{pdf}\NormalTok{(}\StringTok{"diagrams/polishing-subplot-1.pdf"}\NormalTok{, }\DataTypeTok{width =} \DecValTok{4}\NormalTok{, }\DataTypeTok{height =} \DecValTok{4}\NormalTok{)}
\NormalTok{subvp <-}\StringTok{ }\KeywordTok{viewport}\NormalTok{(}\DataTypeTok{width =} \FloatTok{0.4}\NormalTok{, }\DataTypeTok{height =} \FloatTok{0.4}\NormalTok{, }\DataTypeTok{x =} \FloatTok{0.75}\NormalTok{, }\DataTypeTok{y =} \FloatTok{0.35}\NormalTok{)}
\NormalTok{b}
\CommentTok{#> geom_smooth: method="auto" and size of largest group is <1000, so using loess. Use 'method = x' to change the smoothing method.}
\KeywordTok{print}\NormalTok{(c, }\DataTypeTok{vp =} \NormalTok{subvp)}
\KeywordTok{dev.off}\NormalTok{()}
\end{Highlighting}
\end{Shaded}

This gives us what we want, but we need to make a few tweaks to the
appearance: the text should be smaller, we want to remove the axis
labels and shrink the plot margins. The result is shown in
Figure\textasciitilde{}\ref{fig:subplot-2}.

\begin{Shaded}
\begin{Highlighting}[]
\NormalTok{csmall <-}\StringTok{ }\NormalTok{c +}\StringTok{ }
\StringTok{  }\KeywordTok{theme_gray}\NormalTok{(}\DecValTok{9}\NormalTok{) +}\StringTok{ }
\StringTok{  }\KeywordTok{labs}\NormalTok{(}\DataTypeTok{x =} \OtherTok{NULL}\NormalTok{, }\DataTypeTok{y =} \OtherTok{NULL}\NormalTok{) +}\StringTok{ }
\StringTok{  }\KeywordTok{theme}\NormalTok{(}\DataTypeTok{plot.margin =} \KeywordTok{unit}\NormalTok{(}\KeywordTok{rep}\NormalTok{(}\DecValTok{0}\NormalTok{, }\DecValTok{4}\NormalTok{), }\StringTok{"lines"}\NormalTok{))}

\KeywordTok{pdf}\NormalTok{(}\StringTok{"diagrams/polishing-subplot-2.pdf"}\NormalTok{, }\DataTypeTok{width =} \DecValTok{4}\NormalTok{, }\DataTypeTok{height =} \DecValTok{4}\NormalTok{)}
\NormalTok{b}
\CommentTok{#> geom_smooth: method="auto" and size of largest group is <1000, so using loess. Use 'method = x' to change the smoothing method.}
\KeywordTok{print}\NormalTok{(csmall, }\DataTypeTok{vp =} \NormalTok{subvp)}
\KeywordTok{dev.off}\NormalTok{()}
\end{Highlighting}
\end{Shaded}

\begin{figure}[htbp]
  \centering
  \subfigure[Figure with subplot.]{
    \includegraphics[width=0.5\textwidth]{diagrams/polishing-subplot-1}
    \label{fig:subplot-1}
  }%
  \subfigure[Subplot tweaked for better display.]{
    \includegraphics[width=0.5\textwidth]{diagrams/polishing-subplot-2}
    \label{fig:subplot-2}
  }
  \caption{Two examples of a figure with subplot. It will usually be necessary to tweak the theme settings of the subplot for optimum display.}
  \label{fig:subplot}
\end{figure}

Note we need to use \texttt{pdf()} (or \texttt{png()} etc.) to save the
plots to disk because \texttt{ggsave()} only saves a single plot.

\subsection{Rectangular grids}

A more complicated scenario is when you want to arrange a number of
plots in a rectangular grid. Of course you could create a series of
viewports and use what you've learned above, but doing all the
calculations by hand is cumbersome. A better approach is to use
\texttt{grid.layout()}, which sets up a regular grid of viewports with
arbitrary heights and widths. You still need to create each viewport,
but instead of explicitly specifying the position and size, you can
specify the row and column of the layout.

The following example shows how this work. We first create the layout,
here a 2 by 2 grid, then assign it to a viewport and push that viewport
on to the plotting device. Now we are ready to draw each plot into its
own position on the grid. We create a small function to save some
typing, and then draw each plot in the desired place on the grid. You
can supply a vector of rows or columns to span a plot over multiple
cells. The results are shown in
Figure\textasciitilde{}\ref{fig:layout-2}.

\begin{Shaded}
\begin{Highlighting}[]
\KeywordTok{pdf}\NormalTok{(}\StringTok{"diagrams/polishing-layout.pdf"}\NormalTok{, }\DataTypeTok{width =} \DecValTok{8}\NormalTok{, }\DataTypeTok{height =} \DecValTok{6}\NormalTok{)}
\KeywordTok{grid.newpage}\NormalTok{()}
\KeywordTok{pushViewport}\NormalTok{(}\KeywordTok{viewport}\NormalTok{(}\DataTypeTok{layout =} \KeywordTok{grid.layout}\NormalTok{(}\DecValTok{2}\NormalTok{, }\DecValTok{2}\NormalTok{)))}

\NormalTok{vplayout <-}\StringTok{ }\NormalTok{function(x, y) }
  \KeywordTok{viewport}\NormalTok{(}\DataTypeTok{layout.pos.row =} \NormalTok{x, }\DataTypeTok{layout.pos.col =} \NormalTok{y)}
\KeywordTok{print}\NormalTok{(a, }\DataTypeTok{vp =} \KeywordTok{vplayout}\NormalTok{(}\DecValTok{1}\NormalTok{, }\DecValTok{1}\NormalTok{:}\DecValTok{2}\NormalTok{))}
\KeywordTok{print}\NormalTok{(b, }\DataTypeTok{vp =} \KeywordTok{vplayout}\NormalTok{(}\DecValTok{2}\NormalTok{, }\DecValTok{1}\NormalTok{))}
\CommentTok{#> geom_smooth: method="auto" and size of largest group is <1000, so using loess. Use 'method = x' to change the smoothing method.}
\KeywordTok{print}\NormalTok{(c, }\DataTypeTok{vp =} \KeywordTok{vplayout}\NormalTok{(}\DecValTok{2}\NormalTok{, }\DecValTok{2}\NormalTok{))}
\KeywordTok{dev.off}\NormalTok{()}
\CommentTok{#> pdf }
\CommentTok{#>   2}
\end{Highlighting}
\end{Shaded}

\begin{figure}[htbp]
  \centering
    \includegraphics[width=\linewidth]{diagrams/polishing-layout}
  \caption{Three plots laid out in a grid using `grid.layout()`.}
  \label{fig:layout-2}
\end{figure}

By default \texttt{grid.layout()} makes each cell the same size, but you
can use the \texttt{widths} and \texttt{heights} arguments to make them
different sizes. See the documentation for \texttt{grid.layout()} for
more examples.

Brewer, Cynthia A. 1994. ``Color Use Guidelines for Mapping and
Visualization.'' In \emph{Visualization in Modern Cartography}, edited
by A.M. MacEachren and D.R.F. Taylor, 123--47. Elsevier Science.

Cleveland, William. 1993. ``A Model for Studying Display Methods of
Statistical Graphics.'' \emph{Journal of Computational and Graphical
Statistics} 2: 323--64. \url{http://stat.bell-labs.com/doc/93.4.ps}.

Tufte, Edward R. 2006. \emph{Beautiful Evidence}. Graphics Press.

\providecommand{\setflag}{\newif \ifwhole \wholefalse}
\setflag
\ifwhole\else
\documentclass[oneside,letterpaper]{scrbook}
\usepackage{fullpage}
\usepackage[utf8]{inputenc}
\usepackage[pdftex]{graphicx}
\usepackage{hyperref}
\usepackage{minitoc}
\usepackage{pdfsync}
\usepackage{alltt}
\usepackage[round,sort&compress,sectionbib]{natbib}
\bibliographystyle{plainnat}

%\setcounter{secnumdepth}{-1}

\title{ggplot}
\author{Hadley Wickham}

\renewcommand{\topfraction}{0.9}	% max fraction of floats at top
\renewcommand{\bottomfraction}{0.8}	% max fraction of floats at bottom
%   Parameters for TEXT pages (not float pages):
\setcounter{topnumber}{2}
\setcounter{bottomnumber}{2}
\renewcommand{\dbltopfraction}{0.9}	% fit big float above 2-col. text
\renewcommand{\textfraction}{0.07}	% allow minimal text w. figs
%   Parameters for FLOAT pages (not text pages):
\renewcommand{\floatpagefraction}{0.7}	% require fuller float pages
% N.B.: floatpagefraction MUST be less than topfraction !!
\renewcommand{\dblfloatpagefraction}{0.7}	% require fuller float pages

\newcommand{\grobref}[1]{{\tt #1} (page \pageref{sub:#1})}
\newcommand{\secref}[1]{\ref{#1} (\pageref{#1})}


\raggedbottom

\begin{document}
\fi

\setchapterpreamble[u]{% 
\dictum[Anonymous]{Forecasting is the art of saying 
what is going to happen and then to explain 
why it didn’t.}} 

\chapter{Data wrangling}

\section{Introduction}\label{sec:introduction}

To get the most out of ggplot, you need to be able to get your data into the form that {\tt ggplot} wants.  This chapter discusses what the best format is, and some ways to get your data into that form.  It includes an introduction to the {\tt reshape} package, which is a very useful companion to {\tt ggplot}.  I also discuss facetting and its uses in more detail, as well as providing many examples so that you can get more of an idea how it works.

Unlike other graphics packages in R, {\tt ggplot} provides very few ways to get your data in.  For example, lattice functions can take an optional data frame or use vectors direct from the environment.  {\tt ggplot} only works with data in the form of a data.frame (although qplot provides a convenient way of creating a data frame from vectors that exist in your data space).  This means that you have to explicitly arrange your data in the format that ggplot uses.  This makes it a little bit more verbose, but you can be more sure about exactly what is going on with your data.  It also allows a cleaner separation of concerns so that the graphics package deals only with plotting data, not wrangling it into different forms as well.

The package that I recommend for getting data into the right form (because I wrote it!), is {\tt reshape}.

This corresponds to my general philosophy on data.  

The most important thing is that everything should be explicit.  Your data is the most important thing, and if it gets corrupted or arranged in an inappropriate manner everything based on that data will be compromised.  For this reason, you need to do everything yourself.

Use lower case column names (a bit easier to type, but main thing is to be consistent)

\section{Using reshape}\label{sec:using_reshape}

A useful tool for getting data into the right shape to plot.  See the reshape documentation for more details.  Here is a basic introduction and some examples particularly relvent for {\tt ggplot}.

\section{Facetting}\label{sec:facetting}

Facetting is discussed previous in XXX and XXX.  Here I will go into more detail, and provide more examples.

You can set up facetting when you create the plot, {\tt ggplot(df, a ~ b)}, or later using \texttt{setfacets(p, formula = . ~ . , margins = FALSE)}.

\subsection{Continuous variables}\label{sub:continuous_variables}

To use continuous variables as facetting variables, you will first need to convert them to categorical.  Also, the facetting formula currently does not support calculated variables, so you will need to save them in the data frame first.

\subsection{Missing facetting columns}\label{sub:missing_facetting_columsn}

If you facet on the original data set and later add columns


\ifwhole
\else
	\bibliography{bibliography}
  \end{document}
\fi

\chapter{Reducing duplication}\label{cha:duplication}

\section{Introduction}

A major requirement of a good data analysis is flexibility. If the data
changes, or you discover something that makes you rethink your basic
assumptions, you need to be able to easily change many plots at once.
The main inhibitor of flexibility is duplication. If you have the same
plotting statement repeated over and over again, you have to make the
same change in many different places. Often just the thought of making
all those changes is exhausting!

This chapter describes three ways of reducing duplication. In
\hyperref[sec:iteration]{iteration}, you will learn how to iteratively
modify the previous plot, allowing you to build on top of your previous
work without having to retype a lot of code.
\hyperref[sec:templates]{Plot templates} will show you how to produce
plot `templates' that encapsulate repeated components that are defined
once and used in many different places. Finally,
\hyperref[sec:functions]{plot functions} talks about how to create
functions that create or modify plots. \index{Duplication!reducing}
\index{Reducing duplication}

\hyperdef{}{sec:iteration}{\section{Iteration}\label{sec:iteration}}

Whenever you create or modify a plot, \texttt{ggplot} saves a copy of
the result so you can refer to it in later expressions. You can access
this plot with \texttt{last\_plot()}. This is useful in interactive work
as you can start with a basic plot and then iteratively add layers and
tweak the scales until you get to the final result. The following code
demonstrates iteratively zooming in on a plot to find a region of
interest, and then adding a layer which highlights something interesting
that we have found: very few diamonds have equal x and y dimensions. The
plots are shown in Figure\textasciitilde{}\ref{fig:iterate-limits}.
\index{Iteration} \indexf{last_plot} \index{Duplication!iteration}

\begin{Shaded}
\begin{Highlighting}[]
\KeywordTok{qplot}\NormalTok{(x, y, }\DataTypeTok{data =} \NormalTok{diamonds, }\DataTypeTok{na.rm =} \OtherTok{TRUE}\NormalTok{)}
\KeywordTok{last_plot}\NormalTok{() +}\StringTok{ }\KeywordTok{xlim}\NormalTok{(}\DecValTok{3}\NormalTok{, }\DecValTok{11}\NormalTok{) +}\StringTok{ }\KeywordTok{ylim}\NormalTok{(}\DecValTok{3}\NormalTok{, }\DecValTok{11}\NormalTok{)}
\KeywordTok{last_plot}\NormalTok{() +}\StringTok{ }\KeywordTok{xlim}\NormalTok{(}\DecValTok{4}\NormalTok{, }\DecValTok{10}\NormalTok{) +}\StringTok{ }\KeywordTok{ylim}\NormalTok{(}\DecValTok{4}\NormalTok{, }\DecValTok{10}\NormalTok{)}
\KeywordTok{last_plot}\NormalTok{() +}\StringTok{ }\KeywordTok{xlim}\NormalTok{(}\DecValTok{4}\NormalTok{, }\DecValTok{5}\NormalTok{) +}\StringTok{ }\KeywordTok{ylim}\NormalTok{(}\DecValTok{4}\NormalTok{, }\DecValTok{5}\NormalTok{)}
\KeywordTok{last_plot}\NormalTok{() +}\StringTok{ }\KeywordTok{xlim}\NormalTok{(}\DecValTok{4}\NormalTok{, }\FloatTok{4.5}\NormalTok{) +}\StringTok{ }\KeywordTok{ylim}\NormalTok{(}\DecValTok{4}\NormalTok{, }\FloatTok{4.5}\NormalTok{)}
\KeywordTok{last_plot}\NormalTok{() +}\StringTok{ }\KeywordTok{geom_abline}\NormalTok{(}\DataTypeTok{colour =} \StringTok{"red"}\NormalTok{)}
\end{Highlighting}
\end{Shaded}

\begin{figure}
\includegraphics[width=0.32\linewidth]{figures/duplicationiterate-limits-1} \includegraphics[width=0.32\linewidth]{figures/duplicationiterate-limits-2} \includegraphics[width=0.32\linewidth]{figures/duplicationiterate-limits-3} \includegraphics[width=0.32\linewidth]{figures/duplicationiterate-limits-4} \includegraphics[width=0.32\linewidth]{figures/duplicationiterate-limits-5} \includegraphics[width=0.32\linewidth]{figures/duplicationiterate-limits-6} \caption{When 'zooming' in on the plot, it's useful to use \texttt{last\_plot()} iteratively to quickly find the best view. The final plot adds a line with slope 1 and intercept 0, confirming it is the square diamonds that are missing.\label{fig:iterate-limits}}
\end{figure}

Once you have tweaked the plot to your liking, it's a good idea to go
back and create a single expression that generates your final plot. This
is important as when you come back to the plot, you'll be able to
re-create the plot quickly, without having to step through your original
process. You many want to add a comment to your code to indicate exactly
why you chose that final plot. This is good practice in general for R
code: after experimenting interactively, you always want to create a
source file that re-creates your analysis. The following code shows the
final plot after our interactive modifications above.

\begin{Shaded}
\begin{Highlighting}[]
\KeywordTok{qplot}\NormalTok{(x, y, }\DataTypeTok{data =} \NormalTok{diamonds, }\DataTypeTok{na.rm =} \NormalTok{T) +}\StringTok{ }
\StringTok{  }\KeywordTok{geom_abline}\NormalTok{(}\DataTypeTok{colour =} \StringTok{"red"}\NormalTok{) +}
\StringTok{  }\KeywordTok{xlim}\NormalTok{(}\DecValTok{4}\NormalTok{, }\FloatTok{4.5}\NormalTok{) +}\StringTok{ }\KeywordTok{ylim}\NormalTok{(}\DecValTok{4}\NormalTok{, }\FloatTok{4.5}\NormalTok{)}
\end{Highlighting}
\end{Shaded}

\hyperdef{}{sec:templates}{\section{Plot
templates}\label{sec:templates}}

Each component of a \texttt{ggplot} plot is its own object and can be
created, stored and applied independently to a plot. This makes it
possible to create reusable components that can automate common tasks
and helps to offset the cost of typing the long function names. The
following example creates some colour scales and then applies them to
plots. The results are shown in
Figure\textasciitilde{}\ref{fig:gradient-rb}. \index{Templates}
\index{Duplication!templates}

\begin{Shaded}
\begin{Highlighting}[]
\NormalTok{gradient_rb <-}\StringTok{ }\KeywordTok{scale_colour_gradient}\NormalTok{(}\DataTypeTok{low =} \StringTok{"red"}\NormalTok{, }\DataTypeTok{high =} \StringTok{"blue"}\NormalTok{)}
\KeywordTok{qplot}\NormalTok{(cty, hwy, }\DataTypeTok{data =} \NormalTok{mpg, }\DataTypeTok{colour =} \NormalTok{displ) +}\StringTok{ }\NormalTok{gradient_rb}
\KeywordTok{qplot}\NormalTok{(bodywt, brainwt, }\DataTypeTok{data =} \NormalTok{msleep, }\DataTypeTok{colour =} \NormalTok{awake, }\DataTypeTok{log=}\StringTok{"xy"}\NormalTok{) +}
\StringTok{  }\NormalTok{gradient_rb}
\end{Highlighting}
\end{Shaded}

\begin{figure}
\includegraphics[width=0.49\linewidth]{figures/duplicationgradient-rb-1} \includegraphics[width=0.49\linewidth]{figures/duplicationgradient-rb-2} \caption{Saving a scale to a variable makes it easy to apply exactly the same scale to multiple plots.  You can do the same thing with layers and facets too.\label{fig:gradient-rb}}
\end{figure}

As well as saving single objects, you can also save vectors of
\texttt{ggplot} components. Adding a vector of components to a plot is
equivalent to adding each component of the vector in turn. The following
example creates two continuous scales that can be used to turn off the
display of axis labels and ticks. You only need to create these objects
once and you can apply them to many different plots, as shown in the
code below and Figure\textasciitilde{}\ref{fig:quiet}.
\index{Layers!reusing}

\begin{Shaded}
\begin{Highlighting}[]
\NormalTok{xquiet <-}\StringTok{ }\KeywordTok{scale_x_continuous}\NormalTok{(}\DataTypeTok{breaks =} \OtherTok{NULL}\NormalTok{)}
\NormalTok{yquiet <-}\StringTok{ }\KeywordTok{scale_y_continuous}\NormalTok{(}\DataTypeTok{breaks =} \OtherTok{NULL}\NormalTok{)}

\KeywordTok{qplot}\NormalTok{(mpg, wt, }\DataTypeTok{data =} \NormalTok{mtcars) +}\StringTok{ }\NormalTok{xquiet }
\KeywordTok{qplot}\NormalTok{(displ, cty, }\DataTypeTok{data =} \NormalTok{mpg) +}\StringTok{ }\NormalTok{xquiet +}\StringTok{ }\NormalTok{yquiet }
\end{Highlighting}
\end{Shaded}

\begin{figure}
\includegraphics[width=0.49\linewidth]{figures/duplicationquiet-1} \includegraphics[width=0.49\linewidth]{figures/duplicationquiet-2} \caption{Using 'quiet' x and y scales removes the labels and hides ticks and gridlines.\label{fig:quiet}}
\end{figure}

Similarly, it's easy to write simple functions that change the defaults
of a layer. For example, if you wanted to create a function that added
linear models to a plot, you could create a function like the one below.
The results are shown in Figure\textasciitilde{}\ref{fig:geom-lm}.

\begin{Shaded}
\begin{Highlighting}[]
\NormalTok{geom_lm <-}\StringTok{ }\NormalTok{function(}\DataTypeTok{formula =} \NormalTok{y ~}\StringTok{ }\NormalTok{x) \{}
  \KeywordTok{geom_smooth}\NormalTok{(}\DataTypeTok{formula =} \NormalTok{formula, }\DataTypeTok{se =} \OtherTok{FALSE}\NormalTok{, }\DataTypeTok{method =} \StringTok{"lm"}\NormalTok{)}
\NormalTok{\}}
\KeywordTok{qplot}\NormalTok{(mpg, wt, }\DataTypeTok{data =} \NormalTok{mtcars) +}\StringTok{ }\KeywordTok{geom_lm}\NormalTok{()}
\KeywordTok{library}\NormalTok{(}\StringTok{"splines"}\NormalTok{)}
\KeywordTok{qplot}\NormalTok{(mpg, wt, }\DataTypeTok{data =} \NormalTok{mtcars) +}\StringTok{ }\KeywordTok{geom_lm}\NormalTok{(y ~}\StringTok{ }\KeywordTok{ns}\NormalTok{(x, }\DecValTok{3}\NormalTok{))}
\end{Highlighting}
\end{Shaded}

\begin{figure}
\includegraphics[width=0.49\linewidth]{figures/duplicationgeom-lm-1} \includegraphics[width=0.49\linewidth]{figures/duplicationgeom-lm-2} \caption{Creating a custom geom function saves typing when creating plots with similar (but not the same) components.\label{fig:geom-lm}}
\end{figure}

Depending on how complicated your function is, it might even return
multiple components in a vector. You can build up arbitrarily complex
plots this way, reducing duplication wherever you find it. If you want
to create a plot that combines together many different components in a
pre-specified way, you might need to write a function that produces the
entire plot. This is described in the next section.

\hyperdef{}{sec:functions}{\section{Plot
functions}\label{sec:functions}}

If you are using the same basic plot again and again with different
datasets or different parameters, it may be worthwhile to wrap up all
the different options into a single function. Maybe you need to perform
some data restructuring or transformation, or need to combine the data
with a predefined model. In that case you will need to write a function
that produces \texttt{ggplot} plots. It's hard to give advice on how to
go about this because there are so many different possible scenarios,
but this section aims to point out some important things to think about.
\index{Duplication!functions} \index{Functions that create plots}

\begin{itemize}
\itemsep1pt\parskip0pt\parsep0pt
\item
  Since you're creating the plot within the environment of a function,
  you need to be extra careful about supplying the data to
  \texttt{ggplot()} as a data frame, and you need to double check that
  you haven't accidentally referred to any function local variables in
  your aesthetic mappings.
\item
  If you want to allow the user to provide their own variables for
  aesthetic mappings, I'd suggest using \texttt{aes\_string()}. This
  function works just like \texttt{aes}, but uses strings rather than
  unevaluated expressions. \texttt{aes\_string("cty", colour = "hwy")}
  is equivalent to \texttt{aes(cty, colour = hwy)}. Strings are much
  easier to work with than expressions.
  \index{Mappings!creating programmatically} \indexf{aes_string}
\item
  As mentioned in \hyperref[cha:data]{data}, you want to separate your
  plotting code into a function that does any data transformations and
  manipulations and a function that creates the plot. Generally, your
  plotting function should do no data manipulation, just create a plot.
  The following example shows one way to create parallel coordinate plot
  function, wrapping up the code used in
  \hyperref[sub:molten-data]{parallel coordinates plot}.
  \index{Parallel coordinates plot}
\end{itemize}

\begin{Shaded}
\begin{Highlighting}[]
\NormalTok{>}\StringTok{ }\NormalTok{rescale01 <-}\StringTok{ }\NormalTok{function(x) \{}
\NormalTok{+}\StringTok{   }\NormalTok{rng <-}\StringTok{ }\KeywordTok{range}\NormalTok{(x, }\DataTypeTok{na.rm =} \OtherTok{TRUE}\NormalTok{)}
\NormalTok{+}\StringTok{   }\NormalTok{(x -}\StringTok{ }\NormalTok{rng[}\DecValTok{1}\NormalTok{]) /}\StringTok{ }\NormalTok{(rng[}\DecValTok{2}\NormalTok{] -}\StringTok{ }\NormalTok{rng[}\DecValTok{1}\NormalTok{])}
\NormalTok{+}\StringTok{ }\NormalTok{\}}
\NormalTok{>}\StringTok{ }\NormalTok{pcp_data <-}\StringTok{ }\NormalTok{function(df) \{}
\NormalTok{+}\StringTok{   }\NormalTok{numeric <-}\StringTok{ }\KeywordTok{sapply}\NormalTok{(df, is.numeric)}
\NormalTok{+}\StringTok{   }\CommentTok{# Rescale numeric columns}
\NormalTok{+}\StringTok{   }\NormalTok{df[numeric] <-}\StringTok{ }\KeywordTok{lapply}\NormalTok{(df[numeric], rescale01)}
\NormalTok{+}\StringTok{   }\CommentTok{# Add row identified}
\NormalTok{+}\StringTok{   }\NormalTok{df$.row <-}\StringTok{ }\KeywordTok{rownames}\NormalTok{(df)}
\NormalTok{+}\StringTok{   }\CommentTok{# Treat numerics as value (aka measure) variables }
\NormalTok{+}\StringTok{   }\NormalTok{dfg <-}\StringTok{ }\NormalTok{tidyr::}\KeywordTok{gather_}\NormalTok{(df, }\StringTok{"variable"}\NormalTok{, }\StringTok{"value"}\NormalTok{, }\KeywordTok{names}\NormalTok{(df)[numeric])}
\NormalTok{+}\StringTok{   }\CommentTok{# Add pcp to class of the data frame}
\NormalTok{+}\StringTok{   }\KeywordTok{class}\NormalTok{(dfg) <-}\StringTok{ }\KeywordTok{c}\NormalTok{(}\StringTok{"pcp"}\NormalTok{, }\KeywordTok{class}\NormalTok{(dfg))}
\NormalTok{+}\StringTok{   }\NormalTok{dfg}
\NormalTok{+}\StringTok{ }\NormalTok{\}}
\NormalTok{>}\StringTok{ }\NormalTok{pcp <-}\StringTok{ }\NormalTok{function(df, ...) \{}
\NormalTok{+}\StringTok{   }\NormalTok{df <-}\StringTok{ }\KeywordTok{pcp_data}\NormalTok{(df)}
\NormalTok{+}\StringTok{   }\KeywordTok{ggplot}\NormalTok{(df, }\KeywordTok{aes}\NormalTok{(variable, value)) +}\StringTok{ }\KeywordTok{geom_line}\NormalTok{(}\KeywordTok{aes}\NormalTok{(}\DataTypeTok{group =} \NormalTok{.row))}
\NormalTok{+}\StringTok{ }\NormalTok{\}}
\NormalTok{>}\StringTok{ }\KeywordTok{pcp}\NormalTok{(mpg)}
\end{Highlighting}
\end{Shaded}

\includegraphics[width=0.75\linewidth]{figures/duplicationpcp_data-1}

\begin{Shaded}
\begin{Highlighting}[]
\NormalTok{>}\StringTok{ }\KeywordTok{pcp}\NormalTok{(mpg) +}\StringTok{ }\KeywordTok{aes}\NormalTok{(}\DataTypeTok{colour =} \NormalTok{drv)}
\end{Highlighting}
\end{Shaded}

\includegraphics[width=0.75\linewidth]{figures/duplicationpcp_data-2}

The best example of this technique is \texttt{qplot()}, and if you're
interesting in writing your own functions I strongly recommend you have
a look at the source code for this function and step through it line by
line to see how it works. If you've made your way this far through the
book you should have a pretty good grasp of all the \texttt{ggplot}
related code: most of the complexity is R tricks to correctly interpret
all of the possible plot types.


\appendix
\appendixpage
\index{Appendices}

\providecommand{\setflag}{\newif \ifwhole \wholefalse}
\setflag
\ifwhole\else

% Typography and geometry ----------------------------------------------------
\documentclass[letterpaper]{scrbook}
\usepackage[inner=3cm,top=2.5cm,outer=3.5cm]{geometry}

\renewcommand\familydefault{bch}
\usepackage[utf8]{inputenc}
\usepackage{microtype}
\usepackage[small]{caption}
\usepackage[small]{titlesec}
\raggedbottom

% Graphics -------------------------------------------------------------------
\usepackage[pdftex]{graphicx}
\graphicspath{{_include/}}
\DeclareGraphicsExtensions{.png,.pdf}

% Code formatting ------------------------------------------------------------
\usepackage{fancyvrb}
\usepackage{courier}
\usepackage{listings}
\usepackage{color}
\usepackage{alltt}


\definecolor{comment}{rgb}{0.60, 0.60, 0.53}
\definecolor{background}{rgb}{0.97, 0.97, 1.00}
\definecolor{string}{rgb}{0.863, 0.066, 0.266}
\definecolor{number}{rgb}{0.0, 0.6, 0.6}
\definecolor{variable}{rgb}{0.00, 0.52, 0.70}
\lstset{
  basicstyle=\ttfamily,
  keywordstyle=\bfseries, 
  identifierstyle=,  
  commentstyle=\color{comment} \emph,
  stringstyle=\color{string},
  showstringspaces=false,
  columns = fullflexible,
  backgroundcolor=\color{background},
  mathescape = true,
  escapeinside=&&,
  fancyvrb
}
\newcommand{\code}[1]{\lstinline!#1!}
\newcommand{\f}[1]{\lstinline!#1()!}



% Links ----------------------------------------------------------------------

\usepackage{hyperref}
\definecolor{slateblue}{rgb}{0.07,0.07,0.488}
\hypersetup{colorlinks=true,linkcolor=slateblue,anchorcolor=slateblue,citecolor=slateblue,filecolor=slateblue,urlcolor=slateblue,bookmarksnumbered=true,pdfview=FitB}
\usepackage{url}

% Tables ---------------------------------------------------------------------
\usepackage{longtable}
\usepackage{booktabs}

% Miscellaneous --------------------------------------------------------------
\usepackage{pdfsync}
\usepackage{appendix}

\usepackage[round,sort&compress,sectionbib]{natbib}
\bibliographystyle{plainnat}


\title{ggplot2}
\author{Hadley Wickham}

\begin{document}
\fi

\addcontentsline{toc}{part}{Appendices}
\chapter{Translating between different syntaxes}
\label{cha:translating}

\section{Introduction}

\ggplot does not exist in isolation, but is part of long history of graphical tools in R and elsewhere.  This chapter describes how to convert between \ggplot commands and other plotting systems:

\begin{itemize}
  \item Within \ggplot, between the \f{qplot} and \f{ggplot} syntaxes, \secref{sec:qplot-ggplot}
  
  \item From base graphics, \secref{sec:translate-base}.

  \item From lattice graphics, \secref{sec:translate-lattice}.
  
  \item From \sc{gpl}, \secref{sec:translate-gpl}.
\end{itemize} 

Each section gives a general outline on how to convert between the difference types, followed by a number of examples.

\section{Translating between qplot and ggplot}
\label{sec:qplot-ggplot}

Within \ggplot, there are two basic methods to create plots, with \f{qplot} and \f{ggplot}.  \f{qplot} is designed primarily for interactive use: it makes a number of assumptions that speed most cases, but when designing multi-layered plots with different data sources it can get in the way.  This section describes what those defaults are, and how they map to the fuller 
\f{ggplot} syntax.  

By default, \f{qplot} assumes that you want a scatterplot, i.e.\ you want to use \f{geom_point}.

\begin{alltt}
qplot(x, y, data = data)
ggplot(data, aes(x, y)) + geom_point()
\end{alltt}

\subsection{Aesthetics}

If you map additional aesthetics, these will be added to the defaults.  With \f{qplot} there is no way to use different aesthetic mappings (or data) in different layers.

\begin{alltt}
qplot(x, y, data = data, shape = shape, colour = colour)
ggplot(data, aes(x, y, shape = shape, colour = colour)) + geom_point()
\end{alltt}

Aesthetic parameters in \f{qplot} always try and map the aesthetic to a variable.  If the argument is not a variable but a value, effectively a new column is added to the original dataset with that value.  To set an aesthetic to a value and override the default appearance, you surround the value with \f{I} in \f{qplot}, or pass it as a parameter to the layer.  Section~\ref{sub:setting-mapping} expands on the differences between setting and mapping.

\begin{alltt}
qplot(x, y, data = data, colour = I("red"))
ggplot(data, aes(x, y)) + geom_point(colour = "red")
\end{alltt}

\subsection{Layers}

Changing the geom parameter changes the geom added to the plot:

\begin{alltt}
qplot(x, y, data = data, geom = "line")
ggplot(data, aes(x, y)) + geom_line()
\end{alltt}

If a vector of multiple geom names is supplied to the geom argument, each geom will be added in turn:

\begin{alltt}
qplot(x, y, data = data, geom = c("point", "smooth"))
ggplot(data, aes(x, y)) + geom_point() + geom_smooth()
\end{alltt}

Unlike the rest of \ggplot, stats and geoms are independent:

\begin{alltt}
qplot(x, y, data = data, stat = "bin")
ggplot(data, aes(x, y)) + geom_point(stat = "bin")  
\end{alltt}

Any layer parameters will be passed on to all layers.  Most layers will ignore parameters that they don't need.

\begin{alltt}
qplot(x, y, data = data, geom = c("point", "smooth"), method = "lm")
ggplot(data, aes(x, y)) + 
  geom_point(method = "lm") + geom_smooth(method = "lm")
\end{alltt}

\subsection{Scales and axes}

You can control basic properties of the x and y scales with the \code{xlim}, \code{ylim}, \code{xlab} and \code{ylab} arguments:

\begin{alltt}
qplot(x, y, data = data, xlim = c(1, 5), xlab = "my label")
ggplot(data, aes(x, y)) + geom_point() + 
  scale_x_continuous("my label", limits = c(1, 5))

qplot(x, y, data = data, xlim = c(1, 5), ylim = c(10, 20))
ggplot(data, aes(x, y)) + geom_point() + 
  scale_x_continuous(limits = c(1, 5))
  scale_y_continuous(limits = c(10, 20))
\end{alltt}

Like \f{plot}, \f{qplot} has a convenient way of log transforming the axes.  There are many other possible transformations that are not accessible from within \f{qplot}, see Section~\ref{sub:scale-position} for more details.

\begin{alltt}
qplot(x, y, data = data, log="xy")
ggplot(data, aes(x, y)) + geom_point() + scale_x_log10() + scale_y_log10()
\end{alltt}

\subsection{Plot options}

\f{qplot} recognise the same options as plot does, and converts them to their \ggplot equivalents.  Section~\ref{sec:theme_elements} lists all possible plot options and their effects.

\begin{alltt}
qplot(x, y, data = data, main="title", asp = 1)
ggplot(data, aes(x, y)) + geom_point() + 
  opts(title = "title", aspect.ratio = 1)
\end{alltt}

\section{Base graphics}
\label{sec:translate-base}

There are two types of graphics functions in base graphics, those that draw complete graphics and those that add to existing graphics.  

\subsection{High-level plotting commands}

\f{qplot} has been designed to mimic \f{plot}, and can do the job of all other high-level plotting commands.  There are only two graph types from base graphics that can not be replicated with \ggplot: \f{filled.countour} and \f{persp}

\begin{alltt}
plot(x, y);  dotchart(x, y); stripchart(x, y)
qplot(x, y)

plot(x, y, type = "l")
qplot(x, y, geom = "line")

plot(x, y, type = "s")
qplot(x, y, geom = "step")

plot(x, y, type = "b")
qplot(x, y, geom = c("point", "line"))

boxplot(x, y)
qplot(x, y, geom = "boxplot")

hist(x)
qplot(x, geom = "histogram")

cdplot(x, y)
qplot(x, fill = y, geom = "density", position = "fill")

coplot(y ~ x | a + b)
qplot(x, y, facets = a ~ b)
\end{alltt}

Many of the geoms are parameterised differently to base graphics.  For example, \f{hist} is parameterised in terms of the number of bins, while \f{geom_histogram} is parmeterised in terms of the width of each bin.  

\begin{alltt}
hist(x, bins = 100)
qplot(x, geom = "histogram", binwidth = 1)
\end{alltt}

\f{qplot} often requires data in a slightly different format to the base graphics functions.  For example, the bar geom works with untabulated data, not tabulated data like \f{barplot}; the tile and contour geoms expect data in a data frame, not a matrix like \f{image} and \f{contour}.

\begin{alltt}
barplot(table(x))
qplot(x, geom = "bar")

barplot(x)
qplot(names(x), x, geom = "bar", stat = "identity")

image(x)
qplot(X1, X2, data = melt(x), geom = "tile", fill = value)

contour(x)
qplot(X1, X2, data = melt(x), geom = "contour", fill = value)
\end{alltt}

% fourfoldplot -> geom_bar(width=1) + coord_polar()
% pie -> geom_bar(width = 1) + coord_polar()
% stars -> geom_polygon() + coord_polar()

Generally, the base graphics function work with individual vectors, not data frames like \ggplot.  \f{qplot} will try and construct a data frame if one is not specified, but it is not always possible.  If you get strange errors, you may need to create the data frame yourself.

\begin{alltt}
with(df, plot(x, y))
qplot(x, y, data = df)
\end{alltt}

By default, \f{qplot} map values to aesthetics with a scale.  To override this behaviour and set aesthetics, overriding the defaults, you need to use \f{I}.

\begin{alltt}
plot(x, y, col = "red", cex = 1)
qplot(x, y, colour = I("red"), size = I(1))
\end{alltt}

matplot and groups

\subsection{Low-level drawing}

The low level drawing functions which add on to existing plot are equivalent to adding a new layer in \ggplot, described in Table~\ref{tbl:base-equiv}.

\begin{table}
  \begin{center}
    \begin{tabular}{ll}
      \toprule
      Base function & \ggplot layer \\
      \midrule
      \f{curve}    & \f{geom_curve}      \\
      \f{hline}    & \f{geom_hline}      \\
      \f{lines}    & \f{geom_line}       \\
      \f{points}   & \f{geom_point}      \\
      \f{polygon}  & \f{geom_polygon}    \\
      \f{rect}     & \f{geom_rect}       \\
      \f{rug}      & \f{geom_rug}        \\
      \f{segments} & \f{geom_segment}    \\
      \f{text}     & \f{geom_text}       \\
      \f{vline}    & \f{geom_vline}      \\
      \code{abline(lm(y ~ x))}  & \code{geom_smooth(method = "lm")}  \\
      \code{lines(density(x))}  & \f{geom_density}                   \\
      \code{lines(loess(x, y))} & \f{geom_smooth}                    \\
      \bottomrule
    \end{tabular}
  \end{center}
  \caption{Equivalence between base graphics methods that add on to an existing plot, and layers in \ggplot.}
  \label{tbl:base-equiv}
\end{table}

\begin{alltt}
plot(x, y)
lines(x, y)

qplot(x, y) + geom_line()

# Or, building up piece-meal
qplot(x, y)
last_plot() + geom_line()
\end{alltt}

\subsection{Legends, axes and grid lines} 

In \ggplot, the appearance of legends and axes are controlled by the scales.  In base graphics, legends are never displayed automatically, and gain more control over axes you typically do \code{xaxs = F} in the main plot call, and then add the axes yourself with \f{axis} or \f{Axis}.

\begin{itemize}
  \item \code{limits} controls the range of the axis or legend.
  \item \code{breaks} controls which labels appear on the axis or legend.
  \item \code{labels} controls the text of each label.
  \item \code{name} controls the axis or legend title.
\end{itemize}

Because the legend is derived automatically from the plot, there is much less you can do to control it than in base graphics.  Many of the aspects of its appearance can be changed with plot themes, Section~\ref{sec:themes}, while what is displayed in the legend is controlled by the scales, Section~\ref{sec:guides}.

The appearance of grid lines are controlled by the \code{grid.major} and \code{grid.minor} options, and there position by the breaks of the x and y scales.

\subsection{Colour palettes}

Instead of global colour palettes, \ggplot has scales for individual plots.  Much of the time you can rely on the default colour scale (which has somewhat better perceptual properties), but if you want to reuse an existing colour palette, you can use \f{scale_colour_manual}.  You will need to make sure that the colour is a factor for this to work.

\begin{alltt}
palette(rainbow(5))
plot(1:5, 1:5, col = 1:5, pch = 19, cex = 4)

qplot(1:5, 1:5, col = factor(1:5), size = I(4))
last_plot() + scale_colour_manual(values = rainbow(5))
\end{alltt}

In \ggplot, you can also use palettes with continuous values, with intermediate values being linearly interpolated.

\begin{alltt}
qplot(0:100, 0:100, col = 0:100, size = I(4)) +
  scale_colour_gradientn(colours = rainbow(7))
last_plot() +
  scale_colour_gradientn(colours = terrain.colors(7))
\end{alltt}

\subsection{Graphical parameters}

The majority of \code{par} settings have some analogue within the theme system, or in the defaults of the geoms and scales.  The appearance plot border drawn by \f{box} can be controlled in a similar way by the \code{panel.background} and \code{plot.background} theme elements.  Instead of using \f{title}, the plot title is set with the \code{title} option.

\subsection{Specialised graphics} 

Unlike \f{plot}, \f{qplot} doesn't know how to plot anything other than data frames: there is no equivalent to \code{plot(lm)}.  This is a deliberate design decision, to better force the separation between the functions that extract useful data from objects and the functions that plot that data.

% 
% mtext -> grid
% matplot -> melt
% identify, locator -> no equivalent yet
% coplot -> facet_grid
% 
% Changing titles: plot title, legend title, axis title

\section{Lattice graphics}
\label{sec:translate-lattice}

The major difference between lattice and \ggplot is that lattice uses a formula based interface.  \ggplot does not because the formula does not generalise well to more complicated situations.

\begin{alltt}
xyplot(rating ~ year, data=movies)
qplot(year, rating, data=movies)

xyplot(rating ~ year | Comedy + Action, data = movies)
qplot(year, rating, data = movies, facets = ~ Comedy + Action)
# Or maybe
qplot(year, rating, data = movies, facets = Comedy ~ Action)
\end{alltt}

While lattice has many different functions to produce different types of graphics (which are all basically equivalent to setting the panel argument), \ggplot has \f{qplot}.

\begin{alltt}
stripplot(~ rating, data = movies, jitter.data = TRUE)
qplot(rating, 1, data = movies, geom = "jitter")

histogram(~ rating, data = movies)
qplot(rating, data = movies, geom = "histogram")

bwplot(Comedy ~ rating ,data = movies)
qplot(factor(Comedy), rating, data = movies, type = "boxplot")

xyplot(wt ~ mpg, mtcars, type = c("p","smooth"))
qplot(mpg, wt, data = mtcars, geom = c("point","smooth"))

xyplot(wt ~ mpg, mtcars, type = c("p","r"))
qplot(mpg, wt, data = mtcars, geom = c("point","smooth"), method = "lm")
\end{alltt}

The capabilities for scale manipulations are similar in both \ggplot and lattice, although the syntax is a little different.

\begin{alltt}
xyplot(wt ~ mpg | cyl, mtcars, scales = list(y = list(relation = "free")))
qplot(mpg, wt, data = mtcars) + facet_wrap(~ cyl, scales = "free")

xyplot(wt ~ mpg | cyl, mtcars, scales = list(log = 10))
qplot(mpg, wt, data = mtcars, log = "xy") 

xyplot(wt ~ mpg | cyl, mtcars, scales = list(log = 2))
qplot(mpg, wt, data = mtcars) + 
  scale_x_log2() + scale_y_log2()

xyplot(wt ~ mpg, mtcars, group = cyl, auto.key = TRUE)
# Map directly to an aesthetic like colour, size, or shape.
qplot(mpg, wt, data = mtcars, colour = cyl)

xyplot(wt ~ mpg, mtcars, xlim = c(20,30))
# Works like lattice, except you can't specify a different limit 
# for each panel/facet
qplot(mpg, wt, data = mtcars, xlim = c(20,30))
\end{alltt}

Both lattice and \ggplot have similar options for controlling labels on the plot.

\begin{alltt}
xyplot(wt ~ mpg, mtcars, 
  xlab = "Miles per gallon", ylab = "Weight", 
  main = "Weight-efficiency tradeoff")
qplot(mpg, wt, data = mtcars, 
  xlab = "Miles per gallon", ylab = "Weight", 
  main = "Weight-efficiency tradeoff")

xyplot(wt ~ mpg, mtcars, aspect = 1) 
qplot(mpg, wt, data = mtcars, asp = 1)
\end{alltt}

\f{par.settings} is equivalent to \code{+ opts()} and \f{trellis.options.set} and \f{trellis.par.get} to \f{theme_set} and \f{theme_get}.

% TODO: finish this section

More complicated lattice formulas are equivalent to rearranging the data before using ggplot2.

% 
% group -> categorical value on any aesthetic, or if no styling desired, group
% shingle -> no equivalent
% y1 + y2 -> melt
% 
% From Jim Holtman:
% 
% require(reshape)
% cases.melt <- melt(cases[, -ncol(cases)])
% 
% # plot cases/store/day on a box plot
% boxplot(value ~ X2, data=cases.melt, main="Cases Per Day Per Store")
% 
% -----------------------------------------------------------------------------------------
% 
% # partition by section and show the number of picks from each location
% x.m <- melt(table(Aisle, Quad, Section))
% 
% # add in the counts for each section
% col.l <- colorRampPalette(c('white', 'green', 'red'))(100)
% require(lattice)
% x.m$counts <- paste(x.m$Section, " (", ave(x.m$value, x.m$Section,
% FUN=sum), ")", sep="")
% print(levelplot(value ~ Aisle + Quad | counts, x.m, col.regions=col.l,
%    main="Visits Per Section Per Pick Slot (6 Days of Orders)"))

\section{GPL}
\label{sec:translate-gpl}

The Grammar of Graphics uses two specifications.  A concise format is used to caption figures, and a more detailed xml format stored on disk.  The following example of the concise format is adapted from \citet[][Figure 1.5, page 13]{wilkinson:2006}.

\begin{verbatim}
DATA: source("demographics")
DATA: longitude, latitude = map(source("World"))
TRANS: bd = max(birth - death, 0)
COORD: project.mercator()
ELEMENT: point(position(lon * lat), size(bf), color(color.red))
ELEMENT: polygon(position(longitude * latitude))
\end{verbatim}

This is relatively simple to adapt to the syntax of ggplot:

\begin{itemize}
  
  \item {\tt ggplot()} is used to specify the default data and default aesthetic mappings.  {\tt aes} is short for aesthetic mapping and specifies which variables in the data should be mapped to which aesthetic attributes.
  
  \item Data is provided as standard R data.frames existing in the global environment; it does not need to be explicitly loaded.  We also use a slightly different world data set, with columns lat and long.  This lets us use the same aesthetic mappings for both datasets. Layers can override the default data and aesthetic mappings provided by the plot. 
  
  \item We replace {\sf TRANS} with an explicit transformation by R code.

  \item {\sf ELEMENT}s are replaced with layers, which explicitly specify where the data comes from.  Each geom has a default statistic which is used to transform the data prior to plotting.  For the geoms in this example, the default statistic is the identity function.  Fixed aesthetics (the colour red in this example) are supplied as additional arguments to the layer, rather than as special constants.

  \item The {\sf SCALE} component has been omitted from this example (so that the defaults are used)In both the ggplot and GoG examples, scales are defined by default.  In ggplot you can override the defaults by adding a scale object, e.g. {\tt scale\_colour} or {\tt scale\_size}

  \item {\sf COORD} uses a slightly different format.  In general, most of the components specifications in ggplot are slightly different to those in GoG, in order to be more familiar to R users.

  \item Each component is added together with $+$ to create the final plot

\end{itemize}

This gives us:

\begin{verbatim}
demographics <- transform(demographics, bd = max(birth - death, 0))

ggplot(data = demographic, mapping = aes(x = lon, y = lat)) + 
layer(geom = "point", mapping = aes(size = bd), colour="red") +
layer(geom = "polygon", data = world) +
coord_map(projection = "mercator")
\end{verbatim}
\ifwhole
\else
  \nobibliography{/Users/hadley/documents/phd/references}
  \end{document}
\fi

\providecommand{\setflag}{\newif \ifwhole \wholefalse}
\setflag
\ifwhole\else

% Typography and geometry ----------------------------------------------------
\documentclass[letterpaper]{scrbook}
\usepackage[inner=3cm,top=2.5cm,outer=3.5cm]{geometry}

\renewcommand\familydefault{bch}
\usepackage[utf8]{inputenc}
\usepackage{microtype}
\usepackage[small]{caption}
\usepackage[small]{titlesec}
\raggedbottom

% Graphics -------------------------------------------------------------------
\usepackage[pdftex]{graphicx}
\graphicspath{{_include/}}
\DeclareGraphicsExtensions{.png,.pdf}

% Code formatting ------------------------------------------------------------
\usepackage{fancyvrb}
\usepackage{courier}
\usepackage{listings}
\usepackage{color}
\usepackage{alltt}


\definecolor{comment}{rgb}{0.60, 0.60, 0.53}
\definecolor{background}{rgb}{0.97, 0.97, 1.00}
\definecolor{string}{rgb}{0.863, 0.066, 0.266}
\definecolor{number}{rgb}{0.0, 0.6, 0.6}
\definecolor{variable}{rgb}{0.00, 0.52, 0.70}
\lstset{
  basicstyle=\ttfamily,
  keywordstyle=\bfseries, 
  identifierstyle=,  
  commentstyle=\color{comment} \emph,
  stringstyle=\color{string},
  showstringspaces=false,
  columns = fullflexible,
  backgroundcolor=\color{background},
  mathescape = true,
  escapeinside=&&,
  fancyvrb
}
\newcommand{\code}[1]{\lstinline!#1!}
\newcommand{\f}[1]{\lstinline!#1()!}



% Links ----------------------------------------------------------------------

\usepackage{hyperref}
\definecolor{slateblue}{rgb}{0.07,0.07,0.488}
\hypersetup{colorlinks=true,linkcolor=slateblue,anchorcolor=slateblue,citecolor=slateblue,filecolor=slateblue,urlcolor=slateblue,bookmarksnumbered=true,pdfview=FitB}
\usepackage{url}

% Tables ---------------------------------------------------------------------
\usepackage{longtable}
\usepackage{booktabs}

% Miscellaneous --------------------------------------------------------------
\usepackage{pdfsync}
\usepackage{appendix}

\usepackage[round,sort&compress,sectionbib]{natbib}
\bibliographystyle{plainnat}


\title{ggplot2}
\author{Hadley Wickham}

\begin{document}
\fi


\chapter{Aesthetic specifications}
\label{cha:aesthetic_specifications}

\section{Introduction}

Summary in one place of the information available in \code{?par}.

\section{Colour}
\label{sec:colour_spec}

Colours can be specified in several different ways. 

\begin{itemize}
  \item In terms of {\bf rgb components}, with a string of the form \code{"#RRGGBB"} where each of the pairs \code{RR}, \code{GG}, \code{BB} consist of two hexadecimal digits giving a value in the range \code{00} to \code{FF}. 

  \item A {\bf name}, e.g.\ \code{"red"}. The colours are displayed in Figure~\ref{fig:colours}, and can be listed in more detail with \f{colours}. The Stower's institute provides a nice printable pdf that lists all colours:  \url{http://research.stowers-institute.org/efg/R/Color/Chart/}.
    
  \item An {\bf NA}, for a completely transparent colour.  
\end{itemize}

The functions \f{rgb}, \f{hsv}, \f{hcl}, \f{gray} and \f{rainbow} provide additional ways of generating colours.

\begin{figure}[htbp]
  \centering
    \includegraphics[width=0.5\textwidth]{scale-identity}
  \caption{A plot of all named colours in Luv space}
  \label{fig:colours}
\end{figure}


\section{Line type}
\label{sec:line_type_spec}

Line types can either be specified by:

\begin{itemize}
  \item A {\bf integer} or {\bf name}: 0=blank, 1=solid, 2=dashed, 3=dotted, 4=dotdash, 5=longdash, 6=twodash).  Illustrated in Figure~\ref{fig:linetype}

  \item Directly as the lengths of on/off stretches of line. This is done with a string of an even number (up to eight) of characters, namely non-zero (hexadecimal) digits which give the lengths in consecutive positions in the string. For example, the string \code{"33"} specifies three units on followed by three off and \code{"3313"} specifies three units on followed by three off followed by one on and finally three off. The \code{units} here are (on most devices) proportional to \code{lwd}, and with \code{lwd = 1} are in pixels or points or 1/96 inch.
  The five standard dash-dot line types (\code{lty = 2:6}) correspond to \code{c("44", "13", "1343", "73", "2262")}.
\end{itemize} 

\begin{figure}[htbp]
  \centering
    \includegraphics[width=0.5 \textwidth]{spec-linetype}
  \caption{Built-in line types.}
  \label{fig:linetype}
\end{figure}

Note that \code{NA} is not a valid value for \code{lty}.

\section{Shape}
\label{sec:shape_spec}

Shapes take four types of values:

\begin{itemize}
  \item Integer in $[0, 25]$.  These are illustrated in Figure~\ref{fig:shape}.

  \item A single character.  Can be a unicode charater, which 

  \item \code{.} is handled specially. In most devices, it draws the smallest shape that is visible (i.e. about one pixel).
  
  \item \code{NA}.  Point omitted.  
\end{itemize}

\begin{figure}[htbp]
  \centering
    \includegraphics[width=0.5 \textwidth]{spec-shape}
  \caption{R plotting symbols.  Colour is black, and fill is blue.  Symbol 25 has been omitted from this plot, it is symbol 24 rotated 180 degrees.}
  \label{fig:shape}
\end{figure}

While all symbols have a foreground colour, symbols 19--25 also take a background colour (fill).  


\section{Justification}
\label{sec:justification_spec}

The justification of the text relative to its (x, y) location. If there are two values, the first value specifies horizontal justification and the second value specifies vertical justification. Possible string values are: \code{"left"}, \code{"right"}, \code{"centre"}, \code{"center"}, \code{"bottom"}, and \code{"top"}. For numeric values, 0 means left alignment and 1 means right alignment.

\begin{figure}[htbp]
  \centering
    \includegraphics[width=0.5 \textwidth]{spec-justification}
  \caption{Horizontal and vertical justification settings.}
  \label{fig:justification}
\end{figure}

\ifwhole
\else
  \nobibliography{/Users/hadley/documents/phd/references}
  \end{document}
\fi
\chapter{Manipulating plot rendering with grid}\label{cha:grid}

\section{Introduction}

Sometimes you may need to go beyond the theming system and directly
modify the underlying grid graphics output. To do this, you will need a
good understanding of grid, as described in `R Graphics' (Murrell 2005).
If you can't get the book, at least read Chapter 5, `The grid graphics
model', which is available online for free at
\url{http://www.stat.auckland.ac.nz/~paul/RGraphics/chapter5.pdf}. This
appendix outlines the more important viewports and grobs used by
\texttt{ggplot} and should be helpful if you need to interact with the
grobs produced by \texttt{ggplot}. \index{Grid} \index{Package!grid}

\section{Plot viewports}\label{sec:plot-viewports}

Viewports define the basic regions of the plot. The structure will vary
slightly from plot to plot, depending on the type of faceting used, but
the basics will remain the same. \index{Viewports}
\index{Grid!viewports}

The \texttt{panels} viewport contains the meat of the plot: strip
labels, axes and faceted panels. The viewports are named according to
both their job and their position on the plot. A prefix (listed below)
describes the contents of the viewport, and is followed by integer x and
y position (counting from bottom left) separated by \texttt{\_}.
Figure\textasciitilde{}\ref{fig:panelvp} illustrates this naming scheme
for a 2 by 2 plot.

\begin{itemize}
\itemsep1pt\parskip0pt\parsep0pt
\item
  \texttt{strip\_h}: horizontal strip labels
\item
  \texttt{strip\_v}: vertical strip labels
\item
  \texttt{axis\_h}: horizontal axes
\item
  \texttt{axis\_v}: vertical axes
\item
  \texttt{panel}: faceting panels
\end{itemize}

\begin{figure}[htbp]
  \centering
    \includegraphics[width=0.5 \linewidth]{diagrams/grid-panelvp}
  \caption{Naming scheming of the panel viewports.}
  \label{fig:panelvp}
\end{figure}

The \texttt{panels} viewport is contained inside the \texttt{background}
viewport which also contains the following viewports:

\begin{itemize}
\itemsep1pt\parskip0pt\parsep0pt
\item
  \texttt{title}, \texttt{xlabel} and \texttt{ylabel}: for the plot
  title, and x and y axis labels
\item
  \texttt{legend\_box}: for all of the legends for the plot
\end{itemize}

\noindent Figure\textasciitilde{}\ref{fig:viewports} labels a plot with
a representative sample of these viewports. To get a list of all
viewports on the current plot, run \texttt{current.vpTree(all=TRUE)} or
\texttt{grid.ls(grobs = FALSE, viewports = TRUE)}.

\begin{figure}[htbp]
  \centering
    \includegraphics[width=\linewidth]{diagrams/grid-viewports}
  \caption{Diagram showing the structure and names of viewports.}
  \label{fig:viewports}
\end{figure}

\section{Plot grobs \{sec:plot-grobs\}}

Grob names have three components: the name of the grob, the class of the
grob and a unique numeric suffix. The three components are joined
together with \texttt{.} to give a name like \texttt{title.text.435} or
\texttt{ticks.segments.15}. These three components ensure that all grob
names are unique, and allow you to select multiple grobs with the same
name at the same time. Figure\textasciitilde{}\ref{fig:grobs} labels
some of these grobs. The grobs are arranged hierarchically, but it's
hard to capture this in a diagram. You can see a list of all the grobs
in the current plot with \texttt{grid.ls()}. \index{Grid!grobs}

\begin{figure}[htbp]
  \centering
    \includegraphics[width=\linewidth]{diagrams/grid-grobs}
  \caption{A selection of the most important grobs.}
  \label{fig:grobs}
\end{figure}

\section{Saving your work}\label{sec:grid-save}

Using \texttt{grid.gedit()}, and similar functions, works fine if you
are editing the plot on screen, but if you want to save it to disk you
need to take some extra steps, or you will end up with multiple pages of
output, each showing one change. The key is not to modify the plot on
screen, but to modify the plot grob, and then draw it once you have made
all the changes. \index{Saving!grid graphics}

\begin{Shaded}
\begin{Highlighting}[]
\NormalTok{p <-}\StringTok{ }\KeywordTok{qplot}\NormalTok{(wt, mpg, }\DataTypeTok{data=}\NormalTok{mtcars, }\DataTypeTok{colour=}\NormalTok{cyl)}
\CommentTok{# Get the plot grob}
\NormalTok{grob <-}\StringTok{ }\KeywordTok{ggplotGrob}\NormalTok{(p)}
\CommentTok{# Modify in place}
\CommentTok{# DEPRECATED!!! seems as though a proper fix hasn't been found}
\CommentTok{# Study this for a solution -- https://groups.google.com/forum/#!topic/ggplot2/iygHrbUNy7w}
\NormalTok{grob <-}\StringTok{ }\KeywordTok{geditGrob}\NormalTok{(grob, }\KeywordTok{gPath}\NormalTok{(}\StringTok{"strip"}\NormalTok{,}\StringTok{"label"}\NormalTok{), }\DataTypeTok{gp=}\KeywordTok{gpar}\NormalTok{(}\DataTypeTok{fontface=}\StringTok{"bold"}\NormalTok{))}

\CommentTok{# Draw it}
\KeywordTok{grid.newpage}\NormalTok{()}
\KeywordTok{grid.draw}\NormalTok{(grob)}
\end{Highlighting}
\end{Shaded}

An alternative is make all of the changes on screen, and then use
\texttt{dev.copy2pdf()} to copy the final version to disk.

Murrell, Paul. 2005. \emph{R Graphics}. Chapman \& Hall/CRC.


\backmatter
\index{Bibliography}
%\chapter{References}

\cleardoublepage
\markboth{Index}{Index}
\addcontentsline{toc}{chapter}{Index}
\printindex

\end{document}
