\documentclass[graybox,envcountchap,sectrefs]{svmono}

\usepackage[scaled=0.92,varqu]{inconsolata}

\usepackage{float}
\usepackage{index}
% index functions separately
\newindex{code}{adx}{and}{R code index}
\newcommand{\indexf}[1]{\index[code]{#1@\texttt{#1()}}}
\newcommand{\indexc}[1]{\index[code]{#1@\texttt{#1}}}

% Taken from pandoc x.md -o test.tex --standalone
\usepackage{color}
\usepackage{fancyvrb}
\newcommand{\VerbBar}{|}
\newcommand{\VERB}{\Verb[commandchars=\\\{\}]}
\DefineVerbatimEnvironment{Highlighting}{Verbatim}{commandchars=\\\{\}}
\newenvironment{Shaded}{}{}
\newcommand{\KeywordTok} [1]{\textcolor[rgb]{0.00,0.44,0.13}{{#1}}}
\newcommand{\DataTypeTok}[1]{\textcolor[rgb]{0.56,0.13,0.00}{{#1}}}
\newcommand{\DecValTok}  [1]{\textcolor[rgb]{0.25,0.63,0.44}{{#1}}}
\newcommand{\BaseNTok}   [1]{\textcolor[rgb]{0.25,0.63,0.44}{{#1}}}
\newcommand{\FloatTok}   [1]{\textcolor[rgb]{0.25,0.63,0.44}{{#1}}}
\newcommand{\CharTok}    [1]{\textcolor[rgb]{0.25,0.44,0.63}{{#1}}}
\newcommand{\StringTok}  [1]{\textcolor[rgb]{0.25,0.44,0.63}{{#1}}}
\newcommand{\CommentTok} [1]{\textcolor[rgb]{0.38,0.63,0.69}{{#1}}}
\newcommand{\OtherTok}   [1]{\textcolor[rgb]{0.00,0.44,0.13}{{#1}}}
\newcommand{\AlertTok}   [1]{\textcolor[rgb]{1.00,0.00,0.00}{{#1}}}
\newcommand{\FunctionTok}[1]{\textcolor[rgb]{0.02,0.16,0.49}{{#1}}}
\newcommand{\ErrorTok}   [1]{\textcolor[rgb]{1.00,0.00,0.00}{{#1}}}
\newcommand{\NormalTok}  [1]{{#1}}

\newcommand{\OperatorTok}  [1]{{#1}}
\newcommand{\ControlFlowTok}  [1]{{#1}}
%
\usepackage{longtable}
\usepackage{booktabs}
\usepackage{graphicx}
\DeclareGraphicsExtensions{.pdf,.png}
\providecommand{\tightlist}{\setlength{\itemsep}{0pt}\setlength{\parskip}{0pt}}

\usepackage[hyphens]{url}
\usepackage{hyperref}

% Place links in parens
\renewcommand{\href}[2]{#2 (\url{#1})}
% Use auto ref for internal links
\let\oldhyperlink=\hyperlink
\renewcommand{\hyperlink}[2]{\autoref{#1}}
\def\chapterautorefname{Chapter}
\def\sectionautorefname{Section}
\def\subsectionautorefname{Section}
\def\subsubsectionautorefname{Section}

\setlength{\emergencystretch}{3em}  % prevent overfull lines
\vbadness=10000 % suppress underfull \vbox
\hbadness=10000 % suppress underfull \vbox
\hfuzz=10pt

\makeindex
\title{ggplot2}
\subtitle{Elegant Graphics for Data Analysis}
\author{Hadley Wickham}

\begin{document}

\frontmatter
\maketitle

\begin{dedication}
To my parents, Alison \& Brian Wickham. Without them, and their unconditional
love and support, none of this would have been possible.
\end{dedication}

\preface

Welcome to the second edition of ``ggplot2: elegant graphics for data
analysis''. I'm so excited to have an updated book that shows off all
the latest and greatest ggplot2 features, as well as the great things
that have been happening in R and in the ggplot2 community the last five
years. The ggplot2 community is vibrant: the ggplot2 mailing list has
over 7,000 members and there is a very active Stack Overflow community,
with nearly 10,000 questions tagged with ggplot2. While most of my
development effort is no longer going into ggplot2 (more on that below),
there's never been a better time to learn it and use it.

I am tremendously grateful for the success of ggplot2. It's one of the
most commonly downloaded R packages (over a million downloads in the
last year!) and has influenced the design of graphics packages for other
languages. Personally, ggplot2 has bought me many exciting opportunities
to travel the world and meet interesting people. I love hearing how
people are using R and ggplot2 to understand the data that they love.

A big thanks for this edition goes to Carson Sievert, who helped me
modernise the code, including converting the sources to Rmarkdown. He
also updated many of the examples and helped me proofread the book.

\section*{Major changes}

I've spent a lot of effort ensuring that this edition is a true upgrade
over the first edition. As well as updating the code everywhere to make
sure it's fully compatible with the latest version of ggplot2, I have:

\begin{itemize}
\item
  Shown much more code in the book, so it's easier to use as a
  reference. Overall the book has a more ``knitr''-ish sensibility:
  there are fewer floating figures and tables, and more inline code.
  This makes the layout a little less pretty but keeps related items
  closer together. You can find the complete source online at
  \url{https://github.com/hadley/ggplot2-book}.
\item
  Switched from \texttt{qplot()} to \texttt{ggplot()} in the
  introduction, \protect\hyperlink{cha:getting-started}{intro}. Feedback
  indicated that \texttt{qplot()} was a crutch: it makes simple plots a
  little easier, but it doesn't help with mastering the grammar.
\item
  Added practice exercises throughout the book so you can practice new
  techniques immediately after learning about them.
\item
  Added pointers to the rich ecosystem of packages that have built up
  around ggplot2. You'll now see a number of other packages highlighted
  in the book, and get pointers to other packages I think are
  particularly useful.
\item
  Overhauled the toolbox chapter,
  \protect\hyperlink{cha:toolbox}{toolbox}, to cover all the new geoms.
  I've added a completely new section on text labels,
  \protect\hyperlink{sec:labelling}{labels}, since it's important and
  not covered in detail elsewhere. The mapping section,
  \protect\hyperlink{sec:maps}{maps}, has been considerably expanded to
  talk more about the different types of map data, and where you might
  find it.
\item
  Completely rewritten the scales chapter,
  \protect\hyperlink{cha:scales}{scales}, to focus on the most important
  tasks. It also discusses the new features that give finer control over
  legend appearance, \protect\hyperlink{sec:legends}{legends}, and shows
  off some of the new scales added to ggplot2,
  \protect\hyperlink{sec:scale-details}{scales}.
\item
  Split the data analysis chapter into three pieces: data tidying (with
  tidyr), \protect\hyperlink{cha:data}{tidyr}; data manipulation (with
  dplyr), \protect\hyperlink{cha:dplyr}{dplyr}; and model visualisation
  (with broom), \protect\hyperlink{cha:modelling}{models}. I discuss the
  latest iteration of my data manipulation tools, and introduce the
  fantastic broom package by David Robinson.
\end{itemize}

The book is accompanied by a new version of ggplot2: version 2.0.0. This
includes a number of minor tweaks and improvements, and considerable
improvements to the documentation. Coming back to ggplot2 development
after a considerable pause has helped me to see many problems that
previously escaped notice. ggplot2 2.0.0 (finally!) contains an official
extension mechanism so that others can contribute new ggplot2 components
in their own packages. This is documented in a new vignette,
\texttt{vignette("extending-ggplot2")}.

\section*{The future}

ggplot2 is now stable, and is unlikely to change much in the future.
There will be bug fixes and there may be new geoms, but there will be no
large changes to how ggplot2 works. The next iteration of ggplot2 is
ggvis. ggvis is significantly more ambitious because it aims to provide
a grammar of \emph{interactive} graphics. ggvis is still young, and
lacks many of the features of ggplot2 (most notably it currently lacks
facetting and has no way to make static graphics), but over the coming
years the goal is for ggvis to be better than ggplot2.

The syntax of ggvis is a little different to ggplot2. You won't be able
to trivially convert your ggplot2 plots to ggvis, but we think the cost
is worth it: the new syntax is considerably more consistent, and will be
easier for newcomers to learn. If you've mastered ggplot2, you'll find
your skills transfer very well to ggvis and after struggling with the
syntax for a while, it will start to feel quite natural. The important
skills you learn when mastering ggplot2 are not the programmatic details
of describing a plot in code, but the much harder challenge of thinking
about how to turn data into effective visualisations.

\section*{Acknowledgements}

Many people have contributed to this book with high-level structural
insights, spelling and grammar corrections and bug reports. I'd
particularly like to thank William E. J. Doane, Alexander Forrence,
Devin Pastoor, David Robinson, and Guangchuang Yu, for their detailed
technical review of the book.

Many others have contributed over the (now quite long!) lifetime of
ggplot2. I would like to thank: Leland Wilkinson, for discussions and
comments that cemented my understanding of the grammar; Gabor
Grothendieck, for early helpful comments; Heike Hofmann and Di Cook, for
being great advisors and supporting the development of ggplot2 during my
PhD; Charlotte Wickham; the students of stat480 and stat503 at ISU, for
trying it out when it was very young; Debby Swayne, for masses of
helpful feedback and advice; Bob Muenchen, Reinhold Kliegl, Philipp
Pagel, Richard Stahlhut, Baptiste Auguie, Jean-Olivier Irisson, Thierry
Onkelinx and the many others who have read draft versions of the book
and given me feedback; and last, but not least, the members of R-help
and the ggplot2 mailing list, for providing the many interesting and
challenging graphics problems that have helped motivate this book.

\vspace{\baselineskip}\begin{flushright}\noindent
{\it Hadley Wickham}\\
September 2015\\
\end{flushright}


\tableofcontents

\mainmatter

\part{Getting started}

\chapter{Introduction}\label{cha:introduction}

\section{Welcome to ggplot2}\label{welcome-to-ggplot2}

ggplot2 is an R package for producing statistical, or data, graphics,
but it is unlike most other graphics packages because it has a deep
underlying grammar. This grammar, based on the Grammar of Graphics
(Wilkinson 2005), is made up of a set of independent components that can
be composed in many different ways. This makes ggplot2 very powerful
because you are not limited to a set of pre-specified graphics, but you
can create new graphics that are precisely tailored for your problem.
This may sound overwhelming, but because there is a simple set of core
principles and very few special cases, ggplot2 is also easy to learn
(although it may take a little time to forget your preconceptions from
other graphics tools).

Practically, ggplot2 provides beautiful, hassle-free plots that take
care of fiddly details like drawing legends. The plots can be built up
iteratively and edited later. A carefully chosen set of defaults means
that most of the time you can produce a publication-quality graphic in
seconds, but if you do have special formatting requirements, a
comprehensive theming system makes it easy to do what you want. Instead
of spending time making your graph look pretty, you can focus on
creating a graph that best reveals the messages in your data.

ggplot2 is designed to work iteratively. You can start with a layer
showing the raw data then add layers of annotations and statistical
summaries. It allows you to produce graphics using the same structured
thinking that you use to design an analysis, reducing the distance
between a plot in your head and one on the page. It is especially
helpful for students who have not yet developed the structured approach
to analysis used by experts.

Learning the grammar not only will help you create graphics that you
know about now, but will also help you to think about new graphics that
would be even better. Without the grammar, there is no underlying
theory, so most graphics packages are just a big collection of special
cases. For example, in base R, if you design a new graphic, it's
composed of raw plot elements like points and lines, and it's hard to
design new components that combine with existing plots. In ggplot2, the
expressions used to create a new graphic are composed of higher-level
elements like representations of the raw data and statistical
transformations, and can easily be combined with new datasets and other
plots.

This book provides a hands-on introduction to ggplot2 with lots of
example code and graphics. It also explains the grammar on which ggplot2
is based. Like other formal systems, ggplot2 is useful even when you
don't understand the underlying model. However, the more you learn about
it, the more effectively you'll be able to use ggplot2. This book
assumes some basic familiarity with R, to the level described in the
first chapter of Dalgaard's \emph{Introductory Statistics with R}.

This book will introduce you to ggplot2 as a novice, unfamiliar with the
grammar; teach you the basics so that you can re-create plots you are
already familiar with; show you how to use the grammar to create new
types of graphics; and eventually turn you into an expert who can build
new components to extend the grammar.

\section{What is the grammar of
graphics?}\label{what-is-the-grammar-of-graphics}

Wilkinson (2005) created the grammar of graphics to describe the deep
features that underlie all statistical graphics. The grammar of graphics
is an answer to a question: what is a statistical graphic? The layered
grammar of graphics (Wickham 2009) builds on Wilkinson's grammar,
focussing on the primacy of layers and adapting it for embedding within
R. In brief, the grammar tells us that a statistical graphic is a
mapping from data to aesthetic attributes (colour, shape, size) of
geometric objects (points, lines, bars). The plot may also contain
statistical transformations of the data and is drawn on a specific
coordinate system. Facetting can be used to generate the same plot for
different subsets of the dataset. It is the combination of these
independent components that make up a graphic.

As the book progresses, the formal grammar will be explained in
increasing detail. The first description of the components follows
below. It introduces some of the terminology that will be used
throughout the book and outlines the basic responsibilities of each
component. Don't worry if it doesn't all make sense right away: you will
have many more opportunities to learn about the pieces and how they fit
together.

All plots are composed of:

\begin{itemize}
\item
  \textbf{Data} that you want to visualise and a set of aesthetic
  \textbf{mapping}s describing how variables in the data are mapped to
  aesthetic attributes that you can perceive.
\item
  \textbf{Layers} made up of geometric elements and statistical
  transformation. Geometric objects, \textbf{geom}s for short, represent
  what you actually see on the plot: points, lines, polygons, etc.
  Statistical transformations, \textbf{stat}s for short, summarise data
  in many useful ways. For example, binning and counting observations to
  create a histogram, or summarising a 2d relationship with a linear
  model.
\item
  The \textbf{scale}s map values in the data space to values in an
  aesthetic space, whether it be colour, or size, or shape. Scales draw
  a legend or axes, which provide an inverse mapping to make it possible
  to read the original data values from the plot.
\item
  A coordinate system, \textbf{coord} for short, describes how data
  coordinates are mapped to the plane of the graphic. It also provides
  axes and gridlines to make it possible to read the graph. We normally
  use a Cartesian coordinate system, but a number of others are
  available, including polar coordinates and map projections.
\item
  A \textbf{facet}ing specification describes how to break up the data
  into subsets and how to display those subsets as small multiples. This
  is also known as conditioning or latticing/trellising.
\item
  A \textbf{theme} which controls the finer points of display, like the
  font size and background colour. While the defaults in ggplot2 have
  been chosen with care, you may need to consult other references to
  create an attractive plot. A good starting place is Tufte's early
  works (Tufte 1990; Tufte 1997; Tufte 2001).
\end{itemize}

It is also important to talk about what the grammar doesn't do:

\begin{itemize}
\item
  It doesn't suggest what graphics you should use to answer the
  questions you are interested in. While this book endeavours to promote
  a sensible process for producing plots of data, the focus of the book
  is on how to produce the plots you want, not knowing what plots to
  produce. For more advice on this topic, you may want to consult
  Robbins (2013), Cleveland (1993), Chambers et al. (1983), and J. W.
  Tukey (1977).
\item
  It does not describe interactivity: the grammar of graphics describes
  only static graphics and there is essentially no benefit to displaying
  them on a computer screen as opposed to a piece of paper. ggplot2 can
  only create static graphics, so for dynamic and interactive graphics
  you will have to look elsewhere (perhaps at ggvis, described below).
  Cook and Swayne (2007) provides an excellent introduction to the
  interactive graphics package GGobi. GGobi can be connected to R with
  the rggobi package (Wickham et al. 2008).
\end{itemize}

\section{How does ggplot2 fit in with other R
graphics?}\label{how-does-ggplot2-fit-in-with-other-r-graphics}

There are a number of other graphics systems available in R: base
graphics, grid graphics and trellis/lattice graphics. How does ggplot2
differ from them?

\begin{itemize}
\item
  Base graphics were written by Ross Ihaka based on experience
  implementing the S graphics driver and partly looking at Chambers et
  al. (1983). Base graphics has a pen on paper model: you can only draw
  on top of the plot, you cannot modify or delete existing content.
  There is no (user accessible) representation of the graphics, apart
  from their appearance on the screen. Base graphics includes both tools
  for drawing primitives and entire plots. Base graphics functions are
  generally fast, but have limited scope. If you've created a single
  scatterplot, or histogram, or a set of boxplots in the past, you've
  probably used base graphics. \index{Base graphics}
\item
  The development of ``grid'' graphics, a much richer system of
  graphical primitives, started in 2000. Grid is developed by Paul
  Murrell, growing out of his PhD work (Murrell 1998). Grid grobs
  (graphical objects) can be represented independently of the plot and
  modified later. A system of viewports (each containing its own
  coordinate system) makes it easier to lay out complex graphics. Grid
  provides drawing primitives, but no tools for producing statistical
  graphics. \index{grid}
\item
  The lattice package, developed by Deepayan Sarkar, uses grid graphics
  to implement the trellis graphics system of Cleveland (1993) and is a
  considerable improvement over base graphics. You can easily produce
  conditioned plots and some plotting details (e.g., legends) are taken
  care of automatically. However, lattice graphics lacks a formal model,
  which can make it hard to extend. Lattice graphics are explained in
  depth in Sarkar (2008). \index{lattice}
\item
  ggplot2, started in 2005, is an attempt to take the good things about
  base and lattice graphics and improve on them with a strong underlying
  model which supports the production of any kind of statistical
  graphic, based on the principles outlined above. The solid underlying
  model of ggplot2 makes it easy to describe a wide range of graphics
  with a compact syntax, and independent components make extension easy.
  Like lattice, ggplot2 uses grid to draw the graphics, which means you
  can exercise much low-level control over the appearance of the plot.
\item
  Work on ggvis, the successor to ggplot2, started in 2014. It takes the
  foundational ideas of ggplot2 but extends them to the web and
  interactive graphics. The syntax is similar, but it's been re-designed
  from scratch to take advantage of what I've learned in the 10 years
  since creating ggplot2. The most exciting thing about ggvis is that
  it's interactive and dynamic, so plots automatically re-draw
  themselves when the underlying data or plot specification changes.
  However, ggvis is work in progress and currently can create only a
  fraction of the plots in ggplot2 can. Stay tuned for updates!
  \index{ggvis}
\item
  htmlwidgets, \url{http://www.htmlwidgets.org} provides a common
  framework for accessing web visualisation tools from R. Packages built
  on top of htmlwidgets include leaflet
  (\url{https://rstudio.github.io/leaflet/}, maps), dygraph
  (\url{http://rstudio.github.io/dygraphs/}, time series) and networkD3
  (http://christophergandrud.github.io/networkD3/, networks).
  htmlwidgets is to ggvis what the many specialised graphic packages are
  to ggplot2: it provides graphics honed for specific purposes.
  \index{htmlwidgets}
\end{itemize}

Many other R packages, such as vcd (Meyer, Zeileis, and Hornik 2006),
plotrix (Lemon et al. 2008) and gplots (Warnes 2007), implement
specialist graphics, but no others provide a framework for producing
statistical graphics. A comprehensive list of all graphical tools
available in other packages can be found in the graphics task view at
\url{http://cran.r-project.org/web/views/Graphics.html}.

\section{About this book}\label{about-this-book}

The first chapter, \protect\hyperlink{cha:getting-started}{getting
started with ggplot2}, describes how to quickly get started using
ggplot2 to make useful graphics. This chapter introduces several
important ggplot2 concepts: geoms, aesthetic mappings and facetting.
\protect\hyperlink{cha:toolbox}{Toolbox} dives into more details, giving
you a toolbox designed to solve a wide range of problems.

\protect\hyperlink{cha:mastery}{Mastery} describes the layered grammar
of graphics which underlies ggplot2. The theory is illustrated in
\protect\hyperlink{cha:layers}{layers} which demonstrates how to add
additional layers to your plot, exercising full control over the geoms
and stats used within them.

Understanding how scales work is crucial for fine-tuning the perceptual
properties of your plot. Customising scales gives fine control over the
exact appearance of the plot and helps to support the story that you are
telling. \protect\hyperlink{cha:scales}{Scales} will show you what
scales are available, how to adjust their parameters, and how to control
the appearance of axes and legends.

Coordinate systems and facetting control the position of elements of the
plot. These are described in \protect\hyperlink{cha:position}{position}.
Facetting is a very powerful graphical tool as it allows you to rapidly
compare different subsets of your data. Different coordinate systems are
less commonly needed, but are very important for certain types of data.

To polish your plots for publication, you will need to learn about the
tools described in \protect\hyperlink{cha:polishing}{polishing}. There
you will learn about how to control the theming system of ggplot2 and
how to save plots to disk.

The book concludes with four chapters that show how to use ggplot2 as
part of a data analysis pipeline. ggplot2 works best when your data is
tidy, so \protect\hyperlink{cha:data}{Tidying} discusses what that means
and how to make your messy data tidy. \protect\hyperlink{cha:dplyr}{Data
transformation} teaches you how to use the dplyr package to perform the
most common data manipulation operations.
\protect\hyperlink{cha:modelling}{Modelling} shows how to integrate
visualisation and modelling in two useful ways. Duplicated code is a big
inhibitor of flexibility and reduces your ability to respond to changes
in requirements. \protect\hyperlink{cha:programming}{Programming with
ggplot2} covers useful techniques for reducing duplication in your code.

\section{Installation}\label{sec:installation}

\index{Installation}

To use ggplot2, you must first install it. Make sure you have a recent
version of R (at least version 3.2.0) from \url{http://r-project.org}
and then run the following code to download and install ggplot2:

\begin{Shaded}
\begin{Highlighting}[]
\KeywordTok{install.packages}\NormalTok{(}\StringTok{"ggplot2"}\NormalTok{)}
\end{Highlighting}
\end{Shaded}

\section{Other resources}\label{sec:other-resources}

This book teaches you the elements of ggplot2's grammar and how they fit
together, but it does not document every function in complete detail.
You will need additional documentation as your use of ggplot2 becomes
more complex and varied.

The best resource for specific details of ggplot2 functions and their
arguments will always be the built-in documentation. This is accessible
online, \url{http://docs.ggplot2.org/}, and from within R using the
usual help syntax. The advantage of the online documentation is that you
can see all the example plots and navigate between topics more easily.

If you use ggplot2 regularly, it's a good idea to sign up for the
ggplot2 mailing list, \url{http://groups.google.com/group/ggplot2}. The
list has relatively low traffic and is very friendly to new users.
Another useful resource is stackoverflow,
\url{http://stackoverflow.com}. There is an active ggplot2 community on
stackoverflow, and many common questions have already been asked and
answered. In either place, you're much more likely to get help if you
create a minimal reproducible example. The
\href{https://github.com/jennybc/reprex}{reprex} package by Jenny Bryan
provides a convenient way to do this, and also include advice on
creating a good example. The more information you provide, the easier it
is for the community to help you.

The number of functions in ggplot2 can be overwhelming, but RStudio
provides some great cheatsheets to jog your memory at
\url{http://www.rstudio.com/resources/cheatsheets/}.

Finally, the complete source code for the book is available online at
\url{https://github.com/hadley/ggplot2-book}. This contains the complete
text for the book, as well as all the code and data needed to recreate
all the plots.

\section{Colophon}\label{colophon}

This book was written in \href{http://rmarkdown.rstudio.com/}{R
Markdown} inside \href{http://www.rstudio.com/ide/}{RStudio}.
\href{http://yihui.name/knitr/}{knitr} and
\href{http://johnmacfarlane.net/pandoc/}{pandoc} converted the raw
Rmarkdown to html and pdf. The complete source is available from
\href{https://github.com/hadley/ggplot2-book}{github}. This version of
the book was built with:

\begin{Shaded}
\begin{Highlighting}[]
\NormalTok{devtools::}\KeywordTok{session_info}\NormalTok{(}\KeywordTok{c}\NormalTok{(}\StringTok{"ggplot2"}\NormalTok{, }\StringTok{"dplyr"}\NormalTok{, }\StringTok{"broom"}\NormalTok{))}
\CommentTok{#> Session info --------------------------------------------------------}
\CommentTok{#>  setting  value                       }
\CommentTok{#>  version  R version 3.2.3 (2015-12-10)}
\CommentTok{#>  system   x86_64, darwin13.4.0        }
\CommentTok{#>  ui       X11                         }
\CommentTok{#>  language (EN)                        }
\CommentTok{#>  collate  en_US.UTF-8                 }
\CommentTok{#>  tz       America/Chicago             }
\CommentTok{#>  date     2016-02-27}
\CommentTok{#> Packages ------------------------------------------------------------}
\CommentTok{#>  package      * version  date       source        }
\CommentTok{#>  assertthat     0.1      2013-12-06 CRAN (R 3.2.0)}
\CommentTok{#>  BH             1.58.0-1 2015-05-21 CRAN (R 3.2.0)}
\CommentTok{#>  broom          0.4.0    2015-11-30 CRAN (R 3.2.2)}
\CommentTok{#>  colorspace     1.2-6    2015-03-11 CRAN (R 3.2.0)}
\CommentTok{#>  DBI            0.3.1    2014-09-24 CRAN (R 3.2.0)}
\CommentTok{#>  dichromat      2.0-0    2013-01-24 CRAN (R 3.2.0)}
\CommentTok{#>  digest         0.6.9    2016-01-08 CRAN (R 3.2.3)}
\CommentTok{#>  dplyr        * 0.4.3    2015-09-01 CRAN (R 3.2.0)}
\CommentTok{#>  ggplot2      * 2.1.0    2016-02-26 local         }
\CommentTok{#>  gtable         0.1.2    2012-12-05 CRAN (R 3.2.0)}
\CommentTok{#>  labeling       0.3      2014-08-23 CRAN (R 3.2.0)}
\CommentTok{#>  lattice        0.20-33  2015-07-14 CRAN (R 3.2.3)}
\CommentTok{#>  lazyeval       0.1.10   2015-01-02 CRAN (R 3.2.0)}
\CommentTok{#>  magrittr       1.5      2014-11-22 CRAN (R 3.2.0)}
\CommentTok{#>  MASS           7.3-45   2015-11-10 CRAN (R 3.2.3)}
\CommentTok{#>  mnormt         1.5-3    2015-05-25 CRAN (R 3.2.0)}
\CommentTok{#>  munsell        0.4.2    2013-07-11 CRAN (R 3.2.0)}
\CommentTok{#>  nlme           3.1-122  2015-08-19 CRAN (R 3.2.3)}
\CommentTok{#>  plyr           1.8.3    2015-06-12 CRAN (R 3.2.0)}
\CommentTok{#>  psych          1.5.8    2015-08-30 CRAN (R 3.2.0)}
\CommentTok{#>  R6             2.1.2    2016-01-26 CRAN (R 3.2.3)}
\CommentTok{#>  RColorBrewer   1.1-2    2014-12-07 CRAN (R 3.2.0)}
\CommentTok{#>  Rcpp           0.12.3   2016-01-10 CRAN (R 3.2.3)}
\CommentTok{#>  reshape2       1.4.1    2014-12-06 CRAN (R 3.2.0)}
\CommentTok{#>  scales         0.3.0    2015-08-25 CRAN (R 3.2.0)}
\CommentTok{#>  stringi        1.0-1    2015-10-22 CRAN (R 3.2.0)}
\CommentTok{#>  stringr        1.0.0    2015-04-30 CRAN (R 3.2.0)}
\CommentTok{#>  tidyr        * 0.4.1    2016-02-05 CRAN (R 3.2.3)}
\end{Highlighting}
\end{Shaded}

\section*{References}\label{references}
\addcontentsline{toc}{section}{References}

\hypertarget{refs}{}
\hypertarget{ref-chambers:1983}{}
Chambers, John, William Cleveland, Beat Kleiner, and Paul Tukey. 1983.
\emph{Graphical Methods for Data Analysis}. Wadsworth.

\hypertarget{ref-cleveland:1993}{}
Cleveland, William. 1993. \emph{Visualizing Data}. Hobart Press.

\hypertarget{ref-cook:2007}{}
Cook, Dianne, and Deborah F. Swayne. 2007. \emph{Interactive and Dynamic
Graphics for Data Analysis: With Examples Using R and GGobi}. Springer.

\hypertarget{ref-plotrix}{}
Lemon, Jim, Ben Bolker, Sander Oom, Eduardo Klein, Barry Rowlingson,
Hadley Wickham, Anupam Tyagi, et al. 2008. \emph{Plotrix: Various
Plotting Functions}.

\hypertarget{ref-meyer:2006}{}
Meyer, David, Achim Zeileis, and Kurt Hornik. 2006. ``The Strucplot
Framework: Visualizing Multi-Way Contingency Tables with Vcd.''
\emph{Journal of Statistical Software} 17 (3): 1--48.
\url{http://www.jstatsoft.org/v17/i03/}.

\hypertarget{ref-murrell:1998}{}
Murrell, Paul. 1998. ``Investigations in Graphical Statistics.''
PhD thesis, The University of Auckland.

\hypertarget{ref-robbins:2004}{}
Robbins, Naomi. 2013. \emph{Creating More Effective Graphs}. Chart
House.

\hypertarget{ref-sarkar:2008}{}
Sarkar, Deepayan. 2008. \emph{Lattice: Multivariate Data Visualization
with R}. Springer.

\hypertarget{ref-tufte:1990}{}
Tufte, Edward R. 1990. \emph{Envisioning Information}. Graphics Press.

\hypertarget{ref-tufte:1997}{}
---------. 1997. \emph{Visual Explanations}. Graphics Press.

\hypertarget{ref-tufte:2001}{}
---------. 2001. \emph{The Visual Display of Quantitative Information}.
Second. Graphics Press.

\hypertarget{ref-tukey:1977}{}
Tukey, John W. 1977. \emph{Exploratory Data Analysis}. Addison--Wesley.

\hypertarget{ref-gplots}{}
Warnes, Gregory. 2007. \emph{Gplots: Various R Programming Tools for
Plotting Data}.

\hypertarget{ref-wickham:2007d}{}
Wickham, Hadley. 2009. ``A Layered Grammar of Graphics.'' \emph{Journal
of Computational and Graphical Statistics}.

\hypertarget{ref-wickham:2008b}{}
Wickham, Hadley, Michael Lawrence, Duncan Temple Lang, and Deborah F
Swayne. 2008. ``An Introduction to Rggobi.'' \emph{R-News} 8 (2): 3--7.
\url{http://CRAN.R-project.org/doc/Rnews/Rnews_2008-2.pdf}.

\hypertarget{ref-wilkinson:2006}{}
Wilkinson, Leland. 2005. \emph{The Grammar of Graphics}. 2nd ed.
Statistics and Computing. Springer.

\chapter{Getting started with ggplot2}\label{cha:getting-started}

\section{Introduction}\label{introduction}

The goal of this chapter is to teach you how to produce useful graphics
with ggplot2 as quickly as possible. You'll learn the basics of
\texttt{ggplot()} along with some useful ``recipes'' to make the most
important plots. \texttt{ggplot()} allows you to make complex plots with
just a few lines of code because it's based on a rich underlying theory,
the grammar of graphics. Here we'll skip the theory and focus on the
practice, and in later chapters you'll learn how to use the full
expressive power of the grammar.

In this chapter you'll learn:

\begin{itemize}
\item
  About the \texttt{mpg} dataset included with ggplot2,
  \protect\hyperlink{sec:fuel-economy-data}{mpg}.
\item
  The three key components of every plot: data, aesthetics and geoms,
  \protect\hyperlink{sec:basic-use}{key components}.
\item
  How to add additional variables to a plot with aesthetics,
  \protect\hyperlink{aesthetics}{aesthetics}.
\item
  How to display additional categorical variables in a plot using small
  multiples created by facetting,
  \protect\hyperlink{sec:qplot-facetting}{facetting}.
\item
  A variety of different geoms that you can use to create different
  types of plots, \protect\hyperlink{sec:plot-geoms}{geoms}.
\item
  How to modify the axes, \protect\hyperlink{sec:axes}{axes}.
\item
  Things you can do with a plot object other than display it, like save
  it to disk, \protect\hyperlink{sec:output}{output}.
\item
  \texttt{qplot()}, a handy shortcut for when you just want to quickly
  bang out a simple plot without thinking about the grammar at all,
  \protect\hyperlink{qplot}{qplot}.
\end{itemize}

\hypertarget{sec:fuel-economy-data}{\section{Fuel economy
data}\label{sec:fuel-economy-data}}

In this chapter, we'll mostly use one data set that's bundled with
ggplot2: \texttt{mpg}. It includes information about the fuel economy of
popular car models in 1999 and 2008, collected by the US Environmental
Protection Agency, \url{http://fueleconomy.gov}. You can access the data
by loading ggplot2: \index{Data!mpg@\texttt{mpg}}

\begin{Shaded}
\begin{Highlighting}[]
\KeywordTok{library}\NormalTok{(ggplot2)}
\NormalTok{mpg}
\CommentTok{#> Source: local data frame [234 x 11]}
\CommentTok{#> }
\CommentTok{#>    manufacturer model displ  year   cyl      trans   drv   cty   hwy}
\CommentTok{#>           (chr) (chr) (dbl) (int) (int)      (chr) (chr) (int) (int)}
\CommentTok{#> 1          audi    a4   1.8  1999     4   auto(l5)     f    18    29}
\CommentTok{#> 2          audi    a4   1.8  1999     4 manual(m5)     f    21    29}
\CommentTok{#> 3          audi    a4   2.0  2008     4 manual(m6)     f    20    31}
\CommentTok{#> 4          audi    a4   2.0  2008     4   auto(av)     f    21    30}
\CommentTok{#> 5          audi    a4   2.8  1999     6   auto(l5)     f    16    26}
\CommentTok{#> 6          audi    a4   2.8  1999     6 manual(m5)     f    18    26}
\CommentTok{#> ..          ...   ...   ...   ...   ...        ...   ...   ...   ...}
\CommentTok{#> Variables not shown: fl (chr), class (chr)}
\end{Highlighting}
\end{Shaded}

The variables are mostly self-explanatory:

\begin{itemize}
\item
  \texttt{cty} and \texttt{hwy} record miles per gallon (mpg) for city
  and highway driving.
\item
  \texttt{displ} is the engine displacement in litres.
\item
  \texttt{drv} is the drivetrain: front wheel (f), rear wheel (r) or
  four wheel (4).
\item
  \texttt{model} is the model of car. There are 38 models, selected
  because they had a new edition every year between 1999 and 2008.
\item
  \texttt{class} (not shown), is a categorical variable describing the
  ``type'' of car: two seater, SUV, compact, etc.
\end{itemize}

This dataset suggests many interesting questions. How are engine size
and fuel economy related? Do certain manufacturers care more about fuel
economy than others? Has fuel economy improved in the last ten years? We
will try to answer some of these questions, and in the process learn how
to create some basic plots with ggplot2.

\subsection{Exercises}\label{exercises}

\begin{enumerate}
\def\labelenumi{\arabic{enumi}.}
\item
  List five functions that you could use to get more information about
  the \texttt{mpg} dataset.
\item
  How can you find out what other datasets are included with ggplot2?
\item
  Apart from the US, most countries use fuel consumption (fuel consumed
  over fixed distance) rather than fuel economy (distance travelled with
  fixed amount of fuel). How could you convert \texttt{cty} and
  \texttt{hwy} into the European standard of l/100km?
\item
  Which manufacturer has the most the models in this dataset? Which
  model has the most variations? Does your answer change if you remove
  the redundant specification of drive train (e.g. ``pathfinder 4wd'',
  ``a4 quattro'') from the model name?
\end{enumerate}

\hypertarget{sec:basic-use}{\section{Key
components}\label{sec:basic-use}}

Every ggplot2 plot has three key components:

\begin{enumerate}
\def\labelenumi{\arabic{enumi}.}
\item
  \textbf{data},
\item
  A set of \textbf{aesthetic mappings} between variables in the data and
  visual properties, and
\item
  At least one layer which describes how to render each observation.
  Layers are usually created with a \textbf{geom} function.
\end{enumerate}

Here's a simple example: \index{Scatterplot} \indexf{ggplot}

\begin{Shaded}
\begin{Highlighting}[]
\KeywordTok{ggplot}\NormalTok{(mpg, }\KeywordTok{aes}\NormalTok{(}\DataTypeTok{x =} \NormalTok{displ, }\DataTypeTok{y =} \NormalTok{hwy)) +}\StringTok{ }
\StringTok{  }\KeywordTok{geom_point}\NormalTok{()}
\end{Highlighting}
\end{Shaded}

\begin{figure}[H]
  \centering
  \includegraphics[width=0.65\linewidth]{_figures/ggplot/qscatter-1}
\end{figure}

This produces a scatterplot defined by:

\begin{enumerate}
\def\labelenumi{\arabic{enumi}.}
\tightlist
\item
  Data: \texttt{mpg}.
\item
  Aesthetic mapping: engine size mapped to x position, fuel economy to y
  position.
\item
  Layer: points.
\end{enumerate}

Pay attention to the structure of this function call: data and aesthetic
mappings are supplied in \texttt{ggplot()}, then layers are added on
with \texttt{+}. This is an important pattern, and as you learn more
about ggplot2 you'll construct increasingly sophisticated plots by
adding on more types of components.

Almost every plot maps a variable to \texttt{x} and \texttt{y}, so
naming these aesthetics is tedious, so the first two unnamed arguments
to \texttt{aes()} will be mapped to \texttt{x} and \texttt{y}. This
means that the following code is identical to the example above:

\begin{Shaded}
\begin{Highlighting}[]
\KeywordTok{ggplot}\NormalTok{(mpg, }\KeywordTok{aes}\NormalTok{(displ, hwy)) +}
\StringTok{  }\KeywordTok{geom_point}\NormalTok{()}
\end{Highlighting}
\end{Shaded}

I'll stick to that style throughout the book, so don't forget that the
first two arguments to \texttt{aes()} are \texttt{x} and \texttt{y}.
Note that I've put each command on a new line. I recommend doing this in
your own code, so it's easy to scan a plot specification and see exactly
what's there. In this chapter, I'll sometimes use just one line per
plot, because it makes it easier to see the differences between plot
variations.

The plot shows a strong correlation: as the engine size gets bigger, the
fuel economy gets worse. There are also some interesting outliers: some
cars with large engines get higher fuel economy than average. What sort
of cars do you think they are?

\subsection{Exercises}\label{exercises-1}

\begin{enumerate}
\def\labelenumi{\arabic{enumi}.}
\item
  How would you describe the relationship between \texttt{cty} and
  \texttt{hwy}? Do you have any concerns about drawing conclusions from
  that plot?
\item
  What does
  \texttt{ggplot(mpg,\ aes(model,\ manufacturer))\ +\ geom\_point()}
  show? Is it useful? How could you modify the data to make it more
  informative?
\item
  Describe the data, aesthetic mappings and layers used for each of the
  following plots. You'll need to guess a little because you haven't
  seen all the datasets and functions yet, but use your common sense!
  See if you can predict what the plot will look like before running the
  code.

  \begin{enumerate}
  \def\labelenumii{\arabic{enumii}.}
  \tightlist
  \item
    \texttt{ggplot(mpg,\ aes(cty,\ hwy))\ +\ geom\_point()}
  \item
    \texttt{ggplot(diamonds,\ aes(carat,\ price))\ +\ geom\_point()}
  \item
    \texttt{ggplot(economics,\ aes(date,\ unemploy))\ +\ geom\_line()}
  \item
    \texttt{ggplot(mpg,\ aes(cty))\ +\ geom\_histogram()}
  \end{enumerate}
\end{enumerate}

\hypertarget{aesthetics}{\section{Colour, size, shape and other
aesthetic attributes}\label{aesthetics}}

To add additional variables to a plot, we can use other aesthetics like
colour, shape, and size (NB: while I use British spelling throughout
this book, ggplot2 also accepts American spellings). These work in the
same way as the \texttt{x} and \texttt{y} aesthetics, and are added into
the call to \texttt{aes()}: \index{Aesthetics} \indexf{aes}

\begin{itemize}
\tightlist
\item
  \texttt{aes(displ,\ hwy,\ colour\ =\ class)}
\item
  \texttt{aes(displ,\ hwy,\ shape\ =\ drv)}
\item
  \texttt{aes(displ,\ hwy,\ size\ =\ cyl)}
\end{itemize}

ggplot2 takes care of the details of converting data (e.g., `f', `r',
`4') into aesthetics (e.g., `red', `yellow', `green') with a
\textbf{scale}. There is one scale for each aesthetic mapping in a plot.
The scale is also responsible for creating a guide, an axis or legend,
that allows you to read the plot, converting aesthetic values back into
data values. For now, we'll stick with the default scales provided by
ggplot2. You'll learn how to override them in
\protect\hyperlink{cha:scales}{the scales chapter}.

To learn more about those outlying variables in the previous
scatterplot, we could map the class variable to colour:

\begin{Shaded}
\begin{Highlighting}[]
\KeywordTok{ggplot}\NormalTok{(mpg, }\KeywordTok{aes}\NormalTok{(displ, cty, }\DataTypeTok{colour =} \NormalTok{class)) +}\StringTok{ }
\StringTok{  }\KeywordTok{geom_point}\NormalTok{()}
\end{Highlighting}
\end{Shaded}

\begin{figure}[H]
  \centering
  \includegraphics[width=0.65\linewidth]{_figures/ggplot/qplot-aesthetics-1}
\end{figure}

This gives each point a unique colour corresponding to its class. The
legend allows us to read data values from the colour, showing us that
the group of cars with unusually high fuel economy for their engine size
are two seaters: cars with big engines, but lightweight bodies.

If you want to set an aesthetic to a fixed value, without scaling it, do
so in the individual layer outside of \texttt{aes()}. Compare the
following two plots: \index{Aesthetics!setting}

\begin{Shaded}
\begin{Highlighting}[]
\KeywordTok{ggplot}\NormalTok{(mpg, }\KeywordTok{aes}\NormalTok{(displ, hwy)) +}\StringTok{ }\KeywordTok{geom_point}\NormalTok{(}\KeywordTok{aes}\NormalTok{(}\DataTypeTok{colour =} \StringTok{"blue"}\NormalTok{))}
\KeywordTok{ggplot}\NormalTok{(mpg, }\KeywordTok{aes}\NormalTok{(displ, hwy)) +}\StringTok{ }\KeywordTok{geom_point}\NormalTok{(}\DataTypeTok{colour =} \StringTok{"blue"}\NormalTok{)}
\end{Highlighting}
\end{Shaded}

\begin{figure}[H]
  \includegraphics[width=0.5\linewidth]{_figures/ggplot/unnamed-chunk-4-1}%
  \includegraphics[width=0.5\linewidth]{_figures/ggplot/unnamed-chunk-4-2}
\end{figure}

In the first plot, the value ``blue'' is scaled to a pinkish colour, and
a legend is added. In the second plot, the points are given the R colour
blue. This is an important technique and you'll learn more about it in
\protect\hyperlink{sub:setting-mapping}{setting vs.~mapping}. See
\texttt{vignette("ggplot2-specs")} for the values needed for colour and
other aesthetics.

Different types of aesthetic attributes work better with different types
of variables. For example, colour and shape work well with categorical
variables, while size works well for continuous variables. The amount of
data also makes a difference: if there is a lot of data it can be hard
to distinguish different groups. An alternative solution is to use
facetting, as described next.

When using aesthetics in a plot, less is usually more. It's difficult to
see the simultaneous relationships among colour and shape and size, so
exercise restraint when using aesthetics. Instead of trying to make one
very complex plot that shows everything at once, see if you can create a
series of simple plots that tell a story, leading the reader from
ignorance to knowledge.

\subsection{Exercises}\label{exercises-2}

\begin{enumerate}
\def\labelenumi{\arabic{enumi}.}
\item
  Experiment with the colour, shape and size aesthetics. What happens
  when you map them to continuous values? What about categorical values?
  What happens when you use more than one aesthetic in a plot?
\item
  What happens if you map a continuous variable to shape? Why? What
  happens if you map \texttt{trans} to shape? Why?
\item
  How is drive train related to fuel economy? How is drive train related
  to engine size and class?
\end{enumerate}

\hypertarget{sec:qplot-facetting}{\section{Facetting}\label{sec:qplot-facetting}}

Another technique for displaying additional categorical variables on a
plot is facetting. Facetting creates tables of graphics by splitting the
data into subsets and displaying the same graph for each subset. You'll
learn more about facetting in
\protect\hyperlink{sec:facetting}{Facetting}, but it's such a useful
technique that you need to know it right away. \index{Facetting}

There are two types of facetting: grid and wrapped. Wrapped is the most
useful, so we'll discuss it here, and you can learn about grid facetting
later. To facet a plot you simply add a facetting specification with
\texttt{facet\_wrap()}, which takes the name of a variable preceded by
\texttt{\textasciitilde{}}. \indexf{facet\_wrap}

\begin{Shaded}
\begin{Highlighting}[]
\KeywordTok{ggplot}\NormalTok{(mpg, }\KeywordTok{aes}\NormalTok{(displ, hwy)) +}\StringTok{ }
\StringTok{  }\KeywordTok{geom_point}\NormalTok{() +}\StringTok{ }
\StringTok{  }\KeywordTok{facet_wrap}\NormalTok{(~class)}
\end{Highlighting}
\end{Shaded}

\begin{figure}[H]
  \includegraphics[width=1\linewidth]{_figures/ggplot/facet-1}
\end{figure}

You might wonder when to use facetting and when to use aesthetics.
You'll learn more about the relative advantages and disadvantages of
each in \protect\hyperlink{sub:group-vs-facet}{grouping vs.~facetting}.

\subsection{Exercises}\label{exercises-3}

\begin{enumerate}
\def\labelenumi{\arabic{enumi}.}
\item
  What happens if you try to facet by a continuous variable like
  \texttt{hwy}? What about \texttt{cyl}? What's the key difference?
\item
  Use facetting to explore the 3-way relationship between fuel economy,
  engine size, and number of cylinders. How does facetting by number of
  cylinders change your assessement of the relationship between engine
  size and fuel economy?
\item
  Read the documentation for \texttt{facet\_wrap()}. What arguments can
  you use to control how many rows and columns appear in the output?
\item
  What does the \texttt{scales} argument to \texttt{facet\_wrap()} do?
  When might you use it?
\end{enumerate}

\hypertarget{sec:plot-geoms}{\section{Plot geoms}\label{sec:plot-geoms}}

You might guess that by substituting \texttt{geom\_point()} for a
different geom function, you'd get a different type of plot. That's a
great guess! In the following sections, you'll learn about some of the
other important geoms provided in ggplot2. This isn't an exhaustive
list, but should cover the most commonly used plot types. You'll learn
more in \protect\hyperlink{cha:toolbox}{the toolbox}.

\begin{itemize}
\item
  \texttt{geom\_smooth()} fits a smoother to the data and displays the
  smooth and its standard error.
\item
  \texttt{geom\_boxplot()} produces a box-and-whisker plot to summarise
  the distribution of a set of points.
\item
  \texttt{geom\_histogram()} and \texttt{geom\_freqpoly()} show the
  distribution of continuous variables.
\item
  \texttt{geom\_bar()} shows the distribution of categorical variables.
\item
  \texttt{geom\_path()} and \texttt{geom\_line()} draw lines between the
  data points. A line plot is constrained to produce lines that travel
  from left to right, while paths can go in any direction. Lines are
  typically used to explore how things change over time.
\end{itemize}

\subsection{Adding a smoother to a plot}\label{sub:smooth}

If you have a scatterplot with a lot of noise, it can be hard to see the
dominant pattern. In this case it's useful to add a smoothed line to the
plot with \texttt{geom\_smooth()}: \index{Smoothing}
\indexf{geom\_smooth}

\begin{Shaded}
\begin{Highlighting}[]
\KeywordTok{ggplot}\NormalTok{(mpg, }\KeywordTok{aes}\NormalTok{(displ, hwy)) +}\StringTok{ }
\StringTok{  }\KeywordTok{geom_point}\NormalTok{() +}\StringTok{ }
\StringTok{  }\KeywordTok{geom_smooth}\NormalTok{()}
\end{Highlighting}
\end{Shaded}

\begin{figure}[H]
  \centering
  \includegraphics[width=0.65\linewidth]{_figures/ggplot/qplot-smooth-1}
\end{figure}

This overlays the scatterplot with a smooth curve, including an
assessment of uncertainty in the form of point-wise confidence intervals
shown in grey. If you're not interested in the confidence interval, turn
it off with \texttt{geom\_smooth(se\ =\ FALSE)}.

An important argument to \texttt{geom\_smooth()} is the \texttt{method},
which allows you to choose which type of model is used to fit the smooth
curve:

\begin{itemize}
\item
  \texttt{method\ =\ "loess"}, the default for small n, uses a smooth
  local regression (as described in \texttt{?loess}). The wiggliness of
  the line is controlled by the \texttt{span} parameter, which ranges
  from 0 (exceedingly wiggly) to 1 (not so wiggly).

\begin{Shaded}
\begin{Highlighting}[]
\KeywordTok{ggplot}\NormalTok{(mpg, }\KeywordTok{aes}\NormalTok{(displ, hwy)) +}\StringTok{ }
\StringTok{  }\KeywordTok{geom_point}\NormalTok{() +}\StringTok{ }
\StringTok{  }\KeywordTok{geom_smooth}\NormalTok{(}\DataTypeTok{span =} \FloatTok{0.2}\NormalTok{)}

\KeywordTok{ggplot}\NormalTok{(mpg, }\KeywordTok{aes}\NormalTok{(displ, hwy)) +}\StringTok{ }
\StringTok{  }\KeywordTok{geom_point}\NormalTok{() +}\StringTok{ }
\StringTok{  }\KeywordTok{geom_smooth}\NormalTok{(}\DataTypeTok{span =} \DecValTok{1}\NormalTok{)}
\end{Highlighting}
\end{Shaded}

  \begin{figure}[H]
    \includegraphics[width=0.5\linewidth]{_figures/ggplot/smooth-loess-1}%
    \includegraphics[width=0.5\linewidth]{_figures/ggplot/smooth-loess-2}
  \end{figure}

  Loess does not work well for large datasets (it's \(O(n^2)\) in
  memory), so an alternative smoothing algorithm is used when \(n\) is
  greater than 1,000.
\item
  \texttt{method\ =\ "gam"} fits a generalised additive model provided
  by the \textbf{mgcv} package. You need to first load mgcv, then use a
  formula like \texttt{formula\ =\ y\ \textasciitilde{}\ s(x)} or
  \texttt{y\ \textasciitilde{}\ s(x,\ bs\ =\ "cs")} (for large data).
  This is what ggplot2 uses when there are more than 1,000 points.
  \index{mgcv}

\begin{Shaded}
\begin{Highlighting}[]
\KeywordTok{library}\NormalTok{(mgcv)}
\KeywordTok{ggplot}\NormalTok{(mpg, }\KeywordTok{aes}\NormalTok{(displ, hwy)) +}\StringTok{ }
\StringTok{  }\KeywordTok{geom_point}\NormalTok{() +}\StringTok{ }
\StringTok{  }\KeywordTok{geom_smooth}\NormalTok{(}\DataTypeTok{method =} \StringTok{"gam"}\NormalTok{, }\DataTypeTok{formula =} \NormalTok{y ~}\StringTok{ }\KeywordTok{s}\NormalTok{(x))}
\end{Highlighting}
\end{Shaded}

  \begin{figure}[H]
    \includegraphics[width=0.5\linewidth]{_figures/ggplot/smooth-gam-1}
  \end{figure}
\item
  \texttt{method\ =\ "lm"} fits a linear model, giving the line of best
  fit.

\begin{Shaded}
\begin{Highlighting}[]
\KeywordTok{ggplot}\NormalTok{(mpg, }\KeywordTok{aes}\NormalTok{(displ, hwy)) +}\StringTok{ }
\StringTok{  }\KeywordTok{geom_point}\NormalTok{() +}\StringTok{ }
\StringTok{  }\KeywordTok{geom_smooth}\NormalTok{(}\DataTypeTok{method =} \StringTok{"lm"}\NormalTok{)}
\end{Highlighting}
\end{Shaded}

  \begin{figure}[H]
    \includegraphics[width=0.5\linewidth]{_figures/ggplot/smooth-lm-1}
  \end{figure}
\item
  \texttt{method\ =\ "rlm"} works like \texttt{lm()}, but uses a robust
  fitting algorithm so that outliers don't affect the fit as much. It's
  part of the \textbf{MASS} package, so remember to load that first.
  \index{MASS}
\end{itemize}

\subsection{Boxplots and jittered points}\label{sub:boxplot}

When a set of data includes a categorical variable and one or more
continuous variables, you will probably be interested to know how the
values of the continuous variables vary with the levels of the
categorical variable. Say we're interested in seeing how fuel economy
varies within car class. We might start with a scatterplot like this:

\begin{Shaded}
\begin{Highlighting}[]
\KeywordTok{ggplot}\NormalTok{(mpg, }\KeywordTok{aes}\NormalTok{(drv, hwy)) +}\StringTok{ }
\StringTok{  }\KeywordTok{geom_point}\NormalTok{()}
\end{Highlighting}
\end{Shaded}

\begin{figure}[H]
  \centering
  \includegraphics[width=0.5\linewidth]{_figures/ggplot/unnamed-chunk-5-1}
\end{figure}

Because there are few unique values of both class and hwy, there is a
lot of overplotting. Many points are plotted in the same location, and
it's difficult to see the distribution. There are three useful
techniques that help alleviate the problem:

\begin{itemize}
\item
  Jittering, \texttt{geom\_jitter()}, adds a little random noise to the
  data which can help avoid overplotting. \index{Jittering}
  \indexf{geom\_jitter}
\item
  Boxplots, \texttt{geom\_boxplot()}, summarise the shape of the
  distribution with a handful of summary statistics. \index{Boxplot}
  \indexf{geom\_boxplot}
\item
  Violin plots, \texttt{geom\_violin()}, show a compact representation
  of the ``density'' of the distribution, highlighting the areas where
  more points are found. \index{Violin plot} \indexf{geom\_violin}
\end{itemize}

These are illustrated below:

\begin{Shaded}
\begin{Highlighting}[]
\KeywordTok{ggplot}\NormalTok{(mpg, }\KeywordTok{aes}\NormalTok{(drv, hwy)) +}\StringTok{ }\KeywordTok{geom_jitter}\NormalTok{()}
\KeywordTok{ggplot}\NormalTok{(mpg, }\KeywordTok{aes}\NormalTok{(drv, hwy)) +}\StringTok{ }\KeywordTok{geom_boxplot}\NormalTok{()}
\KeywordTok{ggplot}\NormalTok{(mpg, }\KeywordTok{aes}\NormalTok{(drv, hwy)) +}\StringTok{ }\KeywordTok{geom_violin}\NormalTok{()}
\end{Highlighting}
\end{Shaded}

\begin{figure}[H]
  \includegraphics[width=0.333\linewidth]{_figures/ggplot/jitter-boxplot-1}%
  \includegraphics[width=0.333\linewidth]{_figures/ggplot/jitter-boxplot-2}%
  \includegraphics[width=0.333\linewidth]{_figures/ggplot/jitter-boxplot-3}
\end{figure}

Each method has its strengths and weaknesses. Boxplots summarise the
bulk of the distribution with only five numbers, while jittered plots
show every point but only work with relatively small datasets. Violin
plots give the richest display, but rely on the calculation of a density
estimate, which can be hard to interpret.

For jittered points, \texttt{geom\_jitter()} offers the same control
over aesthetics as \texttt{geom\_point()}: \texttt{size},
\texttt{colour}, and \texttt{shape}. For \texttt{geom\_boxplot()} and
\texttt{geom\_violin()}, you can control the outline \texttt{colour} or
the internal \texttt{fill} colour.

\subsection{Histograms and frequency polygons}\label{sub:distribution}

Histograms and frequency polygons show the distribution of a single
numeric variable. They provide more information about the distribution
of a single group than boxplots do, at the expense of needing more
space. \index{Histogram} \indexf{geom\_histogram}

\begin{Shaded}
\begin{Highlighting}[]
\KeywordTok{ggplot}\NormalTok{(mpg, }\KeywordTok{aes}\NormalTok{(hwy)) +}\StringTok{ }\KeywordTok{geom_histogram}\NormalTok{()}
\CommentTok{#> `stat_bin()` using `bins = 30`. Pick better value with `binwidth`.}
\KeywordTok{ggplot}\NormalTok{(mpg, }\KeywordTok{aes}\NormalTok{(hwy)) +}\StringTok{ }\KeywordTok{geom_freqpoly}\NormalTok{()}
\CommentTok{#> `stat_bin()` using `bins = 30`. Pick better value with `binwidth`.}
\end{Highlighting}
\end{Shaded}

\begin{figure}[H]
  \includegraphics[width=0.5\linewidth]{_figures/ggplot/dist-1}%
  \includegraphics[width=0.5\linewidth]{_figures/ggplot/dist-2}
\end{figure}

Both histograms and frequency polygons work in the same way: they bin
the data, then count the number of observations in each bin. The only
difference is the display: histograms use bars and frequency polygons
use lines.

You can control the width of the bins with the \texttt{binwidth}
argument (if you don't want evenly spaced bins you can use the
\texttt{breaks} argument). It is \textbf{very important} to experiment
with the bin width. The default just splits your data into 30 bins,
which is unlikely to be the best choice. You should always try many bin
widths, and you may find you need multiple bin widths to tell the full
story of your data.

\begin{Shaded}
\begin{Highlighting}[]
\KeywordTok{ggplot}\NormalTok{(mpg, }\KeywordTok{aes}\NormalTok{(hwy)) +}\StringTok{ }
\StringTok{  }\KeywordTok{geom_freqpoly}\NormalTok{(}\DataTypeTok{binwidth =} \FloatTok{2.5}\NormalTok{)}
\KeywordTok{ggplot}\NormalTok{(mpg, }\KeywordTok{aes}\NormalTok{(hwy)) +}\StringTok{ }
\StringTok{  }\KeywordTok{geom_freqpoly}\NormalTok{(}\DataTypeTok{binwidth =} \DecValTok{1}\NormalTok{)}
\end{Highlighting}
\end{Shaded}

\begin{figure}[H]
  \includegraphics[width=0.5\linewidth]{_figures/ggplot/unnamed-chunk-6-1}%
  \includegraphics[width=0.5\linewidth]{_figures/ggplot/unnamed-chunk-6-2}
\end{figure}

An alternative to the frequency polygon is the density plot,
\texttt{geom\_density()}. I'm not a fan of density plots because they
are harder to interpret since the underlying computations are more
complex. They also make assumptions that are not true for all data,
namely that the underlying distribution is continuous, unbounded, and
smooth.

To compare the distributions of different subgroups, you can map a
categorical variable to either fill (for \texttt{geom\_histogram()}) or
colour (for \texttt{geom\_freqpoly()}). It's easier to compare
distributions using the frequency polygon because the underlying
perceptual task is easier. You can also use facetting: this makes
comparisons a little harder, but it's easier to see the distribution of
each group.

\begin{Shaded}
\begin{Highlighting}[]
\KeywordTok{ggplot}\NormalTok{(mpg, }\KeywordTok{aes}\NormalTok{(displ, }\DataTypeTok{colour =} \NormalTok{drv)) +}\StringTok{ }
\StringTok{  }\KeywordTok{geom_freqpoly}\NormalTok{(}\DataTypeTok{binwidth =} \FloatTok{0.5}\NormalTok{)}
\KeywordTok{ggplot}\NormalTok{(mpg, }\KeywordTok{aes}\NormalTok{(displ, }\DataTypeTok{fill =} \NormalTok{drv)) +}\StringTok{ }
\StringTok{  }\KeywordTok{geom_histogram}\NormalTok{(}\DataTypeTok{binwidth =} \FloatTok{0.5}\NormalTok{) +}\StringTok{ }
\StringTok{  }\KeywordTok{facet_wrap}\NormalTok{(~drv, }\DataTypeTok{ncol =} \DecValTok{1}\NormalTok{)}
\end{Highlighting}
\end{Shaded}

\begin{figure}[H]
  \includegraphics[width=0.5\linewidth]{_figures/ggplot/dist-fill-1}%
  \includegraphics[width=0.5\linewidth]{_figures/ggplot/dist-fill-2}
\end{figure}

\subsection{Bar charts}\label{sub:bar}

The discrete analogue of the histogram is the bar chart,
\texttt{geom\_bar()}. It's easy to use: \index{Barchart}
\indexf{geom\_bar}

\begin{Shaded}
\begin{Highlighting}[]
\KeywordTok{ggplot}\NormalTok{(mpg, }\KeywordTok{aes}\NormalTok{(manufacturer)) +}\StringTok{ }
\StringTok{  }\KeywordTok{geom_bar}\NormalTok{()}
\end{Highlighting}
\end{Shaded}

\begin{figure}[H]
  \includegraphics[width=1\linewidth]{_figures/ggplot/dist-bar-1}
\end{figure}

(You'll learn how to fix the labels in
\protect\hyperlink{sub:theme-axis}{axis labels}).

Bar charts can be confusing because there are two rather different plots
that are both commonly called bar charts. The above form expects you to
have unsummarised data, and each observation contributes one unit to the
height of each bar. The other form of bar chart is used for
presummarised data. For example, you might have three drugs with their
average effect:

\begin{Shaded}
\begin{Highlighting}[]
\NormalTok{drugs <-}\StringTok{ }\KeywordTok{data.frame}\NormalTok{(}
  \DataTypeTok{drug =} \KeywordTok{c}\NormalTok{(}\StringTok{"a"}\NormalTok{, }\StringTok{"b"}\NormalTok{, }\StringTok{"c"}\NormalTok{),}
  \DataTypeTok{effect =} \KeywordTok{c}\NormalTok{(}\FloatTok{4.2}\NormalTok{, }\FloatTok{9.7}\NormalTok{, }\FloatTok{6.1}\NormalTok{)}
\NormalTok{)}
\end{Highlighting}
\end{Shaded}

To display this sort of data, you need to tell \texttt{geom\_bar()} to
not run the default stat which bins and counts the data. However, I
think it's even better to use \texttt{geom\_point()} because points take
up less space than bars, and don't require that the y axis includes 0.

\begin{Shaded}
\begin{Highlighting}[]
\KeywordTok{ggplot}\NormalTok{(drugs, }\KeywordTok{aes}\NormalTok{(drug, effect)) +}\StringTok{ }\KeywordTok{geom_bar}\NormalTok{(}\DataTypeTok{stat =} \StringTok{"identity"}\NormalTok{)}
\KeywordTok{ggplot}\NormalTok{(drugs, }\KeywordTok{aes}\NormalTok{(drug, effect)) +}\StringTok{ }\KeywordTok{geom_point}\NormalTok{()}
\end{Highlighting}
\end{Shaded}

\begin{figure}[H]
  \includegraphics[width=0.5\linewidth]{_figures/ggplot/unnamed-chunk-8-1}%
  \includegraphics[width=0.5\linewidth]{_figures/ggplot/unnamed-chunk-8-2}
\end{figure}

\subsection{Time series with line and path plots}\label{sub:line}

Line and path plots are typically used for time series data. Line plots
join the points from left to right, while path plots join them in the
order that they appear in the dataset (in other words, a line plot is a
path plot of the data sorted by x value). Line plots usually have time
on the x-axis, showing how a single variable has changed over time. Path
plots show how two variables have simultaneously changed over time, with
time encoded in the way that observations are connected.

Because the year variable in the \texttt{mpg} dataset only has two
values, we'll show some time series plots using the \texttt{economics}
dataset, which contains economic data on the US measured over the last
40 years. The figure below shows two plots of unemployment over time,
both produced using \texttt{geom\_line()}. The first shows the
unemployment rate while the second shows the median number of weeks
unemployed. We can already see some differences in these two variables,
particularly in the last peak, where the unemployment percentage is
lower than it was in the preceding peaks, but the length of unemployment
is high. \indexf{geom\_line} \indexf{geom\_path}
\index{Data!economics@\texttt{economics}}

\begin{Shaded}
\begin{Highlighting}[]
\KeywordTok{ggplot}\NormalTok{(economics, }\KeywordTok{aes}\NormalTok{(date, unemploy /}\StringTok{ }\NormalTok{pop)) +}
\StringTok{  }\KeywordTok{geom_line}\NormalTok{()}
\KeywordTok{ggplot}\NormalTok{(economics, }\KeywordTok{aes}\NormalTok{(date, uempmed)) +}
\StringTok{  }\KeywordTok{geom_line}\NormalTok{()}
\end{Highlighting}
\end{Shaded}

\begin{figure}[H]
  \includegraphics[width=0.5\linewidth]{_figures/ggplot/line-employment-1}%
  \includegraphics[width=0.5\linewidth]{_figures/ggplot/line-employment-2}
\end{figure}

To examine this relationship in greater detail, we would like to draw
both time series on the same plot. We could draw a scatterplot of
unemployment rate vs.~length of unemployment, but then we could no
longer see the evolution over time. The solution is to join points
adjacent in time with line segments, forming a \emph{path} plot.

Below we plot unemployment rate vs.~length of unemployment and join the
individual observations with a path. Because of the many line crossings,
the direction in which time flows isn't easy to see in the first plot.
In the second plot, we colour the points to make it easier to see the
direction of time.

\begin{Shaded}
\begin{Highlighting}[]
\KeywordTok{ggplot}\NormalTok{(economics, }\KeywordTok{aes}\NormalTok{(unemploy /}\StringTok{ }\NormalTok{pop, uempmed)) +}\StringTok{ }
\StringTok{  }\KeywordTok{geom_path}\NormalTok{() +}
\StringTok{  }\KeywordTok{geom_point}\NormalTok{()}

\NormalTok{year <-}\StringTok{ }\NormalTok{function(x) }\KeywordTok{as.POSIXlt}\NormalTok{(x)$year +}\StringTok{ }\DecValTok{1900}
\KeywordTok{ggplot}\NormalTok{(economics, }\KeywordTok{aes}\NormalTok{(unemploy /}\StringTok{ }\NormalTok{pop, uempmed)) +}\StringTok{ }
\StringTok{  }\KeywordTok{geom_path}\NormalTok{(}\DataTypeTok{colour =} \StringTok{"grey50"}\NormalTok{) +}
\StringTok{  }\KeywordTok{geom_point}\NormalTok{(}\KeywordTok{aes}\NormalTok{(}\DataTypeTok{colour =} \KeywordTok{year}\NormalTok{(date)))}
\end{Highlighting}
\end{Shaded}

\begin{figure}[H]
  \includegraphics[width=0.5\linewidth]{_figures/ggplot/path-employ-1}%
  \includegraphics[width=0.5\linewidth]{_figures/ggplot/path-employ-2}
\end{figure}

We can see that unemployment rate and length of unemployment are highly
correlated, but in recent years the length of unemployment has been
increasing relative to the unemployment rate.

With longitudinal data, you often want to display multiple time series
on each plot, each series representing one individual. To do this you
need to map the \texttt{group} aesthetic to a variable encoding the
group membership of each observation. This is explained in more depth in
\protect\hyperlink{sec:grouping}{grouping}.
\index{Longitudinal data|see{Data, longitudinal}}
\index{Data!longitudinal}

\subsection{Exercises}\label{exercises-4}

\begin{enumerate}
\def\labelenumi{\arabic{enumi}.}
\item
  What's the problem with the plot created by
  \texttt{ggplot(mpg,\ aes(cty,\ hwy))\ +\ geom\_point()}? Which of the
  geoms described above is most effective at remedying the problem?
\item
  One challenge with
  \texttt{ggplot(mpg,\ aes(class,\ hwy))\ +\ geom\_boxplot()} is that
  the ordering of \texttt{class} is alphabetical, which is not terribly
  useful. How could you change the factor levels to be more informative?

  Rather than reordering the factor by hand, you can do it automatically
  based on the data:
  \texttt{ggplot(mpg,\ aes(reorder(class,\ hwy),\ hwy))\ +\ geom\_boxplot()}.
  What does \texttt{reorder()} do? Read the documentation.
\item
  Explore the distribution of the carat variable in the
  \texttt{diamonds} dataset. What binwidth reveals the most interesting
  patterns?
\item
  Explore the distribution of the price variable in the
  \texttt{diamonds} data. How does the distribution vary by cut?
\item
  You now know (at least) three ways to compare the distributions of
  subgroups: \texttt{geom\_violin()}, \texttt{geom\_freqpoly()} and the
  colour aesthetic, or \texttt{geom\_histogram()} and facetting. What
  are the strengths and weaknesses of each approach? What other
  approaches could you try?
\item
  Read the documentation for \texttt{geom\_bar()}. What does the
  \texttt{weight} aesthetic do?
\item
  Using the techniques already discussed in this chapter, come up with
  three ways to visualise a 2d categorical distribution. Try them out by
  visualising the distribution of \texttt{model} and
  \texttt{manufacturer}, \texttt{trans} and \texttt{class}, and
  \texttt{cyl} and \texttt{trans}.
\end{enumerate}

\hypertarget{sec:axes}{\section{Modifying the axes}\label{sec:axes}}

You'll learn the full range of options available in
\protect\hyperlink{cha:scales}{scales}, but two families of useful
helpers let you make the most common modifications. \texttt{xlab()} and
\texttt{ylab()} modify the x- and y-axis labels: \indexf{xlab}
\indexf{ylab}

\begin{Shaded}
\begin{Highlighting}[]
\KeywordTok{ggplot}\NormalTok{(mpg, }\KeywordTok{aes}\NormalTok{(cty, hwy)) +}
\StringTok{  }\KeywordTok{geom_point}\NormalTok{(}\DataTypeTok{alpha =} \DecValTok{1} \NormalTok{/}\StringTok{ }\DecValTok{3}\NormalTok{)}

\KeywordTok{ggplot}\NormalTok{(mpg, }\KeywordTok{aes}\NormalTok{(cty, hwy)) +}
\StringTok{  }\KeywordTok{geom_point}\NormalTok{(}\DataTypeTok{alpha =} \DecValTok{1} \NormalTok{/}\StringTok{ }\DecValTok{3}\NormalTok{) +}\StringTok{ }
\StringTok{  }\KeywordTok{xlab}\NormalTok{(}\StringTok{"city driving (mpg)"}\NormalTok{) +}\StringTok{ }
\StringTok{  }\KeywordTok{ylab}\NormalTok{(}\StringTok{"highway driving (mpg)"}\NormalTok{)}

\CommentTok{# Remove the axis labels with NULL}
\KeywordTok{ggplot}\NormalTok{(mpg, }\KeywordTok{aes}\NormalTok{(cty, hwy)) +}
\StringTok{  }\KeywordTok{geom_point}\NormalTok{(}\DataTypeTok{alpha =} \DecValTok{1} \NormalTok{/}\StringTok{ }\DecValTok{3}\NormalTok{) +}\StringTok{ }
\StringTok{  }\KeywordTok{xlab}\NormalTok{(}\OtherTok{NULL}\NormalTok{) +}\StringTok{ }
\StringTok{  }\KeywordTok{ylab}\NormalTok{(}\OtherTok{NULL}\NormalTok{)}
\end{Highlighting}
\end{Shaded}

\begin{figure}[H]
  \includegraphics[width=0.333\linewidth]{_figures/ggplot/unnamed-chunk-9-1}%
  \includegraphics[width=0.333\linewidth]{_figures/ggplot/unnamed-chunk-9-2}%
  \includegraphics[width=0.333\linewidth]{_figures/ggplot/unnamed-chunk-9-3}
\end{figure}

\texttt{xlim()} and \texttt{ylim()} modify the limits of axes:
\indexf{xlim} \indexf{ylim}

\begin{Shaded}
\begin{Highlighting}[]
\KeywordTok{ggplot}\NormalTok{(mpg, }\KeywordTok{aes}\NormalTok{(drv, hwy)) +}
\StringTok{  }\KeywordTok{geom_jitter}\NormalTok{(}\DataTypeTok{width =} \FloatTok{0.25}\NormalTok{)}

\KeywordTok{ggplot}\NormalTok{(mpg, }\KeywordTok{aes}\NormalTok{(drv, hwy)) +}
\StringTok{  }\KeywordTok{geom_jitter}\NormalTok{(}\DataTypeTok{width =} \FloatTok{0.25}\NormalTok{) +}\StringTok{ }
\StringTok{  }\KeywordTok{xlim}\NormalTok{(}\StringTok{"f"}\NormalTok{, }\StringTok{"r"}\NormalTok{) +}\StringTok{ }
\StringTok{  }\KeywordTok{ylim}\NormalTok{(}\DecValTok{20}\NormalTok{, }\DecValTok{30}\NormalTok{)}
\CommentTok{#> Warning: Removed 138 rows containing missing values (geom_point).}
  
\CommentTok{# For continuous scales, use NA to set only one limit}
\KeywordTok{ggplot}\NormalTok{(mpg, }\KeywordTok{aes}\NormalTok{(drv, hwy)) +}
\StringTok{  }\KeywordTok{geom_jitter}\NormalTok{(}\DataTypeTok{width =} \FloatTok{0.25}\NormalTok{, }\DataTypeTok{na.rm =} \OtherTok{TRUE}\NormalTok{) +}\StringTok{ }
\StringTok{  }\KeywordTok{ylim}\NormalTok{(}\OtherTok{NA}\NormalTok{, }\DecValTok{30}\NormalTok{)}
\end{Highlighting}
\end{Shaded}

\begin{figure}[H]
  \includegraphics[width=0.333\linewidth]{_figures/ggplot/unnamed-chunk-10-1}%
  \includegraphics[width=0.333\linewidth]{_figures/ggplot/unnamed-chunk-10-2}%
  \includegraphics[width=0.333\linewidth]{_figures/ggplot/unnamed-chunk-10-3}
\end{figure}

Changing the axes limits sets values outside the range to \texttt{NA}.
You can suppress the associated warning with \texttt{na.rm\ =\ TRUE}.

\hypertarget{sec:output}{\section{Output}\label{sec:output}}

Most of the time you create a plot object and immediately plot it, but
you can also save a plot to a variable and manipulate it:

\begin{Shaded}
\begin{Highlighting}[]
\NormalTok{p <-}\StringTok{ }\KeywordTok{ggplot}\NormalTok{(mpg, }\KeywordTok{aes}\NormalTok{(displ, hwy, }\DataTypeTok{colour =} \KeywordTok{factor}\NormalTok{(cyl))) +}
\StringTok{  }\KeywordTok{geom_point}\NormalTok{()}
\end{Highlighting}
\end{Shaded}

Once you have a plot object, there are a few things you can do with it:

\begin{itemize}
\item
  Render it on screen with \texttt{print()}. This happens automatically
  when running interactively, but inside a loop or function, you'll need
  to \texttt{print()} it yourself. \indexf{print}

\begin{Shaded}
\begin{Highlighting}[]
\KeywordTok{print}\NormalTok{(p)}
\end{Highlighting}
\end{Shaded}

  \begin{figure}[H]
    \centering
    \includegraphics[width=0.65\linewidth]{_figures/ggplot/unnamed-chunk-11-1}
  \end{figure}
\item
  Save it to disk with \texttt{ggsave()}, described in
  \protect\hyperlink{sec:saving}{saving your output}.

\begin{Shaded}
\begin{Highlighting}[]
\CommentTok{# Save png to disk}
\KeywordTok{ggsave}\NormalTok{(}\StringTok{"plot.png"}\NormalTok{, }\DataTypeTok{width =} \DecValTok{5}\NormalTok{, }\DataTypeTok{height =} \DecValTok{5}\NormalTok{)}
\end{Highlighting}
\end{Shaded}
\item
  Briefly describe its structure with \texttt{summary()}.
  \indexf{summary}

\begin{Shaded}
\begin{Highlighting}[]
\KeywordTok{summary}\NormalTok{(p)}
\CommentTok{#> data: manufacturer, model, displ, year, cyl, trans, drv, cty,}
\CommentTok{#>   hwy, fl, class [234x11]}
\CommentTok{#> mapping:  x = displ, y = hwy, colour = factor(cyl)}
\CommentTok{#> faceting: facet_null() }
\CommentTok{#> -----------------------------------}
\CommentTok{#> geom_point: na.rm = FALSE}
\CommentTok{#> stat_identity: na.rm = FALSE}
\CommentTok{#> position_identity}
\end{Highlighting}
\end{Shaded}
\item
  Save a cached copy of it to disk, with \texttt{saveRDS()}. This saves
  a complete copy of the plot object, so you can easily re-create it
  with \texttt{readRDS()}. \indexf{saveRDS} \indexf{readRDS}

\begin{Shaded}
\begin{Highlighting}[]
\KeywordTok{saveRDS}\NormalTok{(p, }\StringTok{"plot.rds"}\NormalTok{)}
\NormalTok{q <-}\StringTok{ }\KeywordTok{readRDS}\NormalTok{(}\StringTok{"plot.rds"}\NormalTok{)}
\end{Highlighting}
\end{Shaded}
\end{itemize}

You'll learn more about how to manipulate these objects in
\protect\hyperlink{cha:programming}{programming with ggplot2}.

\hypertarget{qplot}{\section{Quick plots}\label{qplot}}

In some cases, you will want to create a quick plot with a minimum of
typing. In these cases you may prefer to use \texttt{qplot()} over
\texttt{ggplot()}. \texttt{qplot()} lets you define a plot in a single
call, picking a geom by default if you don't supply one. To use it,
provide a set of aesthetics and a data set: \indexf{qplot}

\begin{Shaded}
\begin{Highlighting}[]
\KeywordTok{qplot}\NormalTok{(displ, hwy, }\DataTypeTok{data =} \NormalTok{mpg)}
\KeywordTok{qplot}\NormalTok{(displ, }\DataTypeTok{data =} \NormalTok{mpg)}
\CommentTok{#> `stat_bin()` using `bins = 30`. Pick better value with `binwidth`.}
\end{Highlighting}
\end{Shaded}

\begin{figure}[H]
  \includegraphics[width=0.5\linewidth]{_figures/ggplot/unnamed-chunk-15-1}%
  \includegraphics[width=0.5\linewidth]{_figures/ggplot/unnamed-chunk-15-2}
\end{figure}

Unless otherwise specified, \texttt{qplot()} tries to pick a sensible
geometry and statistic based on the arguments provided. For example, if
you give \texttt{qplot()} \texttt{x} and \texttt{y} variables, it'll
create a scatterplot. If you just give it an \texttt{x}, it'll create a
histogram or bar chart depending on the type of variable.

\texttt{qplot()} assumes that all variables should be scaled by default.
If you want to set an aesthetic to a constant, you need to use
\texttt{I()}: \indexf{I}

\begin{Shaded}
\begin{Highlighting}[]
\KeywordTok{qplot}\NormalTok{(displ, hwy, }\DataTypeTok{data =} \NormalTok{mpg, }\DataTypeTok{colour =} \StringTok{"blue"}\NormalTok{)}
\KeywordTok{qplot}\NormalTok{(displ, hwy, }\DataTypeTok{data =} \NormalTok{mpg, }\DataTypeTok{colour =} \KeywordTok{I}\NormalTok{(}\StringTok{"blue"}\NormalTok{))}
\end{Highlighting}
\end{Shaded}

\begin{figure}[H]
  \includegraphics[width=0.5\linewidth]{_figures/ggplot/unnamed-chunk-16-1}%
  \includegraphics[width=0.5\linewidth]{_figures/ggplot/unnamed-chunk-16-2}
\end{figure}

If you're used to \texttt{plot()} you may find \texttt{qplot()} to be a
useful crutch to get up and running quickly. However, while it's
possible to use \texttt{qplot()} to access all of the customizability of
ggplot2, I don't recommend it. If you find yourself making a more
complex graph, e.g.~using different aesthetics in different layers or
manually setting visual properties, use \texttt{ggplot()}, not
\texttt{qplot()}.

\providecommand{\setflag}{\newif \ifwhole \wholefalse}
\setflag
\ifwhole\else

% Typography and geometry ----------------------------------------------------
\documentclass[letterpaper]{scrbook}
\usepackage[inner=3cm,top=2.5cm,outer=3.5cm]{geometry}

\renewcommand\familydefault{bch}
\usepackage[utf8]{inputenc}
\usepackage{microtype}
\usepackage[small]{caption}
\usepackage[small]{titlesec}
\raggedbottom

% Graphics -------------------------------------------------------------------
\usepackage[pdftex]{graphicx}
\graphicspath{{_include/}}
\DeclareGraphicsExtensions{.png,.pdf}

% Code formatting ------------------------------------------------------------
\usepackage{fancyvrb}
\usepackage{courier}
\usepackage{listings}
\usepackage{color}
\usepackage{alltt}


\definecolor{comment}{rgb}{0.60, 0.60, 0.53}
\definecolor{background}{rgb}{0.97, 0.97, 1.00}
\definecolor{string}{rgb}{0.863, 0.066, 0.266}
\definecolor{number}{rgb}{0.0, 0.6, 0.6}
\definecolor{variable}{rgb}{0.00, 0.52, 0.70}
\lstset{
  basicstyle=\ttfamily,
  keywordstyle=\bfseries, 
  identifierstyle=,  
  commentstyle=\color{comment} \emph,
  stringstyle=\color{string},
  showstringspaces=false,
  columns = fullflexible,
  backgroundcolor=\color{background},
  mathescape = true,
  escapeinside=&&,
  fancyvrb
}
\newcommand{\code}[1]{\lstinline!#1!}
\newcommand{\f}[1]{\lstinline!#1()!}



% Links ----------------------------------------------------------------------

\usepackage{hyperref}
\definecolor{slateblue}{rgb}{0.07,0.07,0.488}
\hypersetup{colorlinks=true,linkcolor=slateblue,anchorcolor=slateblue,citecolor=slateblue,filecolor=slateblue,urlcolor=slateblue,bookmarksnumbered=true,pdfview=FitB}
\usepackage{url}

% Tables ---------------------------------------------------------------------
\usepackage{longtable}
\usepackage{booktabs}

% Miscellaneous --------------------------------------------------------------
\usepackage{pdfsync}
\usepackage{appendix}

\usepackage[round,sort&compress,sectionbib]{natbib}
\bibliographystyle{plainnat}


\title{ggplot2}
\author{Hadley Wickham}

\begin{document}
\fi


\chapter{Toolbox}
\label{cha:toolbox}

\section{Introduction}\label{sec:introduction}

Graphical objects, or geoms for short, are a key feature of {\tt geom\_plot}.  Geoms create visual objects on the plot that allow you to see you data.  By choosing various types of geoms, you can recreate common plots.  For example, the {\tt geom\_point} geom will make a scatterplot, and the {\tt geom\_line} geom will create a line plot.  The advantage of the geom based system is that you can easily combine different geoms to create almost any plot that you can think of.  This section describes geom functions in detail, including what geoms are currently available in {\tt geom\_plot} and how you can go about creating your own.

Use a mixture of {\tt ggplot()} and {\tt qplot()} calls.  If you need a reminder on how to translate between the two, see Section~\ref{sec:qplot-ggplot}.

\section{Basics}\label{sub:basics}

These geoms are the basic geoms used to build up almost all of the other geoms.  These are useful for creating basic graphics, and when building your own geom function (see section \ref{sec:writing_your_own}).

\begin{itemize}
  \item {\tt geom\_point}: points
  \item {\tt geom\_line}, {\tt geom\_path}: paths and lines.  Lines are paths that have their x-axis values ordered in increasing value.
  \item {\tt geom\_polygon}: polygons
  \item {\tt geom\_bar}: bars
  \item {\tt geom\_text}: text
  \item {\tt geom\_tile}: tiles, rectangles which form a regular tessellation of the plane
\end{itemize}

\section{Displaying distributions}\label{sec:distributions}

There are quite a few geoms associated with displaying distributions:

\begin{itemize}
	\item {\tt geom\_boxplot}: box and whisker plot, for a continuous variable possibly conditioned by a categorical variable
	\item {\tt geom\_jitter}: a crude way of investigating densities
	\item {\tt geom\_quantile}: quantiles, for a continuous variable conditional on another continuous variable.
	\item {\tt geom\_density}: 
	\item {\tt geom\_histogram}: 
	\item {\tt geom\_2ddensity}: for displaying the density of points on the plot surface.
\end{itemize}

(Ask Heike about this)

We have a number of plots available to investigate distributions, depending on the number of variables, and whether we are interested in the conditional or joint distribution.

Single variable
+ cont: histogram, density plot, boxplot
+ cat:  barchart

Two variables: conditional
+ cat  | cat:  mosaic plot
+ cat  | cont: ?
+ cont | cont: quantiles
+ cat  | cont: boxplot

Two variables: joint
+ cat  * cat:  fluctuation diagram
+ cat  * cont: boxplots?
+ cont * cont: bagplot, geom\_2ddensity

Jittered points can be used for any joint distribution (or conditional if one or both variables are categorical)

\section{Dealing with overplotting}\label{sec:overplotting}

The simplest way to deal with overplotting is to bin the plot into small squares and count the number of points that lies in each square, much like a 2D histogram.  This count can then be visualised as the third variable on a plot.  However, breaking the plot into many small squares produces distracting visual artefacts.  Carr (reference) suggests using hexagons instead, and this is implemented with {\tt geom\_hexagon}, using the capabilities of the {\tt hexbin package}.

A continuous analogue of this is to compute a 2D density function and then visualise this as coloured tiles or contour lines.  This can be done with {\tt geom\_2ddensity}.

Another approach to dealing with overplotting is to add supplemental information to help guide the eye to the true shape of or pattern within the data:

\begin{itemize}
	\item {\tt geom\_smooth} add a smooth line showing the mean.
	\item {\tt geom\_quantile} add a smooth line showing any quantile you are interested in.
\end{itemize}

\section{``3d'' plots}

{\tt geom\_plot} currently does not support true 3D plots.  However, it does offer two tools for producing pseudo-3d plots, the imageplot and the contour plot.

\begin{itemize}
	\item {\tt geom\_tile}: map z variable to fill colour
	\item {\tt geom\_contour}: useful for smoother surfaces
	\item {\tt geom\_point}: can map abs(z) variable to size and sign(z) to colour
\end{itemize}

These are often better than ``true'' 3d for static plots anyway, because many perceptual cues necessary for accurate depth perception (eg. occlusion, parallax) are not present in static plots.  You can also manually project data points in a higher dimensional space by multiplying by a projection matrix.  However, correctly representing occlusion or generating correct perspective effects will require considerably more effort.  You may want to look at RGL (\url{http://www.rgl.com}) and rggobi (\url{http://www.ggobi.org/ggobi}) for other solutions.

\section{Revealing uncertainty}\label{sub:displaying_uncertainty}

If you have information about the uncertainty present in your data, possibly from a model or distributional assumptions, it is often useful to visualise.  There are two geoms that allow you to do this depending on whether you have point or functional confidence intervals:

\begin{itemize}
	\item {\tt geom\_errorbar}: for pointwise confidence intervals
	\item {\tt geom\_ribbon}: ribbons of variable width, useful for displaying confidence intervals around functions
\end{itemize}

There are two ways to display standard errors with {\tt ggplot}.  For point standard errors, you can use the {\tt errorbar} geom.  For continuous or functional standard errors, you can use the {\tt ribbon} grob.  We've have already seen an example of this: the {\tt ribbon} grob is used inside {\tt smooth} to display the standard errors of the smooth.  Because there are so many different ways to calculate standard errors, the calculation is up to you.  {\tt ggplot} only provides facilities for displaying them once you have them.

For both {\tt ribbon} and {\tt errobar} you can specify confidence internals using {\tt min} and {\tt lower} which specify the upper and lower edges of the confidence band.

Calculating with {\tt stat\_sum}, {\tt stat\_smooth} etc.

\section{Annotating a plot}\label{sub:annotating_a_plot}

Any of the basic geoms can be used to annotate a plot with additional output, for example, adding text with {\tt geom\_text}, or a point illustrating the mean with {\tt geom\_point}.  Additionally, there are several geoms whose use is almost entirely for annotation.  These are:

\begin{itemize}
	\item {\tt geom\_vline}, {\tt geom\_hline}: add vertical or horizontal lines to a plot
	\item {\tt geom\_abline}: add lines with arbitrary slope and intercept to a plot
\end{itemize}

See also \secref{sec:adding_annotation} for ways to add more general types of annotation using grid graphics.

\section{Plots for weighted data}\label{sec:weighted_data}

When you have aggregated data where each row in the dataset represents multiple observations, you need some way to take into account the weighting variable.  Since there are no variables appropriate for weighting in the diamonds data, we will use some data collected on Midwest states in the 2000 US census.  The data consists mainly of percentages (eg. percent white, percent below poverty line, percentage with college degree) and some information for each county (area, total population, population density).

There are few different things we might want to weight by: 

\begin{itemize}
	\item nothing, to look at county numbers
	\item total population, to work with absolute numbers
	\item area, to investigate geographic effects
\end{itemize}

\noindent The choice of a weighting variable profoundly effects what we are looking at in the plot and the conclusions that we will draw.  There are two aesthetic attributes that can be used to adjust for weights.  Firstly, for simple geoms like lines and points, you can make the size of the grob proportional to the number of points, using the {\tt size} aesthetic, as follows:

% decumar<<< 
% interweave({
% midwest <- read.csv("~/Documents/graphics/weighted/midwest.csv")
% qplot(percwhite, percbelowpoverty, data=midwest)
% qplot(percwhite, percbelowpoverty, data=midwest, size=poptotal)
% qplot(percwhite, percbelowpoverty, data=midwest, size=area)
% })
% |||
\begin{alltt}
> midwest <- read.csv("~/Documents/graphics/weighted/midwest.csv")
> qplot(percwhite, percbelowpoverty, data = midwest)
\includegraphics[scale=0.5]{4b8c600efdc1f31f77c7c3368cac11d2}

> qplot(percwhite, percbelowpoverty, data = midwest, size = poptotal)
\includegraphics[scale=0.5]{f3976daae1f4a7ef912ea259ca3dd8f5}

> qplot(percwhite, percbelowpoverty, data = midwest, size = area)
\includegraphics[scale=0.5]{440b95f469e5b96a08e077d4a177e603}

\end{alltt}
% >>>

For more complicated grobs which involve some statistical transformation, we specify weights with the {\tt weight} aesthetic.  These weights will be passed on to the statistical summary function.  Weights are supported for every case where it makes sense: smoothers, quantile regressions, box plots, histograms, and density plots.  You can't see this weighting variable directly, and it doesn't produce a legend, but it will change the results of the statistical summary.

The following example shows how weighting by population density effects the relationship between percent white and percent below the poverty line.

% decumar<<< 
% interweave({
% qplot(percwhite, percbelowpoverty, data=midwest, geom=c("point","smooth"), method=lm)
% qplot(percwhite, percbelowpoverty, data=midwest, size=popdensity, weight=popdensity,geom=c("point","smooth"), method=lm)
% })
% |||
\begin{alltt}
> qplot(percwhite, percbelowpoverty, data = midwest, geom = c("point", 
+     "smooth"), method = lm)
\includegraphics[scale=0.5]{54d5b5a95badb06ff2339bfd51826af0}

> qplot(percwhite, percbelowpoverty, data = midwest, size = popdensity, 
+     weight = popdensity, geom = c("point", "smooth"), method = lm)
\includegraphics[scale=0.5]{3353081c3ffa3fd62b69fb2c0aef9142}

\end{alltt}
% >>>

When we weight a histogram or density plot by total population, we change from looking at the distribution of the number of counties, to the distribution of the number of people.  This example shows the difference this makes for a histogram and density plot of the percentage below the poverty line.

% decumar<<< 
% interweave({
% qplot(percbelowpoverty, data=midwest, geom="histogram", binwidth=1)
% qplot(percbelowpoverty, data=midwest, geom="histogram", weight=poptotal, binwidth=1)
% })
% |||
\begin{alltt}
> qplot(percbelowpoverty, data = midwest, geom = "histogram", binwidth = 1)
\includegraphics[scale=0.5]{d8cef4ee59175bda59eaa2647a64b930}

> qplot(percbelowpoverty, data = midwest, geom = "histogram", weight = poptotal, 
+     binwidth = 1)
\includegraphics[scale=0.5]{affb33e87c76bc94bd3ea9eb2df0190c}

\end{alltt}
% >>>

\ifwhole
\else
  \nobibliography{/Users/hadley/documents/phd/references}
  \end{document}
\fi


\part{The Grammar}

\providecommand{\setflag}{\newif \ifwhole \wholefalse}
\setflag
\ifwhole\else

% Typography and geometry ----------------------------------------------------
\documentclass[letterpaper]{scrbook}
\usepackage[inner=3cm,top=2.5cm,outer=3.5cm]{geometry}

\renewcommand\familydefault{bch}
\usepackage[utf8]{inputenc}
\usepackage{microtype}
\usepackage[small]{caption}
\usepackage[small]{titlesec}
\raggedbottom

% Graphics -------------------------------------------------------------------
\usepackage[pdftex]{graphicx}
\graphicspath{{_include/}}
\DeclareGraphicsExtensions{.png,.pdf}

% Code formatting ------------------------------------------------------------
\usepackage{fancyvrb}
\usepackage{courier}
\usepackage{listings}
\usepackage{color}
\usepackage{alltt}


\definecolor{comment}{rgb}{0.60, 0.60, 0.53}
\definecolor{background}{rgb}{0.97, 0.97, 1.00}
\definecolor{string}{rgb}{0.863, 0.066, 0.266}
\definecolor{number}{rgb}{0.0, 0.6, 0.6}
\definecolor{variable}{rgb}{0.00, 0.52, 0.70}
\lstset{
  basicstyle=\ttfamily,
  keywordstyle=\bfseries, 
  identifierstyle=,  
  commentstyle=\color{comment} \emph,
  stringstyle=\color{string},
  showstringspaces=false,
  columns = fullflexible,
  backgroundcolor=\color{background},
  mathescape = true,
  escapeinside=&&,
  fancyvrb
}
\newcommand{\code}[1]{\lstinline!#1!}
\newcommand{\f}[1]{\lstinline!#1()!}



% Links ----------------------------------------------------------------------

\usepackage{hyperref}
\definecolor{slateblue}{rgb}{0.07,0.07,0.488}
\hypersetup{colorlinks=true,linkcolor=slateblue,anchorcolor=slateblue,citecolor=slateblue,filecolor=slateblue,urlcolor=slateblue,bookmarksnumbered=true,pdfview=FitB}
\usepackage{url}

% Tables ---------------------------------------------------------------------
\usepackage{longtable}
\usepackage{booktabs}

% Miscellaneous --------------------------------------------------------------
\usepackage{pdfsync}
\usepackage{appendix}

\usepackage[round,sort&compress,sectionbib]{natbib}
\bibliographystyle{plainnat}


\title{ggplot2}
\author{Hadley Wickham}

\begin{document}
\fi


\chapter{Mastering the grammar}
\label{cha:mastery}

% Introduction to the components of the grammar
% Introduction to the data structure
% Roadmap for next few chapters

\section{Introduction}\label{sec:introduction}

You can choose to use just {\tt qplot}, without any understanding of the underlying grammar, but you will not be able to use the full power of ggplot.  By learning more about the grammar, and the components that make it up, you will be able to create a wider range of plots, as well as being able to combine multiple sources of data, and customise to your heart's content.

This chapter describes the theoretical basis of ggplot2: the layered grammar of graphics, a based based on Wilkinson's grammar of graphics \citep{wilkinson:2006}.  The next chapters then describe parts of the grammar in more details: layers (geoms and stats), scales, and positioning (coordinate systems, faceting and position adjustments),

Pull more from layered grammar paper, but needs to be rather different.

\section{Building a plot}
\label{sec:building_a_plot}

Consider the mammal's sleep dataset illustrated in Table~\ref{tbl:sleep}

\begin{alltt}
  qplot(bodywt, brainwt, data=msleep)
  qplot(bodywt, brainwt, data=msleep, log="xy")
  qplot(bodywt, brainwt, data=msleep, log="xy", facets = . ~ sleepyhead)
\end{alltt}

Pick out 5 animals to use throughout in tables.

\section{What is a plot?}
\label{sec:what_is_a_plot}

One way to think about the grammar of graphics is as a question: what is a plot?  The grammar answers this by describing a plot as a collection of independent components, each describing an independent part of the plot.  There are three basic things we need for a plot: one or more layers, scales to map variables from data space to visual space, and a coordinate system.  These are described below.

\begin{itemize}
  \item One or more layers.  A layer is composed of data and a description of which data variables should be mapped to which aesthetic properties, a geometric object, and a statistical transformation:
  
  \begin{itemize}
  	\item Data is obviously the most important part, and it is what you provide.  This is what you are displaying visually to aid communication or analysis.  You also need to describe how variables in the dataset are mapped to visual properties.  For example, in Figure 2.X we mapped diamond price to y position, carat to y position and colour to colour.  Because the data and aesthetic mapping set is usually the same in most layers, these can also be set as defaults at the plot level.
  	
  	\item {\bf Geoms}, short for geometric objects, control the type of plot that you create.  For example, using a point geom will create a scatterplot, while using a line geom will create a line plot.

  	\item {\bf Stats}, or statistical transformations, reduce or augment the data in a statistical manner.  For example, a useful stat is the smoother, which shows the mean of y, conditional on x.  Another common stat is the binner, which bins data in to bins.   Every geom has a default statistic, and every statistic a default geom.  For example, the bin statistic has defaults to using the bar geom to produce a histogram.

  	\item {\bf Position adjustment}
  \end{itemize}

  \item A scale for all the aesthetic properties.  {\bf Scales} control the mapping from data attributes to aesthetic attributes.  They also provide an inverse mapping in the form of a guide, an axis or legend, which facilitates reading the final graphic.  Aesthetic attributes are things like position, size, colour---anything that you can perceive.  The function that maps data to aesthetic attributes is a scale. It takes values in data space (continuous or categorical) and maps them into an aesthetic space (eg. colour, size, shape).  A scale also provides guides to convert back from the aesthetic attribute to the original data.  Guides are either axes (for position) or legends (for everything else)

  \item A coordinate system.  A {\bf coord}, or coordinate systems, maps the position of objects on to the plane of the plot.  Typically we will use the cartesian coordinate system, but sometimes others are useful.
\end{itemize}

There is also another thing that turns out to be sufficiently useful that we should include it in our general framework: faceting (also known as conditioned or trellis plots). This allows us to easily create small multiples of different subsets of an entire dataset. This is a powerful tool when investigating whether patterns hold across all conditions.


\section{Data structures}
\label{sec:data_structures}

These principles are encoded as data structures in a fairly straightforward way.

One thing to note is that all ggplot2 objects (with the exception of the main plot object) are proto objects.  Proto is a package which implements the prototype-style of object-oriented programming.  There are some major differences between this and the typical S3 or S4 style of OO in R, but the good news is that you only need to worry about them if you want to develop your own extensions to ggplot2.  For everyday use, the proto objects are hidden behind a facade which makes them act like normal R objects.

{\tt str} to see full structure (it can be large!)

{\tt summary} briefly describes the structure of the plot

Data stored inside the plot - if you change the data outside of the plot, and then redraw a saved plot, it will not be updated.  Consequence of R copying semantics.

\ifwhole
\else
  \nobibliography{/Users/hadley/documents/phd/references}
  \end{document}
\fi
\chapter{Build a plot layer by layer}\label{cha:layers}

\section{Introduction}

One of the key ideas behind ggplot2 is that it allows you to easily
iterate, building up a complex plot a layer at a time. Each layer can
come from a different dataset and have a different aesthetic mapping,
making it possible to create sophisticated plots that display data from
multiple sources.

You've already created layers with functions like \texttt{geom\_point()}
and \texttt{geom\_histogram()}. In this chapter, you'll dive into the
details of a layer, and how you can control all five components: data,
the aesthetic mappings, the geom, stat, and position adjustments. The
goal here is to give you the tools to build sophisticated plots tailored
to the problem at hand.

\section{Building a plot}

So far, whenever we've created a plot with \texttt{ggplot()}, we've
immediately added on a layer with a geom function. But it's important to
realise that there really are two distinct steps. First we create a plot
with default dataset and aesthetic mappings:

\begin{Shaded}
\begin{Highlighting}[]
\NormalTok{p <-}\StringTok{ }\KeywordTok{ggplot}\NormalTok{(mpg, }\KeywordTok{aes}\NormalTok{(displ, hwy))}
\NormalTok{p}
\end{Highlighting}
\end{Shaded}

\begin{figure}[H]
  \centering
  \includegraphics[width=0.65\linewidth]{_figures/layers/layer1-1}
\end{figure}

There's nothing to see yet, so we need to add a layer:

\begin{Shaded}
\begin{Highlighting}[]
\NormalTok{p +}\StringTok{ }\KeywordTok{geom_point}\NormalTok{()}
\end{Highlighting}
\end{Shaded}

\begin{figure}[H]
  \centering
  \includegraphics[width=0.65\linewidth]{_figures/layers/unnamed-chunk-1-1}
\end{figure}

\texttt{geom\_point()} is a shortcut. Behind the scenes it calls the
\texttt{layer()} function to create a new layer: \indexf{layer}

\begin{Shaded}
\begin{Highlighting}[]
\NormalTok{p +}\StringTok{ }\KeywordTok{layer}\NormalTok{(}
  \DataTypeTok{mapping =} \OtherTok{NULL}\NormalTok{, }
  \DataTypeTok{data =} \OtherTok{NULL}\NormalTok{,}
  \DataTypeTok{geom =} \StringTok{"point"}\NormalTok{, }\DataTypeTok{geom_params =} \KeywordTok{list}\NormalTok{(),}
  \DataTypeTok{stat =} \StringTok{"identity"}\NormalTok{, }\DataTypeTok{stat_params =} \KeywordTok{list}\NormalTok{(),}
  \DataTypeTok{position =} \StringTok{"identity"}
\NormalTok{)}
\end{Highlighting}
\end{Shaded}

This call fully specifies the five components to the layer:
\index{Layers!components}

\begin{itemize}
\item
  \textbf{mapping}: A set of aesthetic mappings, specified using the
  \texttt{aes()} function and combined with the plot defaults as
  described in \hyperref[sec:aes]{aesthetic mappings}. If \texttt{NULL},
  uses the default mapping set in \texttt{ggplot()}.
\item
  \textbf{data}: A dataset which overrides the default plot dataset. It
  is usually omitted (set to \texttt{NULL}), in which case the layer
  will use the default data specified in \texttt{ggplot()}. The
  requirements for data are explained in more detail in
  \hyperref[sec:data]{data}.
\item
  \textbf{geom}: The name of the geometric object to use to draw each
  observation. Geoms are discussed in more detail in
  \hyperref[sec:data]{geom}, and \hyperref[cha:toolbox]{the toolbox}
  explores their use in more depth.

  Geoms can have additional arguments. All geoms take aesthetics as
  parameters. If you supply an aesthetic (e.g.~colour) as a parameter,
  it will not be scaled, allowing you to control the appearance of the
  plot, as described in \hyperref[sub:setting-mapping]{setting
  vs.~mapping}. You can pass params in \texttt{...} (in which case stat
  and geom parameters are automatically teased apart), or in a list
  passed to \texttt{geom\_params}.
\item
  \textbf{stat}: The name of the statistical tranformation to use. A
  statistical transformation performs some useful statistical summary,
  and is key to histograms and smoothers. To keep the data as is, use
  the ``identity'' stat. Learn more in \hyperref[sec:stat]{statistical
  transformations}.

  You only need to set one of stat and geom: every geom has a default
  stat, and every stat a default geom.

  Most stats take additional parameters to specify the details of
  statistical transformation. You can supply params either in
  \texttt{...} (in which case stat and geom parameters are automatically
  teased apart), or in a list called \texttt{stat\_params}.
\item
  \textbf{position}: The method used to adjust overlapping objects, like
  jittering, stacking or dodging. More details in
  \hyperref[sec:position]{position}.
\end{itemize}

It's useful to understand the \texttt{layer()} function so you have a
better mental model of the layer object. But you'll rarely use the full
\texttt{layer()} call because it's so verbose. Instead, you'll use the
shortcut \texttt{geom\_} functions:
\texttt{geom\_point(mapping,\ data,\ ...)} is exactly equivalent to
\texttt{layer(mapping,\ data,\ geom\ =\ "point",\ ...)}.

\hyperdef{}{sec:data}{\section{Data}\label{sec:data}}

Every layer must have some data associated with it, and that data must
be in a tidy data frame. You'll learn about tidy data in
\hyperref[cha:data]{tidy data}, but for now, all you need to know is
that a tidy data frame has variables in the columns and observations in
the rows. This is a strong restriction, but there are good reasons for
it: \index{Data} \indexf{data.frame}

\begin{itemize}
\item
  Your data is very important, so it's best to be explicit about it.
\item
  A single data frame is also easier to save than a multitude of
  vectors, which means it's easier to reproduce your results or send
  your data to someone else.
\item
  It enforces a clean separation of concerns: ggplot2 turns data frames
  into visualisations. Other packages can make data frames in the right
  format (learn more about that in \hyperref[sub:modelvis]{model
  visualisation}).
\end{itemize}

The data on each layer doesn't need to be the same, and it's often
useful to combine multiple datasets in a single plot. To illustrate that
idea I'm going to generate two new datasets related to the mpg dataset.
First I'll fit a loess model and generate predictions from it. (This is
what \texttt{geom\_smooth()} does behind the scenes)

\begin{Shaded}
\begin{Highlighting}[]
\NormalTok{mod <-}\StringTok{ }\KeywordTok{loess}\NormalTok{(hwy ~}\StringTok{ }\NormalTok{displ, }\DataTypeTok{data =} \NormalTok{mpg)}
\NormalTok{grid <-}\StringTok{ }\KeywordTok{data_frame}\NormalTok{(}\DataTypeTok{displ =} \KeywordTok{seq}\NormalTok{(}\KeywordTok{min}\NormalTok{(mpg$displ), }\KeywordTok{max}\NormalTok{(mpg$displ), }\DataTypeTok{length =} \DecValTok{50}\NormalTok{))}
\NormalTok{grid$hwy <-}\StringTok{ }\KeywordTok{predict}\NormalTok{(mod, }\DataTypeTok{newdata =} \NormalTok{grid)}

\NormalTok{grid}
\CommentTok{#> Source: local data frame [50 x 2]}
\CommentTok{#> }
\CommentTok{#>    displ   hwy}
\CommentTok{#>    (dbl) (dbl)}
\CommentTok{#> 1   1.60  33.1}
\CommentTok{#> 2   1.71  32.2}
\CommentTok{#> 3   1.82  31.3}
\CommentTok{#> 4   1.93  30.4}
\CommentTok{#> 5   2.04  29.6}
\CommentTok{#> 6   2.15  28.8}
\CommentTok{#> ..   ...   ...}
\end{Highlighting}
\end{Shaded}

Next, I'll isolate observations that are particularly far away from
their predicted values:

\begin{Shaded}
\begin{Highlighting}[]
\NormalTok{std_resid <-}\StringTok{ }\KeywordTok{resid}\NormalTok{(mod) /}\StringTok{ }\NormalTok{mod$s}
\NormalTok{outlier <-}\StringTok{ }\KeywordTok{filter}\NormalTok{(mpg, }\KeywordTok{abs}\NormalTok{(std_resid) >}\StringTok{ }\DecValTok{2}\NormalTok{)}
\NormalTok{outlier}
\CommentTok{#> Source: local data frame [6 x 11]}
\CommentTok{#> }
\CommentTok{#>   manufacturer      model displ  year   cyl      trans   drv   cty}
\CommentTok{#>          (chr)      (chr) (dbl) (int) (int)      (chr) (chr) (int)}
\CommentTok{#> 1    chevrolet   corvette   5.7  1999     8 manual(m6)     r    16}
\CommentTok{#> 2      pontiac grand prix   3.8  2008     6   auto(l4)     f    18}
\CommentTok{#> 3      pontiac grand prix   5.3  2008     8   auto(s4)     f    16}
\CommentTok{#> 4   volkswagen      jetta   1.9  1999     4 manual(m5)     f    33}
\CommentTok{#> 5   volkswagen new beetle   1.9  1999     4 manual(m5)     f    35}
\CommentTok{#> 6   volkswagen new beetle   1.9  1999     4   auto(l4)     f    29}
\CommentTok{#> Variables not shown: hwy (int), fl (chr), class (chr)}
\end{Highlighting}
\end{Shaded}

I've generated these datasets because it's common to enhance the display
of raw data with a statistical summary and some annotations. With these
new datasets, I can improve our initial scatterplot by overlaying a
smoothed line, and labelling the outlying points:

\begin{Shaded}
\begin{Highlighting}[]
\KeywordTok{ggplot}\NormalTok{(mpg, }\KeywordTok{aes}\NormalTok{(displ, hwy)) +}\StringTok{ }
\StringTok{  }\KeywordTok{geom_point}\NormalTok{() +}\StringTok{ }
\StringTok{  }\KeywordTok{geom_line}\NormalTok{(}\DataTypeTok{data =} \NormalTok{grid, }\DataTypeTok{colour =} \StringTok{"blue"}\NormalTok{, }\DataTypeTok{size =} \FloatTok{1.5}\NormalTok{) +}\StringTok{ }
\StringTok{  }\KeywordTok{geom_text}\NormalTok{(}\DataTypeTok{data =} \NormalTok{outlier, }\KeywordTok{aes}\NormalTok{(}\DataTypeTok{label =} \NormalTok{model))}
\end{Highlighting}
\end{Shaded}

\begin{figure}[H]
  \centering
  \includegraphics[width=0.65\linewidth]{_figures/layers/unnamed-chunk-2-1}
\end{figure}

(The labels aren't particularly easy to read, but you can fix that with
some manual tweaking.)

Note that you need the explicit \texttt{data\ =} in the layers, but not
in the call to \texttt{ggplot()}. That's because the argument order is
different. This is a little inconsistent, but it reduces typing for the
common case where you specify the data once in \texttt{ggplot()} and
modify aesthetics in each layer.

In this example, every layer uses a different dataset. We could define
the same plot in another way, omitting the default dataset, and
specifying a dataset for each layer:

\begin{Shaded}
\begin{Highlighting}[]
\KeywordTok{ggplot}\NormalTok{(}\DataTypeTok{mapping =} \KeywordTok{aes}\NormalTok{(displ, hwy)) +}\StringTok{ }
\StringTok{  }\KeywordTok{geom_point}\NormalTok{(}\DataTypeTok{data =} \NormalTok{mpg) +}\StringTok{ }
\StringTok{  }\KeywordTok{geom_line}\NormalTok{(}\DataTypeTok{data =} \NormalTok{grid) +}\StringTok{ }
\StringTok{  }\KeywordTok{geom_text}\NormalTok{(}\DataTypeTok{data =} \NormalTok{outlier, }\KeywordTok{aes}\NormalTok{(}\DataTypeTok{label =} \NormalTok{model))}
\end{Highlighting}
\end{Shaded}

I don't particularly like this style in this example because it makes it
less clear what the primary dataset is (and because of the way that the
arguments to \texttt{ggplot()} are ordered, it actually requires more
keystrokes). However, you may prefer it in cases where there isn't a
clear primary dataset, or where the aesthetics also vary from layer to
layer.

\subsection{Exercises}

\begin{enumerate}
\def\labelenumi{\arabic{enumi}.}
\item
  The first two arguments to ggplot are \texttt{data} and
  \texttt{mapping}. The first two arguments to all layer functions are
  \texttt{mapping} and \texttt{data}. Why does the order of the
  arguments differ? (Hint: think about what you set most commonly.)
\item
  The following code uses dplyr to generate some summary statistics
  about each class of car (you'll learn how it works in
  \hyperref[cha:dplyr]{data transformation}).

\begin{Shaded}
\begin{Highlighting}[]
\KeywordTok{library}\NormalTok{(dplyr)}
\NormalTok{class <-}\StringTok{ }\NormalTok{mpg %>%}\StringTok{ }
\StringTok{  }\KeywordTok{group_by}\NormalTok{(class) %>%}\StringTok{ }
\StringTok{  }\KeywordTok{summarise}\NormalTok{(}\DataTypeTok{n =} \KeywordTok{n}\NormalTok{(), }\DataTypeTok{hwy =} \KeywordTok{mean}\NormalTok{(hwy))}
\end{Highlighting}
\end{Shaded}

  Use the data to recreate this plot:

  \begin{figure}[H]
    \centering
    \includegraphics[width=0.65\linewidth]{_figures/layers/unnamed-chunk-5-1}
  \end{figure}
\end{enumerate}

\hyperdef{}{sec:aes}{\section{Aesthetic mappings}\label{sec:aes}}

The aesthetic mappings, defined with \texttt{aes()}, describe how
variables are mapped to visual properties or \textbf{aesthetics}.
\texttt{aes()} takes a sequence of aesthetic-variable pairs like this:
\index{Aesthetics!mapping} \indexf{aes}

\begin{Shaded}
\begin{Highlighting}[]
\KeywordTok{aes}\NormalTok{(}\DataTypeTok{x =} \NormalTok{displ, }\DataTypeTok{y =} \NormalTok{hwy, }\DataTypeTok{colour =} \NormalTok{class)}
\end{Highlighting}
\end{Shaded}

(If you're American, you can use \emph{color}, and behind the scenes
ggplot2 will correct your spelling ;)

Here we map x-position to \texttt{displ}, y-position to \texttt{hwy},
and colour to \texttt{class}. The names for the first two arguments can
be omitted, in which case they correspond to the x and y variables. That
makes this specification equivalent to the one above:

\begin{Shaded}
\begin{Highlighting}[]
\KeywordTok{aes}\NormalTok{(displ, hwy, }\DataTypeTok{colour =} \NormalTok{class)}
\end{Highlighting}
\end{Shaded}

While you can do data manipulation in \texttt{aes()}, e.g.
\texttt{aes(log(carat),\ log(price))}, it's best to only do simple
calculations. It's better to move complex transformations out of the
\texttt{aes()} call and into an explicit \texttt{dplyr::mutate()} call,
as you'll learn about in \hyperref[mutate]{mutate}. This makes it easier
to check your work and it's often faster because you need only do the
transformation once, not every time the plot is drawn.

Never refer to a variable with \texttt{\$} (e.g.,
\texttt{diamonds\$carat}) in \texttt{aes()}. This breaks containment, so
that the plot no longer contains everything it needs, and causes
problems if ggplot2 changes the order of the rows, as it does when
facetting. \indexc{\$}

\subsection{Specifying the aesthetics in the plot vs.~in the
layers}\label{sub:plots-and-layers}

Aesthetic mappings can be supplied in the initial \texttt{ggplot()}
call, in individual layers, or in some combination of both. All of these
calls create the same plot specification:
\index{Aesthetics!plot vs. layer}

\begin{Shaded}
\begin{Highlighting}[]
\KeywordTok{ggplot}\NormalTok{(mpg, }\KeywordTok{aes}\NormalTok{(displ, hwy, }\DataTypeTok{colour =} \NormalTok{class)) +}\StringTok{ }
\StringTok{  }\KeywordTok{geom_point}\NormalTok{()}
\KeywordTok{ggplot}\NormalTok{(mpg, }\KeywordTok{aes}\NormalTok{(displ, hwy)) +}\StringTok{ }
\StringTok{  }\KeywordTok{geom_point}\NormalTok{(}\KeywordTok{aes}\NormalTok{(}\DataTypeTok{colour =} \NormalTok{class))}
\KeywordTok{ggplot}\NormalTok{(mpg, }\KeywordTok{aes}\NormalTok{(displ)) +}\StringTok{ }
\StringTok{  }\KeywordTok{geom_point}\NormalTok{(}\KeywordTok{aes}\NormalTok{(}\DataTypeTok{y =} \NormalTok{hwy, }\DataTypeTok{colour =} \NormalTok{class))}
\KeywordTok{ggplot}\NormalTok{(mpg) +}\StringTok{ }
\StringTok{  }\KeywordTok{geom_point}\NormalTok{(}\KeywordTok{aes}\NormalTok{(displ, hwy, }\DataTypeTok{colour =} \NormalTok{class))}
\end{Highlighting}
\end{Shaded}

Within each layer, you can add, override, or remove mappings:

\begin{longtable}[c]{@{}lll@{}}
\toprule
Operation & Layer aesthetics & Result\tabularnewline
\midrule
\endhead
Add & \texttt{aes(colour\ =\ cyl)} &
\texttt{aes(mpg,\ wt,\ colour\ =\ cyl)}\tabularnewline
Override & \texttt{aes(y\ =\ disp)} &
\texttt{aes(mpg,\ disp)}\tabularnewline
Remove & \texttt{aes(y\ =\ NULL)} & \texttt{aes(mpg)}\tabularnewline
\bottomrule
\end{longtable}

If you only have one layer in the plot, the way you specify aesthetics
doesn't make any difference. However, the distinction is important when
you start adding additional layers. These two plots are both valid and
interesting, but focus on quite different aspects of the data:

\begin{Shaded}
\begin{Highlighting}[]
\KeywordTok{ggplot}\NormalTok{(mpg, }\KeywordTok{aes}\NormalTok{(displ, hwy, }\DataTypeTok{colour =} \NormalTok{class)) +}\StringTok{ }
\StringTok{  }\KeywordTok{geom_point}\NormalTok{() +}\StringTok{ }
\StringTok{  }\KeywordTok{geom_smooth}\NormalTok{(}\DataTypeTok{method =} \StringTok{"lm"}\NormalTok{, }\DataTypeTok{se =} \OtherTok{FALSE}\NormalTok{) +}
\StringTok{  }\KeywordTok{theme}\NormalTok{(}\DataTypeTok{legend.position =} \StringTok{"none"}\NormalTok{)}

\KeywordTok{ggplot}\NormalTok{(mpg, }\KeywordTok{aes}\NormalTok{(displ, hwy)) +}\StringTok{ }
\StringTok{  }\KeywordTok{geom_point}\NormalTok{(}\KeywordTok{aes}\NormalTok{(}\DataTypeTok{colour =} \NormalTok{class)) +}\StringTok{ }
\StringTok{  }\KeywordTok{geom_smooth}\NormalTok{(}\DataTypeTok{method =} \StringTok{"lm"}\NormalTok{, }\DataTypeTok{se =} \OtherTok{FALSE}\NormalTok{) +}\StringTok{ }
\StringTok{  }\KeywordTok{theme}\NormalTok{(}\DataTypeTok{legend.position =} \StringTok{"none"}\NormalTok{)}
\end{Highlighting}
\end{Shaded}

\begin{figure}[H]
  \includegraphics[width=0.5\linewidth]{_figures/layers/unnamed-chunk-7-1}%
  \includegraphics[width=0.5\linewidth]{_figures/layers/unnamed-chunk-7-2}
\end{figure}

Generally, you want to set up the mappings to illuminate the structure
underlying the graphic and minimise typing. It may take some time before
the best approach is immediately obvious, so if you've iterated your way
to a complex graphic, it may be worthwhile to rewrite it to make the
structure more clear.

\hyperdef{}{sub:setting-mapping}{\subsection{Setting
vs.~mapping}\label{sub:setting-mapping}}

Instead of mapping an aesthetic property to a variable, you can set it
to a \emph{single} value by specifying it in the layer parameters. We
\textbf{map} an aesthetic to a variable (e.g.,
\texttt{aes(colour\ =\ cut)}) or \textbf{set} it to a constant (e.g.,
\texttt{colour\ =\ "red"}). If you want appearance to be governed by a
variable, put the specification inside \texttt{aes()}; if you want
override the default size or colour, put the value outside of
\texttt{aes()}. \index{Aesthetics!setting}

The following plots are created with similar code, but have rather
different outputs. The second plot \textbf{maps} (not sets) the colour
to the value `darkblue'. This effectively creates a new variable
containing only the value `darkblue' and then scales it with a colour
scale. Because this value is discrete, the default colour scale uses
evenly spaced colours on the colour wheel, and since there is only one
value this colour is pinkish.

\begin{Shaded}
\begin{Highlighting}[]
\KeywordTok{ggplot}\NormalTok{(mpg, }\KeywordTok{aes}\NormalTok{(cty, hwy)) +}\StringTok{ }
\StringTok{  }\KeywordTok{geom_point}\NormalTok{(}\DataTypeTok{colour =} \StringTok{"darkblue"}\NormalTok{) }

\KeywordTok{ggplot}\NormalTok{(mpg, }\KeywordTok{aes}\NormalTok{(cty, hwy)) +}\StringTok{ }
\StringTok{  }\KeywordTok{geom_point}\NormalTok{(}\KeywordTok{aes}\NormalTok{(}\DataTypeTok{colour =} \StringTok{"darkblue"}\NormalTok{))}
\end{Highlighting}
\end{Shaded}

\begin{figure}[H]
  \includegraphics[width=0.5\linewidth]{_figures/layers/layer15-1}%
  \includegraphics[width=0.5\linewidth]{_figures/layers/layer15-2}
\end{figure}

A third approach is to map the value, but override the default scale:

\begin{Shaded}
\begin{Highlighting}[]
\KeywordTok{ggplot}\NormalTok{(mpg, }\KeywordTok{aes}\NormalTok{(cty, hwy)) +}\StringTok{ }
\StringTok{  }\KeywordTok{geom_point}\NormalTok{(}\KeywordTok{aes}\NormalTok{(}\DataTypeTok{colour =} \StringTok{"darkblue"}\NormalTok{)) +}\StringTok{ }
\StringTok{  }\KeywordTok{scale_colour_identity}\NormalTok{()}
\end{Highlighting}
\end{Shaded}

\begin{figure}[H]
  \centering
  \includegraphics[width=0.5\linewidth]{_figures/layers/unnamed-chunk-8-1}
\end{figure}

This is most useful if you always have a column that already contains
colours. You'll learn more about that in
\hyperref[sub:scale-identity]{the identity scale}.

It's sometimes useful to map aesthetics to constants. For example, if
you want to display multiple layers with varying parameters, you can
``name'' each layer:

\begin{Shaded}
\begin{Highlighting}[]
\KeywordTok{ggplot}\NormalTok{(mpg, }\KeywordTok{aes}\NormalTok{(displ, hwy)) +}\StringTok{ }
\StringTok{  }\KeywordTok{geom_point}\NormalTok{() +}
\StringTok{  }\KeywordTok{geom_smooth}\NormalTok{(}\KeywordTok{aes}\NormalTok{(}\DataTypeTok{colour =} \StringTok{"loess"}\NormalTok{), }\DataTypeTok{method =} \StringTok{"loess"}\NormalTok{, }\DataTypeTok{se =} \OtherTok{FALSE}\NormalTok{) +}\StringTok{ }
\StringTok{  }\KeywordTok{geom_smooth}\NormalTok{(}\KeywordTok{aes}\NormalTok{(}\DataTypeTok{colour =} \StringTok{"lm"}\NormalTok{), }\DataTypeTok{method =} \StringTok{"lm"}\NormalTok{, }\DataTypeTok{se =} \OtherTok{FALSE}\NormalTok{) +}
\StringTok{  }\KeywordTok{labs}\NormalTok{(}\DataTypeTok{colour =} \StringTok{"Method"}\NormalTok{)}
\end{Highlighting}
\end{Shaded}

\begin{figure}[H]
  \centering
  \includegraphics[width=0.65\linewidth]{_figures/layers/unnamed-chunk-9-1}
\end{figure}

\subsection{Exercises}

\begin{enumerate}
\def\labelenumi{\arabic{enumi}.}
\item
  Simplify the following plot specifications:

\begin{Shaded}
\begin{Highlighting}[]
\KeywordTok{ggplot}\NormalTok{(mpg) +}\StringTok{ }
\StringTok{  }\KeywordTok{geom_point}\NormalTok{(}\KeywordTok{aes}\NormalTok{(mpg$disp, mpg$hwy))}

\KeywordTok{ggplot}\NormalTok{() +}\StringTok{ }
\StringTok{ }\KeywordTok{geom_point}\NormalTok{(}\DataTypeTok{mapping =} \KeywordTok{aes}\NormalTok{(}\DataTypeTok{y =} \NormalTok{hwy, }\DataTypeTok{x =} \NormalTok{cty), }\DataTypeTok{data =} \NormalTok{mpg) +}
\StringTok{ }\KeywordTok{geom_smooth}\NormalTok{(}\DataTypeTok{data =} \NormalTok{mpg, }\DataTypeTok{mapping =} \KeywordTok{aes}\NormalTok{(cty, hwy))}

\KeywordTok{ggplot}\NormalTok{(diamonds, }\KeywordTok{aes}\NormalTok{(carat, price)) +}\StringTok{ }
\StringTok{  }\KeywordTok{geom_point}\NormalTok{(}\KeywordTok{aes}\NormalTok{(}\KeywordTok{log}\NormalTok{(brainwt), }\KeywordTok{log}\NormalTok{(bodywt)), }\DataTypeTok{data =} \NormalTok{msleep)}
\end{Highlighting}
\end{Shaded}
\item
  What does the following code do? Does it work? Does it make sense?
  Why/why not?

\begin{Shaded}
\begin{Highlighting}[]
\KeywordTok{ggplot}\NormalTok{(mpg) +}
\StringTok{  }\KeywordTok{geom_point}\NormalTok{(}\KeywordTok{aes}\NormalTok{(class, cty)) +}\StringTok{ }
\StringTok{  }\KeywordTok{geom_boxplot}\NormalTok{(}\KeywordTok{aes}\NormalTok{(trans, hwy))}
\end{Highlighting}
\end{Shaded}
\item
  What happens if you try to use a continuous variable on the x axis in
  one layer, and a categorical variable in another layer? What happens
  if you do it in the opposite order?
\end{enumerate}

\section{Geoms}\label{sec:geom}

Geometric objects, or \textbf{geoms} for short, perform the actual
rendering of the layer, controlling the type of plot that you create.
For example, using a point geom will create a scatterplot, while using a
line geom will create a line plot.

\begin{itemize}
\tightlist
\item
  Graphical primitives:

  \begin{itemize}
  \tightlist
  \item
    \texttt{geom\_blank()}: display nothing. Most useful for adjusting
    axes limits using data.
  \item
    \texttt{geom\_point()}: points.
  \item
    \texttt{geom\_path()}: paths.
  \item
    \texttt{geom\_ribbon()}: ribbons, a path with vertical thickness.
  \item
    \texttt{geom\_segment()}: a line segment, specified by start and end
    position.
  \item
    \texttt{geom\_rect()}: rectangles.
  \item
    \texttt{geom\_polyon()}: filled polygons.
  \item
    \texttt{geom\_text()}: text.
  \end{itemize}
\item
  One variable:

  \begin{itemize}
  \tightlist
  \item
    Discrete:

    \begin{itemize}
    \tightlist
    \item
      \texttt{geom\_bar()}: display distribution of discrete variable.
    \end{itemize}
  \item
    Continuous

    \begin{itemize}
    \tightlist
    \item
      \texttt{geom\_histogram()}: bin and count continuous variable,
      display with bars.
    \item
      \texttt{geom\_density()}: smoothed density estimate.
    \item
      \texttt{geom\_dotplot()}: stack individual points into a dot plot.
    \item
      \texttt{geom\_freqpoly()}: bin and count continuous variable,
      display with lines.
    \end{itemize}
  \end{itemize}
\item
  Two variables:

  \begin{itemize}
  \tightlist
  \item
    Both continuous:

    \begin{itemize}
    \tightlist
    \item
      \texttt{geom\_point()}: scatterplot.
    \item
      \texttt{geom\_quantile()}: smoothed quantile regression.
    \item
      \texttt{geom\_rug()}: marginal rug plots.
    \item
      \texttt{geom\_smooth()}: smoothed line of best fit.
    \item
      \texttt{geom\_text()}: text labels.
    \end{itemize}
  \item
    Show distribution:

    \begin{itemize}
    \tightlist
    \item
      \texttt{geom\_bin2d()}: bin into rectangles and count.
    \item
      \texttt{geom\_density2d()}: smoothed 2d density estimate.
    \item
      \texttt{geom\_hex()}: bin into hexagons and count.
    \end{itemize}
  \item
    At least one discrete:

    \begin{itemize}
    \tightlist
    \item
      \texttt{geom\_count()}: count number of point at distinct
      locations
    \item
      \texttt{geom\_jitter()}: randomly jitter overlapping points.
    \end{itemize}
  \item
    One continuous, one discrete:

    \begin{itemize}
    \tightlist
    \item
      \texttt{geom\_bar(stat\ =\ "identity")}: a bar chart of
      precomputed summaries.
    \item
      \texttt{geom\_boxplot()}: boxplots.
    \item
      \texttt{geom\_violin()}: show density of values in each group.
    \end{itemize}
  \item
    One time, one continuous

    \begin{itemize}
    \tightlist
    \item
      \texttt{geom\_area()}: area plot.
    \item
      \texttt{geom\_line()}: line plot.
    \item
      \texttt{geom\_step()}: step plot.
    \end{itemize}
  \item
    Display uncertainty:

    \begin{itemize}
    \tightlist
    \item
      \texttt{geom\_crossbar()}: vertical bar with center.
    \item
      \texttt{geom\_errorbar()}: error bars.
    \item
      \texttt{geom\_linerange()}: vertical line.
    \item
      \texttt{geom\_pointrange()}: vertical line with center.
    \end{itemize}
  \item
    Spatial

    \begin{itemize}
    \tightlist
    \item
      \texttt{geom\_map()}: fast version of \texttt{geom\_polygon()} for
      map data.
    \end{itemize}
  \end{itemize}
\item
  Three variables:

  \begin{itemize}
  \tightlist
  \item
    \texttt{geom\_contour()}: contours.
  \item
    \texttt{geom\_tile()}: tile the plane with rectangles.
  \item
    \texttt{geom\_raster()}: fast version of \texttt{geom\_tile()} for
    equal sized tiles.
  \end{itemize}
\end{itemize}

Each geom has a set of aesthetics that it understands, some of which
\emph{must} be provided. For example, the point geoms requires x and y
position, and understands colour, size and shape aesthetics. A bar
requires height (\texttt{ymax}), and understands width, border colour
and fill colour. Each geom lists its aesthetics in the documentation.

Some geoms differ primarily in the way that they are parameterised. For
example, you can draw a square in three ways:
\index{Geoms!parameterisation}

\begin{itemize}
\item
  By giving \texttt{geom\_tile()} the location (\texttt{x} and
  \texttt{y}) and dimensions (\texttt{width} and \texttt{height}).
  \indexf{geom\_tile}
\item
  By giving \texttt{geom\_rect()} top (\texttt{ymax}), bottom
  (\texttt{ymin}), left (\texttt{xmin}) and right (\texttt{xmax})
  positions. \indexf{geom\_rect}
\item
  By giving \texttt{geom\_polygon()} a four row data frame with the
  \texttt{x} and \texttt{y} positions of each corner.
\end{itemize}

Other related geoms are:

\begin{itemize}
\tightlist
\item
  \texttt{geom\_segment()} and \texttt{geom\_line()}
\item
  \texttt{geom\_area()} and \texttt{geom\_ribbon()}.
\end{itemize}

If alternative parameterisations are available, picking the right one
for your data will usually make it much easier to draw the plot you
want.

\subsection{Exercises}

\begin{enumerate}
\def\labelenumi{\arabic{enumi}.}
\item
  Download and print out the ggplot2 cheatsheet from
  \url{http://www.rstudio.com/resources/cheatsheets/} so you have a
  handy visual reference for all the geoms.
\item
  Look at the documentation for the graphical primitive geoms. Which
  aesthetics do they use? How can you summarise them in a compact form?
\item
  What's the best way to master an unfamiliar geom? List three resources
  to help you get started.
\item
  For each of the plots below, identify the geom used to draw it.

  \begin{figure}[H]
    \includegraphics[width=0.5\linewidth]{_figures/layers/unnamed-chunk-12-1}%
    \includegraphics[width=0.5\linewidth]{_figures/layers/unnamed-chunk-12-2}
  \end{figure}

  \begin{figure}[H]
    \includegraphics[width=0.5\linewidth]{_figures/layers/unnamed-chunk-13-1}%
    \includegraphics[width=0.5\linewidth]{_figures/layers/unnamed-chunk-13-2}
  \end{figure}

  \begin{figure}[H]
    \includegraphics[width=0.5\linewidth]{_figures/layers/unnamed-chunk-14-1}%
    \includegraphics[width=0.5\linewidth]{_figures/layers/unnamed-chunk-14-2}
  \end{figure}
\item
  For each of the following problems, suggest a useful geom:

  \begin{itemize}
  \tightlist
  \item
    Display how a variable has changed over time.
  \item
    Show the detailed distribution of a single variable.
  \item
    Focus attention on the overall trend in a large dataset.
  \item
    Draw a map.
  \item
    Label outlying points.
  \end{itemize}
\end{enumerate}

\hyperdef{}{sec:stat}{\section{Stats}\label{sec:stat}}

A statistical transformation, or \textbf{stat}, transforms the data,
typically by summarising it in some manner. For example, a useful stat
is the smoother, which calculates the smoothed mean of y, conditional on
x. You've already used many of ggplot2's stats because they're used
behind the scenes to generate many important geoms:

\begin{itemize}
\tightlist
\item
  \texttt{stat\_bin()}: \texttt{geom\_bar()}, \texttt{geom\_freqpoly()},
  \texttt{geom\_histogram()}
\item
  \texttt{stat\_bin2d()}: \texttt{geom\_bin2d()}
\item
  \texttt{stat\_bindot()}: \texttt{geom\_dotplot()}
\item
  \texttt{stat\_binhex()}: \texttt{geom\_hex()}
\item
  \texttt{stat\_boxplot()}: \texttt{geom\_boxplot()}
\item
  \texttt{stat\_contour()}: \texttt{geom\_contour()}
\item
  \texttt{stat\_quantile()}: \texttt{geom\_quantile()}
\item
  \texttt{stat\_smooth()}: \texttt{geom\_smooth()}
\item
  \texttt{stat\_sum()}: \texttt{geom\_count()}
\end{itemize}

You'll rarely call these functions directly, but they are useful to know
about because their documentation often provides more detail about the
corresponding statistical transformation.

Other stats can't be created with a \texttt{geom\_} function:

\begin{itemize}
\tightlist
\item
  \texttt{stat\_ecdf()}: compute a empirical cumulative distribution
  plot.
\item
  \texttt{stat\_function()}: compute y values from a function of x
  values.
\item
  \texttt{stat\_summary()}: summarise y values at distinct x values.
\item
  \texttt{stat\_summary2d()}, \texttt{stat\_summary\_hex()}: summarise
  binned values.
\item
  \texttt{stat\_qq()}: perform calculations for a quantile-quantile
  plot.
\item
  \texttt{stat\_spoke()}: convert angle and radius to position.
\item
  \texttt{stat\_unique()}: remove duplicated rows.
\end{itemize}

There are two ways to use these functions. You can either add a
\texttt{stat\_()} function and override the default geom, or add a
\texttt{geom\_()} function and override the default stat:

\begin{Shaded}
\begin{Highlighting}[]
\KeywordTok{ggplot}\NormalTok{(mpg, }\KeywordTok{aes}\NormalTok{(trans, cty)) +}\StringTok{ }
\StringTok{  }\KeywordTok{geom_point}\NormalTok{() +}\StringTok{ }
\StringTok{  }\KeywordTok{stat_summary}\NormalTok{(}\DataTypeTok{geom =} \StringTok{"point"}\NormalTok{, }\DataTypeTok{fun.y =} \StringTok{"mean"}\NormalTok{, }\DataTypeTok{colour =} \StringTok{"red"}\NormalTok{, }\DataTypeTok{size =} \DecValTok{4}\NormalTok{)}

\KeywordTok{ggplot}\NormalTok{(mpg, }\KeywordTok{aes}\NormalTok{(trans, cty)) +}\StringTok{ }
\StringTok{  }\KeywordTok{geom_point}\NormalTok{() +}\StringTok{ }
\StringTok{  }\KeywordTok{geom_point}\NormalTok{(}\DataTypeTok{stat =} \StringTok{"summary"}\NormalTok{, }\DataTypeTok{fun.y =} \StringTok{"mean"}\NormalTok{, }\DataTypeTok{colour =} \StringTok{"red"}\NormalTok{, }\DataTypeTok{size =} \DecValTok{4}\NormalTok{)}
\end{Highlighting}
\end{Shaded}

\begin{figure}[H]
  \centering
  \includegraphics[width=0.75\linewidth]{_figures/layers/unnamed-chunk-15-1}
\end{figure}

I think it's best to use the second form because it makes it more clear
that you're displaying a summary, not the raw data.

\subsection{Generated variables}

Internally, a stat takes a data frame as input and returns a data frame
as output, and so a stat can add new variables to the original dataset.
It is possible to map aesthetics to these new variables. For example,
\texttt{stat\_bin}, the statistic used to make histograms, produces the
following variables: \index{Stats!creating new variables}
\indexf{stat\_bin}

\begin{itemize}
\tightlist
\item
  \texttt{count}, the number of observations in each bin
\item
  \texttt{density}, the density of observations in each bin (percentage
  of total / bar width)
\item
  \texttt{x}, the centre of the bin
\end{itemize}

These generated variables can be used instead of the variables present
in the original dataset. For example, the default histogram geom assigns
the height of the bars to the number of observations (\texttt{count}),
but if you'd prefer a more traditional histogram, you can use the
density (\texttt{density}). To refer to a generated variable like
density, ``\texttt{..}'' must surround the name. This prevents confusion
in case the original dataset includes a variable with the same name as a
generated variable, and it makes it clear to any later reader of the
code that this variable was generated by a stat. Each statistic lists
the variables that it creates in its documentation. \indexc{..} Compare
the y-axes on these two plots:

\begin{Shaded}
\begin{Highlighting}[]
\KeywordTok{ggplot}\NormalTok{(diamonds, }\KeywordTok{aes}\NormalTok{(price)) +}\StringTok{ }
\StringTok{  }\KeywordTok{geom_histogram}\NormalTok{(}\DataTypeTok{binwidth =} \DecValTok{500}\NormalTok{)}
\KeywordTok{ggplot}\NormalTok{(diamonds, }\KeywordTok{aes}\NormalTok{(price)) +}\StringTok{ }
\StringTok{  }\KeywordTok{geom_histogram}\NormalTok{(}\KeywordTok{aes}\NormalTok{(}\DataTypeTok{y =} \NormalTok{..density..), }\DataTypeTok{binwidth =} \DecValTok{500}\NormalTok{)}
\end{Highlighting}
\end{Shaded}

\begin{figure}[H]
  \includegraphics[width=0.5\linewidth]{_figures/layers/hist-1}%
  \includegraphics[width=0.5\linewidth]{_figures/layers/hist-2}
\end{figure}

This technique is particularly useful when you want to compare the
distribution of multiple groups that have very different sizes. For
example, it's hard to compare the distribution of \texttt{price} within
\texttt{cut} because some groups are quite small. It's easier to compare
if we standardise each group to take up the same area:

\begin{Shaded}
\begin{Highlighting}[]
\KeywordTok{ggplot}\NormalTok{(diamonds, }\KeywordTok{aes}\NormalTok{(price, }\DataTypeTok{colour =} \NormalTok{cut)) +}\StringTok{ }
\StringTok{  }\KeywordTok{geom_freqpoly}\NormalTok{(}\DataTypeTok{binwidth =} \DecValTok{500}\NormalTok{) +}
\StringTok{  }\KeywordTok{theme}\NormalTok{(}\DataTypeTok{legend.position =} \StringTok{"none"}\NormalTok{)}

\KeywordTok{ggplot}\NormalTok{(diamonds, }\KeywordTok{aes}\NormalTok{(price, }\DataTypeTok{colour =} \NormalTok{cut)) +}\StringTok{ }
\StringTok{  }\KeywordTok{geom_freqpoly}\NormalTok{(}\KeywordTok{aes}\NormalTok{(}\DataTypeTok{y =} \NormalTok{..density..), }\DataTypeTok{binwidth =} \DecValTok{500}\NormalTok{) +}\StringTok{ }
\StringTok{  }\KeywordTok{theme}\NormalTok{(}\DataTypeTok{legend.position =} \StringTok{"none"}\NormalTok{)}
\end{Highlighting}
\end{Shaded}

\begin{figure}[H]
  \includegraphics[width=0.5\linewidth]{_figures/layers/freqpoly-1}%
  \includegraphics[width=0.5\linewidth]{_figures/layers/freqpoly-2}
\end{figure}

The result of this plot is rather surprising: low quality diamonds seem
to be more expensive on average. We'll come back to this result in
\hyperref[sub:trend]{removing trend}.

\subsection{Exercises}

\begin{enumerate}
\def\labelenumi{\arabic{enumi}.}
\item
  The code below creates a similar dataset to \texttt{stat\_smooth()}.
  Use the appropriate geoms to mimic the default \texttt{geom\_smooth()}
  display.

\begin{Shaded}
\begin{Highlighting}[]
\NormalTok{mod <-}\StringTok{ }\KeywordTok{loess}\NormalTok{(hwy ~}\StringTok{ }\NormalTok{displ, }\DataTypeTok{data =} \NormalTok{mpg)}
\NormalTok{smoothed <-}\StringTok{ }\KeywordTok{data.frame}\NormalTok{(}\DataTypeTok{displ =} \KeywordTok{seq}\NormalTok{(}\FloatTok{1.6}\NormalTok{, }\DecValTok{7}\NormalTok{, }\DataTypeTok{length =} \DecValTok{50}\NormalTok{))}
\NormalTok{pred <-}\StringTok{ }\KeywordTok{predict}\NormalTok{(mod, }\DataTypeTok{newdata =} \NormalTok{smoothed, }\DataTypeTok{se =} \OtherTok{TRUE}\NormalTok{) }
\NormalTok{smoothed$hwy <-}\StringTok{ }\NormalTok{pred$fit}
\NormalTok{smoothed$hwy_lwr <-}\StringTok{ }\NormalTok{pred$fit -}\StringTok{ }\FloatTok{1.96} \NormalTok{*}\StringTok{ }\NormalTok{pred$se.fit}
\NormalTok{smoothed$hwy_upr <-}\StringTok{ }\NormalTok{pred$fit +}\StringTok{ }\FloatTok{1.96} \NormalTok{*}\StringTok{ }\NormalTok{pred$se.fit}
\end{Highlighting}
\end{Shaded}
\item
  What stats were used to create the following plots?

  \begin{figure}[H]
    \includegraphics[width=0.333\linewidth]{_figures/layers/unnamed-chunk-17-1}%
    \includegraphics[width=0.333\linewidth]{_figures/layers/unnamed-chunk-17-2}%
    \includegraphics[width=0.333\linewidth]{_figures/layers/unnamed-chunk-17-3}
  \end{figure}
\item
  Read the help for \texttt{stat\_sum()} then use \texttt{geom\_count()}
  to create a plot that shows the proportion of cars that have each
  combination of \texttt{drv} and \texttt{trans}.
\end{enumerate}

\hyperdef{}{sec:position}{\section{Position
adjustments}\label{sec:position}}

\index{Position adjustments}

Position adjustments apply minor tweaks to the position of elements
within a layer. Three adjustments apply primarily to bars:

\index{Dodging} \index{Side-by-side|see{Dodging}}
\indexf{position\_dodge} \index{Stacking} \indexf{position\_stack}
\indexf{position\_fill}

\begin{itemize}
\tightlist
\item
  \texttt{position\_stack()}: stack overlapping bars (or areas) on top
  of each other.
\item
  \texttt{position\_fill()}: stack overlapping bars, scaling so the top
  is always at 1.
\item
  \texttt{position\_dodge()}: place overlapping bars (or boxplots)
  side-by-side.
\end{itemize}

\begin{Shaded}
\begin{Highlighting}[]
\NormalTok{dplot <-}\StringTok{ }\KeywordTok{ggplot}\NormalTok{(diamonds, }\KeywordTok{aes}\NormalTok{(color, }\DataTypeTok{fill =} \NormalTok{cut)) +}\StringTok{ }
\StringTok{  }\KeywordTok{xlab}\NormalTok{(}\OtherTok{NULL}\NormalTok{) +}\StringTok{ }\KeywordTok{ylab}\NormalTok{(}\OtherTok{NULL}\NormalTok{) +}\StringTok{ }\KeywordTok{theme}\NormalTok{(}\DataTypeTok{legend.position =} \StringTok{"none"}\NormalTok{)}
\CommentTok{# position stack is the default for bars, so `geom_bar()` }
\CommentTok{# is equivalent to `geom_bar(position = "stack")`.}
\NormalTok{dplot +}\StringTok{ }\KeywordTok{geom_bar}\NormalTok{()}
\NormalTok{dplot +}\StringTok{ }\KeywordTok{geom_bar}\NormalTok{(}\DataTypeTok{position =} \StringTok{"fill"}\NormalTok{)}
\NormalTok{dplot +}\StringTok{ }\KeywordTok{geom_bar}\NormalTok{(}\DataTypeTok{position =} \StringTok{"dodge"}\NormalTok{)}
\end{Highlighting}
\end{Shaded}

\begin{figure}[H]
  \includegraphics[width=0.333\linewidth]{_figures/layers/position-bar-1}%
  \includegraphics[width=0.333\linewidth]{_figures/layers/position-bar-2}%
  \includegraphics[width=0.333\linewidth]{_figures/layers/position-bar-3}
\end{figure}

There's also a position adjustment that does nothing:
\texttt{position\_identity()}. The identity position adjustment is not
useful for bars, because each bar obscures the bars behind, but there
are many geoms that don't need adjusting, like the frequency polygon:

\begin{Shaded}
\begin{Highlighting}[]
\NormalTok{dplot +}\StringTok{ }\KeywordTok{geom_bar}\NormalTok{(}\DataTypeTok{position =} \StringTok{"identity"}\NormalTok{, }\DataTypeTok{alpha =} \DecValTok{1} \NormalTok{/}\StringTok{ }\DecValTok{2}\NormalTok{, }\DataTypeTok{colour =} \StringTok{"grey50"}\NormalTok{)}

\KeywordTok{ggplot}\NormalTok{(diamonds, }\KeywordTok{aes}\NormalTok{(color, }\DataTypeTok{colour =} \NormalTok{cut)) +}\StringTok{ }
\StringTok{  }\KeywordTok{geom_freqpoly}\NormalTok{(}\KeywordTok{aes}\NormalTok{(}\DataTypeTok{group =} \NormalTok{cut), }\DataTypeTok{stat =} \StringTok{"count"}\NormalTok{) +}\StringTok{ }
\StringTok{  }\KeywordTok{xlab}\NormalTok{(}\OtherTok{NULL}\NormalTok{) +}\StringTok{ }\KeywordTok{ylab}\NormalTok{(}\OtherTok{NULL}\NormalTok{) +}\StringTok{ }
\StringTok{  }\KeywordTok{theme}\NormalTok{(}\DataTypeTok{legend.position =} \StringTok{"none"}\NormalTok{)}
\end{Highlighting}
\end{Shaded}

\begin{figure}[H]
  \includegraphics[width=0.5\linewidth]{_figures/layers/position-identity-1}%
  \includegraphics[width=0.5\linewidth]{_figures/layers/position-identity-2}
\end{figure}

There are three position adjustments that are primarily useful for
points:

\begin{itemize}
\tightlist
\item
  \texttt{position\_nudge()}: move points by a fixed offset.
\item
  \texttt{position\_jitter()}: add a little random noise to every
  position.
\item
  \texttt{position\_jitterdodge()}: dodge points within groups, then add
  a little random noise.
\end{itemize}

\indexf{position\_nudge} \indexf{position\_jitter}
\indexf{position\_jitterdodge}

Note that the way you pass parameters to position adjustments differs to
stats and geoms. Instead of including additional arguments in
\texttt{...}, you construct a position adjustment object, supplying
additional arguments in the call:

\begin{Shaded}
\begin{Highlighting}[]
\KeywordTok{ggplot}\NormalTok{(mpg, }\KeywordTok{aes}\NormalTok{(displ, hwy)) +}\StringTok{ }
\StringTok{  }\KeywordTok{geom_point}\NormalTok{(}\DataTypeTok{position =} \StringTok{"jitter"}\NormalTok{)}
\KeywordTok{ggplot}\NormalTok{(mpg, }\KeywordTok{aes}\NormalTok{(displ, hwy)) +}\StringTok{ }
\StringTok{  }\KeywordTok{geom_point}\NormalTok{(}\DataTypeTok{position =} \KeywordTok{position_jitter}\NormalTok{(}\DataTypeTok{width =} \FloatTok{0.05}\NormalTok{, }\DataTypeTok{height =} \FloatTok{0.5}\NormalTok{))}
\end{Highlighting}
\end{Shaded}

\begin{figure}[H]
  \includegraphics[width=0.5\linewidth]{_figures/layers/position-point-1}%
  \includegraphics[width=0.5\linewidth]{_figures/layers/position-point-2}
\end{figure}

This is rather verbose, so \texttt{geom\_jitter()} provides a convenient
shortcut:

\begin{Shaded}
\begin{Highlighting}[]
\KeywordTok{ggplot}\NormalTok{(mpg, }\KeywordTok{aes}\NormalTok{(displ, hwy)) +}\StringTok{ }
\StringTok{  }\KeywordTok{geom_jitter}\NormalTok{(}\DataTypeTok{width =} \FloatTok{0.05}\NormalTok{, }\DataTypeTok{height =} \FloatTok{0.5}\NormalTok{)}
\end{Highlighting}
\end{Shaded}

Continuous data typically doesn't overlap exactly, and when it does
(because of high data density) minor adjustments, like jittering, are
often insufficient to fix the problem. For this reason, position
adjustments are generally most useful for discrete data.

\subsection{Exercises}

\begin{enumerate}
\def\labelenumi{\arabic{enumi}.}
\item
  When might you use \texttt{position\_nudge()}? Read the documentation.
\item
  Many position adjustments can only be used with a few geoms. For
  example, you can't stack boxplots or errors bars. Why not? What
  properties must a geom possess in order to be stackable? What
  properties must it possess to be dodgeable?
\item
  Why might you use \texttt{geom\_jitter()} instead of
  \texttt{geom\_count()}? What are the advantages and disadvantages of
  each technique?
\item
  When might you use a stacked area plot? What are the advantages and
  disadvantages compared to a line plot?
\end{enumerate}

\providecommand{\setflag}{\newif \ifwhole \wholefalse}
\setflag
\ifwhole\else
\documentclass[oneside,letterpaper]{scrbook}
\usepackage{fullpage}
\usepackage[utf8]{inputenc}
\usepackage[pdftex]{graphicx}
\usepackage{hyperref}
\usepackage{minitoc}
\usepackage{pdfsync}
\usepackage{alltt}
\usepackage[round,sort&compress,sectionbib]{natbib}
\bibliographystyle{plainnat}

%\setcounter{secnumdepth}{-1}

\title{ggplot}
\author{Hadley Wickham}

\renewcommand{\topfraction}{0.9}	% max fraction of floats at top
\renewcommand{\bottomfraction}{0.8}	% max fraction of floats at bottom
%   Parameters for TEXT pages (not float pages):
\setcounter{topnumber}{2}
\setcounter{bottomnumber}{2}
\renewcommand{\dbltopfraction}{0.9}	% fit big float above 2-col. text
\renewcommand{\textfraction}{0.07}	% allow minimal text w. figs
%   Parameters for FLOAT pages (not text pages):
\renewcommand{\floatpagefraction}{0.7}	% require fuller float pages
% N.B.: floatpagefraction MUST be less than topfraction !!
\renewcommand{\dblfloatpagefraction}{0.7}	% require fuller float pages

\newcommand{\grobref}[1]{{\tt #1} (page \pageref{sub:#1})}
\newcommand{\secref}[1]{\ref{#1} (\pageref{#1})}


\raggedbottom

\begin{document}
\fi

\setchapterpreamble[u]{% 
\dictum[Anonymous]{Forecasting is the art of saying 
what is going to happen and then to explain 
why it didn’t.}} 

\chapter{Scales}

\section{Introduction}\label{sec:introduction}

Scales control the mapping between data space and aesthetic space.  They convert your data values into aesthetic attributes that can be a perceived: colour, shape, size, etc.  Each type of aesthetic attribute has a default scale, and may have other scales that provide different types of mappings.  For example, the default colour scale uses equally space hues, but other scales allow you to generate a gradient between two or three different colours.  This section describes the basic operation of a scale, the details of all the different scales and instructions on how to make your own.

\section{Legend and axis labels}\label{sec:legend_and_axis_labels}

Each scale function has a ``name'' argument, usually the first argument after the plot name.  This determines the label which will appear on the axis or legend. You can supply text strings (using ``\\n'' for line breaks) or mathematical expressions (as described in \verb|plotmath|):

% decumar<<< 
% interweave({
% qplot(tip, total_bill, data=tips, colour=tip/total_bil)
% sccolour(p, "Tip rate")
% sccolour(p, "This is a very long label\nsplit over two lines")
% sccolour(p, expression(beta * x^2))
% })
% |||
% >>>

\verb|qplot| provides some additional shortcuts for axis labels, see \secref{sec:other_options} for details.

\section{Aesthetic attributes}\label{sec:aesthetic_attributes}

The following aesthetic attributes are available in {\tt ggplot}.  This list is not exhaustive, other authors may add grob functions that use other mappings.  

Any aesthetic that a grob does not understand will be silently ignored.  If a scale is not doing what you expect, make sure to check the you don't have any spelling mistakes, as these will not raise an error message.

Position scales
Other scales
 * categorical
 * continuous

\begin{itemize}
	\item colour
	\item fill
	\item group
	\item id
	\item linetype
	\item shape
	\item size
	\item dimension: height, width
	\item position: x, y, z
\end{itemize}

There are also a couple of aesthetic attributes that can't be perceived directly.  The most important of these is the {\tt id} aesthetic, which divides the the data set into discrete components.   This is used in line and path plots to separate the data for different lines, and in the groups grob to divide the different groups. 

Scales may take multiple inputs and return multiple outputs.

\section{Using scales}\label{sec:using_scales}

Scales are added by default whenever you use {\tt qplot} or {\tt ggplot}.  If you want to modify them, you need to manually add a scale to the plot.  This will automatically override any defaults.

% decumar<<< 
% interweave({
% quakes$depthc <- chop(quakes$depth)
% p <- qplot(long, lat, . ~ depthc, data=quakes)
% str(p$scales)
% p <- pscontinuous(p, "x", breaks=c(5,10))
% str(p$scales)
% })
% |||
% >>>

All position scales start with {\tt sc}, all others with {\tt sc}.

\section{How to write your own scales}\label{sec:how_to_write_your_own_scales}


\ifwhole
\else
	\bibliography{bibliography}
  \end{document}
\fi

\providecommand{\setflag}{\newif \ifwhole \wholefalse}
\setflag
\ifwhole\else

% Typography and geometry ----------------------------------------------------
\documentclass[letterpaper]{scrbook}
\usepackage[inner=3cm,top=2.5cm,outer=3.5cm]{geometry}

\renewcommand\familydefault{bch}
\usepackage[utf8]{inputenc}
\usepackage{microtype}
\usepackage[small]{caption}
\usepackage[small]{titlesec}
\raggedbottom

% Graphics -------------------------------------------------------------------
\usepackage[pdftex]{graphicx}
\graphicspath{{_include/}}
\DeclareGraphicsExtensions{.png,.pdf}

% Code formatting ------------------------------------------------------------
\usepackage{fancyvrb}
\usepackage{courier}
\usepackage{listings}
\usepackage{color}
\usepackage{alltt}


\definecolor{comment}{rgb}{0.60, 0.60, 0.53}
\definecolor{background}{rgb}{0.97, 0.97, 1.00}
\definecolor{string}{rgb}{0.863, 0.066, 0.266}
\definecolor{number}{rgb}{0.0, 0.6, 0.6}
\definecolor{variable}{rgb}{0.00, 0.52, 0.70}
\lstset{
  basicstyle=\ttfamily,
  keywordstyle=\bfseries, 
  identifierstyle=,  
  commentstyle=\color{comment} \emph,
  stringstyle=\color{string},
  showstringspaces=false,
  columns = fullflexible,
  backgroundcolor=\color{background},
  mathescape = true,
  escapeinside=&&,
  fancyvrb
}
\newcommand{\code}[1]{\lstinline!#1!}
\newcommand{\f}[1]{\lstinline!#1()!}



% Links ----------------------------------------------------------------------

\usepackage{hyperref}
\definecolor{slateblue}{rgb}{0.07,0.07,0.488}
\hypersetup{colorlinks=true,linkcolor=slateblue,anchorcolor=slateblue,citecolor=slateblue,filecolor=slateblue,urlcolor=slateblue,bookmarksnumbered=true,pdfview=FitB}
\usepackage{url}

% Tables ---------------------------------------------------------------------
\usepackage{longtable}
\usepackage{booktabs}

% Miscellaneous --------------------------------------------------------------
\usepackage{pdfsync}
\usepackage{appendix}

\usepackage[round,sort&compress,sectionbib]{natbib}
\bibliographystyle{plainnat}


\title{ggplot2}
\author{Hadley Wickham}

\begin{document}
\fi

\chapter{Positioning}
\label{cha:position}

\section{Introduction}

This chapter introduces ways of adjusting how plots are laid out on page, and how to manipulate the coordinate system within a plot.

Faceting is a mechanism for automatically laying out multiple plots on a plot.  It splits the data into subsets, and then plots each subset on a different place on the page.  This is only one small type of plot layout that you might want to do.  Section~\ref{sec:grid_layout} shows how to use grid to layout any number of plots in any way that you want.


\section{Position adjustments}
\label{sec:position}

Position adjustments apply minor tweaks to the position of elements within a layer.  Figure~\ref{fig:position} lists all of the position adjustments available with in ggplot.  

They are normally used with discrete data - continuous data typically doesn't overlap exactly, and when it does (because of high data density) minor adjustments are sufficient to fix the problem.

\begin{table}
  \begin{center}
  \begin{tabular}{ll}
    \toprule
    Name      & Description  \\
    \midrule
    dodge    & Adjust position by dodging overlaps to the side \\
    fill     & Stack overlapping objects on top of one another, and standardise have equal height\\
    identity & Don't adjust position \\
    jitter   & Jitter points to avoid overplotting \\
    stack    & Stack overlapping objects on top of one another \\
    \bottomrule
  \end{tabular}
  \end{center}
  \caption{caption}
  \label{fig:position}
\end{table}

Best illustrated with a bar chart.  Figure~\ref{fig:position-bar} shows stacking, filling and dodging.  Stacking puts bars on the same x on top of one-another; filling does the same, but normalises height to 1; and dodging places the bars side-by-side.  The identity adjustment (i.e.\ do nothing) doesn't make much sense for bars, but is shown in Figure~\ref{fig:position-identity} along with a line plot of the same data for reference.

Filling makes a conditional display - removes the marginal effect of column total. 

All geoms have a default position adjustment associated with them.  You can see by printing the geom.

\begin{alltt}
  > geom_point()
  ...
  position_identity: ()
  ...
  > geom_bar()
  ...
  position_stack: ()
  ...
\end{alltt}

\begin{figure}[htbp]
  \centering
    \includegraphics[width = 0.5 \linewidth]{position-stack}%
    \includegraphics[width = 0.5 \linewidth]{position-fill}
  \caption{Position adjustments for a bar chart}
  \label{fig:position-bar}
\end{figure}

\begin{figure}[htbp]
  \centering
    \includegraphics[width = 0.5 \linewidth]{position-dodge}
  \caption{Position adjustments for a bar chart}
  \label{fig:position-dodge}
\end{figure}

\begin{figure}[htbp]
  \centering
    \includegraphics[width = 0.5 \linewidth]{position-identity}%
    \includegraphics[width = 0.5 \linewidth]{position-identity2}
  \caption{The identity position adjustment}
  \label{fig:position-identity}
\end{figure}


\section{Coordinate systems}
\label{sec:coord}

99\% of the time we use a Cartesian coordinate system, where the 2d position of an element is made from the independent combination of the two.

Changing the coordinate system changes the appearance of the grob - a straight line maybe no longer be straight.  Illustrate with polar coordinates and coord\_trans.  

Similar to a scale, but inputs and outputs a tuple.

The only thing that is invariant under transformation is a point.  Assumption that coordinate transformations are smooth in some sense, so that very short lines will still be lines in the new coordinate system.

All coordinate systems are two dimensional.  Maybe 3d graphics one day.

Coordinate systems are the most complicated aspect of the grammar to get absolutely correct and many are still are work in progress.  However, do try them and out and report problems that you experience.


Coordinate systems are responsible for drawing the guides within the plot (i.e.\ grid lines) and 

\begin{table}
  \begin{center}
  \begin{tabular}{ll}
    \toprule
    Name      & Description  \\
    \midrule
    cartesian & Cartesian coordinates                  \\
    equal     & Equal scale cartesian coordinates      \\
    flip      & Flipped cartesian coordinates          \\
    map       & Map projections                        \\
    polar     & Polar coordinates                      \\
    trans     & Transformed cartesian coordinate system\\
    \bottomrule
    
  \end{tabular}
  \end{center}
  \caption{Coordinate systems available in ggplot}
  \label{tbl:coord}
\end{table}

As with the other components in ggplot2, you generate the R name by joining {\tt coord\_} and the name of the coordinate system.  For example, the default Cartesian coordinate system is {\tt coord\_cartesian}.

Writing your own coordinate system is described in Section~\ref{sec:my_coord}.


\section{Faceting}
\label{sec:faceting}

Faceting can be thought of a special type of coordinate system, which is hierarchical.  At the top level we have a coordinate system created by the categorical variables that we are faceting by, and then within each of these regions another coordinate system generated by the x and y position of each graphic.

Faceting is discussed previous in XXX and XXX.  Here I will go into more detail, and provide more examples.  When specifying a faceting formula, you specify a grid of row and column variables.  Variables appearing on a row or column together, is like nesting, only combinations that appear in the data will appear in the plot.  If variables that are specified on rows and columns are crossed: all combinations will be shown, including those that didn't appear in the original data set.

Used to generate small multiples of the same plot for different subsets of the data.  Small multiples are a powerful tool for ...  

\subsection{Facet grid}

You can also specify margins to display, by giving the names of the variables you want margins for.  

% decumar<<< 
% interweave({
% str(smiths)
% })
% |||
%>>>

Currently, {\tt ggplot} doesn't make any distinction between structural and non-structural missings in the facets - you can't tell whether a combination did not appear in the original dataset, or if it had no data.  Dealing with this properly may require some extensions to R's missing value system.  


\subsubsection{Margins}\label{sub:margins}

Faceting a plot is like creating a contingency table.  In contingency tables it is often useful to display marginal totals (totals over a row or column) as well as the individual cells.  It is also useful to be able to do this with graphics.  We can produce graphical margins using the the {\tt margins} argument.  This allows you to compare the conditional patterns with the marginal patterns.

You can either specify that all margins should be displayed, using {\tt margins = TRUE}, or by listing the names of the variables that you want margins for, {\tt margins = c("sex","age")}.  You can also use \verb|"grand_row"| or \verb|"grand_col"| to produce grand row and grand column margins respectively.

This example shows how the margins appear.  In the first plot, there are no margins, and we only see conditional plots.  In the second example, we see margins over columns, but not rows, and in the final example we see all possible margins.  The facet in the lower right corner displays all data points.

% decumar<<< 
% interweave({
% qplot(price, data=diamonds, facets= cut~ ., geom="histogram")
% })
% |||
\begin{alltt}
> qplot(price, data = diamonds, facets = cut ~ ., geom = "histogram")
\includegraphics[scale=0.5]{1a07c6d5903b310295255cfcfa587a9a}

\end{alltt}
% >>>

Plots with many facets and margins may be more appropriate for printing, rather than on screen display, as the higher resolution allows you to compare many more subsets.

\subsection{Facet wrap}
\label{sub:facet_wrap}

An alternative is facet wrap.  Instead of having a 2d grid generate by the combination of two (or more) variables, facet\_wrap essentially makes a long ribbon of panels and wraps it onto the screen.  

This is useful if you have a single variable that with many levels and want to arrange the plots in a more space efficient manner.

This is very similar to how trellising in lattice works.

\subsection{Controlling scales}
\label{sub:controlling_scales}

For both types of faceting you have some control over how the position scales work across plots.  There are two extremes.  One extreme is the default where all plots can share the same scale.  The other extreme is where each plot has it's own set of position scales.  

There also some intermediate levels of sharing:

\begin{itemize}
  \item shared
  \item individual
  \item rows-columns: each row has its own horizontal scale, and each column has its own vertical scale (facet\_grid only)
  \item variable: each plot has its own scale, but they are sizes so that the mapping between the plot and the data is the same in every panel (facet\_wrap only.)
\end{itemize}



\subsection{Missing faceting columns}\label{sub:missing_faceting_columsn}

If you facet on the original data set and add grobs with a new dataset, {\tt ggplot} follows some simple heuristics to try and do what you want.  

If the new dataset contains all of the faceting columns then each facet will have a specialised grob. If the new dataset contains none of the faceting columns, then each facet will use the same grob.  For anything in between, i.e. the new dataset contains some of the existing faceting columns, then the grobs will be duplicated over the faceting columns that are not present.  This sounds complicated, but most of the time it should do what you want.  If you need precise control, just make sure that every new dataset contains all the faceting columns used in the original dataset.

% decumar<<< 
% interweave({
% auckland <- data.frame(lat=-20 , lat=-25, long=175, label="Auck" )
% p <- ggpoint(ggplot(quakes, formula=. ~ depth, aes=list(x=lat, y=long)))
% ggtext(p, data=auckland, aes=list(label=label))
% })
% |||
% >>>  


\subsection{Grouping vs. faceting}

Faceting is an alternative to using aesthetics (like colour, shape or size) to distinguish different groups.  Each has strengths and weaknesses, centred around the distance between the groups.

With faceting, each group is quite far apart, in its own panel - there is no overlap between the groups.  This is good if the groups overlap a lot, but it does make small differences harder to see.  The groups are in the same panel when using aesthetics, so may overlap, but otherwise small differences are easier to see.  Illustrated in Figure~\ref{fig:facet-vs-groups}

Faceting will also work with much larger number of groups, and because you can split in two dimensions, you can compare two variables simultaneously more easily than using two different aesthetics (conjunctions are not preattentive).

\begin{figure}[htbp]
  \centering
    \includegraphics[width=0.19\textwidth]{position-fvg-1}
    \includegraphics[width=0.79\textwidth]{position-fvg-2}

    \includegraphics[width=0.19\textwidth]{position-fvg-3}
    \includegraphics[width=0.79\textwidth]{position-fvg-4}
  \caption{caption}
  \label{fig:facet-vs-groups}
\end{figure}


% \subsection{Continuous variables}\label{sub:continuous_variables}
% 
% To use continuous variables as faceting variables, you will first need to convert them to categorical.  Also, as the faceting formula currently does not support calculated variables, so you will need to save them in the data frame first.
% 
% {\tt ggplot} provides a convenient function for converting a continuous variable to a categorical variable: {\tt chop}.  Chop takes the following arguments:
% 
% \begin{itemize}
%   \item {\tt x}: vector of numbers to chop up into categorical variable
%   \item {\tt n}: number of bins to cut the variable into
%   \item {\tt method}: method to use, equal ranges, equal numbers in each range or pretty ranges
% \end{itemize}
% 
% The following examples shows the use of chop to explore the earthquakes data set.
% 
% % decumar<<< 
% % ggopt(aspect.ratio = 1)
% % interweave({
% % quakes$depthc <- chop(quakes$depth)
% % qplot(long, lat, . ~ depthc, data=quakes)
% % quakes$magc <- chop(quakes$mag, n=6, method="pretty")
% % qplot(long, lat, . ~ magc, data=quakes)
% % })
% % |||
% % >>>
% 
% Breaking by quantiles, so that each facet has approximately the same number of points, is usually the best choice as it is easier to compare the facets because you know the same number of points are in each, even in the presences of overplotting.

\ifwhole
\else
  \nobibliography{/Users/hadley/documents/phd/references}
  \end{document}
\fi
\chapter{Themes}\label{cha:polishing}

\section{Introduction}

In this chapter you will learn how to use the ggplot2 theme system,
which allows you to exercise fine control over the non-data elements of
your plot. The theme system does not affect how the data is rendered by
geoms, or how it is transformed by scales. Themes don't change the
perceptual properties of the plot, but they do help you make the plot
aesthetically pleasing or match an existing style guide. Themes give you
control over things like fonts, ticks, panel strips, and backgrounds.
\index{Themes}

This separation of control into data and non-data parts is quite
different from base and lattice graphics. In base and lattice graphics,
most functions take a large number of arguments that specify both data
and non-data appearance, which makes the functions complicated and
harder to learn. ggplot2 takes a different approach: when creating the
plot you determine how the data is displayed, then \emph{after} it has
been created you can edit every detail of the rendering, using the
theming system.

The theming system is composed of four main components:

\begin{itemize}
\item
  Theme \emph{elements} specify the non-data elements that you can
  control. For example, the \texttt{plot.title} element controls the
  appearance of the plot title; \texttt{axis.ticks.x}, the ticks on the
  x axis; \texttt{legend.key.height}, the height of the keys in the
  legend.
\item
  Each element is associated with an \emph{element function}, which
  describes the visual properties of the element. For example,
  \texttt{element\_text()} sets the font size, colour and face of text
  elements like \texttt{plot.title}.
\item
  The \texttt{theme()} function which allows you to override the default
  theme elements by calling element functions, like
  \texttt{theme(plot.title\ =\ element\_text(colour\ =\ "red"))}.
\item
  Complete \emph{themes}, like \texttt{theme\_grey()} set all of the
  theme elements to values designed to work together harmoniously.
\end{itemize}

For example, imagine you've made the following plot of your data.

\begin{Shaded}
\begin{Highlighting}[]
\NormalTok{base <-}\StringTok{ }\KeywordTok{ggplot}\NormalTok{(mpg, }\KeywordTok{aes}\NormalTok{(cty, hwy, }\DataTypeTok{color =} \KeywordTok{factor}\NormalTok{(cyl))) +}
\StringTok{  }\KeywordTok{geom_jitter}\NormalTok{() +}\StringTok{ }
\StringTok{  }\KeywordTok{geom_abline}\NormalTok{(}\DataTypeTok{colour =} \StringTok{"grey50"}\NormalTok{, }\DataTypeTok{size =} \DecValTok{2}\NormalTok{)}
\NormalTok{base}
\end{Highlighting}
\end{Shaded}

\begin{figure}[H]
  \centering
  \includegraphics[width=0.75\linewidth]{_figures/themes/motivation-1-1}
\end{figure}

It's served it's purpose for you: you've learned that \texttt{cty} and
\texttt{hwy} are highly correlated, both are tightly coupled with to
\texttt{cyl}, and that \texttt{hwy} is always greater than \texttt{cty}
(and the difference increases as \texttt{cty} increases). Now you want
to share the plot with others, perhaps by publishing it in a paper. That
requires some changes. First, you need to make sure the plot can stand
alone by:

\begin{itemize}
\tightlist
\item
  Improving the axes and legend labels.
\item
  Adding a title for the plot.
\item
  Tweaking the colour scale.
\end{itemize}

Fortunately you know how to do that already because you've read
\hyperref[cha:scales]{the scales chapter}:

\begin{Shaded}
\begin{Highlighting}[]
\NormalTok{labelled <-}\StringTok{ }\NormalTok{base +}
\StringTok{  }\KeywordTok{labs}\NormalTok{(}
    \DataTypeTok{x =} \StringTok{"City mileage/gallon"}\NormalTok{,}
    \DataTypeTok{y =} \StringTok{"Highway mileage/gallon"}\NormalTok{,}
    \DataTypeTok{colour =} \StringTok{"Cylinders"}\NormalTok{,}
    \DataTypeTok{title =} \StringTok{"Highway and city mileage are highly correlated"}
  \NormalTok{) +}
\StringTok{  }\KeywordTok{scale_colour_brewer}\NormalTok{(}\DataTypeTok{type =} \StringTok{"seq"}\NormalTok{, }\DataTypeTok{palette =} \StringTok{"Spectral"}\NormalTok{)}
\NormalTok{labelled}
\end{Highlighting}
\end{Shaded}

\begin{figure}[H]
  \centering
  \includegraphics[width=0.75\linewidth]{_figures/themes/motivation-2-1}
\end{figure}

Next, you need to make sure the plot matches the style guidelines of
your journal:

\begin{itemize}
\tightlist
\item
  The background should be white, not pale grey.
\item
  The legend should be placed inside the plot if there's room.
\item
  Major gridlines should be a pale grey and minor gridlines should be
  removed.
\item
  The plot title should be 12pt bold text.
\end{itemize}

In this chapter, you'll learn how to use the theming system to make
those changes, as shown below:

\begin{Shaded}
\begin{Highlighting}[]
\NormalTok{styled <-}\StringTok{ }\NormalTok{labelled +}
\StringTok{  }\KeywordTok{theme_bw}\NormalTok{() +}\StringTok{ }
\StringTok{  }\KeywordTok{theme}\NormalTok{(}
    \DataTypeTok{plot.title =} \KeywordTok{element_text}\NormalTok{(}\DataTypeTok{face =} \StringTok{"bold"}\NormalTok{, }\DataTypeTok{size =} \DecValTok{12}\NormalTok{),}
    \DataTypeTok{legend.background =} \KeywordTok{element_rect}\NormalTok{(}\DataTypeTok{fill =} \StringTok{"white"}\NormalTok{, }\DataTypeTok{size =} \DecValTok{4}\NormalTok{, }\DataTypeTok{colour =} \StringTok{"white"}\NormalTok{),}
    \DataTypeTok{legend.justification =} \KeywordTok{c}\NormalTok{(}\DecValTok{0}\NormalTok{, }\DecValTok{1}\NormalTok{),}
    \DataTypeTok{legend.position =} \KeywordTok{c}\NormalTok{(}\DecValTok{0}\NormalTok{, }\DecValTok{1}\NormalTok{),}
    \DataTypeTok{axis.ticks =} \KeywordTok{element_line}\NormalTok{(}\DataTypeTok{colour =} \StringTok{"grey70"}\NormalTok{, }\DataTypeTok{size =} \FloatTok{0.2}\NormalTok{),}
    \DataTypeTok{panel.grid.major =} \KeywordTok{element_line}\NormalTok{(}\DataTypeTok{colour =} \StringTok{"grey70"}\NormalTok{, }\DataTypeTok{size =} \FloatTok{0.2}\NormalTok{),}
    \DataTypeTok{panel.grid.minor =} \KeywordTok{element_blank}\NormalTok{()}
  \NormalTok{)}
\NormalTok{styled}
\end{Highlighting}
\end{Shaded}

\begin{figure}[H]
  \centering
  \includegraphics[width=0.75\linewidth]{_figures/themes/motivation-3-1}
\end{figure}

Finally, the journal wants the figure as a 600 dpi TIFF file. You'll
learn the fine details of \texttt{ggsave()} in
\hyperref[sec:saving]{saving your output}.

\section{Complete themes}\label{sec:themes}

ggplot2 comes with a number of built in themes. The most important is
\texttt{theme\_grey()}, the signature ggplot2 theme with a light grey
background and white gridlines. The theme is designed to put the data
forward while supporting comparisons, following the advice of (Tufte
2006; Brewer 1994; Carr 2002; Carr 1994; Carr and Sun 1999). We can
still see the gridlines to aid in the judgement of position (Cleveland
1993), but they have little visual impact and we can easily `tune' them
out. The grey background gives the plot a similar typographic colour to
the text, ensuring that the graphics fit in with the flow of a document
without jumping out with a bright white background. Finally, the grey
background creates a continuous field of colour which ensures that the
plot is perceived as a single visual entity. \index{Themes!built-in}
\indexf{theme\_grey}

There are seven other themes built in to ggplot2 1.1.0:

\begin{itemize}
\item
  \texttt{theme\_bw()}: a variation on \texttt{theme\_grey()} that uses
  a white background and thin grey grid lines. \indexf{theme\_bw}
\item
  \texttt{theme\_linedraw()}: A theme with only black lines of various
  widths on white backgrounds, reminiscent of a line drawing.
  \indexf{theme\_linedraw}
\item
  \texttt{theme\_light()}: similar to \texttt{theme\_linedraw()} but
  with light grey lines and axes, to direct more attention towards the
  data. \indexf{theme\_light}
\item
  \texttt{theme\_dark()}: the dark cousin of \texttt{theme\_light()},
  with similar line sizes but a dark background. Useful to make thin
  coloured lines pop out. \indexf{theme\_dark}
\item
  \texttt{theme\_minimal()}: A minimalistic theme with no background
  annotations. \indexf{theme\_minimal}
\item
  \texttt{theme\_classic()}: A classic-looking theme, with x and y axis
  lines and no gridlines. \indexf{theme\_classic}
\item
  \texttt{theme\_void()}: A completely empty theme. \indexf{theme\_void}
\end{itemize}

\begin{Shaded}
\begin{Highlighting}[]
\NormalTok{df <-}\StringTok{ }\KeywordTok{data.frame}\NormalTok{(}\DataTypeTok{x =} \DecValTok{1}\NormalTok{:}\DecValTok{3}\NormalTok{, }\DataTypeTok{y =} \DecValTok{1}\NormalTok{:}\DecValTok{3}\NormalTok{)}
\NormalTok{base <-}\StringTok{ }\KeywordTok{ggplot}\NormalTok{(df, }\KeywordTok{aes}\NormalTok{(x, y)) +}\StringTok{ }\KeywordTok{geom_point}\NormalTok{()}
\NormalTok{base +}\StringTok{ }\KeywordTok{theme_grey}\NormalTok{() +}\StringTok{ }\KeywordTok{ggtitle}\NormalTok{(}\StringTok{"theme_grey()"}\NormalTok{)}
\NormalTok{base +}\StringTok{ }\KeywordTok{theme_bw}\NormalTok{() +}\StringTok{ }\KeywordTok{ggtitle}\NormalTok{(}\StringTok{"theme_bw()"}\NormalTok{)}
\NormalTok{base +}\StringTok{ }\KeywordTok{theme_linedraw}\NormalTok{() +}\StringTok{ }\KeywordTok{ggtitle}\NormalTok{(}\StringTok{"theme_linedraw()"}\NormalTok{)}
\end{Highlighting}
\end{Shaded}

\begin{figure}[H]
  \includegraphics[width=0.333\linewidth]{_figures/themes/built-in-1}%
  \includegraphics[width=0.333\linewidth]{_figures/themes/built-in-2}%
  \includegraphics[width=0.333\linewidth]{_figures/themes/built-in-3}
\end{figure}

\begin{Shaded}
\begin{Highlighting}[]
\NormalTok{base +}\StringTok{ }\KeywordTok{theme_light}\NormalTok{() +}\StringTok{ }\KeywordTok{ggtitle}\NormalTok{(}\StringTok{"theme_light()"}\NormalTok{)}
\NormalTok{base +}\StringTok{ }\KeywordTok{theme_dark}\NormalTok{() +}\StringTok{ }\KeywordTok{ggtitle}\NormalTok{(}\StringTok{"theme_dark()"}\NormalTok{)}
\NormalTok{base +}\StringTok{ }\KeywordTok{theme_minimal}\NormalTok{()  +}\StringTok{ }\KeywordTok{ggtitle}\NormalTok{(}\StringTok{"theme_minimal()"}\NormalTok{)}
\end{Highlighting}
\end{Shaded}

\begin{figure}[H]
  \includegraphics[width=0.333\linewidth]{_figures/themes/unnamed-chunk-1-1}%
  \includegraphics[width=0.333\linewidth]{_figures/themes/unnamed-chunk-1-2}%
  \includegraphics[width=0.333\linewidth]{_figures/themes/unnamed-chunk-1-3}
\end{figure}

\begin{Shaded}
\begin{Highlighting}[]
\NormalTok{base +}\StringTok{ }\KeywordTok{theme_classic}\NormalTok{() +}\StringTok{ }\KeywordTok{ggtitle}\NormalTok{(}\StringTok{"theme_classic()"}\NormalTok{)}
\NormalTok{base +}\StringTok{ }\KeywordTok{theme_void}\NormalTok{() +}\StringTok{ }\KeywordTok{ggtitle}\NormalTok{(}\StringTok{"theme_void()"}\NormalTok{)}
\end{Highlighting}
\end{Shaded}

\begin{figure}[H]
  \includegraphics[width=0.333\linewidth]{_figures/themes/unnamed-chunk-2-1}%
  \includegraphics[width=0.333\linewidth]{_figures/themes/unnamed-chunk-2-2}
\end{figure}

All themes have a \texttt{base\_size} parameter which controls the base
font size. The base font size is the size that the axis titles use: the
plot title is usually bigger (1.2x), and the tick and strip labels are
smaller (0.8x). If you want to control these sizes separately, you'll
need to modify the individual elements as described below.

As well as applying themes a plot at-a-time, you can change the default
theme with \texttt{theme\_set()}. For example, if you really hate the
default grey background, run \texttt{theme\_set(theme\_bw())} to use a
white background for all plots. \indexf{theme\_set}

You're not limited to the themes built-in to ggplot2. Other packages,
like ggtheme by Jeffrey Arnold, add even more. Here's a few of my
favourite from ggtheme: \index{ggtheme}

\begin{Shaded}
\begin{Highlighting}[]
\KeywordTok{library}\NormalTok{(ggthemes)}
\CommentTok{#> Warning: replacing previous import by 'grid::arrow' when loading}
\CommentTok{#> 'ggthemes'}
\CommentTok{#> Warning: replacing previous import by 'grid::unit' when loading}
\CommentTok{#> 'ggthemes'}
\CommentTok{#> Warning: replacing previous import by 'scales::alpha' when loading}
\CommentTok{#> 'ggthemes'}
\NormalTok{base +}\StringTok{ }\KeywordTok{theme_tufte}\NormalTok{() +}\StringTok{ }\KeywordTok{ggtitle}\NormalTok{(}\StringTok{"theme_tufte()"}\NormalTok{)}
\NormalTok{base +}\StringTok{ }\KeywordTok{theme_solarized}\NormalTok{() +}\StringTok{ }\KeywordTok{ggtitle}\NormalTok{(}\StringTok{"theme_solarized()"}\NormalTok{)}
\NormalTok{base +}\StringTok{ }\KeywordTok{theme_excel}\NormalTok{() +}\StringTok{ }\KeywordTok{ggtitle}\NormalTok{(}\StringTok{"theme_excel()"}\NormalTok{) }\CommentTok{# ;)}
\end{Highlighting}
\end{Shaded}

\begin{figure}[H]
  \includegraphics[width=0.333\linewidth]{_figures/themes/ggtheme-1}%
  \includegraphics[width=0.333\linewidth]{_figures/themes/ggtheme-2}%
  \includegraphics[width=0.333\linewidth]{_figures/themes/ggtheme-3}
\end{figure}

The complete themes are a great place to start but don't give you a lot
of control. To modify individual elements, you need to use
\texttt{theme()} to override the default setting for an element with an
element function.

\subsection{Exercises}

\begin{enumerate}
\def\labelenumi{\arabic{enumi}.}
\item
  Try out all the themes in ggthemes. Which do you like the best?
\item
  What aspects of the default theme do you like? What don't you like?\\
  What would you change?
\item
  Look at the plots in your favourite scientific journal. What theme do
  they most resemble? What are the main differences?
\end{enumerate}

\section{Modifying theme components}

Modifying an individual theme component looks like
\texttt{plot\ +\ theme(element.name\ =\ element\_function())}. In this
section you'll learn about the basic element functions, and then in the
next section, you'll see all the elements that you can modify.
\indexf{theme}

There are four basic types of built-in element functions: text, lines,
rectangles, and blank. Each element function has a set of parameters
that control the appearance:

\begin{itemize}
\item
  \texttt{element\_text()} draws labels and headings. You can control
  the font \texttt{family}, \texttt{face}, \texttt{colour},
  \texttt{size} (in points), \texttt{hjust}, \texttt{vjust},
  \texttt{angle} (in degrees) and \texttt{lineheight} (as ratio of
  \texttt{fontcase}). More details on the parameters can be found in
  \texttt{vignette("ggplot2-specs")}. Setting the font face is
  particularly challenging. \index{Themes!labels} \indexf{element\_text}

\begin{Shaded}
\begin{Highlighting}[]
\NormalTok{base_t <-}\StringTok{ }\NormalTok{base +}\StringTok{ }\KeywordTok{labs}\NormalTok{(}\DataTypeTok{title =} \StringTok{"This is a ggplot"}\NormalTok{) +}\StringTok{ }\KeywordTok{xlab}\NormalTok{(}\OtherTok{NULL}\NormalTok{) +}\StringTok{ }\KeywordTok{ylab}\NormalTok{(}\OtherTok{NULL}\NormalTok{)}
\NormalTok{base_t +}\StringTok{ }\KeywordTok{theme}\NormalTok{(}\DataTypeTok{plot.title =} \KeywordTok{element_text}\NormalTok{(}\DataTypeTok{size =} \DecValTok{16}\NormalTok{))}
\NormalTok{base_t +}\StringTok{ }\KeywordTok{theme}\NormalTok{(}\DataTypeTok{plot.title =} \KeywordTok{element_text}\NormalTok{(}\DataTypeTok{face =} \StringTok{"bold"}\NormalTok{, }\DataTypeTok{colour =} \StringTok{"red"}\NormalTok{))}
\NormalTok{base_t +}\StringTok{ }\KeywordTok{theme}\NormalTok{(}\DataTypeTok{plot.title =} \KeywordTok{element_text}\NormalTok{(}\DataTypeTok{hjust =} \DecValTok{1}\NormalTok{))}
\end{Highlighting}
\end{Shaded}

  \begin{figure}[H]
    \includegraphics[width=0.333\linewidth]{_figures/themes/element_text-1}%
    \includegraphics[width=0.333\linewidth]{_figures/themes/element_text-2}%
    \includegraphics[width=0.333\linewidth]{_figures/themes/element_text-3}
  \end{figure}

  You can control the margins around the text with the \texttt{margin}
  argument and \texttt{margin()} function. \texttt{margin()} has four
  arguments: the amount of space (in points) to add to the top, right,
  bottom and left sides of the text. Any elements not specified default
  to 0.

\begin{Shaded}
\begin{Highlighting}[]
\CommentTok{# The margins here look asymmetric because there are also plot margins}
\NormalTok{base_t +}\StringTok{ }\KeywordTok{theme}\NormalTok{(}\DataTypeTok{plot.title =} \KeywordTok{element_text}\NormalTok{(}\DataTypeTok{margin =} \KeywordTok{margin}\NormalTok{()))}
\NormalTok{base_t +}\StringTok{ }\KeywordTok{theme}\NormalTok{(}\DataTypeTok{plot.title =} \KeywordTok{element_text}\NormalTok{(}\DataTypeTok{margin =} \KeywordTok{margin}\NormalTok{(}\DataTypeTok{t =} \DecValTok{10}\NormalTok{, }\DataTypeTok{b =} \DecValTok{10}\NormalTok{)))}
\NormalTok{base_t +}\StringTok{ }\KeywordTok{theme}\NormalTok{(}\DataTypeTok{axis.title.y =} \KeywordTok{element_text}\NormalTok{(}\DataTypeTok{margin =} \KeywordTok{margin}\NormalTok{(}\DataTypeTok{r =} \DecValTok{10}\NormalTok{)))}
\end{Highlighting}
\end{Shaded}

  \begin{figure}[H]
    \includegraphics[width=0.333\linewidth]{_figures/themes/element_text-margin-1}%
    \includegraphics[width=0.333\linewidth]{_figures/themes/element_text-margin-2}%
    \includegraphics[width=0.333\linewidth]{_figures/themes/element_text-margin-3}
  \end{figure}
\item
  \texttt{element\_line()} draws lines parameterised by \texttt{colour},
  \texttt{size} and \texttt{linetype}: \indexf{element\_line}
  \index{Themes!lines}

\begin{Shaded}
\begin{Highlighting}[]
\NormalTok{base +}\StringTok{ }\KeywordTok{theme}\NormalTok{(}\DataTypeTok{panel.grid.major =} \KeywordTok{element_line}\NormalTok{(}\DataTypeTok{colour =} \StringTok{"black"}\NormalTok{))}
\NormalTok{base +}\StringTok{ }\KeywordTok{theme}\NormalTok{(}\DataTypeTok{panel.grid.major =} \KeywordTok{element_line}\NormalTok{(}\DataTypeTok{size =} \DecValTok{2}\NormalTok{))}
\NormalTok{base +}\StringTok{ }\KeywordTok{theme}\NormalTok{(}\DataTypeTok{panel.grid.major =} \KeywordTok{element_line}\NormalTok{(}\DataTypeTok{linetype =} \StringTok{"dotted"}\NormalTok{))}
\end{Highlighting}
\end{Shaded}

  \begin{figure}[H]
    \includegraphics[width=0.333\linewidth]{_figures/themes/element_line-1}%
    \includegraphics[width=0.333\linewidth]{_figures/themes/element_line-2}%
    \includegraphics[width=0.333\linewidth]{_figures/themes/element_line-3}
  \end{figure}
\item
  \texttt{element\_rect()} draws rectangles, mostly used for
  backgrounds, parameterised by \texttt{fill} colour and border
  \texttt{colour}, \texttt{size} and \texttt{linetype}.\\
   \index{Background} \index{Themes!background} \indexf{theme\_rect}

\begin{Shaded}
\begin{Highlighting}[]
\NormalTok{base +}\StringTok{ }\KeywordTok{theme}\NormalTok{(}\DataTypeTok{plot.background =} \KeywordTok{element_rect}\NormalTok{(}\DataTypeTok{fill =} \StringTok{"grey80"}\NormalTok{, }\DataTypeTok{colour =} \OtherTok{NA}\NormalTok{))}
\NormalTok{base +}\StringTok{ }\KeywordTok{theme}\NormalTok{(}\DataTypeTok{plot.background =} \KeywordTok{element_rect}\NormalTok{(}\DataTypeTok{colour =} \StringTok{"red"}\NormalTok{, }\DataTypeTok{size =} \DecValTok{2}\NormalTok{))}
\NormalTok{base +}\StringTok{ }\KeywordTok{theme}\NormalTok{(}\DataTypeTok{panel.background =} \KeywordTok{element_rect}\NormalTok{(}\DataTypeTok{fill =} \StringTok{"linen"}\NormalTok{))}
\end{Highlighting}
\end{Shaded}

  \begin{figure}[H]
    \includegraphics[width=0.333\linewidth]{_figures/themes/element_rect-1}%
    \includegraphics[width=0.333\linewidth]{_figures/themes/element_rect-2}%
    \includegraphics[width=0.333\linewidth]{_figures/themes/element_rect-3}
  \end{figure}
\item
  \texttt{element\_blank()} draws nothing. Use this if you don't want
  anything drawn, and no space allocated for that element. The following
  example uses \texttt{element\_blank()} to progressively suppress the
  appearance of elements we're not interested in. Notice how the plot
  automatically reclaims the space previously used by these elements: if
  you don't want this to happen (perhaps because they need to line up
  with other plots on the page), use
  \texttt{colour\ =\ NA,\ fill\ =\ NA} as parameter to create invisible
  elements that still take up space. \indexf{element\_blank}

\begin{Shaded}
\begin{Highlighting}[]
\NormalTok{base}
\KeywordTok{last_plot}\NormalTok{() +}\StringTok{ }\KeywordTok{theme}\NormalTok{(}\DataTypeTok{panel.grid.minor =} \KeywordTok{element_blank}\NormalTok{())}
\KeywordTok{last_plot}\NormalTok{() +}\StringTok{ }\KeywordTok{theme}\NormalTok{(}\DataTypeTok{panel.grid.major =} \KeywordTok{element_blank}\NormalTok{())}
\end{Highlighting}
\end{Shaded}

  \begin{figure}[H]
    \includegraphics[width=0.333\linewidth]{_figures/themes/element_blank-1}%
    \includegraphics[width=0.333\linewidth]{_figures/themes/element_blank-2}%
    \includegraphics[width=0.333\linewidth]{_figures/themes/element_blank-3}
  \end{figure}

\begin{Shaded}
\begin{Highlighting}[]
\KeywordTok{last_plot}\NormalTok{() +}\StringTok{ }\KeywordTok{theme}\NormalTok{(}\DataTypeTok{panel.background =} \KeywordTok{element_blank}\NormalTok{())}
\KeywordTok{last_plot}\NormalTok{() +}\StringTok{ }\KeywordTok{theme}\NormalTok{(}
  \DataTypeTok{axis.title.x =} \KeywordTok{element_blank}\NormalTok{(), }
  \DataTypeTok{axis.title.y =} \KeywordTok{element_blank}\NormalTok{()}
\NormalTok{)}
\KeywordTok{last_plot}\NormalTok{() +}\StringTok{ }\KeywordTok{theme}\NormalTok{(}\DataTypeTok{axis.line =} \KeywordTok{element_line}\NormalTok{(}\DataTypeTok{colour =} \StringTok{"grey50"}\NormalTok{))}
\end{Highlighting}
\end{Shaded}

  \begin{figure}[H]
    \includegraphics[width=0.333\linewidth]{_figures/themes/element_blank-2-1}%
    \includegraphics[width=0.333\linewidth]{_figures/themes/element_blank-2-2}%
    \includegraphics[width=0.333\linewidth]{_figures/themes/element_blank-2-3}
  \end{figure}
\end{itemize}

A few other settings take grid units. Create them with
\texttt{unit(1,\ "cm")} or \texttt{unit(0.25,\ "in")}.

To modify theme elements for all future plots, use
\texttt{theme\_update()}. It returns the previous theme settings, so you
can easily restore the original parameters once you're done.
\index{Themes!updating} \indexf{theme\_set}

\begin{Shaded}
\begin{Highlighting}[]
\NormalTok{old_theme <-}\StringTok{ }\KeywordTok{theme_update}\NormalTok{(}
  \DataTypeTok{plot.background =} \KeywordTok{element_rect}\NormalTok{(}\DataTypeTok{fill =} \StringTok{"lightblue3"}\NormalTok{, }\DataTypeTok{colour =} \OtherTok{NA}\NormalTok{),}
  \DataTypeTok{panel.background =} \KeywordTok{element_rect}\NormalTok{(}\DataTypeTok{fill =} \StringTok{"lightblue"}\NormalTok{, }\DataTypeTok{colour =} \OtherTok{NA}\NormalTok{),}
  \DataTypeTok{axis.text =} \KeywordTok{element_text}\NormalTok{(}\DataTypeTok{colour =} \StringTok{"linen"}\NormalTok{),}
  \DataTypeTok{axis.title =} \KeywordTok{element_text}\NormalTok{(}\DataTypeTok{colour =} \StringTok{"linen"}\NormalTok{)}
\NormalTok{)}
\NormalTok{base}
\KeywordTok{theme_set}\NormalTok{(old_theme)}
\NormalTok{base}
\end{Highlighting}
\end{Shaded}

\begin{figure}[H]
  \centering
  \includegraphics[width=0.333\linewidth]{_figures/themes/theme-update-1}%
  \includegraphics[width=0.333\linewidth]{_figures/themes/theme-update-2}
\end{figure}

\section{Theme elements}\label{sec:theme-elements}

There are around 40 unique elements that control the appearance of the
plot. They can be roughly grouped into five categories: plot, axis,
legend, panel and facet. The following sections describe each in turn.
\index{Themes!elements}

\subsection{Plot elements}

\index{Themes!plot}

Some elements affect the plot as a whole:

\begin{longtable}[c]{@{}lll@{}}
\toprule
Element & Setter & Description\tabularnewline
\midrule
\endhead
plot.background & \texttt{element\_rect()} & plot
background\tabularnewline
plot.title & \texttt{element\_text()} & plot title\tabularnewline
plot.margin & \texttt{margin()} & margins around plot\tabularnewline
\bottomrule
\end{longtable}

\texttt{plot.background} draws a rectangle that underlies everything
else on the plot. By default, ggplot2 uses a white background which
ensures that the plot is usable wherever it might end up (e.g.~even if
you save as a png and put on a slide with a black background). When
exporting plots to use in other systems, you might want to make the
background transparent with \texttt{fill\ =\ NA}. Similarly, if you're
embedding a plot in a system that already has margins you might want to
eliminate the built-in margins. Note that a small margin is still
necessary if you want to draw a border around the plot.

\begin{Shaded}
\begin{Highlighting}[]
\NormalTok{base +}\StringTok{ }\KeywordTok{theme}\NormalTok{(}\DataTypeTok{plot.background =} \KeywordTok{element_rect}\NormalTok{(}\DataTypeTok{colour =} \StringTok{"grey50"}\NormalTok{, }\DataTypeTok{size =} \DecValTok{2}\NormalTok{))}
\NormalTok{base +}\StringTok{ }\KeywordTok{theme}\NormalTok{(}
  \DataTypeTok{plot.background =} \KeywordTok{element_rect}\NormalTok{(}\DataTypeTok{colour =} \StringTok{"grey50"}\NormalTok{, }\DataTypeTok{size =} \DecValTok{2}\NormalTok{),}
  \DataTypeTok{plot.margin =} \KeywordTok{margin}\NormalTok{(}\DecValTok{2}\NormalTok{, }\DecValTok{2}\NormalTok{, }\DecValTok{2}\NormalTok{, }\DecValTok{2}\NormalTok{)}
\NormalTok{)}
\NormalTok{base +}\StringTok{ }\KeywordTok{theme}\NormalTok{(}\DataTypeTok{plot.background =} \KeywordTok{element_rect}\NormalTok{(}\DataTypeTok{fill =} \StringTok{"lightblue"}\NormalTok{))}
\end{Highlighting}
\end{Shaded}

\begin{figure}[H]
  \includegraphics[width=0.333\linewidth]{_figures/themes/plot-1}%
  \includegraphics[width=0.333\linewidth]{_figures/themes/plot-2}%
  \includegraphics[width=0.333\linewidth]{_figures/themes/plot-3}
\end{figure}

\subsection{Axis elements}\label{sub:theme-axis}

\index{Themes!axis} \index{Axis!styling}

The axis elements control the apperance of the axes:

\begin{longtable}[c]{@{}lll@{}}
\toprule
Element & Setter & Description\tabularnewline
\midrule
\endhead
axis.line & \texttt{element\_line()} & line parallel to axis (hidden in
default themes)\tabularnewline
axis.text & \texttt{element\_text()} & tick labels\tabularnewline
axis.text.x & \texttt{element\_text()} & x-axis tick
labels\tabularnewline
axis.text.y & \texttt{element\_text()} & y-axis tick
labels\tabularnewline
axis.title & \texttt{element\_text()} & axis titles\tabularnewline
axis.title.x & \texttt{element\_text()} & x-axis title\tabularnewline
axis.title.y & \texttt{element\_text()} & y-axis title\tabularnewline
axis.ticks & \texttt{element\_line()} & axis tick marks\tabularnewline
axis.ticks.length & \texttt{unit()} & length of tick
marks\tabularnewline
\bottomrule
\end{longtable}

Note that \texttt{axis.text} (and \texttt{axis.title}) comes in three
forms: \texttt{axis.text}, \texttt{axis.text.x}, and
\texttt{axis.text.y}. Use the first form if you want to modify the
properties of both axes at once: any properties that you don't
explicitly set in \texttt{axis.text.x} and \texttt{axis.text.y} will be
inherited from \texttt{axis.text}.

\begin{Shaded}
\begin{Highlighting}[]
\NormalTok{df <-}\StringTok{ }\KeywordTok{data.frame}\NormalTok{(}\DataTypeTok{x =} \DecValTok{1}\NormalTok{:}\DecValTok{3}\NormalTok{, }\DataTypeTok{y =} \DecValTok{1}\NormalTok{:}\DecValTok{3}\NormalTok{)}
\NormalTok{base <-}\StringTok{ }\KeywordTok{ggplot}\NormalTok{(df, }\KeywordTok{aes}\NormalTok{(x, y)) +}\StringTok{ }\KeywordTok{geom_point}\NormalTok{()}

\CommentTok{# Accentuate the axes}
\NormalTok{base +}\StringTok{ }\KeywordTok{theme}\NormalTok{(}\DataTypeTok{axis.line =} \KeywordTok{element_line}\NormalTok{(}\DataTypeTok{colour =} \StringTok{"grey50"}\NormalTok{, }\DataTypeTok{size =} \DecValTok{1}\NormalTok{))}
\CommentTok{# Style both x and y axis labels}
\NormalTok{base +}\StringTok{ }\KeywordTok{theme}\NormalTok{(}\DataTypeTok{axis.text =} \KeywordTok{element_text}\NormalTok{(}\DataTypeTok{color =} \StringTok{"blue"}\NormalTok{, }\DataTypeTok{size =} \DecValTok{12}\NormalTok{))}
\CommentTok{# Useful for long labels}
\NormalTok{base +}\StringTok{ }\KeywordTok{theme}\NormalTok{(}\DataTypeTok{axis.text.x =} \KeywordTok{element_text}\NormalTok{(}\DataTypeTok{angle =} \NormalTok{-}\DecValTok{90}\NormalTok{, }\DataTypeTok{vjust =} \FloatTok{0.5}\NormalTok{))}
\end{Highlighting}
\end{Shaded}

\begin{figure}[H]
  \includegraphics[width=0.333\linewidth]{_figures/themes/axis-1}%
  \includegraphics[width=0.333\linewidth]{_figures/themes/axis-2}%
  \includegraphics[width=0.333\linewidth]{_figures/themes/axis-3}
\end{figure}

The most common adjustment is to rotate the x-axis labels to avoid long
overlapping labels. If you do this, note negative angles tend to look
best and you should set \texttt{hjust\ =\ 0} and \texttt{vjust\ =\ 1}:

\begin{Shaded}
\begin{Highlighting}[]
\NormalTok{df <-}\StringTok{ }\KeywordTok{data.frame}\NormalTok{(}
  \DataTypeTok{x =} \KeywordTok{c}\NormalTok{(}\StringTok{"label"}\NormalTok{, }\StringTok{"a long label"}\NormalTok{, }\StringTok{"an even longer label"}\NormalTok{), }
  \DataTypeTok{y =} \DecValTok{1}\NormalTok{:}\DecValTok{3}
\NormalTok{)}
\NormalTok{base <-}\StringTok{ }\KeywordTok{ggplot}\NormalTok{(df, }\KeywordTok{aes}\NormalTok{(x, y)) +}\StringTok{ }\KeywordTok{geom_point}\NormalTok{()}
\NormalTok{base}
\NormalTok{base +}\StringTok{ }
\StringTok{  }\KeywordTok{theme}\NormalTok{(}\DataTypeTok{axis.text.x =} \KeywordTok{element_text}\NormalTok{(}\DataTypeTok{angle =} \NormalTok{-}\DecValTok{30}\NormalTok{, }\DataTypeTok{vjust =} \DecValTok{1}\NormalTok{, }\DataTypeTok{hjust =} \DecValTok{0}\NormalTok{)) +}\StringTok{ }
\StringTok{  }\KeywordTok{xlab}\NormalTok{(}\OtherTok{NULL}\NormalTok{) +}\StringTok{ }
\StringTok{  }\KeywordTok{ylab}\NormalTok{(}\OtherTok{NULL}\NormalTok{)}
\end{Highlighting}
\end{Shaded}

\begin{figure}[H]
  \includegraphics[width=0.5\linewidth]{_figures/themes/axis-labels-1}%
  \includegraphics[width=0.5\linewidth]{_figures/themes/axis-labels-2}
\end{figure}

\subsection{Legend elements}

\index{Themes!legend} \index{Legend!styling}

The legend elements control the apperance of all legends. You can also
modify the appearance of individual legends, by modifying the same
elements in \texttt{guide\_legend()} or \texttt{guide\_colourbar()}.

\begin{longtable}[c]{@{}lll@{}}
\toprule
Element & Setter & Description\tabularnewline
\midrule
\endhead
legend.background & \texttt{element\_rect()} & legend
background\tabularnewline
legend.key & \texttt{element\_rect()} & background of legend
keys\tabularnewline
legend.key.size & \texttt{unit()} & legend key size\tabularnewline
legend.key.height & \texttt{unit()} & legend key height\tabularnewline
legend.key.width & \texttt{unit()} & legend key width\tabularnewline
legend.margin & \texttt{unit()} & legend margin\tabularnewline
legend.text & \texttt{element\_text()} & legend labels\tabularnewline
legend.text.align & 0--1 & legend label alignment (0 = right, 1 =
left)\tabularnewline
legend.title & \texttt{element\_text()} & legend name\tabularnewline
legend.title.align & 0--1 & legend name alignment (0 = right, 1 =
left)\tabularnewline
\bottomrule
\end{longtable}

These options are illustrated below:

\begin{Shaded}
\begin{Highlighting}[]
\NormalTok{df <-}\StringTok{ }\KeywordTok{data.frame}\NormalTok{(}\DataTypeTok{x =} \DecValTok{1}\NormalTok{:}\DecValTok{4}\NormalTok{, }\DataTypeTok{y =} \DecValTok{1}\NormalTok{:}\DecValTok{4}\NormalTok{, }\DataTypeTok{z =} \KeywordTok{rep}\NormalTok{(}\KeywordTok{c}\NormalTok{(}\StringTok{"a"}\NormalTok{, }\StringTok{"b"}\NormalTok{), }\DataTypeTok{each =} \DecValTok{2}\NormalTok{))}
\NormalTok{base <-}\StringTok{ }\KeywordTok{ggplot}\NormalTok{(df, }\KeywordTok{aes}\NormalTok{(x, y, }\DataTypeTok{colour =} \NormalTok{z)) +}\StringTok{ }\KeywordTok{geom_point}\NormalTok{()}

\NormalTok{base +}\StringTok{ }\KeywordTok{theme}\NormalTok{(}
  \DataTypeTok{legend.background =} \KeywordTok{element_rect}\NormalTok{(}
    \DataTypeTok{fill =} \StringTok{"lemonchiffon"}\NormalTok{, }
    \DataTypeTok{colour =} \StringTok{"grey50"}\NormalTok{, }
    \DataTypeTok{size =} \DecValTok{1}
  \NormalTok{)}
\NormalTok{)}
\NormalTok{base +}\StringTok{ }\KeywordTok{theme}\NormalTok{(}
  \DataTypeTok{legend.key =} \KeywordTok{element_rect}\NormalTok{(}\DataTypeTok{color =} \StringTok{"grey50"}\NormalTok{),}
  \DataTypeTok{legend.key.width =} \KeywordTok{unit}\NormalTok{(}\FloatTok{0.9}\NormalTok{, }\StringTok{"cm"}\NormalTok{),}
  \DataTypeTok{legend.key.height =} \KeywordTok{unit}\NormalTok{(}\FloatTok{0.75}\NormalTok{, }\StringTok{"cm"}\NormalTok{)}
\NormalTok{)}
\NormalTok{base +}\StringTok{ }\KeywordTok{theme}\NormalTok{(}
  \DataTypeTok{legend.text =} \KeywordTok{element_text}\NormalTok{(}\DataTypeTok{size =} \DecValTok{15}\NormalTok{),}
  \DataTypeTok{legend.title =} \KeywordTok{element_text}\NormalTok{(}\DataTypeTok{size =} \DecValTok{15}\NormalTok{, }\DataTypeTok{face =} \StringTok{"bold"}\NormalTok{)}
\NormalTok{)}
\end{Highlighting}
\end{Shaded}

\begin{figure}[H]
  \includegraphics[width=0.333\linewidth]{_figures/themes/legend-1}%
  \includegraphics[width=0.333\linewidth]{_figures/themes/legend-2}%
  \includegraphics[width=0.333\linewidth]{_figures/themes/legend-3}
\end{figure}

There are four other properties that control how legends are laid out in
the context of the plot (\texttt{legend.position},
\texttt{legend.direction}, \texttt{legend.justification},
\texttt{legend.box}). They are described in
\hyperref[sub:legend-layout]{legend layout}.

\subsection{Panel elements}

\index{Themes!panel} \index{Aspect ratio}

Panel elements control the appearance of the plotting panels:

\begin{longtable}[c]{@{}lll@{}}
\toprule
Element & Setter & Description\tabularnewline
\midrule
\endhead
panel.background & \texttt{element\_rect()} & panel background (under
data)\tabularnewline
panel.border & \texttt{element\_rect()} & panel border (over
data)\tabularnewline
panel.grid.major & \texttt{element\_line()} & major grid
lines\tabularnewline
panel.grid.major.x & \texttt{element\_line()} & vertical major grid
lines\tabularnewline
panel.grid.major.y & \texttt{element\_line()} & horizontal major grid
lines\tabularnewline
panel.grid.minor & \texttt{element\_line()} & minor grid
lines\tabularnewline
panel.grid.minor.x & \texttt{element\_line()} & vertical minor grid
lines\tabularnewline
panel.grid.minor.y & \texttt{element\_line()} & horizontal minor grid
lines\tabularnewline
aspect.ratio & numeric & plot aspect ratio\tabularnewline
\bottomrule
\end{longtable}

The main difference is between \texttt{panel.background} and
\texttt{panel.border} is that the background is drawn underneath the
data, and the border is drawn on top of it. For that reason, you'll
always need to assign \texttt{fill\ =\ NA} when overriding
\texttt{panel.border}.

\begin{Shaded}
\begin{Highlighting}[]
\NormalTok{base <-}\StringTok{ }\KeywordTok{ggplot}\NormalTok{(df, }\KeywordTok{aes}\NormalTok{(x, y)) +}\StringTok{ }\KeywordTok{geom_point}\NormalTok{()}
\CommentTok{# Modify background}
\NormalTok{base +}\StringTok{ }\KeywordTok{theme}\NormalTok{(}\DataTypeTok{panel.background =} \KeywordTok{element_rect}\NormalTok{(}\DataTypeTok{fill =} \StringTok{"lightblue"}\NormalTok{))}

\CommentTok{# Tweak major grid lines}
\NormalTok{base +}\StringTok{ }\KeywordTok{theme}\NormalTok{(}
  \DataTypeTok{panel.grid.major =} \KeywordTok{element_line}\NormalTok{(}\DataTypeTok{color =} \StringTok{"gray60"}\NormalTok{, }\DataTypeTok{size =} \FloatTok{0.8}\NormalTok{)}
\NormalTok{)}
\CommentTok{# Just in one direction  }
\NormalTok{base +}\StringTok{ }\KeywordTok{theme}\NormalTok{(}
  \DataTypeTok{panel.grid.major.x =} \KeywordTok{element_line}\NormalTok{(}\DataTypeTok{color =} \StringTok{"gray60"}\NormalTok{, }\DataTypeTok{size =} \FloatTok{0.8}\NormalTok{)}
\NormalTok{)}
\end{Highlighting}
\end{Shaded}

\begin{figure}[H]
  \includegraphics[width=0.333\linewidth]{_figures/themes/panel-1}%
  \includegraphics[width=0.333\linewidth]{_figures/themes/panel-2}%
  \includegraphics[width=0.333\linewidth]{_figures/themes/panel-3}
\end{figure}

Note that aspect ratio controls the aspect ratio of the \emph{panel},
not the overall plot:

\begin{Shaded}
\begin{Highlighting}[]
\NormalTok{base2 <-}\StringTok{ }\NormalTok{base +}\StringTok{ }\KeywordTok{theme}\NormalTok{(}\DataTypeTok{plot.background =} \KeywordTok{element_rect}\NormalTok{(}\DataTypeTok{colour =} \StringTok{"grey50"}\NormalTok{))}
\CommentTok{# Wide screen}
\NormalTok{base2 +}\StringTok{ }\KeywordTok{theme}\NormalTok{(}\DataTypeTok{aspect.ratio =} \DecValTok{9} \NormalTok{/}\StringTok{ }\DecValTok{16}\NormalTok{)}
\CommentTok{# Long and skiny}
\NormalTok{base2 +}\StringTok{ }\KeywordTok{theme}\NormalTok{(}\DataTypeTok{aspect.ratio =} \DecValTok{2} \NormalTok{/}\StringTok{ }\DecValTok{1}\NormalTok{)}
\CommentTok{# Square}
\NormalTok{base2 +}\StringTok{ }\KeywordTok{theme}\NormalTok{(}\DataTypeTok{aspect.ratio =} \DecValTok{1}\NormalTok{)}
\end{Highlighting}
\end{Shaded}

\begin{figure}[H]
  \includegraphics[width=0.333\linewidth]{_figures/themes/aspect-ratio-1}%
  \includegraphics[width=0.333\linewidth]{_figures/themes/aspect-ratio-2}%
  \includegraphics[width=0.333\linewidth]{_figures/themes/aspect-ratio-3}
\end{figure}

\subsection{Facetting elements}

\index{Themes!facets} \index{Faceting!styling}

The following theme elements are associated with faceted ggplots:

\begin{longtable}[c]{@{}lll@{}}
\toprule
Element & Setter & Description\tabularnewline
\midrule
\endhead
strip.background & \texttt{element\_rect()} & background of panel
strips\tabularnewline
strip.text & \texttt{element\_text()} & strip text\tabularnewline
strip.text.x & \texttt{element\_text()} & horizontal strip
text\tabularnewline
strip.text.y & \texttt{element\_text()} & vertical strip
text\tabularnewline
panel.margin & \texttt{unit()} & margin between facets\tabularnewline
panel.margin.x & \texttt{unit()} & margin between facets
(vertical)\tabularnewline
panel.margin.y & \texttt{unit()} & margin between facets
(horizontal)\tabularnewline
\bottomrule
\end{longtable}

Element \texttt{strip.text.x} affects both \texttt{facet\_wrap()} or
\texttt{facet\_grid()}; \texttt{strip.text.y} only affects
\texttt{facet\_grid()}.

\begin{Shaded}
\begin{Highlighting}[]
\NormalTok{df <-}\StringTok{ }\KeywordTok{data.frame}\NormalTok{(}\DataTypeTok{x =} \DecValTok{1}\NormalTok{:}\DecValTok{4}\NormalTok{, }\DataTypeTok{y =} \DecValTok{1}\NormalTok{:}\DecValTok{4}\NormalTok{, }\DataTypeTok{z =} \KeywordTok{c}\NormalTok{(}\StringTok{"a"}\NormalTok{, }\StringTok{"a"}\NormalTok{, }\StringTok{"b"}\NormalTok{, }\StringTok{"b"}\NormalTok{))}
\NormalTok{base_f <-}\StringTok{ }\KeywordTok{ggplot}\NormalTok{(df, }\KeywordTok{aes}\NormalTok{(x, y)) +}\StringTok{ }\KeywordTok{geom_point}\NormalTok{() +}\StringTok{ }\KeywordTok{facet_wrap}\NormalTok{(~z)}

\NormalTok{base_f}
\NormalTok{base_f +}\StringTok{ }\KeywordTok{theme}\NormalTok{(}\DataTypeTok{panel.margin =} \KeywordTok{unit}\NormalTok{(}\FloatTok{0.5}\NormalTok{, }\StringTok{"in"}\NormalTok{))}
\NormalTok{base_f +}\StringTok{ }\KeywordTok{theme}\NormalTok{(}
  \DataTypeTok{strip.background =} \KeywordTok{element_rect}\NormalTok{(}\DataTypeTok{fill =} \StringTok{"grey20"}\NormalTok{, }\DataTypeTok{color =} \StringTok{"grey80"}\NormalTok{, }\DataTypeTok{size =} \DecValTok{1}\NormalTok{),}
  \DataTypeTok{strip.text =} \KeywordTok{element_text}\NormalTok{(}\DataTypeTok{colour =} \StringTok{"white"}\NormalTok{)}
\NormalTok{)}
\end{Highlighting}
\end{Shaded}

\begin{figure}[H]
  \includegraphics[width=0.333\linewidth]{_figures/themes/faceting-1}%
  \includegraphics[width=0.333\linewidth]{_figures/themes/faceting-2}%
  \includegraphics[width=0.333\linewidth]{_figures/themes/faceting-3}
\end{figure}

\subsection{Exercises}

\begin{enumerate}
\def\labelenumi{\arabic{enumi}.}
\item
  Create the ugliest plot possible! (Contributed by Andrew D. Steen,
  University of Tennessee - Knoxville)
\item
  \texttt{theme\_dark()} makes the inside of the plot dark, but not the
  outside. Change the plot background to black, and then update the text
  settings so you can still read the labels.
\item
  Make an elegant theme that uses ``linen'' as the background colour and
  a serif font for the text.
\item
  Systematically explore the effects of \texttt{hjust} when you have a
  multiline title. Why doesn't \texttt{vjust} do anything?
\end{enumerate}

\hyperdef{}{sec:saving}{\section{Saving your output}\label{sec:saving}}

When saving a plot to use in another program, you have two basic choices
of output: raster or vector: \index{Exporting} \index{Saving output}

\begin{itemize}
\item
  Vector graphics describe a plot as sequence of operations: draw a line
  from \((x_1, y_1)\) to \((x_2, y_2)\), draw a circle at \((x_3, x_4)\)
  with radius \(r\). This means that they are effectively `infinitely'
  zoomable; there is no loss of detail. The most useful vector graphic
  formats are pdf and svg.
\item
  Raster graphics are stored as an array of pixel colours and have a
  fixed optimal viewing size. The most useful raster graphic format is
  png.
\end{itemize}

Figure \ref{fig:vector-raster} illustrates the basic differences in
these formats for a circle. A good description is available at
\url{http://tinyurl.com/rstrvctr}.

\begin{figure}[htbp]
  \centering
    \includegraphics[width= 0.5\linewidth]{diagrams/vector-raster}
  \caption{The schematic difference between raster (left) and vector (right) graphics. }
  \label{fig:vector-raster}
\end{figure}

Unless there is a compelling reason not to, use vector graphics: they
look better in more places. There are two main reasons to use raster
graphics:

\begin{itemize}
\item
  You have a plot (e.g.~a scatterplot) with thousands of graphical
  objects (i.e.~points). A vector version will be large and slow to
  render.
\item
  You want to embed the graphic in MS Office. MS has poor support for
  vector graphics (except for their own DrawingXML format which is not
  currently easy to make from R), so raster graphics are easier.
\end{itemize}

There are two ways to save output from ggplot2. You can use the standard
R approach where you open a graphics device, generate the plot, then
close the device: \indexf{pdf}

\begin{Shaded}
\begin{Highlighting}[]
\KeywordTok{pdf}\NormalTok{(}\StringTok{"output.pdf"}\NormalTok{, }\DataTypeTok{width =} \DecValTok{6}\NormalTok{, }\DataTypeTok{height =} \DecValTok{6}\NormalTok{)}
\KeywordTok{ggplot}\NormalTok{(mpg, }\KeywordTok{aes}\NormalTok{(displ, cty)) +}\StringTok{ }\KeywordTok{geom_point}\NormalTok{()}
\KeywordTok{dev.off}\NormalTok{()}
\end{Highlighting}
\end{Shaded}

This works for all packages, but is verbose. ggplot2 provides a
convenient shorthand with \texttt{ggsave()}:

\begin{Shaded}
\begin{Highlighting}[]
\KeywordTok{ggplot}\NormalTok{(mpg, }\KeywordTok{aes}\NormalTok{(displ, cty)) +}\StringTok{ }\KeywordTok{geom_point}\NormalTok{()}
\KeywordTok{ggsave}\NormalTok{(}\StringTok{"output.pdf"}\NormalTok{)}
\end{Highlighting}
\end{Shaded}

\texttt{ggsave()} is optimised for interactive use: you can use it after
you've drawn a plot. It has the following important arguments:
\indexf{ggsave}

\begin{itemize}
\item
  The first argument, \texttt{path}, specifies the path where the image
  should be saved. The file extension will be used to automatically
  select the correct graphics device. \texttt{ggsave()} can produce
  \texttt{.eps}, \texttt{.pdf}, \texttt{.svg}, \texttt{.wmf},
  \texttt{.png}, \texttt{.jpg}, \texttt{.bmp}, and \texttt{.tiff}.
\item
  \texttt{width} and \texttt{height} control the output size. If left
  blank, they'll use the size of the on-screen graphics device.
\item
  For raster graphics (i.e. \texttt{.png}), the \texttt{dpi} argument
  controls the resolution of the plot. It defaults to 300, which is
  appropriate for most printers, but you may want to use 600 for
  particularly high-resolution output, or 96 for on-screen (e.g., web)
  display.
\end{itemize}

See \texttt{?ggsave} for more details.

\section*{References}
\addcontentsline{toc}{section}{References}

\hyperdef{}{ref-brewer:1994}{\label{ref-brewer:1994}}
Brewer, Cynthia A. 1994. ``Color Use Guidelines for Mapping and
Visualization.'' In \emph{Visualization in Modern Cartography}, edited
by A.M. MacEachren and D.R.F. Taylor, 123--47. Elsevier Science.

\hyperdef{}{ref-carr:1994}{\label{ref-carr:1994}}
Carr, Dan. 1994. ``Using Gray in Plots.'' \emph{ASA Statistical
Computing and Graphics Newsletter} 2 (5): 11--14.
\url{http://www.galaxy.gmu.edu/~dcarr/lib/v5n2.pdf}.

\hyperdef{}{ref-carr:2002}{\label{ref-carr:2002}}
---------. 2002. ``Graphical Displays.'' In \emph{Encyclopedia of
Environmetrics}, edited by Abdel H. El-Shaarawi and Walter W. Piegorsch,
2:933--60. John Wiley \& Sons.
\url{http://www.galaxy.gmu.edu/~dcarr/lib/EnvironmentalGraphics.pdf}.

\hyperdef{}{ref-carr:1999}{\label{ref-carr:1999}}
Carr, Dan, and Ru Sun. 1999. ``Using Layering and Perceptual Grouping in
Statistical Graphics.'' \emph{ASA Statistical Computing and Graphics
Newsletter} 10 (1): 25--31.

\hyperdef{}{ref-cleveland:1993a}{\label{ref-cleveland:1993a}}
Cleveland, William. 1993. ``A Model for Studying Display Methods of
Statistical Graphics.'' \emph{Journal of Computational and Graphical
Statistics} 2: 323--64. \url{http://stat.bell-labs.com/doc/93.4.ps}.

\hyperdef{}{ref-tufte:2006}{\label{ref-tufte:2006}}
Tufte, Edward R. 2006. \emph{Beautiful Evidence}. Graphics Press.


\part{Data analysis}

\chapter{Data analysis}\label{cha:data}

\section{Introduction}

So far, every example in this book has started with a nice dataset
that's easy to plot. That's great for learning (because you don't want
to struggle with data handling while you're learning visualisation), but
in real life, datasets hardly ever come in exactly the right structure.
To use ggplot2 in practice, you'll need to learn some data wrangling
skills. Indeed, in my experience, visualisation is often the easiest
part of the data analysis process: once you have the right data, in the
right format, aggregated in the right way, the right visualisation is
often obvious.

The goal of this part of the book is to show you how to integrate
ggplot2 with other tools needed for a complete data analysis:

\begin{itemize}
\item
  In this chapter, you'll learn the principles of tidy data (Wickham
  2014), which help you organise your data in a way that makes it easy
  to visualise with ggplot2, manipulate with dplyr and model with the
  many modelling packages. The principles of tidy data are supported by
  the \textbf{tidyr} package, which helps you tidy messy datasets.
\item
  Most visualisations require some data transformation whether it's
  creating a new variable from existing variables, or performing simple
  aggregations so you can see the forest for the trees.
  \hyperref[cha:dplyr]{dplyr} will show you how to do this with the
  \textbf{dplyr} package.
\item
  If you're using R, you're almost certainly using it for it's fantastic
  modelling capabilities. While there's an R package for almost every
  type of model that you can think of, the results of these models can
  be hard to visualise. In \hyperref[cha:modelling]{modelling}, you'll
  learn about the \textbf{broom} package, by David Robinson, to convert
  models into tidy datasets so you can easily visualise them with
  ggplot2.
\end{itemize}

Tidy data is the foundation for data manipulation and visualising
models. In the following sections, you'll learn the definition of tidy
data, and the tools you need to make messy data tidy. The chapter
concludes with two case studies that show how to apply the tools in
sequence to work with real(istic) data.

\section{Tidy data}\label{sec:tidy-data}

The principle behind tidy data is simple: storing your data in a
consistent way makes it easier to work with it. Tidy data is a mapping
between the statistical structure of a data frame (variables and
observations) and the physical structure (columns and rows). Tidy data
follows two main principles: \index{Tidy data}
\index{Data!best form for ggplot2}

\begin{enumerate}
\def\labelenumi{\arabic{enumi}.}
\tightlist
\item
  Variables go in columns.
\item
  Observations go in rows.
\end{enumerate}

Tidy data is particularly important for ggplot2 because the job of
ggplot2 is to map variables to visual properties: if your data isn't
tidy, you'll have a hard time visualising it.

Sometimes you'll find a dataset that you have no idea how to plot.
That's normally because it's not tidy. For example, take this data frame
that contains monthly employment data for the United States:

\begin{Shaded}
\begin{Highlighting}[]
\NormalTok{ec2}
\CommentTok{#> Source: local data frame [12 x 11]}
\CommentTok{#> }
\CommentTok{#>    month  2006  2007  2008  2009  2010  2011  2012  2013  2014  2015}
\CommentTok{#>    (dbl) (dbl) (dbl) (dbl) (dbl) (dbl) (dbl) (dbl) (dbl) (dbl) (dbl)}
\CommentTok{#> 1      1   8.6   8.3   9.0  10.7  20.0  21.6  21.0  16.2  15.9  13.4}
\CommentTok{#> 2      2   9.1   8.5   8.7  11.7  19.9  21.1  19.8  17.5  16.2  13.1}
\CommentTok{#> 3      3   8.7   9.1   8.7  12.3  20.4  21.5  19.2  17.7  15.9  12.2}
\CommentTok{#> 4      4   8.4   8.6   9.4  13.1  22.1  20.9  19.1  17.1  15.6  11.7}
\CommentTok{#> 5      5   8.5   8.2   7.9  14.2  22.3  21.6  19.9  17.0  14.5    NA}
\CommentTok{#> 6      6   7.3   7.7   9.0  17.2  25.2  22.3  20.1  16.6  13.2    NA}
\CommentTok{#> ..   ...   ...   ...   ...   ...   ...   ...   ...   ...   ...   ...}
\end{Highlighting}
\end{Shaded}

(If it looks familiar, it's because it's derived from the
\texttt{economics} dataset that we used earlier in the book.)

Imagine you want to plot a time series showing how unemployment has
changed over the last 10 years. Can you picture the ggplot2 command
you'd need to do it? What if you wanted to focus on the seasonal
component of unemployment by putting months on the x-axis and drawing
one line for each year? It's difficult to see how to create those plots
because the data is not tidy. There are three variables, month, year and
unemployment rate, but each variable is stored in a different way:

\begin{itemize}
\tightlist
\item
  \texttt{month} is stored in a column.
\item
  \texttt{year} is spread across the column names.
\item
  \texttt{rate} is the value of each cell.
\end{itemize}

To make it possible to plot this data we first need to tidy it. There
are two important pairs of tools:

\begin{itemize}
\tightlist
\item
  Spread \& gather.
\item
  Separate \& unite.
\end{itemize}

\section{Spread and gather}\label{sec:spread-gather}

Take a look at the two tables below:

\begin{longtable}[c]{@{}llr@{}}
\toprule
x & y & z\tabularnewline
\midrule
\endhead
a & A & 1\tabularnewline
b & D & 5\tabularnewline
c & A & 4\tabularnewline
c & B & 10\tabularnewline
d & C & 9\tabularnewline
\bottomrule
\end{longtable}

\begin{longtable}[c]{@{}lrrrr@{}}
\toprule
x & A & B & C & D\tabularnewline
\midrule
\endhead
a & 1 & NA & NA & NA\tabularnewline
b & NA & NA & NA & 5\tabularnewline
c & 4 & 10 & NA & NA\tabularnewline
d & NA & NA & 9 & NA\tabularnewline
\bottomrule
\end{longtable}

If you study them for a little while, you'll notice that they contain
the same data in different forms. I call first form \textbf{indexed}
data, because you look up a value using an index (the values of the
\texttt{x} and \texttt{y} variables). I call the second form
\textbf{Cartesian} data, because you find a value by looking at
intersection of a row and a column. We can't tell if these datasets are
tidy or not. Either form could be tidy depending on what the values
``A'', ``B'', ``C'', ``D'' mean.

(Also note the missing values: missing values that are explicit in one
form maybe implicit in the other. An \texttt{NA} is the presence of an
absense; but sometimes a missing value is the absense of a presence.)

Tidying your data will often require translating Cartesian → indexed
forms, called \textbf{gathering}, and less commonly, indexed →
Cartesian, calling \textbf{spreading}. The tidyr package provides the
\texttt{spread()} and \texttt{gather()} functions to perform these
operations, as described below.

(You can imagine generalising these ideas to higher dimensions. However,
data is almost always stored in 2d (rows \& columns), so these
generalisations are fun to think about, but not that practical. I
explore the idea more in Wickham (2007).

\subsection{Gather}

\texttt{gather()} has four main arguments: \indexf{gather}

\begin{itemize}
\item
  \texttt{data}: the dataset to translate.
\item
  \texttt{key} \& \texttt{value}: the key is the name of the variable
  that will be created from the column names, and the value is the name
  of the variable that will be created from the cell values.
\item
  \texttt{...}: which variables to gather. You can specify individually,
  \texttt{A,\ B,\ C,\ D}, or as a range \texttt{A:D}. Alternatively, you
  can specify which columns are \emph{not} to be gathered with
  \texttt{-}: \texttt{-E,\ -F}.
\end{itemize}

To tidy the economics dataset shown above, you first need to identify
the variables: \texttt{year}, \texttt{month} and \texttt{rate}.
\texttt{month} is already in a column, but \texttt{year} and
\texttt{rate} are in Cartesian form, and we want them in indexed form,
so we need to use \texttt{gather()}. In this example, the key is
\texttt{year}, the value is \texttt{unemp} and we want to select columns
from \texttt{2006} to \texttt{2015}:

\begin{Shaded}
\begin{Highlighting}[]
\KeywordTok{gather}\NormalTok{(ec2, }\DataTypeTok{key =} \NormalTok{year, }\DataTypeTok{value =} \NormalTok{unemp, }\StringTok{`}\DataTypeTok{2006}\StringTok{`}\NormalTok{:}\StringTok{`}\DataTypeTok{2015}\StringTok{`}\NormalTok{)}
\CommentTok{#> Source: local data frame [120 x 3]}
\CommentTok{#> }
\CommentTok{#>    month   year unemp}
\CommentTok{#>    (dbl) (fctr) (dbl)}
\CommentTok{#> 1      1   2006   8.6}
\CommentTok{#> 2      2   2006   9.1}
\CommentTok{#> 3      3   2006   8.7}
\CommentTok{#> 4      4   2006   8.4}
\CommentTok{#> 5      5   2006   8.5}
\CommentTok{#> 6      6   2006   7.3}
\CommentTok{#> ..   ...    ...   ...}
\end{Highlighting}
\end{Shaded}

Note that the columns have names that are not standard variable names in
R (they don't start with a letter). This means that we need to surround
them in backticks, i.e. \texttt{`2006`} to refer to them.

Alternatively, we could gather all columns except \texttt{month}:

\begin{Shaded}
\begin{Highlighting}[]
\KeywordTok{gather}\NormalTok{(ec2, }\DataTypeTok{key =} \NormalTok{year, }\DataTypeTok{value =} \NormalTok{unemp, -month)}
\CommentTok{#> Source: local data frame [120 x 3]}
\CommentTok{#> }
\CommentTok{#>    month   year unemp}
\CommentTok{#>    (dbl) (fctr) (dbl)}
\CommentTok{#> 1      1   2006   8.6}
\CommentTok{#> 2      2   2006   9.1}
\CommentTok{#> 3      3   2006   8.7}
\CommentTok{#> 4      4   2006   8.4}
\CommentTok{#> 5      5   2006   8.5}
\CommentTok{#> 6      6   2006   7.3}
\CommentTok{#> ..   ...    ...   ...}
\end{Highlighting}
\end{Shaded}

To be most useful, we can provide two extra arguments:

\begin{Shaded}
\begin{Highlighting}[]
\NormalTok{economics_2 <-}\StringTok{ }\KeywordTok{gather}\NormalTok{(ec2, year, rate, }\StringTok{`}\DataTypeTok{2006}\StringTok{`}\NormalTok{:}\StringTok{`}\DataTypeTok{2015}\StringTok{`}\NormalTok{, }
  \DataTypeTok{convert =} \OtherTok{TRUE}\NormalTok{, }\DataTypeTok{na.rm =} \OtherTok{TRUE}\NormalTok{)}
\NormalTok{economics_2}
\CommentTok{#> Source: local data frame [112 x 3]}
\CommentTok{#> }
\CommentTok{#>    month  year  rate}
\CommentTok{#>    (dbl) (int) (dbl)}
\CommentTok{#> 1      1  2006   8.6}
\CommentTok{#> 2      2  2006   9.1}
\CommentTok{#> 3      3  2006   8.7}
\CommentTok{#> 4      4  2006   8.4}
\CommentTok{#> 5      5  2006   8.5}
\CommentTok{#> 6      6  2006   7.3}
\CommentTok{#> ..   ...   ...   ...}
\end{Highlighting}
\end{Shaded}

We use \texttt{convert\ =\ TRUE} to automatically convert the years from
character strings to numbers, and \texttt{na.rm\ =\ TRUE} to remove the
months with no data. (In some sense the data isn't actually missing
because it represents dates that haven't occurred yet.)

When the data is in this form, it's easy to visualise in many different
ways. For example, we can choose to emphasise either long term trend or
seasonal variations:

\begin{Shaded}
\begin{Highlighting}[]
\KeywordTok{ggplot}\NormalTok{(economics_2, }\KeywordTok{aes}\NormalTok{(year +}\StringTok{ }\NormalTok{(month -}\StringTok{ }\DecValTok{1}\NormalTok{) /}\StringTok{ }\DecValTok{12}\NormalTok{, rate)) +}
\StringTok{  }\KeywordTok{geom_line}\NormalTok{()}

\KeywordTok{ggplot}\NormalTok{(economics_2, }\KeywordTok{aes}\NormalTok{(month, rate, }\DataTypeTok{group =} \NormalTok{year)) +}
\StringTok{  }\KeywordTok{geom_line}\NormalTok{(}\KeywordTok{aes}\NormalTok{(}\DataTypeTok{colour =} \NormalTok{year), }\DataTypeTok{size =} \DecValTok{1}\NormalTok{)}
\end{Highlighting}
\end{Shaded}

\begin{figure}[H]
  \includegraphics[width=0.5\linewidth]{_figures/tidy_data/ec2-plots-1}%
  \includegraphics[width=0.5\linewidth]{_figures/tidy_data/ec2-plots-2}
\end{figure}

\subsection{Spread}

\texttt{spread()} is the opposite of \texttt{gather()}. You use it when
you have a pair of columns that are in indexed form, instead of
Cartesian form. For example, the following example dataset contains
three variables (\texttt{day}, \texttt{rain} and \texttt{temp}), but
\texttt{rain} and \texttt{temp} are stored in indexed form.
\indexf{spread}

\begin{Shaded}
\begin{Highlighting}[]
\NormalTok{weather <-}\StringTok{ }\NormalTok{dplyr::}\KeywordTok{data_frame}\NormalTok{(}
  \DataTypeTok{day =} \KeywordTok{rep}\NormalTok{(}\DecValTok{1}\NormalTok{:}\DecValTok{3}\NormalTok{, }\DecValTok{2}\NormalTok{),}
  \DataTypeTok{obs =} \KeywordTok{rep}\NormalTok{(}\KeywordTok{c}\NormalTok{(}\StringTok{"temp"}\NormalTok{, }\StringTok{"rain"}\NormalTok{), }\DataTypeTok{each =} \DecValTok{3}\NormalTok{),}
  \DataTypeTok{val =} \KeywordTok{c}\NormalTok{(}\KeywordTok{c}\NormalTok{(}\DecValTok{23}\NormalTok{, }\DecValTok{22}\NormalTok{, }\DecValTok{20}\NormalTok{), }\KeywordTok{c}\NormalTok{(}\DecValTok{0}\NormalTok{, }\DecValTok{0}\NormalTok{, }\DecValTok{5}\NormalTok{))}
\NormalTok{)}
\NormalTok{weather}
\CommentTok{#> Source: local data frame [6 x 3]}
\CommentTok{#> }
\CommentTok{#>     day   obs   val}
\CommentTok{#>   (int) (chr) (dbl)}
\CommentTok{#> 1     1  temp    23}
\CommentTok{#> 2     2  temp    22}
\CommentTok{#> 3     3  temp    20}
\CommentTok{#> 4     1  rain     0}
\CommentTok{#> 5     2  rain     0}
\CommentTok{#> 6     3  rain     5}
\end{Highlighting}
\end{Shaded}

Spread allows us to turn this messy indexed form into a tidy Cartesian
form. It shares many of the arguments with \texttt{gather()}. You'll
need to supply the \texttt{data} to translate, as well as the name of
the \texttt{key} column which gives the variable names, the
\texttt{value} column which contains the cell values. Here the key is
\texttt{obs} and the value is \texttt{val}:

\begin{Shaded}
\begin{Highlighting}[]
\KeywordTok{spread}\NormalTok{(weather, }\DataTypeTok{key =} \NormalTok{obs, }\DataTypeTok{value =} \NormalTok{val)}
\CommentTok{#> Source: local data frame [3 x 3]}
\CommentTok{#> }
\CommentTok{#>     day  rain  temp}
\CommentTok{#>   (int) (dbl) (dbl)}
\CommentTok{#> 1     1     0    23}
\CommentTok{#> 2     2     0    22}
\CommentTok{#> 3     3     5    20}
\end{Highlighting}
\end{Shaded}

\subsection{Exercises}

\begin{enumerate}
\def\labelenumi{\arabic{enumi}.}
\item
  How can you translate each of the initial example datasets into the
  other form?
\item
  How can you convert back and forth between the \texttt{economics} and
  \texttt{economics\_long} datasets built into ggplot2?
\item
  Install the EDAWR package from \url{https://github.com/rstudio/EDAWR}.
  Tidy the \texttt{storms}, \texttt{population} and \texttt{tb}
  datasets.
\end{enumerate}

\section{Separate and unite}\label{sec:separate-unite}

Spread and gather help when the variables are in the wrong place in the
dataset. Separate and unite help when multiple variables are crammed
into one column, or spread across multiple columns. \indexf{separate}
\indexf{unite}

For example, the following dataset stores some information about the
response to a medical treatment. There are three variables (time,
treatment and value), but time and treatment are jammed in one variable
together:

\begin{Shaded}
\begin{Highlighting}[]
\NormalTok{trt <-}\StringTok{ }\NormalTok{dplyr::}\KeywordTok{data_frame}\NormalTok{(}
  \DataTypeTok{var =} \KeywordTok{paste0}\NormalTok{(}\KeywordTok{rep}\NormalTok{(}\KeywordTok{c}\NormalTok{(}\StringTok{"beg"}\NormalTok{, }\StringTok{"end"}\NormalTok{), }\DataTypeTok{each =} \DecValTok{3}\NormalTok{), }\StringTok{"_"}\NormalTok{, }\KeywordTok{rep}\NormalTok{(}\KeywordTok{c}\NormalTok{(}\StringTok{"a"}\NormalTok{, }\StringTok{"b"}\NormalTok{, }\StringTok{"c"}\NormalTok{))),}
  \DataTypeTok{val =} \KeywordTok{c}\NormalTok{(}\DecValTok{1}\NormalTok{, }\DecValTok{4}\NormalTok{, }\DecValTok{2}\NormalTok{, }\DecValTok{10}\NormalTok{, }\DecValTok{5}\NormalTok{, }\DecValTok{11}\NormalTok{)}
\NormalTok{)}
\NormalTok{trt}
\CommentTok{#> Source: local data frame [6 x 2]}
\CommentTok{#> }
\CommentTok{#>     var   val}
\CommentTok{#>   (chr) (dbl)}
\CommentTok{#> 1 beg_a     1}
\CommentTok{#> 2 beg_b     4}
\CommentTok{#> 3 beg_c     2}
\CommentTok{#> 4 end_a    10}
\CommentTok{#> 5 end_b     5}
\CommentTok{#> 6 end_c    11}
\end{Highlighting}
\end{Shaded}

The \texttt{separate()} function makes it easy to tease apart multiple
variables stored in one column. It takes four arguments:

\begin{itemize}
\item
  \texttt{data}: the data frame to modify.
\item
  \texttt{col}: the name of the variable to split into pieces.
\item
  \texttt{into}: a character vector giving the names of the new
  variables.
\item
  \texttt{sep}: a description of how to split the variable apart. This
  can either be a regular expression, e.g. \texttt{\_} to split by
  underscores, or \texttt{{[}\^{}a-z{]}} to split by any non-letter, or
  an integer giving a position.
\end{itemize}

In this case, we want to split by the \texttt{\_} character:

\begin{Shaded}
\begin{Highlighting}[]
\KeywordTok{separate}\NormalTok{(trt, var, }\KeywordTok{c}\NormalTok{(}\StringTok{"time"}\NormalTok{, }\StringTok{"treatment"}\NormalTok{), }\StringTok{"_"}\NormalTok{)}
\CommentTok{#> Source: local data frame [6 x 3]}
\CommentTok{#> }
\CommentTok{#>    time treatment   val}
\CommentTok{#>   (chr)     (chr) (dbl)}
\CommentTok{#> 1   beg         a     1}
\CommentTok{#> 2   beg         b     4}
\CommentTok{#> 3   beg         c     2}
\CommentTok{#> 4   end         a    10}
\CommentTok{#> 5   end         b     5}
\CommentTok{#> 6   end         c    11}
\end{Highlighting}
\end{Shaded}

(If the variables are combined in a more complex form, have a look at
\texttt{extract()}. Alternatively, you might need to create columns
individually yourself using other calculations. A useful tool for this
is \texttt{mutate()} which you'll learn about in the next chapter.)

\texttt{unite()} is the inverse of \texttt{separate()} - it joins
together multiple columns into one column. This is less common, but it's
useful to know about as the inverse of \texttt{separate()}.

\subsection{Exercises}

\begin{enumerate}
\def\labelenumi{\arabic{enumi}.}
\item
  Install the EDAWR package from \url{https://github.com/rstudio/EDAWR}.
  Tidy the \texttt{who} dataset.
\item
  Work through the demos included in the tidyr package
  (\texttt{demo(package\ =\ "tidyr")})
\end{enumerate}

\section{Case studies}\label{sec:tidy-case-study}

For most real datasets, you'll need to use more than one tidying verb.
There many be multiple ways to get there, but as long as each step makes
the data tidier, you'll eventually get to the tidy dataset. That said,
you typically apply the functions in the same order: \texttt{gather()},
\texttt{separate()} and \texttt{spread()} (although you might not use
all three).

\subsection{Blood pressure}

The first step when tidying a new dataset is always to identify the
variables. Take the following simulated medical data. There are seven
variables in this dataset: name, age, start date, week, systolic \&
diastolic blood pressure. Can you see how they're stored?

\begin{Shaded}
\begin{Highlighting}[]
\CommentTok{# Adapted from example by Barry Rowlingson, }
\CommentTok{# http://barryrowlingson.github.io/hadleyverse/}
\NormalTok{bpd <-}\StringTok{ }\NormalTok{readr::}\KeywordTok{read_table}\NormalTok{(}
\StringTok{"name age      start  week1  week2  week3}
\StringTok{Anne  35 2014-03-27 100/80 100/75 120/90}
\StringTok{ Ben  41 2014-03-09 110/65 100/65 135/70}
\StringTok{Carl  33 2014-04-02 125/80   <NA>   <NA>}
\StringTok{"}\NormalTok{, }\DataTypeTok{na =} \StringTok{"<NA>"}\NormalTok{)}
\end{Highlighting}
\end{Shaded}

The first step is to convert from Cartesian to indexed form:

\begin{Shaded}
\begin{Highlighting}[]
\NormalTok{bpd_1 <-}\StringTok{ }\KeywordTok{gather}\NormalTok{(bpd, week, bp, week1:week3)}
\NormalTok{bpd_1}
\CommentTok{#> Source: local data frame [9 x 5]}
\CommentTok{#> }
\CommentTok{#>     name   age start   week     bp}
\CommentTok{#>    (chr) (int) (int) (fctr)  (chr)}
\CommentTok{#> 1   Anne    35 16156  week1 100/80}
\CommentTok{#> 2    Ben    41 16138  week1 110/65}
\CommentTok{#> 3   Carl    33 16162  week1 125/80}
\CommentTok{#> 4   Anne    35 16156  week2 100/75}
\CommentTok{#> 5    Ben    41 16138  week2 100/65}
\CommentTok{#> 6   Carl    33 16162  week2     NA}
\CommentTok{#> ..   ...   ...   ...    ...    ...}
\end{Highlighting}
\end{Shaded}

This is tidier, but we have two variables combined together in the
\texttt{bp} variable. This is a common way of writing down the blood
pressure, but analysis is easier if we break into two variables. That's
the job of separate:

\begin{Shaded}
\begin{Highlighting}[]
\NormalTok{bpd_2 <-}\StringTok{ }\KeywordTok{separate}\NormalTok{(bpd_1, bp, }\KeywordTok{c}\NormalTok{(}\StringTok{"sys"}\NormalTok{, }\StringTok{"dia"}\NormalTok{), }\StringTok{"/"}\NormalTok{)}
\NormalTok{bpd_2}
\CommentTok{#> Source: local data frame [9 x 6]}
\CommentTok{#> }
\CommentTok{#>     name   age start   week   sys   dia}
\CommentTok{#>    (chr) (int) (int) (fctr) (chr) (chr)}
\CommentTok{#> 1   Anne    35 16156  week1   100    80}
\CommentTok{#> 2    Ben    41 16138  week1   110    65}
\CommentTok{#> 3   Carl    33 16162  week1   125    80}
\CommentTok{#> 4   Anne    35 16156  week2   100    75}
\CommentTok{#> 5    Ben    41 16138  week2   100    65}
\CommentTok{#> 6   Carl    33 16162  week2    NA    NA}
\CommentTok{#> ..   ...   ...   ...    ...   ...   ...}
\end{Highlighting}
\end{Shaded}

This dataset is now tidy, but we could do a little more to make it
easier to use. The following code uses \texttt{extract()} to pull the
week number out into its own variable (using regular expressions is
beyond the scope of the book, but
\texttt{\textbackslash{}\textbackslash{}d} stands for any digit). I also
use \texttt{arrange()} (which you'll learn about in the next chapter) to
order the rows to keep the records for each person together.

\begin{Shaded}
\begin{Highlighting}[]
\NormalTok{bpd_3 <-}\StringTok{ }\KeywordTok{extract}\NormalTok{(bpd_2, week, }\StringTok{"week"}\NormalTok{, }\StringTok{"(}\CharTok{\textbackslash{}\textbackslash{}}\StringTok{d)"}\NormalTok{, }\DataTypeTok{convert =} \OtherTok{TRUE}\NormalTok{)}
\NormalTok{bpd_4 <-}\StringTok{ }\NormalTok{dplyr::}\KeywordTok{arrange}\NormalTok{(bpd_3, name, week)}
\NormalTok{bpd_4}
\CommentTok{#> Source: local data frame [9 x 6]}
\CommentTok{#> }
\CommentTok{#>     name   age start  week   sys   dia}
\CommentTok{#>    (chr) (int) (int) (int) (chr) (chr)}
\CommentTok{#> 1   Anne    35 16156     1   100    80}
\CommentTok{#> 2   Anne    35 16156     2   100    75}
\CommentTok{#> 3   Anne    35 16156     3   120    90}
\CommentTok{#> 4    Ben    41 16138     1   110    65}
\CommentTok{#> 5    Ben    41 16138     2   100    65}
\CommentTok{#> 6    Ben    41 16138     3   135    70}
\CommentTok{#> ..   ...   ...   ...   ...   ...   ...}
\end{Highlighting}
\end{Shaded}

You might notice that there's some repetition in this dataset: if you
know the name, then you also know the age and start date. This reflects
a third condition of tidyness that I don't discuss here: each data frame
should contain one and only one data set. Here there are really two
datasets: information about each person that doesn't change over time,
and their weekly blood pressure measurements. You can learn more about
this sort of messiness in the resources mentioned at the end of the
chapter.

\subsection{Test scores}

Imagine you're interested in the effect of an intervention on test
scores. You've collected the following data. What are the variables?

\begin{Shaded}
\begin{Highlighting}[]
\CommentTok{# Adapted from http://stackoverflow.com/questions/29775461}
\NormalTok{scores <-}\StringTok{ }\NormalTok{dplyr::}\KeywordTok{data_frame}\NormalTok{(}
  \DataTypeTok{person =} \KeywordTok{rep}\NormalTok{(}\KeywordTok{c}\NormalTok{(}\StringTok{"Greg"}\NormalTok{, }\StringTok{"Sally"}\NormalTok{, }\StringTok{"Sue"}\NormalTok{), }\DataTypeTok{each =} \DecValTok{2}\NormalTok{),}
  \DataTypeTok{time   =} \KeywordTok{rep}\NormalTok{(}\KeywordTok{c}\NormalTok{(}\StringTok{"pre"}\NormalTok{, }\StringTok{"post"}\NormalTok{), }\DecValTok{3}\NormalTok{),}
  \DataTypeTok{test1  =} \KeywordTok{round}\NormalTok{(}\KeywordTok{rnorm}\NormalTok{(}\DecValTok{6}\NormalTok{, }\DataTypeTok{mean =} \DecValTok{80}\NormalTok{, }\DataTypeTok{sd =} \DecValTok{4}\NormalTok{), }\DecValTok{0}\NormalTok{),}
  \DataTypeTok{test2  =} \KeywordTok{round}\NormalTok{(}\KeywordTok{jitter}\NormalTok{(test1, }\DecValTok{15}\NormalTok{), }\DecValTok{0}\NormalTok{)}
\NormalTok{)}
\NormalTok{scores}
\CommentTok{#> Source: local data frame [6 x 4]}
\CommentTok{#> }
\CommentTok{#>   person  time test1 test2}
\CommentTok{#>    (chr) (chr) (dbl) (dbl)}
\CommentTok{#> 1   Greg   pre    84    83}
\CommentTok{#> 2   Greg  post    76    75}
\CommentTok{#> 3  Sally   pre    80    78}
\CommentTok{#> 4  Sally  post    78    77}
\CommentTok{#> 5    Sue   pre    83    80}
\CommentTok{#> 6    Sue  post    76    75}
\end{Highlighting}
\end{Shaded}

I think the variables are person, test, pre-test score and post-test
score. As usual, we start by converting columns in Cartesian form
(\texttt{test1} and \texttt{test2}) to indexed form (\texttt{test} and
\texttt{score}):

\begin{Shaded}
\begin{Highlighting}[]
\NormalTok{scores_1 <-}\StringTok{ }\KeywordTok{gather}\NormalTok{(scores, test, score, test1:test2)}
\NormalTok{scores_1}
\CommentTok{#> Source: local data frame [12 x 4]}
\CommentTok{#> }
\CommentTok{#>    person  time   test score}
\CommentTok{#>     (chr) (chr) (fctr) (dbl)}
\CommentTok{#> 1    Greg   pre  test1    84}
\CommentTok{#> 2    Greg  post  test1    76}
\CommentTok{#> 3   Sally   pre  test1    80}
\CommentTok{#> 4   Sally  post  test1    78}
\CommentTok{#> 5     Sue   pre  test1    83}
\CommentTok{#> 6     Sue  post  test1    76}
\CommentTok{#> ..    ...   ...    ...   ...}
\end{Highlighting}
\end{Shaded}

Now we need to do the opposite: \texttt{pre} and \texttt{post} should be
variables, not values, so we need to spread \texttt{time} and
\texttt{score}:

\begin{Shaded}
\begin{Highlighting}[]
\NormalTok{scores_2 <-}\StringTok{ }\KeywordTok{spread}\NormalTok{(scores_1, time, score)}
\NormalTok{scores_2}
\CommentTok{#> Source: local data frame [6 x 4]}
\CommentTok{#> }
\CommentTok{#>   person   test  post   pre}
\CommentTok{#>    (chr) (fctr) (dbl) (dbl)}
\CommentTok{#> 1   Greg  test1    76    84}
\CommentTok{#> 2   Greg  test2    75    83}
\CommentTok{#> 3  Sally  test1    78    80}
\CommentTok{#> 4  Sally  test2    77    78}
\CommentTok{#> 5    Sue  test1    76    83}
\CommentTok{#> 6    Sue  test2    75    80}
\end{Highlighting}
\end{Shaded}

A good indication that we have made a tidy dataset is that it's now easy
to calculate the statistic of interest: the difference between pre- and
post-intervention scores:

\begin{Shaded}
\begin{Highlighting}[]
\NormalTok{scores_3 <-}\StringTok{ }\KeywordTok{mutate}\NormalTok{(scores_2, }\DataTypeTok{diff =} \NormalTok{post -}\StringTok{ }\NormalTok{pre)}
\NormalTok{scores_3}
\CommentTok{#> Source: local data frame [6 x 5]}
\CommentTok{#> }
\CommentTok{#>   person   test  post   pre  diff}
\CommentTok{#>    (chr) (fctr) (dbl) (dbl) (dbl)}
\CommentTok{#> 1   Greg  test1    76    84    -8}
\CommentTok{#> 2   Greg  test2    75    83    -8}
\CommentTok{#> 3  Sally  test1    78    80    -2}
\CommentTok{#> 4  Sally  test2    77    78    -1}
\CommentTok{#> 5    Sue  test1    76    83    -7}
\CommentTok{#> 6    Sue  test2    75    80    -5}
\end{Highlighting}
\end{Shaded}

And it's similarly easy to plot:

\begin{Shaded}
\begin{Highlighting}[]
\KeywordTok{ggplot}\NormalTok{(scores_3, }\KeywordTok{aes}\NormalTok{(person, diff, }\DataTypeTok{color =} \NormalTok{test)) +}
\StringTok{  }\KeywordTok{geom_hline}\NormalTok{(}\DataTypeTok{size =} \DecValTok{2}\NormalTok{, }\DataTypeTok{colour =} \StringTok{"white"}\NormalTok{, }\DataTypeTok{yintercept =} \DecValTok{0}\NormalTok{) +}
\StringTok{  }\KeywordTok{geom_point}\NormalTok{() +}
\StringTok{  }\KeywordTok{geom_path}\NormalTok{(}\KeywordTok{aes}\NormalTok{(}\DataTypeTok{group =} \NormalTok{person), }\DataTypeTok{colour =} \StringTok{"grey50"}\NormalTok{, }
    \DataTypeTok{arrow =} \KeywordTok{arrow}\NormalTok{(}\DataTypeTok{length =} \KeywordTok{unit}\NormalTok{(}\FloatTok{0.25}\NormalTok{, }\StringTok{"cm"}\NormalTok{)))}
\end{Highlighting}
\end{Shaded}

\begin{figure}[H]
  \includegraphics[width=0.5\linewidth]{_figures/tidy_data/scores4-1}
\end{figure}

(Again, you'll learn about \texttt{mutate()} in the next chapter.)

\section{Learning more}

Data tidying is a big topic and this chapter only scratches the surface.
I recommend the following references which go into considerably more
depth on this topic:

\begin{itemize}
\item
  The tidyr documentation. I've described the most important arguments,
  but most functions have other arguments that help deal with less
  common situations. If you're struggling, make sure to read the
  documentation to see if there's an argument that might help you.
\item
  ``\href{http://www.jstatsoft.org/v59/i10/}{Tidy data}'', an article in
  the \emph{Journal of Statistical Software}. It describes the ideas of
  tidy data in more depth and shows other types of messy data.
  Unfortunately the paper was written before tidyr existed, so to see
  how to use tidyr instead of reshape2, consult the
  \href{http://cran.r-project.org/web/packages/tidyr/vignettes/tidy-data.html}{tidyr
  vignette}.
\item
  The \href{http://rstudio.com/cheatsheets}{data wrangling cheatsheet}
  by RStudio, includes the most common tidyr verbs in a form designed to
  jog your memory when you're stuck.
\end{itemize}

\section*{References}
\addcontentsline{toc}{section}{References}

\hyperdef{}{ref-wickham:2007b}{\label{ref-wickham:2007b}}
Wickham, Hadley. 2007. ``Reshaping Data with the Reshape Package.''
\emph{Journal of Statistical Software} 21 (12).
\url{http://www.jstatsoft.org/v21/i12/paper}.

\hyperdef{}{ref-tidy-data}{\label{ref-tidy-data}}
---------. 2014. ``Tidy Data.'' \emph{The Journal of Statistical
Software} 59. \url{http://www.jstatsoft.org/v59/i10/}.

\chapter{Data transformation}\label{cha:dplyr}

\section{Introduction}

Tidy data is important, but it's not the end of the road. Often you
won't have quite the right variables, or your data might need a little
aggregation before you visualise it. This chapter will show you how to
solve these problems (and more!) with the \textbf{dplyr} package.
\index{Data!manipulating} \index{dplyr}
\index{Grammar!of data manipulation}

The goal of dplyr is to provide verbs (functions) that help you solve
the most common 95\% of data manipulation problems. dplyr is similar to
ggplot2, but instead of providing a grammar of graphics, it provides a
grammar of data manipulation. Like ggplot2, dplyr helps you not just by
giving you functions, but it also helps you think about data
manipulation. In particular, dplyr helps by constraining you: instead of
struggling to think about which of the thousands of functions that might
help, you can just pick from a handful that are design to be very likely
to be helpful. In this chapter you'll learn four of the most important
dplyr verbs:

\begin{itemize}
\tightlist
\item
  \texttt{filter()}
\item
  \texttt{mutate()}
\item
  \texttt{group\_by()} \& \texttt{summarise()}
\end{itemize}

These verbs are easy to learn because they all work the same way: they
take a data frame as the first argument, and return a modified data
frame. The other arguments control the details of the transformation,
and are always interpreted in the context of the data frame so you can
refer to variables directly. I'll also explain each in the same way:
I'll show you a motivating examples using the \texttt{diamonds} data,
give you more details about how the function works, and finish up with
some exercises for you to practice your skills with.

You'll also learn how to create data transformation pipelines using
\texttt{\%\textgreater{}\%}. \texttt{\%\textgreater{}\%} plays a similar
role to \texttt{+} in ggplot2: it allows you to solve complex problems
by combining small pieces that are easily understood in isolation.

This chapter only scratches the surface of dplyr's capabilities but it
should be enough to get help you with visualisation problems. You can
learn more by using the resources discussed at the end of the chapter.

\section{Filter observations}

It's common to only want to explore one part of a dataset. A great data
analysis strategy is to start with just one observation unit (one
person, one city, etc), and understand how it works before attempting to
generalise the conclusion to others. This is a great technique if you
ever feel overwhelmed by an analysis: zoom down to a small subset,
master it, and then zoom back out, to apply your conclusions to the full
dataset. \indexf{filter}

Filtering is also useful for extracting outliers. Generally, you don't
want to just throw outliers away, as they're often highly revealing, but
it's useful to think about partitioning the data into the common and the
unusual. You summarise the common to look at the broad trends; you
examine the outliers individually to see if you can figure out what's
going on.

For example, look at this plot that shows how the x and y dimensions of
the diamonds are related:

\begin{Shaded}
\begin{Highlighting}[]
\KeywordTok{ggplot}\NormalTok{(diamonds, }\KeywordTok{aes}\NormalTok{(x, y)) +}\StringTok{ }
\StringTok{  }\KeywordTok{geom_bin2d}\NormalTok{()}
\end{Highlighting}
\end{Shaded}

\begin{figure}[H]
  \centering
  \includegraphics[width=0.65\linewidth]{_figures/data-manip/diamonds-x-y-1}
\end{figure}

There are around 50,000 points in this dataset: most of them lie along
the diagonal, but there are a handful of outliers. One clear set of
incorrect values are those diamonds with zero dimensions. We can use
\texttt{filter()} to pull them out:

\begin{Shaded}
\begin{Highlighting}[]
\KeywordTok{filter}\NormalTok{(diamonds, x ==}\StringTok{ }\DecValTok{0} \NormalTok{|}\StringTok{ }\NormalTok{y ==}\StringTok{ }\DecValTok{0}\NormalTok{)}
\CommentTok{#> Source: local data frame [8 x 10]}
\CommentTok{#> }
\CommentTok{#>    carat       cut  color clarity depth table price     x     y}
\CommentTok{#>    (dbl)    (fctr) (fctr)  (fctr) (dbl) (dbl) (int) (dbl) (dbl)}
\CommentTok{#> 1   1.07     Ideal      F     SI2  61.6    56  4954     0  6.62}
\CommentTok{#> 2   1.00 Very Good      H     VS2  63.3    53  5139     0  0.00}
\CommentTok{#> 3   1.14      Fair      G     VS1  57.5    67  6381     0  0.00}
\CommentTok{#> 4   1.56     Ideal      G     VS2  62.2    54 12800     0  0.00}
\CommentTok{#> 5   1.20   Premium      D    VVS1  62.1    59 15686     0  0.00}
\CommentTok{#> 6   2.25   Premium      H     SI2  62.8    59 18034     0  0.00}
\CommentTok{#> ..   ...       ...    ...     ...   ...   ...   ...   ...   ...}
\CommentTok{#>        z}
\CommentTok{#>    (dbl)}
\CommentTok{#> 1      0}
\CommentTok{#> 2      0}
\CommentTok{#> 3      0}
\CommentTok{#> 4      0}
\CommentTok{#> 5      0}
\CommentTok{#> 6      0}
\CommentTok{#> ..   ...}
\end{Highlighting}
\end{Shaded}

This is equivalent to the base R code
\texttt{diamonds{[}diamonds\$x\ ==\ 0\ \textbar{}\ diamonds\$y\ ==\ 0,\ {]}},
but is more concise because \texttt{filter()} when you refer to (e.g.)
the bare \texttt{x}, dplyr knows to look first in the data frame.

(If you've used \texttt{subset()} before, you'll notice that it has very
similar behaviour. The biggest difference is that \texttt{subset()} can
select both observations and variables, where in dplyr,
\texttt{filter()} works exclusively with observations and
\texttt{select()} with variables. There are some other subtle
differences, but the main advantage to using \texttt{filter()} is that
it behaves identically to the other dplyr verbs and it tends to be a bit
faster than \texttt{subset()}.)

In a real analysis, you'd look at the outliers in more detail to see if
you can find the root cause of the data quality problem. In this case,
we're just going to throw them out and focus on what remains. To save
some typing, we may provide multiple arguments to \texttt{filter()}
which combines them.

\begin{Shaded}
\begin{Highlighting}[]
\NormalTok{diamonds_ok <-}\StringTok{ }\KeywordTok{filter}\NormalTok{(diamonds, x >}\StringTok{ }\DecValTok{0}\NormalTok{, y >}\StringTok{ }\DecValTok{0}\NormalTok{, y <}\StringTok{ }\DecValTok{20}\NormalTok{)}
\KeywordTok{ggplot}\NormalTok{(diamonds_ok, }\KeywordTok{aes}\NormalTok{(x, y)) +}
\StringTok{  }\KeywordTok{geom_bin2d}\NormalTok{() +}
\StringTok{  }\KeywordTok{geom_abline}\NormalTok{(}\DataTypeTok{slope =} \DecValTok{1}\NormalTok{, }\DataTypeTok{colour =} \StringTok{"white"}\NormalTok{, }\DataTypeTok{size =} \DecValTok{1}\NormalTok{, }\DataTypeTok{alpha =} \FloatTok{0.5}\NormalTok{)}
\end{Highlighting}
\end{Shaded}

\begin{figure}[H]
  \centering
  \includegraphics[width=0.65\linewidth]{_figures/data-manip/diamonds-ok-1}
\end{figure}

This plot is now more informative - we can see a very strong
relationship between \texttt{x} and \texttt{y}. I've added the reference
line to make it clear that for most diamonds, \texttt{x} and \texttt{y}
are very similar. However, this plot still has problems:

\begin{itemize}
\item
  The plot is mostly empty, because most of the data lies along the
  diagonal.
\item
  There are some clear bivariate outliers, but it's hard to select them
  with a simple filter.
\end{itemize}

We'll solve both of these problem in the next section by adding a new
variable that's a transformation of x and y. But before we continue on
to that, lets talk more about the details of \texttt{filter()}.

\subsection{Useful tools}

The first argument to \texttt{filter()} is a data frame. The second and
subsequent arguments must be logical vectors: \texttt{filter()} selects
every row where all the logical expressions are \texttt{TRUE}. The
logical vectors must always be the same length as the data frame: if
not, you'll get an error. Typically you create the logical vector with
the comparison operators:

\begin{itemize}
\tightlist
\item
  \texttt{x\ ==\ y}: x and y are equal.
\item
  \texttt{x\ !=\ y}: x and y are not equal.
\item
  \texttt{x\ \%in\%\ c("a",\ "b",\ "c")}: \texttt{x} is one of the
  values in the right hand side.
\item
  \texttt{x\ \textgreater{}\ y}, \texttt{x\ \textgreater{}=\ y},
  \texttt{x\ \textless{}\ y}, \texttt{x\ \textless{}=\ y}: greater than,
  greater than or equal to, less than, less than or equal to.
\end{itemize}

And combine them with logical operators:

\begin{itemize}
\tightlist
\item
  \texttt{!x} (pronounced ``not x''), flips \texttt{TRUE} and
  \texttt{FALSE} so it keeps all the values where \texttt{x} is
  \texttt{FALSE}.
\item
  \texttt{x\ \&\ y}: \texttt{TRUE} if both \texttt{x} and \texttt{y} are
  \texttt{TRUE}.
\item
  \texttt{x\ \textbar{}\ y}: \texttt{TRUE} if either \texttt{x} or
  \texttt{y} are \texttt{TRUE}.
\item
  \texttt{xor(x,\ y)}: \texttt{TRUE} if either \texttt{x} or \texttt{y}
  are \texttt{TRUE}, but not both (exclusive or).
\end{itemize}

Most real queries involve some combination of both:

\begin{itemize}
\tightlist
\item
  Price less than \$500: \texttt{price\ \textless{}\ 500}
\item
  Size between 1 and 2 carats:
  \texttt{carat\ \textgreater{}=\ 1\ \&\ carat\ \textless{}\ 2}
\item
  Cut is ideal or premium:
  \texttt{cut\ ==\ "Premium"\ \textbar{}\ cut\ ==\ "Ideal"}, or
  \texttt{cut\ \%in\%\ c("Premium",\ "Ideal")} (note that R is case
  sensitive)
\item
  Worst colour, cut and clarify:
  \texttt{cut\ ==\ "Fair"\ \&\ color\ ==\ "J"\ \&\ clarity\ ==\ "SI2"}
\end{itemize}

You can also use functions in the filtering expression:

\begin{itemize}
\tightlist
\item
  Size is between 1 and 2 carats: \texttt{floor(carat)\ ==\ 1}
\item
  An average dimension greater than 3:
  \texttt{(x\ +\ y\ +\ z)\ /\ 3\ \textgreater{}\ 3}
\end{itemize}

This is useful for simple expressions, but as things get more
complicated it's better to create a new variable first so you can check
that you've done the computation correctly before doing the subsetting.
You'll learn how to do that in the next section.

The rules for \texttt{NA} are a bit trickier, so I'll explain them next.

\subsection{Missing values}

\texttt{NA}, R's missing value indicator, can be frustrating to work
with. R's underlying philosophy is to force you to recognise that you
have missing values, and make a deliberate choice to deal with them:
missing values never silently go missing. This is a pain because you
almost always want to just get rid of them, but it's a good principle to
force you to think about the correct option. \indexc{NA}
\index{Missing values}

The most important thing to understand about missing values is that they
are infectious: with few exceptions, the result of any operation that
includes a missing value will be a missing value. This happens because
\texttt{NA} represents an unknown value, and there are few operations
that turn an unknown value into a known value.

\begin{Shaded}
\begin{Highlighting}[]
\NormalTok{x <-}\StringTok{ }\KeywordTok{c}\NormalTok{(}\DecValTok{1}\NormalTok{, }\OtherTok{NA}\NormalTok{, }\DecValTok{2}\NormalTok{)}
\NormalTok{x ==}\StringTok{ }\DecValTok{1}
\CommentTok{#> [1]  TRUE    NA FALSE}
\NormalTok{x >}\StringTok{ }\DecValTok{2}
\CommentTok{#> [1] FALSE    NA FALSE}
\NormalTok{x +}\StringTok{ }\DecValTok{10}
\CommentTok{#> [1] 11 NA 12}
\end{Highlighting}
\end{Shaded}

When you first learn R, you might be tempted to find missing values
using \texttt{==}:

\begin{Shaded}
\begin{Highlighting}[]
\NormalTok{x ==}\StringTok{ }\OtherTok{NA}
\CommentTok{#> [1] NA NA NA}
\NormalTok{x !=}\StringTok{ }\OtherTok{NA}
\CommentTok{#> [1] NA NA NA}
\end{Highlighting}
\end{Shaded}

But that doesn't work! A little thought reveals why: there's no reason
why two unknown values should be the same. Instead, use
\texttt{is.na(X)} to determine if a value is missing: \indexf{is.na}

\begin{Shaded}
\begin{Highlighting}[]
\KeywordTok{is.na}\NormalTok{(x)}
\CommentTok{#> [1] FALSE  TRUE FALSE}
\end{Highlighting}
\end{Shaded}

\texttt{filter()} only includes observations where all arguments are
\texttt{TRUE}, so \texttt{NA} values are automatically dropped. If you
want to include missing values, be explicit:
\texttt{x\ \textgreater{}\ 10\ \textbar{}\ is.na(x)}. In other parts of
R, you'll sometimes need to convert missing values into \texttt{FALSE}.
You can do that with \texttt{x\ \textgreater{}\ 10\ \&\ !is.na(x)}

\subsection{Exercises}

\begin{enumerate}
\def\labelenumi{\arabic{enumi}.}
\item
  Practice your filtering skills by:

  \begin{itemize}
  \tightlist
  \item
    Finding all the diamonds with equal x and y dimensions.
  \item
    A depth between 55 and 70.
  \item
    A carat smaller than the median carat.
  \item
    Cost more than \$10,000 per carat
  \item
    Are of good or better quality
  \end{itemize}
\item
  Fill in the question marks in this table:

  \begin{longtable}[c]{@{}llll@{}}
  \toprule
  Expression & \texttt{TRUE} & \texttt{FALSE} &
  \texttt{NA}\tabularnewline
  \midrule
  \endhead
  \texttt{x} & • &\tabularnewline
  ? & & •\tabularnewline
  \texttt{is.na(x)} & & & •\tabularnewline
  \texttt{!is.na(x)} & ? & ? & ?\tabularnewline
  ? & • & & •\tabularnewline
  ? & & • & •\tabularnewline
  \bottomrule
  \end{longtable}
\item
  Repeat the analysis of outlying values with \texttt{z}. Compared to
  \texttt{x} and \texttt{y}, how would you characterise the relationship
  of \texttt{x} and \texttt{z}, or \texttt{y} and \texttt{z}?
\item
  Install the \textbf{ggplot2movies} package and look at the movies that
  have a missing budget. How are they different from the movies with a
  budget? (Hint: try a frequency polygon plus
  \texttt{colour\ =\ is.na(budget)}.)
\item
  What is \texttt{NA\ \&\ FALSE} and \texttt{NA\ \textbar{}\ TRUE}? Why?
  Why doesn't \texttt{NA\ *\ 0} equal zero? What number times zero does
  not equal 0? What do you expect \texttt{NA\ \^{}\ 0} to equal? Why?
\end{enumerate}

\section{Create new variables}\label{mutate}

To better explore the relationship between \texttt{x} and \texttt{y},
it's useful to ``rotate'' the plot so that the data is flat, not
diagonal. We can do that by creating two new variables: one that
represents the difference between \texttt{x} and \texttt{y} (which in
this context represents the symmetry of the diamond) and one that
represents its size (the length of the diagonal). \indexf{mutate}
\index{Data!creating new variables}

To create new variables use \texttt{mutate()}. Like \texttt{filter()} it
takes a data frame as its first argument and returns a data frame. It's
second and subsequent arguments are named expressions that generate new
variables. Like \texttt{filter()} you can refer to variables just by
their name, you don't need to also include the name of the dataset.

\begin{Shaded}
\begin{Highlighting}[]
\NormalTok{diamonds_ok2 <-}\StringTok{ }\KeywordTok{mutate}\NormalTok{(diamonds_ok,}
  \DataTypeTok{sym =} \NormalTok{x -}\StringTok{ }\NormalTok{y,}
  \DataTypeTok{size =} \KeywordTok{sqrt}\NormalTok{(x ^}\StringTok{ }\DecValTok{2} \NormalTok{+}\StringTok{ }\NormalTok{y ^}\StringTok{ }\DecValTok{2}\NormalTok{)}
\NormalTok{)}
\NormalTok{diamonds_ok2}
\CommentTok{#> Source: local data frame [53,930 x 12]}
\CommentTok{#> }
\CommentTok{#>    carat       cut  color clarity depth table price     x     y}
\CommentTok{#>    (dbl)    (fctr) (fctr)  (fctr) (dbl) (dbl) (int) (dbl) (dbl)}
\CommentTok{#> 1   0.23     Ideal      E     SI2  61.5    55   326  3.95  3.98}
\CommentTok{#> 2   0.21   Premium      E     SI1  59.8    61   326  3.89  3.84}
\CommentTok{#> 3   0.23      Good      E     VS1  56.9    65   327  4.05  4.07}
\CommentTok{#> 4   0.29   Premium      I     VS2  62.4    58   334  4.20  4.23}
\CommentTok{#> 5   0.31      Good      J     SI2  63.3    58   335  4.34  4.35}
\CommentTok{#> 6   0.24 Very Good      J    VVS2  62.8    57   336  3.94  3.96}
\CommentTok{#> ..   ...       ...    ...     ...   ...   ...   ...   ...   ...}
\CommentTok{#>        z}
\CommentTok{#>    (dbl)}
\CommentTok{#> 1   2.43}
\CommentTok{#> 2   2.31}
\CommentTok{#> 3   2.31}
\CommentTok{#> 4   2.63}
\CommentTok{#> 5   2.75}
\CommentTok{#> 6   2.48}
\CommentTok{#> ..   ...}
\CommentTok{#> Variables not shown: sym (dbl), size (dbl)}

\KeywordTok{ggplot}\NormalTok{(diamonds_ok2, }\KeywordTok{aes}\NormalTok{(size, sym)) +}\StringTok{ }
\StringTok{  }\KeywordTok{stat_bin2d}\NormalTok{()}
\end{Highlighting}
\end{Shaded}

\begin{figure}[H]
  \centering
  \includegraphics[width=0.65\linewidth]{_figures/data-manip/mutate1-1}
\end{figure}

This plot has two advantages: we can more easily see the pattern
followed by most diamonds, and we can easily select outliers. Here, it
doesn't seem important whether the outliers are positive (i.e.
\texttt{x} is bigger than \texttt{y}) or negative (i.e. \texttt{y} is
bigger \texttt{x}). So we can use the absolute value of the symmetry
variable to pull out the outliers. Based on the plot, and a little
experimentation, I came up with a threshold of 0.20. We'll check out the
results with a histogram.

\begin{Shaded}
\begin{Highlighting}[]
\KeywordTok{ggplot}\NormalTok{(diamonds_ok2, }\KeywordTok{aes}\NormalTok{(}\KeywordTok{abs}\NormalTok{(sym))) +}\StringTok{ }
\StringTok{  }\KeywordTok{geom_histogram}\NormalTok{(}\DataTypeTok{binwidth =} \FloatTok{0.10}\NormalTok{)}

\NormalTok{diamonds_ok3 <-}\StringTok{ }\KeywordTok{filter}\NormalTok{(diamonds_ok2, }\KeywordTok{abs}\NormalTok{(sym) <}\StringTok{ }\FloatTok{0.20}\NormalTok{)}
\KeywordTok{ggplot}\NormalTok{(diamonds_ok3, }\KeywordTok{aes}\NormalTok{(}\KeywordTok{abs}\NormalTok{(sym))) +}\StringTok{ }
\StringTok{  }\KeywordTok{geom_histogram}\NormalTok{(}\DataTypeTok{binwidth =} \FloatTok{0.01}\NormalTok{)}
\end{Highlighting}
\end{Shaded}

\begin{figure}[H]
  \includegraphics[width=0.5\linewidth]{_figures/data-manip/sym-hist-1}%
  \includegraphics[width=0.5\linewidth]{_figures/data-manip/sym-hist-2}
\end{figure}

That's an interesting histogram! While most diamonds are close to being
symmetric there are very few that are perfectly symmetric (i.e.
\texttt{x\ ==\ y}).

\subsection{Useful tools}

Typically, transformations will be suggested by your domain knowledge.
However, there are a few transformations that are useful in a
surprisingly wide domain range of circumstances.

\begin{itemize}
\item
  Log-transformations are often useful. They turn multiplicative
  relationships into additive relationships; they compress data that
  varies over orders of magnitude; they convert power relationships to
  linear relationship. See examples at
  \url{http://stats.stackexchange.com/questions/27951}
\item
  Relative difference: If you're interested in the relative difference
  between two variables, use \texttt{log(x\ /\ y)}. It's better than
  \texttt{x\ /\ y} because it's symmetric: if x \textless{} y,
  \texttt{x\ /\ y} takes values {[}0, 1), but if x \textgreater{} y,
  \texttt{x\ /\ y} takes values (1, Inf). See Törnqvist, Vartia, and
  Vartia (1985) for more details. \indexf{log}
\item
  Sometimes integrating or differentiating might make the data more
  interpretable: if you have distance and time, would speed or
  acceleration be more useful? (or vice versa). (Note that integration
  makes data more smooth; differentiation makes it less smooth.)
\item
  Partition a number into magnitude and direction with \texttt{abs(x)}
  and \texttt{sign(x)}.
\end{itemize}

There are also a few useful ways to transform pairs of variables:

\begin{itemize}
\item
  Partitioning into overall size and difference is often useful, as seen
  above.
\item
  If you see a strong trend, use a model to partition it into pattern
  and residuals is often useful. You'll learn more about that in the
  next chapter.
\item
  Sometimes it's useful to change positions to polar coordinates (or
  vice versa): distance (\texttt{sqrt(x\^{}2\ +\ y\^{}2)}) and angle
  (\texttt{atan2(y,\ x)}).
\end{itemize}

\subsection{Exercises}

\begin{enumerate}
\def\labelenumi{\arabic{enumi}.}
\item
  Practice your variable creation skills by creating the following new
  variables:

  \begin{itemize}
  \tightlist
  \item
    The approximate volume of the diamond (using x, y, and z).
  \item
    The approximate density of the diamond.
  \item
    The price per carat.
  \item
    Log transformation of carat and price.
  \end{itemize}
\item
  How can you improve the data density of
  \texttt{ggplot(diamonds,\ aes(x,\ z))\ +\ stat\_bin2d()}. What
  transformation makes it easier to extract outliers?
\item
  The depth variable is just the width of the diamond (average of
  \texttt{x} and \texttt{y}) divided by its height (\texttt{z})
  multiplied by 100 and round to the nearest integer. Compute the depth
  yourself and compare it to the existing depth variable. Summarise your
  findings with a plot.
\item
  Compare the distribution of symmetry for diamonds with \(x > y\) vs.
  \(y < x\).
\end{enumerate}

\section{Group-wise summaries}\label{sec:summarise}

Many insightful visualisations require that you reduce the full dataset
down to a meaningful summary. ggplot2 provides a number of geoms that
will do summaries for you. But it's often useful to do summaries by
hand: that gives you more flexibility and you can use the summaries for
other purposes. \indexf{group\_by} \indexf{summarise}

dplyr does summaries in two steps:

\begin{enumerate}
\def\labelenumi{\arabic{enumi}.}
\tightlist
\item
  Define the grouping variables with \texttt{group\_by()}.
\item
  Describe how to summarise each group with a single row with
  \texttt{summarise()}
\end{enumerate}

For example, to look at the average price per clarity, we first group by
clarity, then summarise:

\begin{Shaded}
\begin{Highlighting}[]
\NormalTok{by_clarity <-}\StringTok{ }\KeywordTok{group_by}\NormalTok{(diamonds, clarity)}
\NormalTok{sum_clarity <-}\StringTok{ }\KeywordTok{summarise}\NormalTok{(by_clarity, }\DataTypeTok{price =} \KeywordTok{mean}\NormalTok{(price))}
\NormalTok{sum_clarity}
\CommentTok{#> Source: local data frame [8 x 2]}
\CommentTok{#> }
\CommentTok{#>    clarity price}
\CommentTok{#>     (fctr) (dbl)}
\CommentTok{#> 1       I1  3924}
\CommentTok{#> 2      SI2  5063}
\CommentTok{#> 3      SI1  3996}
\CommentTok{#> 4      VS2  3925}
\CommentTok{#> 5      VS1  3839}
\CommentTok{#> 6     VVS2  3284}
\CommentTok{#> ..     ...   ...}

\KeywordTok{ggplot}\NormalTok{(sum_clarity, }\KeywordTok{aes}\NormalTok{(clarity, price)) +}\StringTok{ }
\StringTok{  }\KeywordTok{geom_line}\NormalTok{(}\KeywordTok{aes}\NormalTok{(}\DataTypeTok{group =} \DecValTok{1}\NormalTok{), }\DataTypeTok{colour =} \StringTok{"grey80"}\NormalTok{) +}
\StringTok{  }\KeywordTok{geom_point}\NormalTok{(}\DataTypeTok{size =} \DecValTok{2}\NormalTok{)}
\end{Highlighting}
\end{Shaded}

\begin{figure}[H]
  \centering
  \includegraphics[width=0.65\linewidth]{_figures/data-manip/price-by-clarity-1}
\end{figure}

You might be surprised by this pattern: why do diamonds with better
clarity have lower prices? We'll see why this is the case and what to do
about it in \hyperref[sub:trend]{removing trend}.

Supply additional variables to \texttt{group\_by()} to create groups
based on more than one variable. The next example shows how we can
compute (by hand) a frequency polygon that shows how cut and depth
interact. The special summary function \texttt{n()} counts the number of
observations in each group.

\begin{Shaded}
\begin{Highlighting}[]
\NormalTok{cut_depth <-}\StringTok{ }\KeywordTok{summarise}\NormalTok{(}\KeywordTok{group_by}\NormalTok{(diamonds, cut, depth), }\DataTypeTok{n =} \KeywordTok{n}\NormalTok{())}
\NormalTok{cut_depth <-}\StringTok{ }\KeywordTok{filter}\NormalTok{(cut_depth, depth >}\StringTok{ }\DecValTok{55}\NormalTok{, depth <}\StringTok{ }\DecValTok{70}\NormalTok{)}
\NormalTok{cut_depth}
\CommentTok{#> Source: local data frame [455 x 3]}
\CommentTok{#> Groups: cut [5]}
\CommentTok{#> }
\CommentTok{#>       cut depth     n}
\CommentTok{#>    (fctr) (dbl) (int)}
\CommentTok{#> 1    Fair  55.1     3}
\CommentTok{#> 2    Fair  55.2     6}
\CommentTok{#> 3    Fair  55.3     5}
\CommentTok{#> 4    Fair  55.4     2}
\CommentTok{#> 5    Fair  55.5     3}
\CommentTok{#> 6    Fair  55.6     4}
\CommentTok{#> ..    ...   ...   ...}

\KeywordTok{ggplot}\NormalTok{(cut_depth, }\KeywordTok{aes}\NormalTok{(depth, n, }\DataTypeTok{colour =} \NormalTok{cut)) +}\StringTok{ }
\StringTok{  }\KeywordTok{geom_line}\NormalTok{()}
\end{Highlighting}
\end{Shaded}

\begin{figure}[H]
  \centering
  \includegraphics[width=0.65\linewidth]{_figures/data-manip/freqpoly-by-hand-1}
\end{figure}

We can use a grouped \texttt{mutate()} to convert counts to proportions,
so it's easier to compare across the cuts. \texttt{summarise()} strips
one level of grouping off, so \texttt{cut\_depth()} will be grouped by
cut.

\begin{Shaded}
\begin{Highlighting}[]
\NormalTok{cut_depth <-}\StringTok{ }\KeywordTok{mutate}\NormalTok{(cut_depth, }\DataTypeTok{prop =} \NormalTok{n /}\StringTok{ }\KeywordTok{sum}\NormalTok{(n))}
\KeywordTok{ggplot}\NormalTok{(cut_depth, }\KeywordTok{aes}\NormalTok{(depth, prop, }\DataTypeTok{colour =} \NormalTok{cut)) +}\StringTok{ }
\StringTok{  }\KeywordTok{geom_line}\NormalTok{()}
\end{Highlighting}
\end{Shaded}

\begin{figure}[H]
  \centering
  \includegraphics[width=0.65\linewidth]{_figures/data-manip/freqpoly-scaled-1}
\end{figure}

\subsection{Useful tools}

\texttt{summarise()} needs to be used with functions that take a vector
of \(n\) values and always return a single value. Those functions
include:

\begin{itemize}
\tightlist
\item
  Counts: \texttt{n()}, \texttt{n\_distinct(x)}.
\item
  Middle: \texttt{mean(x)}, \texttt{median(x)}.
\item
  Spread: \texttt{sd(x)}, \texttt{mad(x)}, \texttt{IQR(x)}.
\item
  Extremes: \texttt{quartile(x)}, \texttt{min(x)}, \texttt{max(x)}.
\item
  Positions: \texttt{first(x)}, \texttt{last(x)}, \texttt{nth(x,\ 2)}.
\end{itemize}

Another extremely useful technique is to use \texttt{sum()} or
\texttt{mean()} with a logical vector. When logical vector is treated as
numeric, \texttt{TRUE} becomes 1 and \texttt{FALSE} becomes 0. This
means that \texttt{sum()} tells you the number of \texttt{TRUE}s, and
\texttt{mean()} tells you the proportion of \texttt{TRUE}s. For example,
the following code counts the number of diamonds with carat greater than
or equal to 4, and the proportion of diamonds that cost less than
\$1000.

\begin{Shaded}
\begin{Highlighting}[]
\KeywordTok{summarise}\NormalTok{(diamonds, }
  \DataTypeTok{n_big =} \KeywordTok{sum}\NormalTok{(carat >=}\StringTok{ }\DecValTok{4}\NormalTok{), }
  \DataTypeTok{prop_cheap =} \KeywordTok{mean}\NormalTok{(price <}\StringTok{ }\DecValTok{1000}\NormalTok{)}
\NormalTok{)}
\CommentTok{#> Source: local data frame [1 x 2]}
\CommentTok{#> }
\CommentTok{#>   n_big prop_cheap}
\CommentTok{#>   (int)      (dbl)}
\CommentTok{#> 1     6      0.269}
\end{Highlighting}
\end{Shaded}

Most summary functions have a \texttt{na.rm} argument:
\texttt{na.rm\ =\ TRUE} tells the summary function to remove any missing
values prior to summiarisation. This is a convenient shortcut: rather
than removing the missing values then summarising, you can do it in one
step.

\subsection{Statistical considerations}

When summarising with the mean or median, it's always a good idea to
include a count and a measure of spread. This helps you calibrate your
assessments - if you don't include them you're likely to think that the
data is less variable than it really is, and potentially draw
unwarranted conclusions.

The following example extends our previous summary of the average price
by clarity to also include the number of observations in each group, and
the upper and lower quartiles. It suggests the mean might be a bad
summary for this data - the distributions of price are so highly skewed
that the mean is higher than the upper quartile for some of the groups!

\begin{Shaded}
\begin{Highlighting}[]
\NormalTok{by_clarity <-}\StringTok{ }\NormalTok{diamonds %>%}
\StringTok{  }\KeywordTok{group_by}\NormalTok{(clarity) %>%}
\StringTok{  }\KeywordTok{summarise}\NormalTok{(}
    \DataTypeTok{n =} \KeywordTok{n}\NormalTok{(), }
    \DataTypeTok{mean =} \KeywordTok{mean}\NormalTok{(price), }
    \DataTypeTok{lq =} \KeywordTok{quantile}\NormalTok{(price, }\FloatTok{0.25}\NormalTok{), }
    \DataTypeTok{uq =} \KeywordTok{quantile}\NormalTok{(price, }\FloatTok{0.75}\NormalTok{)}
  \NormalTok{)}
\NormalTok{by_clarity}
\CommentTok{#> Source: local data frame [8 x 5]}
\CommentTok{#> }
\CommentTok{#>    clarity     n  mean    lq    uq}
\CommentTok{#>     (fctr) (int) (dbl) (dbl) (dbl)}
\CommentTok{#> 1       I1   741  3924  2080  5161}
\CommentTok{#> 2      SI2  9194  5063  2264  5777}
\CommentTok{#> 3      SI1 13065  3996  1089  5250}
\CommentTok{#> 4      VS2 12258  3925   900  6024}
\CommentTok{#> 5      VS1  8171  3839   876  6023}
\CommentTok{#> 6     VVS2  5066  3284   794  3638}
\CommentTok{#> ..     ...   ...   ...   ...   ...}
\KeywordTok{ggplot}\NormalTok{(by_clarity, }\KeywordTok{aes}\NormalTok{(clarity, mean)) +}\StringTok{ }
\StringTok{  }\KeywordTok{geom_linerange}\NormalTok{(}\KeywordTok{aes}\NormalTok{(}\DataTypeTok{ymin =} \NormalTok{lq, }\DataTypeTok{ymax =} \NormalTok{uq)) +}\StringTok{ }
\StringTok{  }\KeywordTok{geom_line}\NormalTok{(}\KeywordTok{aes}\NormalTok{(}\DataTypeTok{group =} \DecValTok{1}\NormalTok{), }\DataTypeTok{colour =} \StringTok{"grey50"}\NormalTok{) +}
\StringTok{  }\KeywordTok{geom_point}\NormalTok{(}\KeywordTok{aes}\NormalTok{(}\DataTypeTok{size =} \NormalTok{n))}
\end{Highlighting}
\end{Shaded}

\begin{figure}[H]
  \centering
  \includegraphics[width=0.65\linewidth]{_figures/data-manip/unnamed-chunk-2-1}
\end{figure}

Another example of this comes from baseball. Let's take the MLB batting
data from the Lahman package and calculate the batting average, the
number of hits divided by the number of at bats. Who's the best batter
according to this metric?

\begin{Shaded}
\begin{Highlighting}[]
\KeywordTok{data}\NormalTok{(Batting, }\DataTypeTok{package =} \StringTok{"Lahman"}\NormalTok{)}
\NormalTok{batters <-}\StringTok{ }\KeywordTok{filter}\NormalTok{(Batting, AB >}\StringTok{ }\DecValTok{0}\NormalTok{)}
\NormalTok{per_player <-}\StringTok{ }\KeywordTok{group_by}\NormalTok{(batters, playerID)}
\NormalTok{ba <-}\StringTok{ }\KeywordTok{summarise}\NormalTok{(per_player, }
  \DataTypeTok{ba =} \KeywordTok{sum}\NormalTok{(H, }\DataTypeTok{na.rm =} \OtherTok{TRUE}\NormalTok{) /}\StringTok{ }\KeywordTok{sum}\NormalTok{(AB, }\DataTypeTok{na.rm =} \OtherTok{TRUE}\NormalTok{)}
\NormalTok{)}
\KeywordTok{ggplot}\NormalTok{(ba, }\KeywordTok{aes}\NormalTok{(ba)) +}\StringTok{ }
\StringTok{  }\KeywordTok{geom_histogram}\NormalTok{(}\DataTypeTok{binwidth =} \FloatTok{0.01}\NormalTok{)}
\end{Highlighting}
\end{Shaded}

\begin{figure}[H]
  \centering
  \includegraphics[width=0.65\linewidth]{_figures/data-manip/unnamed-chunk-3-1}
\end{figure}

Wow, there are a lot of players who can hit the ball every single time!
Would you want them on your fantasy baseball team? Let's double check
they're really that good by calibrating also showing the total number of
at bats:

\begin{Shaded}
\begin{Highlighting}[]
\NormalTok{ba <-}\StringTok{ }\KeywordTok{summarise}\NormalTok{(per_player, }
  \DataTypeTok{ba =} \KeywordTok{sum}\NormalTok{(H, }\DataTypeTok{na.rm =} \OtherTok{TRUE}\NormalTok{) /}\StringTok{ }\KeywordTok{sum}\NormalTok{(AB, }\DataTypeTok{na.rm =} \OtherTok{TRUE}\NormalTok{),}
  \DataTypeTok{ab =} \KeywordTok{sum}\NormalTok{(AB, }\DataTypeTok{na.rm =} \OtherTok{TRUE}\NormalTok{)}
\NormalTok{)}
\KeywordTok{ggplot}\NormalTok{(ba, }\KeywordTok{aes}\NormalTok{(ab, ba)) +}\StringTok{ }
\StringTok{  }\KeywordTok{geom_bin2d}\NormalTok{(}\DataTypeTok{bins =} \DecValTok{100}\NormalTok{) +}\StringTok{ }
\StringTok{  }\KeywordTok{geom_smooth}\NormalTok{()}
\end{Highlighting}
\end{Shaded}

\begin{figure}[H]
  \centering
  \includegraphics[width=0.65\linewidth]{_figures/data-manip/unnamed-chunk-4-1}
\end{figure}

The highest batting averages occur for the players with the smallest
number of at bats - it's not hard to hit the ball every time if you've
only had two pitches. We can make the pattern a little more clear by
getting rid of the players with less than 10 at bats.

\begin{Shaded}
\begin{Highlighting}[]
\KeywordTok{ggplot}\NormalTok{(}\KeywordTok{filter}\NormalTok{(ba, ab >=}\StringTok{ }\DecValTok{10}\NormalTok{), }\KeywordTok{aes}\NormalTok{(ab, ba)) +}\StringTok{ }
\StringTok{  }\KeywordTok{geom_bin2d}\NormalTok{() +}\StringTok{ }
\StringTok{  }\KeywordTok{geom_smooth}\NormalTok{()}
\end{Highlighting}
\end{Shaded}

\begin{figure}[H]
  \centering
  \includegraphics[width=0.65\linewidth]{_figures/data-manip/unnamed-chunk-5-1}
\end{figure}

You'll often see a similar pattern whenever you plot number of
observations vs.~an average. Be aware!

\subsection{Exercises}

\begin{enumerate}
\def\labelenumi{\arabic{enumi}.}
\item
  For each year in the \texttt{ggplot2movies::movies} data determine the
  percent of movies with missing budgets. Visualise the result.
\item
  How does the average length of a movie change over time? Display your
  answer with a plot, including some display of uncertainty.
\item
  For each combination of diamond quality (e.g.~cut, colour and
  clarity), count the number of diamonds, the average price and the
  average size. Visualise the results.
\item
  Compute a histogram of carat by ``hand'' using a binwidth of 0.1.
  Display the results with \texttt{geom\_bar(stat\ =\ "identity")}.
  (Hint: you might need to create a new variable first).
\item
  In the baseball example, the batting average seems to increase as the
  number of at bats increases. Why?
\end{enumerate}

\section{Transformation pipelines}

Most real analyses require you to string together multiple
\texttt{mutate()}s, \texttt{filter()}s, \texttt{group\_by()}s , and
\texttt{summarise()}s. For example, above, we created a frequency
polygon by hand by with a combination of all four verbs: \indexc{\%>\%}

\begin{Shaded}
\begin{Highlighting}[]
\CommentTok{# By using intermediate values}
\NormalTok{cut_depth <-}\StringTok{ }\KeywordTok{group_by}\NormalTok{(diamonds, cut, depth)}
\NormalTok{cut_depth <-}\StringTok{ }\KeywordTok{summarise}\NormalTok{(cut_depth, }\DataTypeTok{n =} \KeywordTok{n}\NormalTok{())}
\NormalTok{cut_depth <-}\StringTok{ }\KeywordTok{filter}\NormalTok{(cut_depth, depth >}\StringTok{ }\DecValTok{55}\NormalTok{, depth <}\StringTok{ }\DecValTok{70}\NormalTok{)}
\NormalTok{cut_depth <-}\StringTok{ }\KeywordTok{mutate}\NormalTok{(cut_depth, }\DataTypeTok{prop =} \NormalTok{n /}\StringTok{ }\KeywordTok{sum}\NormalTok{(n))}
\end{Highlighting}
\end{Shaded}

This sequence of operations is a bit painful because we repeated the
name of the data frame many times. An alternative is just to do it with
one sequence of function calls:

\begin{Shaded}
\begin{Highlighting}[]
\CommentTok{# By "composing" functions}
\KeywordTok{mutate}\NormalTok{(}
  \KeywordTok{filter}\NormalTok{(}
    \KeywordTok{summarise}\NormalTok{(}
      \KeywordTok{group_by}\NormalTok{(}
        \NormalTok{diamonds, }
        \NormalTok{cut, }
        \NormalTok{depth}
      \NormalTok{), }
      \DataTypeTok{n =} \KeywordTok{n}\NormalTok{()}
    \NormalTok{), }
    \NormalTok{depth >}\StringTok{ }\DecValTok{55}\NormalTok{, }
    \NormalTok{depth <}\StringTok{ }\DecValTok{70}
  \NormalTok{), }
  \DataTypeTok{prop =} \NormalTok{n /}\StringTok{ }\KeywordTok{sum}\NormalTok{(n)}
\NormalTok{)}
\end{Highlighting}
\end{Shaded}

But this is also hard to read because the sequence of operations is
inside out, and the arguments to each function can be quite far apart.
dplyr provides an alternative approach with the \textbf{pipe},
\texttt{\%\textgreater{}\%}. With the pipe, we can write the above
sequence of operations as:

\begin{Shaded}
\begin{Highlighting}[]
\NormalTok{cut_depth <-}\StringTok{ }\NormalTok{diamonds %>%}\StringTok{ }
\StringTok{  }\KeywordTok{group_by}\NormalTok{(cut, depth) %>%}\StringTok{ }
\StringTok{  }\KeywordTok{summarise}\NormalTok{(}\DataTypeTok{n =} \KeywordTok{n}\NormalTok{()) %>%}\StringTok{ }
\StringTok{  }\KeywordTok{filter}\NormalTok{(depth >}\StringTok{ }\DecValTok{55}\NormalTok{, depth <}\StringTok{ }\DecValTok{70}\NormalTok{) %>%}\StringTok{ }
\StringTok{  }\KeywordTok{mutate}\NormalTok{(}\DataTypeTok{prop =} \NormalTok{n /}\StringTok{ }\KeywordTok{sum}\NormalTok{(n))}
\end{Highlighting}
\end{Shaded}

This makes it easier to understand what's going on as we can read it
almost like an English sentence: first group, then summarise, then
filter, then mutate. In fact, the best way to pronounce
\texttt{\%\textgreater{}\%} when reading a sequence of code is as
``then''. \texttt{\%\textgreater{}\%} comes from the magrittr package,
by Stefan Milton Bache. It provides a number of other tools that dplyr
doesn't expose by default, so I highly recommend that you check out the
\href{https://github.com/smbache/magrittr}{magrittr website}.
\index{magrittr}

\texttt{\%\textgreater{}\%} works by taking the thing on the left hand
side (LHS) and supplying it as the first argument to the function on the
right hand side (RHS). Each of these pairs of calls is equivalent:

\begin{Shaded}
\begin{Highlighting}[]
\KeywordTok{f}\NormalTok{(x, y)}
\CommentTok{# is the same as}
\NormalTok{x %>%}\StringTok{ }\KeywordTok{f}\NormalTok{(y)}

\KeywordTok{g}\NormalTok{(}\KeywordTok{f}\NormalTok{(x, y), z)}
\CommentTok{# is the same as}
\NormalTok{x %>%}\StringTok{ }\KeywordTok{f}\NormalTok{(y) %>%}\StringTok{ }\KeywordTok{g}\NormalTok{(z)}
\end{Highlighting}
\end{Shaded}

\subsection{Exercises}

\begin{enumerate}
\def\labelenumi{\arabic{enumi}.}
\item
  Translate each of the examples in this chapter to use the pipe.
\item
  What does the following pipe do?

\begin{Shaded}
\begin{Highlighting}[]
\KeywordTok{library}\NormalTok{(magrittr)}
\NormalTok{x <-}\StringTok{ }\KeywordTok{runif}\NormalTok{(}\DecValTok{100}\NormalTok{)}
\NormalTok{x %>%}
\StringTok{  }\KeywordTok{subtract}\NormalTok{(}\KeywordTok{mean}\NormalTok{(.)) %>%}
\StringTok{  }\KeywordTok{raise_to_power}\NormalTok{(}\DecValTok{2}\NormalTok{) %>%}
\StringTok{  }\KeywordTok{mean}\NormalTok{() %>%}
\StringTok{  }\KeywordTok{sqrt}\NormalTok{()}
\end{Highlighting}
\end{Shaded}
\item
  Which player in the \texttt{Batting} dataset has had the most
  consistently good performance over the course of their career?
\end{enumerate}

\section{Learning more}

dplyr provides a number of other verbs that are less useful for
visualisation, but important to know about in general:

\begin{itemize}
\item
  \texttt{arrange()} orders observations according to variable(s). This
  is most useful when you're looking at the data from the console. It
  can also be useful for visualisations if you want to control which
  points are plotted on top.
\item
  \texttt{select()} picks variables based on their names. Useful when
  you have many variables and want to focus on just a few for analysis.
\item
  \texttt{rename()} allows you to change the name of variables.
\item
  Grouped mutates and filters are also useful, but more advanced. See
  \texttt{vignette("window-functions",\ package\ =\ "dplyr")} for more
  details.
\item
  There are a number of verbs designed to work with two tables of data
  at a time. These include SQL joins (like the base \texttt{merge()}
  function) and set operations. Learn more about them in
  \texttt{vignette("two-table",\ package\ =\ "dplyr")}.
\item
  dplyr can work directly with data stored in a database - you use the
  same R code as you do for local data and dplyr generates SQL to send
  to the database. See
  \texttt{vignette("databases",\ package\ =\ "dplyr")} for the details.
\end{itemize}

Finally, RStudio provides a handy dplyr cheatsheet that will help jog
your memory when you're wondering which function to use. Get it from
\url{http://rstudio.com/cheatsheets}.

\section*{References}
\addcontentsline{toc}{section}{References}

\hyperdef{}{ref-tornqvist:1985}{\label{ref-tornqvist:1985}}
Törnqvist, Leo, Pentti Vartia, and Yrjö O Vartia. 1985. ``How Should
Relative Changes Be Measured?'' \emph{The American Statistician} 39 (1):
43--46.

\chapter{Modelling for visualisation}\label{cha:modelling}

\section{Introduction}\label{introduction}

Modelling is an essential tool for visualisation. There are two
particularly strong connections between modelling and visualisation that
I want to explore in this chapter: \index{Modelling}

\begin{itemize}
\item
  Using models as a tool to remove obvious patterns in your plots. This
  is useful because strong patterns mask subtler effects. Often the
  strongest effects are already known and expected, and removing them
  allows you to see surprises more easily.
\item
  Other times you have a lot of data, too much to show on a handful of
  plots. Models can be a powerful tool for summarising data so that you
  get a higher level view.
\end{itemize}

In this chapter, I'm going to focus on the use of linear models to
acheive these goals. Linear models are a basic, but powerful, tool of
statistics, and I recommend that everyone serious about visualisation
learns at least the basics of how to use them. To this end, I highly
recommend two books by Julian J. Faraway:

\begin{itemize}
\tightlist
\item
  Linear Models with R \url{http://amzn.com/1439887330}
\item
  Extending the Linear Model with R \url{http://amzn.com/158488424X}
\end{itemize}

These books cover some of the theory of linear models, but are pragmatic
and focussed on how to actually use linear models (and their extensions)
in R. \index{Linear models}

There are many other modelling tools, which I don't have the space to
show. If you understand how linear models can help improve your
visualisations, you should be able to translate the basic idea to other
families of models. This chapter just scratches the surface of what you
can do. But hopefully it reinforces how visualisation can combine with
modelling to help you build a powerful data analysis toolbox. For more
ideas, check out Wickham, Cook, and Hofmann (2015).

This chapter only scratches the surface of the intersection between
visualisation and modelling. In my opinion, mastering the combination of
visualisations and models is key to being an effective data scientist.
Unfortunately most books (like this one!) only focus on either
visualisation or modelling, but not both. There's a lot of interesting
work to be done.

\section{Removing trend}\label{sub:trend}

So far our analysis of the diamonds data has been plagued by the
powerful relationship between size and price. It makes it very difficult
to see the impact of cut, colour and clarity because higher quality
diamonds tend to be smaller, and hence cheaper. This challenge is often
called confounding. We can use a linear model to remove the effect of
size on price. Instead of looking at the raw price, we can look at the
relative price: how valuable is this diamond relative to the average
diamond of the same size. \index{Removing trend}

To get started, we'll focus on diamonds of size two carats or less (96\%
of the dataset). This avoids some incidental problems that you can
explore in the exercises if you're interested. We'll also create two new
variables: log price and log carat. These variables are useful because
they produce a plot with a strong linear trend.

\begin{Shaded}
\begin{Highlighting}[]
\NormalTok{diamonds2 <-}\StringTok{ }\NormalTok{diamonds %>%}\StringTok{ }
\StringTok{  }\KeywordTok{filter}\NormalTok{(carat <=}\StringTok{ }\DecValTok{2}\NormalTok{) %>%}
\StringTok{  }\KeywordTok{mutate}\NormalTok{(}
    \DataTypeTok{lcarat =} \KeywordTok{log2}\NormalTok{(carat),}
    \DataTypeTok{lprice =} \KeywordTok{log2}\NormalTok{(price)}
  \NormalTok{)}
\NormalTok{diamonds2}
\CommentTok{#> Source: local data frame [52,051 x 12]}
\CommentTok{#> }
\CommentTok{#>    carat       cut  color clarity depth table price     x     y}
\CommentTok{#>    (dbl)    (fctr) (fctr)  (fctr) (dbl) (dbl) (int) (dbl) (dbl)}
\CommentTok{#> 1   0.23     Ideal      E     SI2  61.5    55   326  3.95  3.98}
\CommentTok{#> 2   0.21   Premium      E     SI1  59.8    61   326  3.89  3.84}
\CommentTok{#> 3   0.23      Good      E     VS1  56.9    65   327  4.05  4.07}
\CommentTok{#> 4   0.29   Premium      I     VS2  62.4    58   334  4.20  4.23}
\CommentTok{#> 5   0.31      Good      J     SI2  63.3    58   335  4.34  4.35}
\CommentTok{#> 6   0.24 Very Good      J    VVS2  62.8    57   336  3.94  3.96}
\CommentTok{#> ..   ...       ...    ...     ...   ...   ...   ...   ...   ...}
\CommentTok{#> Variables not shown: z (dbl), lcarat (dbl), lprice (dbl)}

\KeywordTok{ggplot}\NormalTok{(diamonds2, }\KeywordTok{aes}\NormalTok{(lcarat, lprice)) +}\StringTok{ }
\StringTok{  }\KeywordTok{geom_bin2d}\NormalTok{() +}\StringTok{ }
\StringTok{  }\KeywordTok{geom_smooth}\NormalTok{(}\DataTypeTok{method =} \StringTok{"lm"}\NormalTok{, }\DataTypeTok{se =} \OtherTok{FALSE}\NormalTok{, }\DataTypeTok{size =} \DecValTok{2}\NormalTok{, }\DataTypeTok{colour =} \StringTok{"yellow"}\NormalTok{)}
\end{Highlighting}
\end{Shaded}

\begin{figure}[H]
  \centering
  \includegraphics[width=0.75\linewidth]{_figures/modelling/unnamed-chunk-1-1}
\end{figure}

In the graphic we used \texttt{geom\_smooth()} to overlay the line of
best fit to the data. We can replicate this outside of ggplot2 by
fitting a linear model with \texttt{lm()}. This allows us to find out
the slope and intercept of the line: \indexf{lm} \indexf{coef}

\begin{Shaded}
\begin{Highlighting}[]
\NormalTok{mod <-}\StringTok{ }\KeywordTok{lm}\NormalTok{(lprice ~}\StringTok{ }\NormalTok{lcarat, }\DataTypeTok{data =} \NormalTok{diamonds2)}
\KeywordTok{coef}\NormalTok{(}\KeywordTok{summary}\NormalTok{(mod))}
\CommentTok{#>             Estimate Std. Error t value Pr(>|t|)}
\CommentTok{#> (Intercept)     12.2    0.00211    5789        0}
\CommentTok{#> lcarat           1.7    0.00208     816        0}
\end{Highlighting}
\end{Shaded}

If you're familiar with linear models, you might want to interpret those
coefficients: \(\log_2(price) = 12.2 + 1.7 \cdot \log_2(carat)\), which
implies \(price = 4900 \cdot carat ^ {1.7}\). Interpreting those
coefficients certainly is useful, but even if you don't understand them,
the model can still be useful. We can use it to subtract the trend away
by looking at the residuals: the price of each diamond minus its
predicted price, based on weight alone. Geometrically, the residuals are
the vertical distance between each point and the line of best fit. They
tell us the price relative to the ``average'' diamond of that size.
\indexf{resid}

\begin{Shaded}
\begin{Highlighting}[]
\NormalTok{diamonds2 <-}\StringTok{ }\NormalTok{diamonds2 %>%}\StringTok{ }\KeywordTok{mutate}\NormalTok{(}\DataTypeTok{rel_price =} \KeywordTok{resid}\NormalTok{(mod))}
\KeywordTok{ggplot}\NormalTok{(diamonds2, }\KeywordTok{aes}\NormalTok{(carat, rel_price)) +}\StringTok{ }
\StringTok{  }\KeywordTok{geom_bin2d}\NormalTok{()}
\end{Highlighting}
\end{Shaded}

\begin{figure}[H]
  \centering
  \includegraphics[width=0.75\linewidth]{_figures/modelling/unnamed-chunk-3-1}
\end{figure}

A relative price of zero means that the diamond was at the average
price; positive means that it's more expensive than expected (based on
its size), and negative means that it's cheaper than expected.

Interpreting the values precisely is a little tricky here because we've
log-transformed price. The residuals give the absolute difference
(\(x - expected\)), but here we have
\(\log_2(price) - \log_2(expected price)\), or equivalently
\(\log_2(price / expected price)\). If we ``back-transform'' to the
original scale by applying the opposite transformation (\(2 ^ x\)) we
get \(price / expected price\). This makes the values more
interpretable, at the cost of the nice symmetry property of the logged
values (i.e.~both relatively cheaper and relatively more expensive
diamonds have the same range). We can make a little table to help
interpret the values:

\begin{Shaded}
\begin{Highlighting}[]
\NormalTok{xgrid <-}\StringTok{ }\KeywordTok{seq}\NormalTok{(-}\DecValTok{2}\NormalTok{, }\DecValTok{1}\NormalTok{, }\DataTypeTok{by =} \DecValTok{1}\NormalTok{/}\DecValTok{3}\NormalTok{)}
\KeywordTok{data.frame}\NormalTok{(}\DataTypeTok{logx =} \NormalTok{xgrid, }\DataTypeTok{x =} \KeywordTok{round}\NormalTok{(}\DecValTok{2} \NormalTok{^}\StringTok{ }\NormalTok{xgrid, }\DecValTok{2}\NormalTok{))}
\CommentTok{#>      logx    x}
\CommentTok{#> 1  -2.000 0.25}
\CommentTok{#> 2  -1.667 0.31}
\CommentTok{#> 3  -1.333 0.40}
\CommentTok{#> 4  -1.000 0.50}
\CommentTok{#> 5  -0.667 0.63}
\CommentTok{#> 6  -0.333 0.79}
\CommentTok{#> 7   0.000 1.00}
\CommentTok{#> 8   0.333 1.26}
\CommentTok{#> 9   0.667 1.59}
\CommentTok{#> 10  1.000 2.00}
\end{Highlighting}
\end{Shaded}

This table illustrates why we used \texttt{log2()} rather than
\texttt{log()}: a change of 1 unit on the logged scale, corresponding to
a doubling on the original scale. For example, a \texttt{rel\_price} of
-1 means that it's half of the expected price; a relative price of 1
means that it's twice the expected price. \index{Log!transform}

Let's use both price and relative price to see how colour and cut affect
the value of a diamond. We'll compute the average price and average
relative price for each combination of colour and cut:

\begin{Shaded}
\begin{Highlighting}[]
\NormalTok{color_cut <-}\StringTok{ }\NormalTok{diamonds2 %>%}\StringTok{ }
\StringTok{  }\KeywordTok{group_by}\NormalTok{(color, cut) %>%}
\StringTok{  }\KeywordTok{summarise}\NormalTok{(}
    \DataTypeTok{price =} \KeywordTok{mean}\NormalTok{(price), }
    \DataTypeTok{rel_price =} \KeywordTok{mean}\NormalTok{(rel_price)}
  \NormalTok{)}
\NormalTok{color_cut}
\CommentTok{#> Source: local data frame [35 x 4]}
\CommentTok{#> Groups: color [?]}
\CommentTok{#> }
\CommentTok{#>     color       cut price rel_price}
\CommentTok{#>    (fctr)    (fctr) (dbl)     (dbl)}
\CommentTok{#> 1       D      Fair  3939   -0.0755}
\CommentTok{#> 2       D      Good  3309   -0.0472}
\CommentTok{#> 3       D Very Good  3368    0.1038}
\CommentTok{#> 4       D   Premium  3513    0.1093}
\CommentTok{#> 5       D     Ideal  2595    0.2173}
\CommentTok{#> 6       E      Fair  3516   -0.1720}
\CommentTok{#> ..    ...       ...   ...       ...}
\end{Highlighting}
\end{Shaded}

If we look at price, it's hard to see how the quality of the diamond
affects the price. The lowest quality diamonds (fair cut with colour J)
have the highest average value! This is because those diamonds also tend
to be larger: size and quality are confounded.

\begin{Shaded}
\begin{Highlighting}[]
\KeywordTok{ggplot}\NormalTok{(color_cut, }\KeywordTok{aes}\NormalTok{(color, price)) +}\StringTok{ }
\StringTok{  }\KeywordTok{geom_line}\NormalTok{(}\KeywordTok{aes}\NormalTok{(}\DataTypeTok{group =} \NormalTok{cut), }\DataTypeTok{colour =} \StringTok{"grey80"}\NormalTok{) +}
\StringTok{  }\KeywordTok{geom_point}\NormalTok{(}\KeywordTok{aes}\NormalTok{(}\DataTypeTok{colour =} \NormalTok{cut))}
\end{Highlighting}
\end{Shaded}

\begin{figure}[H]
  \centering
  \includegraphics[width=0.75\linewidth]{_figures/modelling/unnamed-chunk-6-1}
\end{figure}

If however, we plot the relative price, you see the pattern that you
expect: as the quality of the diamonds decreases, the relative price
decreases. The worst quality diamond is 0.61x (\(2 ^ {-0.7}\)) the price
of an ``average'' diamond.

\begin{Shaded}
\begin{Highlighting}[]
\KeywordTok{ggplot}\NormalTok{(color_cut, }\KeywordTok{aes}\NormalTok{(color, rel_price)) +}\StringTok{ }
\StringTok{  }\KeywordTok{geom_line}\NormalTok{(}\KeywordTok{aes}\NormalTok{(}\DataTypeTok{group =} \NormalTok{cut), }\DataTypeTok{colour =} \StringTok{"grey80"}\NormalTok{) +}
\StringTok{  }\KeywordTok{geom_point}\NormalTok{(}\KeywordTok{aes}\NormalTok{(}\DataTypeTok{colour =} \NormalTok{cut))}
\end{Highlighting}
\end{Shaded}

\begin{figure}[H]
  \centering
  \includegraphics[width=0.75\linewidth]{_figures/modelling/unnamed-chunk-7-1}
\end{figure}

This technique can be employed in a wide range of situations. Wherever
you can explicitly model a strong pattern that you see in a plot, it's
worthwhile to use a model to remove that strong pattern so that you can
see what interesting trends remain.

\subsection{Exercises}\label{exercises}

\begin{enumerate}
\def\labelenumi{\arabic{enumi}.}
\item
  What happens if you repeat the above analysis with all diamonds? (Not
  just all diamonds with two or fewer carats). What does the strange
  geometry of \texttt{log(carat)} vs relative price represent? What does
  the diagonal line without any points represent?
\item
  I made an unsupported assertion that lower-quality diamonds tend to be
  larger. Support my claim with a plot.
\item
  Can you create a plot that simultaneously shows the effect of colour,
  cut, and clarity on relative price? If there's too much information to
  show on one plot, think about how you might create a sequence of plots
  to convey the same message.
\item
  How do depth and table relate to the relative price?
\end{enumerate}

\section{Texas housing data}\label{texas-housing-data}

We'll continue to explore the connection between modelling and
visualisation with the \texttt{txhousing} dataset:

\begin{Shaded}
\begin{Highlighting}[]
\NormalTok{txhousing}
\CommentTok{#> Source: local data frame [8,034 x 9]}
\CommentTok{#> }
\CommentTok{#>       city  year month sales   volume median listings inventory}
\CommentTok{#>      (chr) (int) (int) (dbl)    (dbl)  (dbl)    (dbl)     (dbl)}
\CommentTok{#> 1  Abilene  2000     1    72  5380000  71400      701       6.3}
\CommentTok{#> 2  Abilene  2000     2    98  6505000  58700      746       6.6}
\CommentTok{#> 3  Abilene  2000     3   130  9285000  58100      784       6.8}
\CommentTok{#> 4  Abilene  2000     4    98  9730000  68600      785       6.9}
\CommentTok{#> 5  Abilene  2000     5   141 10590000  67300      794       6.8}
\CommentTok{#> 6  Abilene  2000     6   156 13910000  66900      780       6.6}
\CommentTok{#> ..     ...   ...   ...   ...      ...    ...      ...       ...}
\CommentTok{#> Variables not shown: date (dbl)}
\end{Highlighting}
\end{Shaded}

This data was collected by the Real Estate Center at Texas A\&M
University, \url{http://recenter.tamu.edu/Data/hs/}. The data contains
information about 46 Texas cities, recording the number of house sales
(\texttt{sales}), the total volume of sales (\texttt{volume}), the
\texttt{average} and \texttt{median} sale prices, the number of houses
listed for sale (\texttt{listings}) and the number of months inventory
(\texttt{inventory}). Data is recorded monthly from Jan 2000 to Apr
2015, 187 entries for each city.
\index{Data!txhousing@\texttt{txhousing}}

We're going to explore how sales have varied over time for each city as
it shows some interesting trends and poses some interesting challenges.
Let's start with an overview: a time series of sales for each city:
\index{Data!longitudinal}

\begin{Shaded}
\begin{Highlighting}[]
\KeywordTok{ggplot}\NormalTok{(txhousing, }\KeywordTok{aes}\NormalTok{(date, sales)) +}\StringTok{ }
\StringTok{  }\KeywordTok{geom_line}\NormalTok{(}\KeywordTok{aes}\NormalTok{(}\DataTypeTok{group =} \NormalTok{city), }\DataTypeTok{alpha =} \DecValTok{1}\NormalTok{/}\DecValTok{2}\NormalTok{)}
\end{Highlighting}
\end{Shaded}

\begin{figure}[H]
  \includegraphics[width=1\linewidth]{_figures/modelling/unnamed-chunk-9-1}
\end{figure}

Two factors make it hard to see the long-term trend in this plot:

\begin{enumerate}
\def\labelenumi{\arabic{enumi}.}
\item
  The range of sales varies over multiple orders of magnitude. The
  biggest city, Houston, averages over \textasciitilde{}4000 sales per
  month; the smallest city, San Marcos, only averages
  \textasciitilde{}20 sales per month.
\item
  There is a strong seasonal trend: sales are much higher in the summer
  than in the winter.
\end{enumerate}

We can fix the first problem by plotting log sales:

\begin{Shaded}
\begin{Highlighting}[]
\KeywordTok{ggplot}\NormalTok{(txhousing, }\KeywordTok{aes}\NormalTok{(date, }\KeywordTok{log}\NormalTok{(sales))) +}\StringTok{ }
\StringTok{  }\KeywordTok{geom_line}\NormalTok{(}\KeywordTok{aes}\NormalTok{(}\DataTypeTok{group =} \NormalTok{city), }\DataTypeTok{alpha =} \DecValTok{1}\NormalTok{/}\DecValTok{2}\NormalTok{)}
\end{Highlighting}
\end{Shaded}

\begin{figure}[H]
  \includegraphics[width=1\linewidth]{_figures/modelling/unnamed-chunk-10-1}
\end{figure}

We can fix the second problem using the same technique we used for
removing the trend in the diamonds data: we'll fit a linear model and
look at the residuals. This time we'll use a categorical predictor to
remove the month effect. First we check that the technique works by
applying it to a single city. It's always a good idea to start simple so
that if something goes wrong you can more easily pinpoint the problem.

\begin{Shaded}
\begin{Highlighting}[]
\NormalTok{abilene <-}\StringTok{ }\NormalTok{txhousing %>%}\StringTok{ }\KeywordTok{filter}\NormalTok{(city ==}\StringTok{ "Abilene"}\NormalTok{)}
\KeywordTok{ggplot}\NormalTok{(abilene, }\KeywordTok{aes}\NormalTok{(date, }\KeywordTok{log}\NormalTok{(sales))) +}\StringTok{ }
\StringTok{  }\KeywordTok{geom_line}\NormalTok{()}

\NormalTok{mod <-}\StringTok{ }\KeywordTok{lm}\NormalTok{(}\KeywordTok{log}\NormalTok{(sales) ~}\StringTok{ }\KeywordTok{factor}\NormalTok{(month), }\DataTypeTok{data =} \NormalTok{abilene)}
\NormalTok{abilene$rel_sales <-}\StringTok{ }\KeywordTok{resid}\NormalTok{(mod)}
\KeywordTok{ggplot}\NormalTok{(abilene, }\KeywordTok{aes}\NormalTok{(date, rel_sales)) +}\StringTok{ }
\StringTok{  }\KeywordTok{geom_line}\NormalTok{()}
\end{Highlighting}
\end{Shaded}

\begin{figure}[H]
  \includegraphics[width=0.5\linewidth]{_figures/modelling/unnamed-chunk-11-1}%
  \includegraphics[width=0.5\linewidth]{_figures/modelling/unnamed-chunk-11-2}
\end{figure}

We can apply this transformation to every city with \texttt{group\_by()}
and \texttt{mutate()}. Note the use of \texttt{na.action\ =\ na.exclude}
argument to \texttt{lm()}. Counterintuitively this ensures that missing
values in the input are matched with missing values in the output
predictions and residuals. Without this argument, missing values are
just dropped, and the residuals don't line up with the inputs.

\begin{Shaded}
\begin{Highlighting}[]
\NormalTok{deseas <-}\StringTok{ }\NormalTok{function(x, month) \{}
  \KeywordTok{resid}\NormalTok{(}\KeywordTok{lm}\NormalTok{(x ~}\StringTok{ }\KeywordTok{factor}\NormalTok{(month), }\DataTypeTok{na.action =} \NormalTok{na.exclude))}
\NormalTok{\}}

\NormalTok{txhousing <-}\StringTok{ }\NormalTok{txhousing %>%}\StringTok{ }
\StringTok{  }\KeywordTok{group_by}\NormalTok{(city) %>%}\StringTok{ }
\StringTok{  }\KeywordTok{mutate}\NormalTok{(}\DataTypeTok{rel_sales =} \KeywordTok{deseas}\NormalTok{(}\KeywordTok{log}\NormalTok{(sales), month))}
\end{Highlighting}
\end{Shaded}

With this data in hand, we can re-plot the data. Now that we have
log-transformed the data and removed the strong seasonal effects we can
see there is a strong common pattern: a consistent increase from
2000-2007, a drop until 2010 (with quite some noise), and then a gradual
rebound. To make that more clear, I included a summary line that shows
the mean relative sales across all cities.

\begin{Shaded}
\begin{Highlighting}[]
\KeywordTok{ggplot}\NormalTok{(txhousing, }\KeywordTok{aes}\NormalTok{(date, rel_sales)) +}
\StringTok{  }\KeywordTok{geom_line}\NormalTok{(}\KeywordTok{aes}\NormalTok{(}\DataTypeTok{group =} \NormalTok{city), }\DataTypeTok{alpha =} \DecValTok{1}\NormalTok{/}\DecValTok{5}\NormalTok{) +}\StringTok{ }
\StringTok{  }\KeywordTok{geom_line}\NormalTok{(}\DataTypeTok{stat =} \StringTok{"summary"}\NormalTok{, }\DataTypeTok{fun.y =} \StringTok{"mean"}\NormalTok{, }\DataTypeTok{colour =} \StringTok{"red"}\NormalTok{)}
\end{Highlighting}
\end{Shaded}

\begin{figure}[H]
  \includegraphics[width=1\linewidth]{_figures/modelling/unnamed-chunk-13-1}
\end{figure}

(Note that removing the seasonal effect also removed the intercept - we
see the trend for each city relative to its average number of sales.)

\subsection{Exercises}\label{exercises-1}

\begin{enumerate}
\def\labelenumi{\arabic{enumi}.}
\item
  The final plot shows a lot of short-term noise in the overall trend.
  How could you smooth this further to focus on long-term changes?
\item
  If you look closely (e.g. \texttt{+\ xlim(2008,\ 2012)}) at the
  long-term trend you'll notice a weird pattern in 2009-2011. It looks
  like there was a big dip in 2010. Is this dip ``real''? (i.e.~can you
  spot it in the original data)
\item
  What other variables in the TX housing data show strong seasonal
  effects? Does this technique help to remove them?
\item
  Not all the cities in this data set have complete time series. Use
  your dplyr skills to figure out how much data each city is missing.
  Display the results with a visualisation.
\item
  Replicate the computation that \texttt{stat\_summary()} did with dplyr
  so you can plot the data ``by hand''.
\end{enumerate}

\section{Visualising models}\label{sub:modelvis}

The previous examples used the linear model just as a tool for removing
trend: we fit the model and immediately threw it away. We didn't care
about the model itself, just what it could do for us. But the models
themselves contain useful information and if we keep them around, there
are many new problems that we can solve:

\begin{itemize}
\item
  We might be interested in cities where the model didn't fit well: a
  poorly fitting model suggests that there isn't much of a seasonal
  pattern, which contradicts our implicit hypothesis that all cities
  share a similar pattern.
\item
  The coefficients themselves might be interesting. In this case,
  looking at the coefficients will show us how the seasonal pattern
  varies between cities.
\item
  We may want to dive into the details of the model itself, and see
  exactly what it says about each observation. For this data, it might
  help us find suspicious data points that might reflect data entry
  errors.
\end{itemize}

To take advantage of this data, we need to store the models. We can do
this using a new dplyr verb: \texttt{do()}. It allows us to store the
result of arbitrary computation in a column. Here we'll use it to store
that linear model: \indexf{do}

\begin{Shaded}
\begin{Highlighting}[]
\NormalTok{models <-}\StringTok{ }\NormalTok{txhousing %>%}\StringTok{ }
\StringTok{  }\KeywordTok{group_by}\NormalTok{(city) %>%}
\StringTok{  }\KeywordTok{do}\NormalTok{(}\DataTypeTok{mod =} \KeywordTok{lm}\NormalTok{(}
    \KeywordTok{log2}\NormalTok{(sales) ~}\StringTok{ }\KeywordTok{factor}\NormalTok{(month), }
    \DataTypeTok{data =} \NormalTok{., }
    \DataTypeTok{na.action =} \NormalTok{na.exclude}
  \NormalTok{))}
\NormalTok{models}
\CommentTok{#> Source: local data frame [46 x 2]}
\CommentTok{#> Groups: <by row>}
\CommentTok{#> }
\CommentTok{#>         city     mod}
\CommentTok{#>        (chr)   (chr)}
\CommentTok{#> 1    Abilene <S3:lm>}
\CommentTok{#> 2   Amarillo <S3:lm>}
\CommentTok{#> 3  Arlington <S3:lm>}
\CommentTok{#> 4     Austin <S3:lm>}
\CommentTok{#> 5   Bay Area <S3:lm>}
\CommentTok{#> 6   Beaumont <S3:lm>}
\CommentTok{#> ..       ...     ...}
\end{Highlighting}
\end{Shaded}

There are two important things to note in this code:

\begin{itemize}
\item
  \texttt{do()} creates a new column called \texttt{mod.} This is a
  special type of column: instead of containing an atomic vector (a
  logical, integer, numeric, or character) like usual, it's a list.
  Lists are R's most flexible data structure and can hold anything,
  including linear models.
\item
  \texttt{.} is a special pronoun used by \texttt{do()}. It refers to
  the ``current'' data frame. In this example, \texttt{do()} fits the
  model 46 times (once for each city), each time replacing \texttt{.}
  with the data for one city. \indexc{.}
\end{itemize}

If you're an experienced modeller, you might wonder why I didn't fit one
model to all cities simultaneously. That's a great next step, but it's
often useful to start off simple. Once we have a model that works for
each city individually, you can figure out how to generalise it to fit
all cities simultaneously.

To visualise these models, we'll turn them into tidy data frames. We'll
do that with the \textbf{broom} package by David Robinson. \index{broom}
\index{Tidy models} \index{Model data}

\begin{Shaded}
\begin{Highlighting}[]
\KeywordTok{library}\NormalTok{(broom)}
\end{Highlighting}
\end{Shaded}

Broom provides three key verbs, each corresponding to one of the
challenges outlined above:

\begin{itemize}
\item
  \texttt{glance()} extracts \textbf{model}-level summaries with one row
  of data for each model. It contains summary statistics like the
  \(R^2\) and degrees of freedom.
\item
  \texttt{tidy()} extracts \textbf{coefficient}-level summaries with one
  row of data for each coefficient in each model. It contains
  information about individual coefficients like their estimate and
  standard error.
\item
  \texttt{augment()} extracts \textbf{observation}-level summaries with
  one row of data for each observation in each model. It includes
  variables like the residual and influence metrics useful for
  diagnosing outliers.
\end{itemize}

We'll learn more about each of these functions in the following three
sections.

\section{Model-level summaries}\label{model-level-summaries}

We'll begin by looking at how well the model fit to each city with
\texttt{glance()}: \indexf{glance}

\begin{Shaded}
\begin{Highlighting}[]
\NormalTok{model_sum <-}\StringTok{ }\NormalTok{models %>%}\StringTok{ }\KeywordTok{glance}\NormalTok{(mod)}
\NormalTok{model_sum}
\CommentTok{#> Source: local data frame [46 x 12]}
\CommentTok{#> Groups: city [46]}
\CommentTok{#> }
\CommentTok{#>         city r.squared adj.r.squared sigma statistic  p.value}
\CommentTok{#>        (chr)     (dbl)         (dbl) (dbl)     (dbl)    (dbl)}
\CommentTok{#> 1    Abilene     0.530         0.500 0.282      17.9 1.50e-23}
\CommentTok{#> 2   Amarillo     0.449         0.415 0.302      13.0 7.41e-18}
\CommentTok{#> 3  Arlington     0.513         0.483 0.267      16.8 2.75e-22}
\CommentTok{#> 4     Austin     0.487         0.455 0.310      15.1 2.04e-20}
\CommentTok{#> 5   Bay Area     0.555         0.527 0.265      19.9 1.45e-25}
\CommentTok{#> 6   Beaumont     0.430         0.395 0.275      12.0 1.18e-16}
\CommentTok{#> ..       ...       ...           ...   ...       ...      ...}
\CommentTok{#>       df}
\CommentTok{#>    (int)}
\CommentTok{#> 1     12}
\CommentTok{#> 2     12}
\CommentTok{#> 3     12}
\CommentTok{#> 4     12}
\CommentTok{#> 5     12}
\CommentTok{#> 6     12}
\CommentTok{#> ..   ...}
\CommentTok{#> Variables not shown: logLik (dbl), AIC (dbl), BIC (dbl), deviance}
\CommentTok{#>   (dbl), df.residual (int)}
\end{Highlighting}
\end{Shaded}

This creates a variable with one row for each city, and variables that
either summarise complexity (e.g. \texttt{df}) or fit (e.g.
\texttt{r.squared}, \texttt{p.value}, \texttt{AIC}). Since all the
models we fit have the same complexity (12 terms: one for each month),
we'll focus on the model fit summaries. \(R^2\) is a reasonable place to
start because it's well known. We can use a dot plot to see the
variation across cities:

\begin{Shaded}
\begin{Highlighting}[]
\KeywordTok{ggplot}\NormalTok{(model_sum, }\KeywordTok{aes}\NormalTok{(r.squared, }\KeywordTok{reorder}\NormalTok{(city, r.squared))) +}\StringTok{ }
\StringTok{  }\KeywordTok{geom_point}\NormalTok{()}
\end{Highlighting}
\end{Shaded}

\begin{figure}[H]
  \centering
  \includegraphics[width=0.65\linewidth]{_figures/modelling/unnamed-chunk-17-1}
\end{figure}

It's hard to picture exactly what those values of \(R^2\) mean, so it's
helpful to pick out a few exemplars. The following code extracts and
plots out the three cities with the highest and lowest \(R^2\):

\begin{Shaded}
\begin{Highlighting}[]
\NormalTok{top3 <-}\StringTok{ }\KeywordTok{c}\NormalTok{(}\StringTok{"Bryan-College Station"}\NormalTok{, }\StringTok{"Lubbock"}\NormalTok{, }\StringTok{"NE Tarrant County"}\NormalTok{)}
\NormalTok{bottom3 <-}\StringTok{ }\KeywordTok{c}\NormalTok{(}\StringTok{"McAllen"}\NormalTok{, }\StringTok{"Brownsville"}\NormalTok{, }\StringTok{"Harlingen"}\NormalTok{)}
\NormalTok{extreme <-}\StringTok{ }\NormalTok{txhousing %>%}\StringTok{ }\KeywordTok{ungroup}\NormalTok{() %>%}
\StringTok{  }\KeywordTok{filter}\NormalTok{(city %in%}\StringTok{ }\KeywordTok{c}\NormalTok{(top3, bottom3), !}\KeywordTok{is.na}\NormalTok{(sales)) %>%}
\StringTok{  }\KeywordTok{mutate}\NormalTok{(}\DataTypeTok{city =} \KeywordTok{factor}\NormalTok{(city, }\KeywordTok{c}\NormalTok{(top3, bottom3)))}

\KeywordTok{ggplot}\NormalTok{(extreme, }\KeywordTok{aes}\NormalTok{(month, }\KeywordTok{log}\NormalTok{(sales))) +}\StringTok{ }
\StringTok{  }\KeywordTok{geom_line}\NormalTok{(}\KeywordTok{aes}\NormalTok{(}\DataTypeTok{group =} \NormalTok{year)) +}\StringTok{ }
\StringTok{  }\KeywordTok{facet_wrap}\NormalTok{(~city)}
\end{Highlighting}
\end{Shaded}

\begin{figure}[H]
  \centering
  \includegraphics[width=0.65\linewidth]{_figures/modelling/unnamed-chunk-18-1}
\end{figure}

The cities with low \(R^2\) have weaker seasonal patterns and more
variation between years. The data for Harlingen seems particularly
noisy.

\subsection{Exercises}\label{exercises-2}

\begin{enumerate}
\def\labelenumi{\arabic{enumi}.}
\item
  Do your conclusions change if you use a different measurement of model
  fit like AIC or deviance? Why/why not?
\item
  One possible hypothesis that explains why McAllen, Harlingen and
  Brownsville have lower \(R^2\) is that they're smaller towns so there
  are fewer sales and more noise. Confirm or refute this hypothesis.
\item
  McAllen, Harlingen and Brownsville seem to have much more year-to-year
  variation than Bryan-College Station, Lubbock, and NE Tarrant County.
  How does the model change if you also include a linear trend for year?
  (i.e. \texttt{log(sales)\ \textasciitilde{}\ factor(month)\ +\ year}).
\item
  Create a faceted plot that shows the seasonal patterns for all
  cities.\\
  Order the facets by the \(R^2\) for the city.
\end{enumerate}

\section{Coefficient-level summaries}\label{coefficient-level-summaries}

The model fit summaries suggest that there are some important
differences in seasonality between the different cities. Let's dive into
those differences by using \texttt{tidy()} to extract detail about each
individual coefficient: \indexf{tidy}

\begin{Shaded}
\begin{Highlighting}[]
\NormalTok{coefs <-}\StringTok{ }\NormalTok{models %>%}\StringTok{ }\KeywordTok{tidy}\NormalTok{(mod)}
\NormalTok{coefs}
\CommentTok{#> Source: local data frame [552 x 6]}
\CommentTok{#> Groups: city [46]}
\CommentTok{#> }
\CommentTok{#>       city           term estimate std.error statistic   p.value}
\CommentTok{#>      (chr)          (chr)    (dbl)     (dbl)     (dbl)     (dbl)}
\CommentTok{#> 1  Abilene    (Intercept)    6.542    0.0704     92.88 7.90e-151}
\CommentTok{#> 2  Abilene factor(month)2    0.354    0.0996      3.55  4.91e-04}
\CommentTok{#> 3  Abilene factor(month)3    0.675    0.0996      6.77  1.83e-10}
\CommentTok{#> 4  Abilene factor(month)4    0.749    0.0996      7.52  2.76e-12}
\CommentTok{#> 5  Abilene factor(month)5    0.916    0.0996      9.20  1.06e-16}
\CommentTok{#> 6  Abilene factor(month)6    1.002    0.0996     10.06  4.37e-19}
\CommentTok{#> ..     ...            ...      ...       ...       ...       ...}
\end{Highlighting}
\end{Shaded}

We're more interested in the month effect, so we'll do a little extra
tidying to only look at the month coefficients, and then to extract the
month value into a numeric variable:

\begin{Shaded}
\begin{Highlighting}[]
\NormalTok{months <-}\StringTok{ }\NormalTok{coefs %>%}
\StringTok{  }\KeywordTok{filter}\NormalTok{(}\KeywordTok{grepl}\NormalTok{(}\StringTok{"factor"}\NormalTok{, term)) %>%}
\StringTok{  }\NormalTok{tidyr::}\KeywordTok{extract}\NormalTok{(term, }\StringTok{"month"}\NormalTok{, }\StringTok{"(}\CharTok{\textbackslash{}\textbackslash{}}\StringTok{d+)"}\NormalTok{, }\DataTypeTok{convert =} \OtherTok{TRUE}\NormalTok{)}
\NormalTok{months}
\CommentTok{#> Source: local data frame [506 x 6]}
\CommentTok{#> Groups: city [46]}
\CommentTok{#> }
\CommentTok{#>       city month estimate std.error statistic  p.value}
\CommentTok{#>      (chr) (int)    (dbl)     (dbl)     (dbl)    (dbl)}
\CommentTok{#> 1  Abilene     2    0.354    0.0996      3.55 4.91e-04}
\CommentTok{#> 2  Abilene     3    0.675    0.0996      6.77 1.83e-10}
\CommentTok{#> 3  Abilene     4    0.749    0.0996      7.52 2.76e-12}
\CommentTok{#> 4  Abilene     5    0.916    0.0996      9.20 1.06e-16}
\CommentTok{#> 5  Abilene     6    1.002    0.0996     10.06 4.37e-19}
\CommentTok{#> 6  Abilene     7    0.954    0.0996      9.58 9.81e-18}
\CommentTok{#> ..     ...   ...      ...       ...       ...      ...}
\end{Highlighting}
\end{Shaded}

This is a common pattern. You need to use your data tidying skills at
many points in an analysis. Once you have the correct tidy dataset,
creating the plot is usually easy. Here we'll put month on the x-axis,
estimate on the y-axis, and draw one line for each city. I've
back-transformed to make the coefficients more interpretable: these are
now ratios of sales compared to January.

\begin{Shaded}
\begin{Highlighting}[]
\KeywordTok{ggplot}\NormalTok{(months, }\KeywordTok{aes}\NormalTok{(month, }\DecValTok{2} \NormalTok{^}\StringTok{ }\NormalTok{estimate)) +}
\StringTok{  }\KeywordTok{geom_line}\NormalTok{(}\KeywordTok{aes}\NormalTok{(}\DataTypeTok{group =} \NormalTok{city))}
\end{Highlighting}
\end{Shaded}

\begin{figure}[H]
  \centering
  \includegraphics[width=0.65\linewidth]{_figures/modelling/unnamed-chunk-21-1}
\end{figure}

The pattern seems similar across the cities. The main difference is the
strength of the seasonal effect. Let's pull that out and plot it:

\begin{Shaded}
\begin{Highlighting}[]
\NormalTok{coef_sum <-}\StringTok{ }\NormalTok{months %>%}
\StringTok{  }\KeywordTok{group_by}\NormalTok{(city) %>%}
\StringTok{  }\KeywordTok{summarise}\NormalTok{(}\DataTypeTok{max =} \KeywordTok{max}\NormalTok{(estimate))}
\KeywordTok{ggplot}\NormalTok{(coef_sum, }\KeywordTok{aes}\NormalTok{(}\DecValTok{2} \NormalTok{^}\StringTok{ }\NormalTok{max, }\KeywordTok{reorder}\NormalTok{(city, max))) +}\StringTok{ }
\StringTok{  }\KeywordTok{geom_point}\NormalTok{()}
\end{Highlighting}
\end{Shaded}

\begin{figure}[H]
  \centering
  \includegraphics[width=0.65\linewidth]{_figures/modelling/unnamed-chunk-22-1}
\end{figure}

The cities with the strongest seasonal effect are College Station and
San Marcos (both college towns) and Galveston and South Padre Island
(beach cities). It makes sense that these cities would have very strong
seasonal effects.

\subsection{Exercises}\label{exercises-3}

\begin{enumerate}
\def\labelenumi{\arabic{enumi}.}
\item
  Pull out the three cities with highest and lowest seasonal effect.
  Plot their coefficients.
\item
  How does strength of seasonal effect relate to the \(R^2\) for the
  model? Answer with a plot.
\item
  You should be extra cautious when your results agree with your prior
  beliefs. How can you confirm or refute my hypothesis about the causes
  of strong seasonal patterns?
\item
  Group the diamonds data by cut, clarity and colour. Fit a linear model
  \texttt{log(price)\ \textasciitilde{}\ log(carat)}. What does the
  intercept tell you? What does the slope tell you? How do the slope and
  intercept vary across the groups? Answer with a plot.
\end{enumerate}

\section{Observation data}\label{observation-data}

Observation-level data, which include residual diagnostics, is most
useful in the traditional model fitting scenario, because it can helps
you find ``high-leverage'' points, point that have a big influence on
the final model. It's also useful in conjunction with visualisation,
particularly because it provides an alternative way to access the
residuals.

Extracting observation-level data is the job of the \texttt{augment()}
function. This adds one row for each observation. It includes the
variables used in the original model, the residuals, and a number of
common influence statistics (see \texttt{?augment.lm} for more details):
\indexf{augment}

\begin{Shaded}
\begin{Highlighting}[]
\NormalTok{obs_sum <-}\StringTok{ }\NormalTok{models %>%}\StringTok{ }\KeywordTok{augment}\NormalTok{(mod)}
\NormalTok{obs_sum}
\CommentTok{#> Source: local data frame [8,034 x 10]}
\CommentTok{#> Groups: city [46]}
\CommentTok{#> }
\CommentTok{#>       city log2.sales. factor.month. .fitted .se.fit .resid   .hat}
\CommentTok{#>      (chr)       (dbl)        (fctr)   (dbl)   (dbl)  (dbl)  (dbl)}
\CommentTok{#> 1  Abilene        6.17             1    6.54  0.0704 -0.372 0.0625}
\CommentTok{#> 2  Abilene        6.61             2    6.90  0.0704 -0.281 0.0625}
\CommentTok{#> 3  Abilene        7.02             3    7.22  0.0704 -0.194 0.0625}
\CommentTok{#> 4  Abilene        6.61             4    7.29  0.0704 -0.676 0.0625}
\CommentTok{#> 5  Abilene        7.14             5    7.46  0.0704 -0.319 0.0625}
\CommentTok{#> 6  Abilene        7.29             6    7.54  0.0704 -0.259 0.0625}
\CommentTok{#> ..     ...         ...           ...     ...     ...    ...    ...}
\CommentTok{#> Variables not shown: .sigma (dbl), .cooksd (dbl), .std.resid (dbl)}
\end{Highlighting}
\end{Shaded}

For example, it might be interesting to look at the distribution of
standardised residuals. (These are residuals standardised to have a
variance of one in each model, making them more comparable). We're
looking for unusual values that might need deeper exploration:

\begin{Shaded}
\begin{Highlighting}[]
\KeywordTok{ggplot}\NormalTok{(obs_sum, }\KeywordTok{aes}\NormalTok{(.std.resid)) +}\StringTok{ }
\StringTok{  }\KeywordTok{geom_histogram}\NormalTok{(}\DataTypeTok{binwidth =} \FloatTok{0.1}\NormalTok{)}
\KeywordTok{ggplot}\NormalTok{(obs_sum, }\KeywordTok{aes}\NormalTok{(}\KeywordTok{abs}\NormalTok{(.std.resid))) +}\StringTok{ }
\StringTok{  }\KeywordTok{geom_histogram}\NormalTok{(}\DataTypeTok{binwidth =} \FloatTok{0.1}\NormalTok{)}
\end{Highlighting}
\end{Shaded}

\begin{figure}[H]
  \includegraphics[width=0.5\linewidth]{_figures/modelling/unnamed-chunk-25-1}%
  \includegraphics[width=0.5\linewidth]{_figures/modelling/unnamed-chunk-25-2}
\end{figure}

A threshold of 2 seems like a reasonable threshold to explore
individually:

\begin{Shaded}
\begin{Highlighting}[]
\NormalTok{obs_sum %>%}\StringTok{ }
\StringTok{  }\KeywordTok{filter}\NormalTok{(}\KeywordTok{abs}\NormalTok{(.std.resid) >}\StringTok{ }\DecValTok{2}\NormalTok{) %>%}
\StringTok{  }\KeywordTok{group_by}\NormalTok{(city) %>%}
\StringTok{  }\KeywordTok{summarise}\NormalTok{(}\DataTypeTok{n =} \KeywordTok{n}\NormalTok{(), }\DataTypeTok{avg =} \KeywordTok{mean}\NormalTok{(}\KeywordTok{abs}\NormalTok{(.std.resid))) %>%}
\StringTok{  }\KeywordTok{arrange}\NormalTok{(}\KeywordTok{desc}\NormalTok{(n))}
\CommentTok{#> Source: local data frame [43 x 3]}
\CommentTok{#> }
\CommentTok{#>               city     n   avg}
\CommentTok{#>              (chr) (int) (dbl)}
\CommentTok{#> 1        Texarkana    12  2.43}
\CommentTok{#> 2        Harlingen    11  2.73}
\CommentTok{#> 3             Waco    11  2.96}
\CommentTok{#> 4         Victoria    10  2.49}
\CommentTok{#> 5  Brazoria County     9  2.31}
\CommentTok{#> 6      Brownsville     9  2.48}
\CommentTok{#> ..             ...   ...   ...}
\end{Highlighting}
\end{Shaded}

In a real analysis, you'd want to look into these cities in more detail.

\subsection{Exercises}\label{exercises-4}

\begin{enumerate}
\def\labelenumi{\arabic{enumi}.}
\item
  A common diagnotic plot is fitted values (\texttt{.fitted})
  vs.~residuals (\texttt{.resid}). Do you see any patterns? What if you
  include the city or month on the same plot?
\item
  Create a time series of log(sales) for each city. Highlight points
  that have a standardised residual of greater than 2.
\end{enumerate}

\section*{References}\label{references}
\addcontentsline{toc}{section}{References}

\hypertarget{refs}{}
\hypertarget{ref-model-vis-paper}{}
Wickham, Hadley, Dianne Cook, and Heike Hofmann. 2015. ``Visualizing
Statistical Models: Removing the Blindfold.'' \emph{Statistical Analysis
and Data Mining: The ASA Data Science Journal} 8 (4): 203--25.

\chapter{Programming with ggplot2}\label{cha:programming}

\section{Introduction}

A major requirement of a good data analysis is flexibility. If your data
changes, or you discover something that makes you rethink your basic
assumptions, you need to be able to easily change many plots at once.
The main inhibitor of flexibility is code duplication. If you have the
same plotting statement repeated over and over again, you'll have to
make the same change in many different places. Often just the thought of
making all those changes is exhausting! This chapter will help you
overcome that by problem by showing you how to program with ggplot2.
\index{Programming}

To make your code more flexible, you need to reduce duplicated code by
writing functions. When you notice you're doing the same thing over and
over again, think about how you might generalise it and turn it into a
function. If you're not that familiar with how functions work in R, you
might want to brush up your knowledge at
\url{http://adv-r.had.co.nz/Functions.html}.

In this chapter I'll show how to write functions that create:

\begin{itemize}
\tightlist
\item
  A single ggplot2 component.
\item
  Multiple ggplot2 components.
\item
  A complete plot
\end{itemize}

And then I'll finish off with a brief illustration of how you can apply
functional programming techniques to ggplot2 objects.

You might also find the
\href{https://github.com/wilkelab/cowplot}{cowplot} and
\href{https://github.com/jrnold/ggthemes}{ggthemes} packages helpful. As
well as providing reusable components that help you directly, you can
also read the source code of the packages to figure out how they work.

\section{Single components}

Each component of a ggplot plot is an object. Most of the time you
create the component and immediately add it to a plot, but you don't
have to. Instead, you can save any component to a variable (giving it a
name), and then add it to multiple plots:

\begin{Shaded}
\begin{Highlighting}[]
\NormalTok{bestfit <-}\StringTok{ }\KeywordTok{geom_smooth}\NormalTok{(}
  \DataTypeTok{method =} \StringTok{"lm"}\NormalTok{, }
  \DataTypeTok{se =} \OtherTok{FALSE}\NormalTok{, }
  \DataTypeTok{colour =} \KeywordTok{alpha}\NormalTok{(}\StringTok{"steelblue"}\NormalTok{, }\FloatTok{0.5}\NormalTok{), }
  \DataTypeTok{size =} \DecValTok{2}
\NormalTok{)}
\KeywordTok{ggplot}\NormalTok{(mpg, }\KeywordTok{aes}\NormalTok{(cty, hwy)) +}\StringTok{ }
\StringTok{  }\KeywordTok{geom_point}\NormalTok{() +}\StringTok{ }
\StringTok{  }\NormalTok{bestfit}
\KeywordTok{ggplot}\NormalTok{(mpg, }\KeywordTok{aes}\NormalTok{(displ, hwy)) +}\StringTok{ }
\StringTok{  }\KeywordTok{geom_point}\NormalTok{() +}\StringTok{ }
\StringTok{  }\NormalTok{bestfit}
\end{Highlighting}
\end{Shaded}

\begin{figure}[H]
  \centering
  \includegraphics[width=0.375\linewidth]{_figures/programming/layer9-1}%
  \includegraphics[width=0.375\linewidth]{_figures/programming/layer9-2}
\end{figure}

That's a great way to reduce simple types of duplication (it's much
better than copying-and-pasting!), but requires that the component be
exactly the same each time. If you need more flexibility, you can wrap
these reusable snippets in a function. For example, we could extend our
\texttt{bestfit} object to a more general function for adding lines of
best fit to a plot. The following code creates a \texttt{geom\_lm()}
with three parameters: the model \texttt{formula}, the line
\texttt{colour} and the line \texttt{size}:

\begin{Shaded}
\begin{Highlighting}[]
\NormalTok{geom_lm <-}\StringTok{ }\NormalTok{function(}\DataTypeTok{formula =} \NormalTok{y ~}\StringTok{ }\NormalTok{x, }\DataTypeTok{colour =} \KeywordTok{alpha}\NormalTok{(}\StringTok{"steelblue"}\NormalTok{, }\FloatTok{0.5}\NormalTok{), }
                    \DataTypeTok{size =} \DecValTok{2}\NormalTok{, ...)  \{}
  \KeywordTok{geom_smooth}\NormalTok{(}\DataTypeTok{formula =} \NormalTok{formula, }\DataTypeTok{se =} \OtherTok{FALSE}\NormalTok{, }\DataTypeTok{method =} \StringTok{"lm"}\NormalTok{, }\DataTypeTok{colour =} \NormalTok{colour,}
    \DataTypeTok{size =} \NormalTok{size, ...)}
\NormalTok{\}}
\KeywordTok{ggplot}\NormalTok{(mpg, }\KeywordTok{aes}\NormalTok{(displ, }\DecValTok{1} \NormalTok{/}\StringTok{ }\NormalTok{hwy)) +}\StringTok{ }
\StringTok{  }\KeywordTok{geom_point}\NormalTok{() +}\StringTok{ }
\StringTok{  }\KeywordTok{geom_lm}\NormalTok{()}
\KeywordTok{ggplot}\NormalTok{(mpg, }\KeywordTok{aes}\NormalTok{(displ, }\DecValTok{1} \NormalTok{/}\StringTok{ }\NormalTok{hwy)) +}\StringTok{ }
\StringTok{  }\KeywordTok{geom_point}\NormalTok{() +}\StringTok{ }
\StringTok{  }\KeywordTok{geom_lm}\NormalTok{(y ~}\StringTok{ }\KeywordTok{poly}\NormalTok{(x, }\DecValTok{2}\NormalTok{), }\DataTypeTok{size =} \DecValTok{1}\NormalTok{, }\DataTypeTok{colour =} \StringTok{"red"}\NormalTok{)}
\end{Highlighting}
\end{Shaded}

\begin{figure}[H]
  \centering
  \includegraphics[width=0.375\linewidth]{_figures/programming/geom-lm-1}%
  \includegraphics[width=0.375\linewidth]{_figures/programming/geom-lm-2}
\end{figure}

Pay close attention to the use of \texttt{...}. When included in the
function definition \texttt{...} allows a function to accept arbitrary
additional arguments. Inside the function, you can then use \texttt{...}
to pass those arguments on to another function. Here we pass
\texttt{...} onto \texttt{geom\_smooth()} so the user can still modify
all the other arguments we haven't explicit overridden. When you write
your own component functions, it's a good idea to always use
\texttt{...} is this way. \indexc{...}

Finally, note that you can only \emph{add} components to a plot: you
can't modify or remove existing objects.

\subsection{Exercises}

\begin{enumerate}
\def\labelenumi{\arabic{enumi}.}
\item
  Create an object that represents a pink histogram with 100 bins.
\item
  Create an object that represents a fill scale with the Blues
  ColorBrewer palette.
\item
  Read the source code for \texttt{theme\_grey()}. What are its
  arguments? How does it work?
\item
  Create \texttt{scale\_colour\_wesanderson()}. It should have a
  parameter to pick the palette from the wesanderson package, and create
  either a continuous or discrete scale.
\end{enumerate}

\section{Multiple components}

It's not always possible to achieve your goals with a single component.
Fortunately, ggplot2 has a convenient way of adding multiple components
to a plot in one step with a list. The following function adds two
layers: one to show the mean, and one to show its standard error:

\begin{Shaded}
\begin{Highlighting}[]
\NormalTok{geom_mean <-}\StringTok{ }\NormalTok{function() \{}
  \KeywordTok{list}\NormalTok{(}
    \KeywordTok{stat_summary}\NormalTok{(}\DataTypeTok{fun.y =} \StringTok{"mean"}\NormalTok{, }\DataTypeTok{geom =} \StringTok{"bar"}\NormalTok{, }\DataTypeTok{fill =} \StringTok{"grey70"}\NormalTok{),}
    \KeywordTok{stat_summary}\NormalTok{(}\DataTypeTok{fun.data =} \StringTok{"mean_cl_normal"}\NormalTok{, }\DataTypeTok{geom =} \StringTok{"errorbar"}\NormalTok{, }\DataTypeTok{width =} \FloatTok{0.4}\NormalTok{)}
  \NormalTok{)}
\NormalTok{\}}
\KeywordTok{ggplot}\NormalTok{(mpg, }\KeywordTok{aes}\NormalTok{(class, cty)) +}\StringTok{ }\KeywordTok{geom_mean}\NormalTok{()}
\CommentTok{#> Warning: replacing previous import by 'ggplot2::unit' when loading}
\CommentTok{#> 'Hmisc'}
\CommentTok{#> Warning: replacing previous import by 'ggplot2::arrow' when loading}
\CommentTok{#> 'Hmisc'}
\CommentTok{#> Warning: replacing previous import by 'scales::alpha' when loading}
\CommentTok{#> 'Hmisc'}
\KeywordTok{ggplot}\NormalTok{(mpg, }\KeywordTok{aes}\NormalTok{(drv, cty)) +}\StringTok{ }\KeywordTok{geom_mean}\NormalTok{()}
\end{Highlighting}
\end{Shaded}

\begin{figure}[H]
  \centering
  \includegraphics[width=0.375\linewidth]{_figures/programming/geom-mean-1-1}%
  \includegraphics[width=0.375\linewidth]{_figures/programming/geom-mean-1-2}
\end{figure}

If the list contains any \texttt{NULL} elements, they're ignored. This
makes it easy to conditionally add components:

\begin{Shaded}
\begin{Highlighting}[]
\NormalTok{geom_mean <-}\StringTok{ }\NormalTok{function(}\DataTypeTok{se =} \OtherTok{TRUE}\NormalTok{) \{}
  \KeywordTok{list}\NormalTok{(}
    \KeywordTok{stat_summary}\NormalTok{(}\DataTypeTok{fun.y =} \StringTok{"mean"}\NormalTok{, }\DataTypeTok{geom =} \StringTok{"bar"}\NormalTok{, }\DataTypeTok{fill =} \StringTok{"grey70"}\NormalTok{),}
    \NormalTok{if (se) }
      \KeywordTok{stat_summary}\NormalTok{(}\DataTypeTok{fun.data =} \StringTok{"mean_cl_normal"}\NormalTok{, }\DataTypeTok{geom =} \StringTok{"errorbar"}\NormalTok{, }\DataTypeTok{width =} \FloatTok{0.4}\NormalTok{)}
  \NormalTok{)}
\NormalTok{\}}
\KeywordTok{ggplot}\NormalTok{(mpg, }\KeywordTok{aes}\NormalTok{(drv, cty)) +}\StringTok{ }\KeywordTok{geom_mean}\NormalTok{()}
\CommentTok{#> Warning: replacing previous import by 'ggplot2::unit' when loading}
\CommentTok{#> 'Hmisc'}
\CommentTok{#> Warning: replacing previous import by 'ggplot2::arrow' when loading}
\CommentTok{#> 'Hmisc'}
\CommentTok{#> Warning: replacing previous import by 'scales::alpha' when loading}
\CommentTok{#> 'Hmisc'}
\KeywordTok{ggplot}\NormalTok{(mpg, }\KeywordTok{aes}\NormalTok{(drv, cty)) +}\StringTok{ }\KeywordTok{geom_mean}\NormalTok{(}\DataTypeTok{se =} \OtherTok{FALSE}\NormalTok{)}
\end{Highlighting}
\end{Shaded}

\begin{figure}[H]
  \centering
  \includegraphics[width=0.375\linewidth]{_figures/programming/geom-mean-2-1}%
  \includegraphics[width=0.375\linewidth]{_figures/programming/geom-mean-2-2}
\end{figure}

\subsection{Plot components}

You're not just limited to adding layers in this way. You can also
include any of the following object types in the list:

\begin{itemize}
\item
  A data.frame, which will override the default dataset associated with
  the plot. (If you add a data frame by itself, you'll need to use
  \texttt{\%+\%}, but this is not necessary if the data frame is in a
  list)
\item
  An \texttt{aes()} object, which will combined with the existing
  default aesthetic mapping.
\item
  Scales, which override existing scales, with a warning if they've
  already been set by the user.
\item
  Coordinate systems and facetting specification, which override the
  existing settings.
\item
  Theme components, which override the specified components.
\end{itemize}

\subsection{Annotation}

It's often useful to add standard annotations to a plot. In this case,
your function will also set the data in the layer function, rather than
inheriting it from the plot. There are two other options that you should
set when you do this. These ensure that the layer is self-contained:
\index{Annotation!functions}

\begin{itemize}
\item
  \texttt{inherit.aes\ =\ FALSE} prevents the layer from inheriting
  aesthetics from the parent plot. This ensures your annotation works
  regardless of what else is on the plot. \indexc{inherit.aes}
\item
  \texttt{show.legend\ =\ FALSE} ensures that your annotation won't
  appear in the legend. \indexc{show.legend}
\end{itemize}

One example of this technique is the \texttt{borders()} function built
into ggplot2. It's designed to add map borders from one of the datasets
in the maps package: \indexf{borders}

\begin{Shaded}
\begin{Highlighting}[]
\NormalTok{borders <-}\StringTok{ }\NormalTok{function(}\DataTypeTok{database =} \StringTok{"world"}\NormalTok{, }\DataTypeTok{regions =} \StringTok{"."}\NormalTok{, }\DataTypeTok{fill =} \OtherTok{NA}\NormalTok{, }
                    \DataTypeTok{colour =} \StringTok{"grey50"}\NormalTok{, ...) \{}
  \NormalTok{df <-}\StringTok{ }\KeywordTok{map_data}\NormalTok{(database, regions)}
  \KeywordTok{geom_polygon}\NormalTok{(}
    \KeywordTok{aes_}\NormalTok{(~lat, ~long, }\DataTypeTok{group =} \NormalTok{~group), }
    \DataTypeTok{data =} \NormalTok{df, }\DataTypeTok{fill =} \NormalTok{fill, }\DataTypeTok{colour =} \NormalTok{colour, ..., }
    \DataTypeTok{inherit.aes =} \OtherTok{FALSE}\NormalTok{, }\DataTypeTok{show.legend =} \OtherTok{FALSE}
  \NormalTok{)}
\NormalTok{\}}
\end{Highlighting}
\end{Shaded}

\subsection{Additional arguments}

If you want to pass additional arguments to the components in your
function, \texttt{...} is no good: there's no way to direct different
arguments to different components. Instead, you'll need to think about
how you want your function to work, balancing the benefits of having one
function that does it all vs.~the cost of having a complex function
that's harder to understand. \indexc{...}

To get you started, here's one approach using \texttt{modifyList()} and
\texttt{do.call()}: \indexf{modifyList} \indexf{do.call}

\begin{Shaded}
\begin{Highlighting}[]
\NormalTok{geom_mean <-}\StringTok{ }\NormalTok{function(..., }\DataTypeTok{bar.params =} \KeywordTok{list}\NormalTok{(), }\DataTypeTok{errorbar.params =} \KeywordTok{list}\NormalTok{()) \{}
  \NormalTok{params <-}\StringTok{ }\KeywordTok{list}\NormalTok{(...)}
  \NormalTok{bar.params <-}\StringTok{ }\KeywordTok{modifyList}\NormalTok{(params, bar.params)}
  \NormalTok{errorbar.params  <-}\StringTok{ }\KeywordTok{modifyList}\NormalTok{(params, errorbar.params)}
  
  \NormalTok{bar <-}\StringTok{ }\KeywordTok{do.call}\NormalTok{(}\StringTok{"stat_summary"}\NormalTok{, }\KeywordTok{modifyList}\NormalTok{(}
    \KeywordTok{list}\NormalTok{(}\DataTypeTok{fun.y =} \StringTok{"mean"}\NormalTok{, }\DataTypeTok{geom =} \StringTok{"bar"}\NormalTok{, }\DataTypeTok{fill =} \StringTok{"grey70"}\NormalTok{),}
    \NormalTok{bar.params)}
  \NormalTok{)}
  \NormalTok{errorbar <-}\StringTok{ }\KeywordTok{do.call}\NormalTok{(}\StringTok{"stat_summary"}\NormalTok{, }\KeywordTok{modifyList}\NormalTok{(}
    \KeywordTok{list}\NormalTok{(}\DataTypeTok{fun.data =} \StringTok{"mean_cl_normal"}\NormalTok{, }\DataTypeTok{geom =} \StringTok{"errorbar"}\NormalTok{, }\DataTypeTok{width =} \FloatTok{0.4}\NormalTok{),}
    \NormalTok{errorbar.params)}
  \NormalTok{)}

  \KeywordTok{list}\NormalTok{(bar, errorbar)}
\NormalTok{\}}

\KeywordTok{ggplot}\NormalTok{(mpg, }\KeywordTok{aes}\NormalTok{(class, cty)) +}\StringTok{ }
\StringTok{  }\KeywordTok{geom_mean}\NormalTok{(}
    \DataTypeTok{colour =} \StringTok{"steelblue"}\NormalTok{,}
    \DataTypeTok{errorbar.params =} \KeywordTok{list}\NormalTok{(}\DataTypeTok{width =} \FloatTok{0.5}\NormalTok{, }\DataTypeTok{size =} \DecValTok{1}\NormalTok{)}
  \NormalTok{)}
\KeywordTok{ggplot}\NormalTok{(mpg, }\KeywordTok{aes}\NormalTok{(class, cty)) +}\StringTok{ }
\StringTok{  }\KeywordTok{geom_mean}\NormalTok{(}
    \DataTypeTok{bar.params =} \KeywordTok{list}\NormalTok{(}\DataTypeTok{fill =} \StringTok{"steelblue"}\NormalTok{),}
    \DataTypeTok{errorbar.params =} \KeywordTok{list}\NormalTok{(}\DataTypeTok{colour =} \StringTok{"blue"}\NormalTok{)}
  \NormalTok{)}
\end{Highlighting}
\end{Shaded}

\begin{figure}[H]
  \centering
  \includegraphics[width=0.375\linewidth]{_figures/programming/unnamed-chunk-3-1}%
  \includegraphics[width=0.375\linewidth]{_figures/programming/unnamed-chunk-3-2}
\end{figure}

If you need more complex behaviour, it might be easier to create a
custom geom or stat. You can learn about that in the extending ggplot2
vignette included with the package. Read it by running
\texttt{vignette("extending-ggplot2")}.

\subsection{Exercises}

\begin{enumerate}
\def\labelenumi{\arabic{enumi}.}
\item
  To make the best use of space, many examples in this book hide the
  axes labels and legend. I've just copied-and-pasted the same code into
  multiple places, but it would make more sense to create a reusable
  function. What would that function look like?
\item
  Extend the \texttt{borders()} function to also add
  \texttt{coord\_quickmap()} to the plot.
\item
  Look through your own code. What combinations of geoms or scales do
  you use all the time? How could you extract the pattern into a
  reusable function?
\end{enumerate}

\section{Plot functions}\label{sec:functions}

Creating small reusable components is most in line with the ggplot2
spirit: you can recombine them flexibly to create whatever plot you
want. But sometimes you're creating the same plot over and over again,
and you don't need that flexibility. Instead of creating components, you
might want to write a function that takes data and parameters and
returns a complete plot. \index{Plot functions}

For example, you could wrap up the complete code needed to make a
piechart:

\begin{Shaded}
\begin{Highlighting}[]
\NormalTok{piechart <-}\StringTok{ }\NormalTok{function(data, mapping) \{}
  \KeywordTok{ggplot}\NormalTok{(data, mapping) +}
\StringTok{    }\KeywordTok{geom_bar}\NormalTok{(}\DataTypeTok{width =} \DecValTok{1}\NormalTok{) +}\StringTok{ }
\StringTok{    }\KeywordTok{coord_polar}\NormalTok{(}\DataTypeTok{theta =} \StringTok{"y"}\NormalTok{) +}\StringTok{ }
\StringTok{    }\KeywordTok{xlab}\NormalTok{(}\OtherTok{NULL}\NormalTok{) +}\StringTok{ }
\StringTok{    }\KeywordTok{ylab}\NormalTok{(}\OtherTok{NULL}\NormalTok{)}
\NormalTok{\}}
\KeywordTok{piechart}\NormalTok{(mpg, }\KeywordTok{aes}\NormalTok{(}\KeywordTok{factor}\NormalTok{(}\DecValTok{1}\NormalTok{), }\DataTypeTok{fill =} \NormalTok{class))}
\end{Highlighting}
\end{Shaded}

\begin{figure}[H]
  \centering
  \includegraphics[width=0.5\linewidth]{_figures/programming/unnamed-chunk-4-1}
\end{figure}

This is much less flexible than the component based approach, but
equally, it's much more concise.

You can take a similar approach to drawing parallel coordinates plots
(PCPs). PCPs require a transformation of the data, so I recommend
writing two functions: one that does the transformation and one that
generates the plot. Keeping these two piece separate makes life much
easier if you later want to reuse the same transformation for a
different visualisation. \index{Parallel coordinate plots}

\begin{Shaded}
\begin{Highlighting}[]
\NormalTok{pcp_data <-}\StringTok{ }\NormalTok{function(df) \{}
  \NormalTok{is_numeric <-}\StringTok{ }\KeywordTok{vapply}\NormalTok{(df, is.numeric, }\KeywordTok{logical}\NormalTok{(}\DecValTok{1}\NormalTok{))}

  \CommentTok{# Rescale numeric columns}
  \NormalTok{rescale01 <-}\StringTok{ }\NormalTok{function(x) \{}
    \NormalTok{rng <-}\StringTok{ }\KeywordTok{range}\NormalTok{(x, }\DataTypeTok{na.rm =} \OtherTok{TRUE}\NormalTok{)}
    \NormalTok{(x -}\StringTok{ }\NormalTok{rng[}\DecValTok{1}\NormalTok{]) /}\StringTok{ }\NormalTok{(rng[}\DecValTok{2}\NormalTok{] -}\StringTok{ }\NormalTok{rng[}\DecValTok{1}\NormalTok{])}
  \NormalTok{\}}
  \NormalTok{df[is_numeric] <-}\StringTok{ }\KeywordTok{lapply}\NormalTok{(df[is_numeric], rescale01)}
  
  \CommentTok{# Add row identifier}
  \NormalTok{df$.row <-}\StringTok{ }\KeywordTok{rownames}\NormalTok{(df)}
  
  \CommentTok{# Treat numerics as value (aka measure) variables }
  \NormalTok{tidyr::}\KeywordTok{gather_}\NormalTok{(df, }\StringTok{"variable"}\NormalTok{, }\StringTok{"value"}\NormalTok{, }\KeywordTok{names}\NormalTok{(df)[is_numeric])}
\NormalTok{\}}
\NormalTok{pcp <-}\StringTok{ }\NormalTok{function(df, ...) \{}
  \NormalTok{df <-}\StringTok{ }\KeywordTok{pcp_data}\NormalTok{(df)}
  \KeywordTok{ggplot}\NormalTok{(df, }\KeywordTok{aes}\NormalTok{(variable, value, }\DataTypeTok{group =} \NormalTok{.row)) +}\StringTok{ }\KeywordTok{geom_line}\NormalTok{(...)}
\NormalTok{\}}
\KeywordTok{pcp}\NormalTok{(mpg)}
\KeywordTok{pcp}\NormalTok{(mpg, }\KeywordTok{aes}\NormalTok{(}\DataTypeTok{colour =} \NormalTok{drv))}
\end{Highlighting}
\end{Shaded}

\begin{figure}[H]
  \includegraphics[width=0.5\linewidth]{_figures/programming/pcp_data-1}%
  \includegraphics[width=0.5\linewidth]{_figures/programming/pcp_data-2}
\end{figure}

A complete exploration of this idea is \texttt{qplot()}, which provides
a fairly deep wrapper around the most common \texttt{ggplot()} options.
I recommend studying the source code if you want to see how far these
basic techniques can take you. \indexf{qplot}

\subsection{Indirectly referring to variables}

The \texttt{piechart()} function above is a little unappealing because
it requires the user to know the exact \texttt{aes()} specification that
generates a pie chart. It would be more convenient if the user could
simply specify the name of the variable to plot. To do that you'll need
to learn a bit more about how \texttt{aes()} works.

\texttt{aes()} uses non-standard evaluation: rather than looking at the
values of its arguments, it looks at their expressions. This makes it
difficult to work with programmatically - there's no way to store the
name of a variable in an object and then refer to it later:

\begin{Shaded}
\begin{Highlighting}[]
\NormalTok{x_var <-}\StringTok{ "displ"}
\KeywordTok{aes}\NormalTok{(x_var)}
\CommentTok{#> * x -> x_var}
\end{Highlighting}
\end{Shaded}

Instead we need to use \texttt{aes\_()}, which uses regular evaluation.
There are two basic ways to create a mapping with \texttt{aes\_()}:
\indexf{aes\_}

\begin{itemize}
\item
  Using a \emph{quoted call}, created by \texttt{quote()},
  \texttt{substitute()}, \texttt{as.name()}, or \texttt{parse()}.
  \indexf{quote} \indexf{substitute} \indexf{parse} \indexf{as.name}

\begin{Shaded}
\begin{Highlighting}[]
\KeywordTok{aes_}\NormalTok{(}\KeywordTok{quote}\NormalTok{(displ))}
\CommentTok{#> * x -> displ}
\KeywordTok{aes_}\NormalTok{(}\KeywordTok{as.name}\NormalTok{(x_var))}
\CommentTok{#> * x -> displ}
\KeywordTok{aes_}\NormalTok{(}\KeywordTok{parse}\NormalTok{(}\DataTypeTok{text =} \NormalTok{x_var)[[}\DecValTok{1}\NormalTok{]])}
\CommentTok{#> * x -> displ}

\NormalTok{f <-}\StringTok{ }\NormalTok{function(x_var) \{}
  \KeywordTok{aes_}\NormalTok{(}\KeywordTok{substitute}\NormalTok{(x_var))}
\NormalTok{\}}
\KeywordTok{f}\NormalTok{(displ)}
\CommentTok{#> * x -> displ}
\end{Highlighting}
\end{Shaded}

  The difference between \texttt{as.name()} and \texttt{parse()} is
  subtle. If \texttt{x\_var} is ``a + b'', \texttt{as.name()} will turn
  it into a variable called \texttt{`a\ +\ b`}, \texttt{parse()} will
  turn it into the function call \texttt{a\ +\ b}. (If this is confusing
  \url{http://adv-r.had.co.nz/Expressions.html} might help).
\item
  Using a formula, created with \texttt{\textasciitilde{}}.
  \indexc{\textasciitilde}

\begin{Shaded}
\begin{Highlighting}[]
\KeywordTok{aes_}\NormalTok{(~displ)}
\CommentTok{#> * x -> displ}
\end{Highlighting}
\end{Shaded}
\end{itemize}

\texttt{aes\_()} gives us three options for how a user can supply
variables: as a string, as a formula, or as a bare expression. These
three options are illustrated below

\begin{Shaded}
\begin{Highlighting}[]
\NormalTok{piechart1 <-}\StringTok{ }\NormalTok{function(data, var, ...) \{}
  \KeywordTok{piechart}\NormalTok{(data, }\KeywordTok{aes_}\NormalTok{(~}\KeywordTok{factor}\NormalTok{(}\DecValTok{1}\NormalTok{), }\DataTypeTok{fill =} \KeywordTok{as.name}\NormalTok{(var)))}
\NormalTok{\}}
\KeywordTok{piechart1}\NormalTok{(mpg, }\StringTok{"class"}\NormalTok{) +}\StringTok{ }\KeywordTok{theme}\NormalTok{(}\DataTypeTok{legend.position =} \StringTok{"none"}\NormalTok{)}

\NormalTok{piechart2 <-}\StringTok{ }\NormalTok{function(data, var, ...) \{}
  \KeywordTok{piechart}\NormalTok{(data, }\KeywordTok{aes_}\NormalTok{(~}\KeywordTok{factor}\NormalTok{(}\DecValTok{1}\NormalTok{), }\DataTypeTok{fill =} \NormalTok{var))}
\NormalTok{\}}
\KeywordTok{piechart2}\NormalTok{(mpg, ~class) +}\StringTok{ }\KeywordTok{theme}\NormalTok{(}\DataTypeTok{legend.position =} \StringTok{"none"}\NormalTok{)}

\NormalTok{piechart3 <-}\StringTok{ }\NormalTok{function(data, var, ...) \{}
  \KeywordTok{piechart}\NormalTok{(data, }\KeywordTok{aes_}\NormalTok{(~}\KeywordTok{factor}\NormalTok{(}\DecValTok{1}\NormalTok{), }\DataTypeTok{fill =} \KeywordTok{substitute}\NormalTok{(var)))}
\NormalTok{\}}
\KeywordTok{piechart3}\NormalTok{(mpg, class) +}\StringTok{ }\KeywordTok{theme}\NormalTok{(}\DataTypeTok{legend.position =} \StringTok{"none"}\NormalTok{)}
\end{Highlighting}
\end{Shaded}

\begin{figure}[H]
  \includegraphics[width=0.333\linewidth]{_figures/programming/unnamed-chunk-8-1}%
  \includegraphics[width=0.333\linewidth]{_figures/programming/unnamed-chunk-8-2}%
  \includegraphics[width=0.333\linewidth]{_figures/programming/unnamed-chunk-8-3}
\end{figure}

There's another advantage to \texttt{aes\_()} over \texttt{aes()} if
you're writing ggplot2 plots inside a package: using
\texttt{aes\_(\textasciitilde{}x,\ \textasciitilde{}y)} instead of
\texttt{aes(x,\ y)} avoids the global variables NOTE in
\texttt{R\ CMD\ check}. \index{Global variables}

\subsection{The plot environment}

As you create more sophisticated plotting functions, you'll need to
understand a bit more about ggplot2's scoping rules. ggplot2 was written
well before I understood the full intricacies of non-standard evaluation
so it has a rather simple scoping system. If a variable is not found in
the \texttt{data}, it is looked for in \emph{the} plot environment.
There is only one environment for a plot (not one for each layer), and
it is the environment in which \texttt{ggplot()} is called from
(i.e.~the \texttt{parent.frame()}). \index{Environments}
\indexf{parent.frame}

This means that the following function won't work because \texttt{n} is
not stored in an environment accessible when the expressions in
\texttt{aes()} are evaluated.

\begin{Shaded}
\begin{Highlighting}[]
\NormalTok{f <-}\StringTok{ }\NormalTok{function() \{}
  \NormalTok{n <-}\StringTok{ }\DecValTok{10}
  \KeywordTok{geom_line}\NormalTok{(}\KeywordTok{aes}\NormalTok{(x /}\StringTok{ }\NormalTok{n)) }
\NormalTok{\}}
\NormalTok{df <-}\StringTok{ }\KeywordTok{data.frame}\NormalTok{(}\DataTypeTok{x =} \DecValTok{1}\NormalTok{:}\DecValTok{3}\NormalTok{, }\DataTypeTok{y =} \DecValTok{1}\NormalTok{:}\DecValTok{3}\NormalTok{)}
\KeywordTok{ggplot}\NormalTok{(df, }\KeywordTok{aes}\NormalTok{(x, y)) +}\StringTok{ }\KeywordTok{f}\NormalTok{()}
\CommentTok{#> Error in x/n: non-numeric argument to binary operator}
\end{Highlighting}
\end{Shaded}

Note that this is only a problem with the \texttt{mapping} argument. All
other arguments are evaluated immediately so their values (not a
reference to a name) are stored in the plot object. This means the
following function will work:

\begin{Shaded}
\begin{Highlighting}[]
\NormalTok{f <-}\StringTok{ }\NormalTok{function() \{}
  \NormalTok{colour <-}\StringTok{ "blue"}
  \KeywordTok{geom_line}\NormalTok{(}\DataTypeTok{colour =} \NormalTok{colour) }
\NormalTok{\}}
\KeywordTok{ggplot}\NormalTok{(df, }\KeywordTok{aes}\NormalTok{(x, y)) +}\StringTok{ }\KeywordTok{f}\NormalTok{()}
\end{Highlighting}
\end{Shaded}

If you need to use a different environment for the plot, you can specify
it with the \texttt{environment} argument to \texttt{ggplot()}. You'll
need to do this if you're creating a plot function that takes user
provided data. See \texttt{qplot()} for an example.

\subsection{Exercises}

\begin{enumerate}
\def\labelenumi{\arabic{enumi}.}
\item
  Create a \texttt{distribution()} function specially designed for
  visualising continuous distributions. Allow the user to supply a
  dataset and the name of a variable to visualise. Let them choose
  between histograms, frequency polygons, and density plots. What other
  arguments might you want to include?
\item
  What additional arguments should \texttt{pcp()} take? What are the
  downsides of how \texttt{...} is used in the current code?
\item
  Advanced: why doesn't this code work? How can you fix it?

\begin{Shaded}
\begin{Highlighting}[]
\NormalTok{f <-}\StringTok{ }\NormalTok{function() \{}
  \NormalTok{levs <-}\StringTok{ }\KeywordTok{c}\NormalTok{(}\StringTok{"2seater"}\NormalTok{, }\StringTok{"compact"}\NormalTok{, }\StringTok{"midsize"}\NormalTok{, }\StringTok{"minivan"}\NormalTok{, }\StringTok{"pickup"}\NormalTok{, }
    \StringTok{"subcompact"}\NormalTok{, }\StringTok{"suv"}\NormalTok{)}
  \KeywordTok{piechart3}\NormalTok{(mpg, }\KeywordTok{factor}\NormalTok{(class, }\DataTypeTok{levels =} \NormalTok{levs))}
\NormalTok{\}}
\KeywordTok{f}\NormalTok{()}
\CommentTok{#> Error in factor(class, levels = levs): object 'levs' not found}
\end{Highlighting}
\end{Shaded}
\end{enumerate}

\section{Functional programming}

Since ggplot2 objects are just regular R objects, you can put them in a
list. This means you can apply all of R's great functional programming
tools. For example, if you wanted to add different geoms to the same
base plot, you could put them in a list and use \texttt{lapply()}.
\index{Functional programming} \indexf{lapply}

\begin{Shaded}
\begin{Highlighting}[]
\NormalTok{geoms <-}\StringTok{ }\KeywordTok{list}\NormalTok{(}
  \KeywordTok{geom_point}\NormalTok{(),}
  \KeywordTok{geom_boxplot}\NormalTok{(}\KeywordTok{aes}\NormalTok{(}\DataTypeTok{group =} \KeywordTok{cut_width}\NormalTok{(displ, }\DecValTok{1}\NormalTok{))),}
  \KeywordTok{list}\NormalTok{(}\KeywordTok{geom_point}\NormalTok{(), }\KeywordTok{geom_smooth}\NormalTok{())}
\NormalTok{)}

\NormalTok{p <-}\StringTok{ }\KeywordTok{ggplot}\NormalTok{(mpg, }\KeywordTok{aes}\NormalTok{(displ, hwy))}
\KeywordTok{lapply}\NormalTok{(geoms, function(g) p +}\StringTok{ }\NormalTok{g)}
\CommentTok{#> [[1]]}
\CommentTok{#> }
\CommentTok{#> [[2]]}
\CommentTok{#> }
\CommentTok{#> [[3]]}
\end{Highlighting}
\end{Shaded}

\begin{figure}[H]
  \includegraphics[width=0.333\linewidth]{_figures/programming/unnamed-chunk-12-1}%
  \includegraphics[width=0.333\linewidth]{_figures/programming/unnamed-chunk-12-2}%
  \includegraphics[width=0.333\linewidth]{_figures/programming/unnamed-chunk-12-3}
\end{figure}

If you're not familiar with functional programming, read through
\url{http://adv-r.had.co.nz/Functional-programming.html} and think about
how you might apply the techniques to your duplicated plotting code.

\subsection{Exercises}

\begin{enumerate}
\def\labelenumi{\arabic{enumi}.}
\item
  How could you add a \texttt{geom\_point()} layer to each element of
  the following list?

\begin{Shaded}
\begin{Highlighting}[]
\NormalTok{plots <-}\StringTok{ }\KeywordTok{list}\NormalTok{(}
  \KeywordTok{ggplot}\NormalTok{(mpg, }\KeywordTok{aes}\NormalTok{(displ, hwy)),}
  \KeywordTok{ggplot}\NormalTok{(diamonds, }\KeywordTok{aes}\NormalTok{(carat, price)),}
  \KeywordTok{ggplot}\NormalTok{(faithfuld, }\KeywordTok{aes}\NormalTok{(waiting, eruptions, }\DataTypeTok{size =} \NormalTok{density))}
\NormalTok{)}
\end{Highlighting}
\end{Shaded}
\item
  What does the following function do? What's a better name for it?

\begin{Shaded}
\begin{Highlighting}[]
\NormalTok{mystery <-}\StringTok{ }\NormalTok{function(...) \{}
  \KeywordTok{Reduce}\NormalTok{(}\StringTok{`}\DataTypeTok{+}\StringTok{`}\NormalTok{, }\KeywordTok{list}\NormalTok{(...), }\DataTypeTok{accumulate =} \OtherTok{TRUE}\NormalTok{)}
\NormalTok{\}}

\KeywordTok{mystery}\NormalTok{(}
  \KeywordTok{ggplot}\NormalTok{(mpg, }\KeywordTok{aes}\NormalTok{(displ, hwy)) +}\StringTok{ }\KeywordTok{geom_point}\NormalTok{(), }
  \KeywordTok{geom_smooth}\NormalTok{(), }
  \KeywordTok{xlab}\NormalTok{(}\OtherTok{NULL}\NormalTok{), }
  \KeywordTok{ylab}\NormalTok{(}\OtherTok{NULL}\NormalTok{)}
\NormalTok{)}
\end{Highlighting}
\end{Shaded}
\end{enumerate}


\backmatter

\let\hyperlink=\oldhyperlink % Restore old hyperlink behaviour
\cleardoublepage
\markboth{Index}{Index}
\addcontentsline{toc}{chapter}{Index}
\printindex

\addcontentsline{toc}{chapter}{Code index}
\printindex[code]

\end{document}
