\chapter{Introduction}\label{cha:introduction}

\section{Welcome to ggplot2}\label{welcome-to-ggplot2}

ggplot2 is an R package for producing statistical, or data, graphics,
but it is unlike most other graphics packages because it has a deep
underlying grammar. This grammar, based on the Grammar of Graphics
(Wilkinson 2005), is made up of a set of independent components that can
be composed in many different ways. This makes ggplot2 very powerful
because you are not limited to a set of pre-specified graphics, but you
can create new graphics that are precisely tailored for your problem.
This may sound overwhelming, but because there is a simple set of core
principles and very few special cases, ggplot2 is also easy to learn
(although it may take a little time to forget your preconceptions from
other graphics tools).

Practically, ggplot2 provides beautiful, hassle-free plots that take
care of fiddly details like drawing legends. The plots can be built up
iteratively and edited later. A carefully chosen set of defaults means
that most of the time you can produce a publication-quality graphic in
seconds, but if you do have special formatting requirements, a
comprehensive theming system makes it easy to do what you want. Instead
of spending time making your graph look pretty, you can focus on
creating a graph that best reveals the messages in your data.

ggplot2 is designed to work iteratively. You can start with a layer
showing the raw data then add layers of annotations and statistical
summaries. It allows you to produce graphics using the same structured
thinking that you use to design an analysis, reducing the distance
between a plot in your head and one on the page. It is especially
helpful for students who have not yet developed the structured approach
to analysis used by experts.

Learning the grammar not only will help you create graphics that you
know about now, but will also help you to think about new graphics that
would be even better. Without the grammar, there is no underlying
theory, so most graphics packages are just a big collection of special
cases. For example, in base R, if you design a new graphic, it's
composed of raw plot elements like points and lines, and it's hard to
design new components that combine with existing plots. In ggplot2, the
expressions used to create a new graphic are composed of higher-level
elements like representations of the raw data and statistical
transformations, and can easily be combined with new datasets and other
plots.

This book provides a hands-on introduction to ggplot2 with lots of
example code and graphics. It also explains the grammar on which ggplot2
is based. Like other formal systems, ggplot2 is useful even when you
don't understand the underlying model. However, the more you learn about
it, the more effectively you'll be able to use ggplot2. This book
assumes some basic familiarity with R, to the level described in the
first chapter of Dalgaard's \emph{Introductory Statistics with R}.

This book will introduce you to ggplot2 as a novice, unfamiliar with the
grammar; teach you the basics so that you can re-create plots you are
already familiar with; show you how to use the grammar to create new
types of graphics; and eventually turn you into an expert who can build
new components to extend the grammar.

\section{What is the grammar of
graphics?}\label{what-is-the-grammar-of-graphics}

Wilkinson (2005) created the grammar of graphics to describe the deep
features that underlie all statistical graphics. The grammar of graphics
is an answer to a question: what is a statistical graphic? The layered
grammar of graphics (Wickham 2009) builds on Wilkinson's grammar,
focussing on the primacy of layers and adapting it for embedding within
R. In brief, the grammar tells us that a statistical graphic is a
mapping from data to aesthetic attributes (colour, shape, size) of
geometric objects (points, lines, bars). The plot may also contain
statistical transformations of the data and is drawn on a specific
coordinate system. Facetting can be used to generate the same plot for
different subsets of the dataset. It is the combination of these
independent components that make up a graphic.

As the book progresses, the formal grammar will be explained in
increasing detail. The first description of the components follows
below. It introduces some of the terminology that will be used
throughout the book and outlines the basic responsibilities of each
component. Don't worry if it doesn't all make sense right away: you will
have many more opportunities to learn about the pieces and how they fit
together.

All plots are composed of:

\begin{itemize}
\item
  \textbf{Data} that you want to visualise and a set of aesthetic
  \textbf{mapping}s describing how variables in the data are mapped to
  aesthetic attributes that you can perceive.
\item
  \textbf{Layers} made up of geometric elements and statistical
  transformation. Geometric objects, \textbf{geom}s for short, represent
  what you actually see on the plot: points, lines, polygons, etc.
  Statistical transformations, \textbf{stat}s for short, summarise data
  in many useful ways. For example, binning and counting observations to
  create a histogram, or summarising a 2d relationship with a linear
  model.
\item
  The \textbf{scale}s map values in the data space to values in an
  aesthetic space, whether it be colour, or size, or shape. Scales draw
  a legend or axes, which provide an inverse mapping to make it possible
  to read the original data values from the plot.
\item
  A coordinate system, \textbf{coord} for short, describes how data
  coordinates are mapped to the plane of the graphic. It also provides
  axes and gridlines to make it possible to read the graph. We normally
  use a Cartesian coordinate system, but a number of others are
  available, including polar coordinates and map projections.
\item
  A \textbf{facet}ing specification describes how to break up the data
  into subsets and how to display those subsets as small multiples. This
  is also known as conditioning or latticing/trellising.
\item
  A \textbf{theme} which controls the finer points of display, like the
  font size and background colour. While the defaults in ggplot2 have
  been chosen with care, you may need to consult other references to
  create an attractive plot. A good starting place is Tufte's early
  works (Tufte 1990; Tufte 1997; Tufte 2001).
\end{itemize}

It is also important to talk about what the grammar doesn't do:

\begin{itemize}
\item
  It doesn't suggest what graphics you should use to answer the
  questions you are interested in. While this book endeavours to promote
  a sensible process for producing plots of data, the focus of the book
  is on how to produce the plots you want, not knowing what plots to
  produce. For more advice on this topic, you may want to consult
  Robbins (2013), Cleveland (1993), Chambers et al. (1983), and J. W.
  Tukey (1977).
\item
  It does not describe interactivity: the grammar of graphics describes
  only static graphics and there is essentially no benefit to displaying
  them on a computer screen as opposed to a piece of paper. ggplot2 can
  only create static graphics, so for dynamic and interactive graphics
  you will have to look elsewhere (perhaps at ggvis, described below).
  Cook and Swayne (2007) provides an excellent introduction to the
  interactive graphics package GGobi. GGobi can be connected to R with
  the rggobi package (Wickham et al. 2008).
\end{itemize}

\section{How does ggplot2 fit in with other R
graphics?}\label{how-does-ggplot2-fit-in-with-other-r-graphics}

There are a number of other graphics systems available in R: base
graphics, grid graphics and trellis/lattice graphics. How does ggplot2
differ from them?

\begin{itemize}
\item
  Base graphics were written by Ross Ihaka based on experience
  implementing the S graphics driver and partly looking at Chambers et
  al. (1983). Base graphics has a pen on paper model: you can only draw
  on top of the plot, you cannot modify or delete existing content.
  There is no (user accessible) representation of the graphics, apart
  from their appearance on the screen. Base graphics includes both tools
  for drawing primitives and entire plots. Base graphics functions are
  generally fast, but have limited scope. If you've created a single
  scatterplot, or histogram, or a set of boxplots in the past, you've
  probably used base graphics. \index{Base graphics}
\item
  The development of ``grid'' graphics, a much richer system of
  graphical primitives, started in 2000. Grid is developed by Paul
  Murrell, growing out of his PhD work (Murrell 1998). Grid grobs
  (graphical objects) can be represented independently of the plot and
  modified later. A system of viewports (each containing its own
  coordinate system) makes it easier to lay out complex graphics. Grid
  provides drawing primitives, but no tools for producing statistical
  graphics. \index{grid}
\item
  The lattice package, developed by Deepayan Sarkar, uses grid graphics
  to implement the trellis graphics system of Cleveland (1993) and is a
  considerable improvement over base graphics. You can easily produce
  conditioned plots and some plotting details (e.g., legends) are taken
  care of automatically. However, lattice graphics lacks a formal model,
  which can make it hard to extend. Lattice graphics are explained in
  depth in Sarkar (2008). \index{lattice}
\item
  ggplot2, started in 2005, is an attempt to take the good things about
  base and lattice graphics and improve on them with a strong underlying
  model which supports the production of any kind of statistical
  graphic, based on the principles outlined above. The solid underlying
  model of ggplot2 makes it easy to describe a wide range of graphics
  with a compact syntax, and independent components make extension easy.
  Like lattice, ggplot2 uses grid to draw the graphics, which means you
  can exercise much low-level control over the appearance of the plot.
\item
  Work on ggvis, the successor to ggplot2, started in 2014. It takes the
  foundational ideas of ggplot2 but extends them to the web and
  interactive graphics. The syntax is similar, but it's been re-designed
  from scratch to take advantage of what I've learned in the 10 years
  since creating ggplot2. The most exciting thing about ggvis is that
  it's interactive and dynamic, so plots automatically re-draw
  themselves when the underlying data or plot specification changes.
  However, ggvis is work in progress and currently can create only a
  fraction of the plots in ggplot2 can. Stay tuned for updates!
  \index{ggvis}
\item
  htmlwidgets, \url{http://www.htmlwidgets.org} provides a common
  framework for accessing web visualisation tools from R. Packages built
  on top of htmlwidgets include leaflet
  (\url{https://rstudio.github.io/leaflet/}, maps), dygraph
  (\url{http://rstudio.github.io/dygraphs/}, time series) and networkD3
  (http://christophergandrud.github.io/networkD3/, networks).
  htmlwidgets is to ggvis what the many specialised graphic packages are
  to ggplot2: it provides graphics honed for specific purposes.
  \index{htmlwidgets}
\end{itemize}

Many other R packages, such as vcd (Meyer, Zeileis, and Hornik 2006),
plotrix (Lemon et al. 2008) and gplots (Warnes 2007), implement
specialist graphics, but no others provide a framework for producing
statistical graphics. A comprehensive list of all graphical tools
available in other packages can be found in the graphics task view at
\url{http://cran.r-project.org/web/views/Graphics.html}.

\section{About this book}\label{about-this-book}

The first chapter, \protect\hyperlink{cha:getting-started}{getting
started with ggplot2}, describes how to quickly get started using
ggplot2 to make useful graphics. This chapter introduces several
important ggplot2 concepts: geoms, aesthetic mappings and facetting.
\protect\hyperlink{cha:toolbox}{Toolbox} dives into more details, giving
you a toolbox designed to solve a wide range of problems.

\protect\hyperlink{cha:mastery}{Mastery} describes the layered grammar
of graphics which underlies ggplot2. The theory is illustrated in
\protect\hyperlink{cha:layers}{layers} which demonstrates how to add
additional layers to your plot, exercising full control over the geoms
and stats used within them.

Understanding how scales work is crucial for fine-tuning the perceptual
properties of your plot. Customising scales gives fine control over the
exact appearance of the plot and helps to support the story that you are
telling. \protect\hyperlink{cha:scales}{Scales} will show you what
scales are available, how to adjust their parameters, and how to control
the appearance of axes and legends.

Coordinate systems and facetting control the position of elements of the
plot. These are described in \protect\hyperlink{cha:position}{position}.
Facetting is a very powerful graphical tool as it allows you to rapidly
compare different subsets of your data. Different coordinate systems are
less commonly needed, but are very important for certain types of data.

To polish your plots for publication, you will need to learn about the
tools described in \protect\hyperlink{cha:polishing}{polishing}. There
you will learn about how to control the theming system of ggplot2 and
how to save plots to disk.

The book concludes with four chapters that show how to use ggplot2 as
part of a data analysis pipeline. ggplot2 works best when your data is
tidy, so \protect\hyperlink{cha:data}{Tidying} discusses what that means
and how to make your messy data tidy. \protect\hyperlink{cha:dplyr}{Data
transformation} teaches you how to use the dplyr package to perform the
most common data manipulation operations.
\protect\hyperlink{cha:modelling}{Modelling} shows how to integrate
visualisation and modelling in two useful ways. Duplicated code is a big
inhibitor of flexibility and reduces your ability to respond to changes
in requirements. \protect\hyperlink{cha:programming}{Programming with
ggplot2} covers useful techniques for reducing duplication in your code.

\section{Installation}\label{sec:installation}

\index{Installation}

To use ggplot2, you must first install it. Make sure you have a recent
version of R (at least version 3.2.0) from \url{http://r-project.org}
and then run the following code to download and install ggplot2:

\begin{Shaded}
\begin{Highlighting}[]
\KeywordTok{install.packages}\NormalTok{(}\StringTok{"ggplot2"}\NormalTok{)}
\end{Highlighting}
\end{Shaded}

\section{Other resources}\label{sec:other-resources}

This book teaches you the elements of ggplot2's grammar and how they fit
together, but it does not document every function in complete detail.
You will need additional documentation as your use of ggplot2 becomes
more complex and varied.

The best resource for specific details of ggplot2 functions and their
arguments will always be the built-in documentation. This is accessible
online, \url{http://docs.ggplot2.org/}, and from within R using the
usual help syntax. The advantage of the online documentation is that you
can see all the example plots and navigate between topics more easily.

If you use ggplot2 regularly, it's a good idea to sign up for the
ggplot2 mailing list, \url{http://groups.google.com/group/ggplot2}. The
list has relatively low traffic and is very friendly to new users.
Another useful resource is stackoverflow,
\url{http://stackoverflow.com}. There is an active ggplot2 community on
stackoverflow, and many common questions have already been asked and
answered. In either place, you're much more likely to get help if you
create a minimal reproducible example. The
\href{https://github.com/jennybc/reprex}{reprex} package by Jenny Bryan
provides a convenient way to do this, and also include advice on
creating a good example. The more information you provide, the easier it
is for the community to help you.

The number of functions in ggplot2 can be overwhelming, but RStudio
provides some great cheatsheets to jog your memory at
\url{http://www.rstudio.com/resources/cheatsheets/}.

Finally, the complete source code for the book is available online at
\url{https://github.com/hadley/ggplot2-book}. This contains the complete
text for the book, as well as all the code and data needed to recreate
all the plots.

\section{Colophon}\label{colophon}

This book was written in \href{http://rmarkdown.rstudio.com/}{R
Markdown} inside \href{http://www.rstudio.com/ide/}{RStudio}.
\href{http://yihui.name/knitr/}{knitr} and
\href{http://johnmacfarlane.net/pandoc/}{pandoc} converted the raw
Rmarkdown to html and pdf. The complete source is available from
\href{https://github.com/hadley/ggplot2-book}{github}. This version of
the book was built with:

\begin{Shaded}
\begin{Highlighting}[]
\NormalTok{devtools::}\KeywordTok{session_info}\NormalTok{(}\KeywordTok{c}\NormalTok{(}\StringTok{"ggplot2"}\NormalTok{, }\StringTok{"dplyr"}\NormalTok{, }\StringTok{"broom"}\NormalTok{))}
\CommentTok{#> Session info --------------------------------------------------------}
\CommentTok{#>  setting  value                       }
\CommentTok{#>  version  R version 3.2.3 (2015-12-10)}
\CommentTok{#>  system   x86_64, darwin13.4.0        }
\CommentTok{#>  ui       X11                         }
\CommentTok{#>  language (EN)                        }
\CommentTok{#>  collate  en_US.UTF-8                 }
\CommentTok{#>  tz       America/Chicago             }
\CommentTok{#>  date     2016-02-27}
\CommentTok{#> Packages ------------------------------------------------------------}
\CommentTok{#>  package      * version  date       source        }
\CommentTok{#>  assertthat     0.1      2013-12-06 CRAN (R 3.2.0)}
\CommentTok{#>  BH             1.58.0-1 2015-05-21 CRAN (R 3.2.0)}
\CommentTok{#>  broom          0.4.0    2015-11-30 CRAN (R 3.2.2)}
\CommentTok{#>  colorspace     1.2-6    2015-03-11 CRAN (R 3.2.0)}
\CommentTok{#>  DBI            0.3.1    2014-09-24 CRAN (R 3.2.0)}
\CommentTok{#>  dichromat      2.0-0    2013-01-24 CRAN (R 3.2.0)}
\CommentTok{#>  digest         0.6.9    2016-01-08 CRAN (R 3.2.3)}
\CommentTok{#>  dplyr        * 0.4.3    2015-09-01 CRAN (R 3.2.0)}
\CommentTok{#>  ggplot2      * 2.1.0    2016-02-26 local         }
\CommentTok{#>  gtable         0.1.2    2012-12-05 CRAN (R 3.2.0)}
\CommentTok{#>  labeling       0.3      2014-08-23 CRAN (R 3.2.0)}
\CommentTok{#>  lattice        0.20-33  2015-07-14 CRAN (R 3.2.3)}
\CommentTok{#>  lazyeval       0.1.10   2015-01-02 CRAN (R 3.2.0)}
\CommentTok{#>  magrittr       1.5      2014-11-22 CRAN (R 3.2.0)}
\CommentTok{#>  MASS           7.3-45   2015-11-10 CRAN (R 3.2.3)}
\CommentTok{#>  mnormt         1.5-3    2015-05-25 CRAN (R 3.2.0)}
\CommentTok{#>  munsell        0.4.2    2013-07-11 CRAN (R 3.2.0)}
\CommentTok{#>  nlme           3.1-122  2015-08-19 CRAN (R 3.2.3)}
\CommentTok{#>  plyr           1.8.3    2015-06-12 CRAN (R 3.2.0)}
\CommentTok{#>  psych          1.5.8    2015-08-30 CRAN (R 3.2.0)}
\CommentTok{#>  R6             2.1.2    2016-01-26 CRAN (R 3.2.3)}
\CommentTok{#>  RColorBrewer   1.1-2    2014-12-07 CRAN (R 3.2.0)}
\CommentTok{#>  Rcpp           0.12.3   2016-01-10 CRAN (R 3.2.3)}
\CommentTok{#>  reshape2       1.4.1    2014-12-06 CRAN (R 3.2.0)}
\CommentTok{#>  scales         0.3.0    2015-08-25 CRAN (R 3.2.0)}
\CommentTok{#>  stringi        1.0-1    2015-10-22 CRAN (R 3.2.0)}
\CommentTok{#>  stringr        1.0.0    2015-04-30 CRAN (R 3.2.0)}
\CommentTok{#>  tidyr        * 0.4.1    2016-02-05 CRAN (R 3.2.3)}
\end{Highlighting}
\end{Shaded}

\section*{References}\label{references}
\addcontentsline{toc}{section}{References}

\hypertarget{refs}{}
\hypertarget{ref-chambers:1983}{}
Chambers, John, William Cleveland, Beat Kleiner, and Paul Tukey. 1983.
\emph{Graphical Methods for Data Analysis}. Wadsworth.

\hypertarget{ref-cleveland:1993}{}
Cleveland, William. 1993. \emph{Visualizing Data}. Hobart Press.

\hypertarget{ref-cook:2007}{}
Cook, Dianne, and Deborah F. Swayne. 2007. \emph{Interactive and Dynamic
Graphics for Data Analysis: With Examples Using R and GGobi}. Springer.

\hypertarget{ref-plotrix}{}
Lemon, Jim, Ben Bolker, Sander Oom, Eduardo Klein, Barry Rowlingson,
Hadley Wickham, Anupam Tyagi, et al. 2008. \emph{Plotrix: Various
Plotting Functions}.

\hypertarget{ref-meyer:2006}{}
Meyer, David, Achim Zeileis, and Kurt Hornik. 2006. ``The Strucplot
Framework: Visualizing Multi-Way Contingency Tables with Vcd.''
\emph{Journal of Statistical Software} 17 (3): 1--48.
\url{http://www.jstatsoft.org/v17/i03/}.

\hypertarget{ref-murrell:1998}{}
Murrell, Paul. 1998. ``Investigations in Graphical Statistics.''
PhD thesis, The University of Auckland.

\hypertarget{ref-robbins:2004}{}
Robbins, Naomi. 2013. \emph{Creating More Effective Graphs}. Chart
House.

\hypertarget{ref-sarkar:2008}{}
Sarkar, Deepayan. 2008. \emph{Lattice: Multivariate Data Visualization
with R}. Springer.

\hypertarget{ref-tufte:1990}{}
Tufte, Edward R. 1990. \emph{Envisioning Information}. Graphics Press.

\hypertarget{ref-tufte:1997}{}
---------. 1997. \emph{Visual Explanations}. Graphics Press.

\hypertarget{ref-tufte:2001}{}
---------. 2001. \emph{The Visual Display of Quantitative Information}.
Second. Graphics Press.

\hypertarget{ref-tukey:1977}{}
Tukey, John W. 1977. \emph{Exploratory Data Analysis}. Addison--Wesley.

\hypertarget{ref-gplots}{}
Warnes, Gregory. 2007. \emph{Gplots: Various R Programming Tools for
Plotting Data}.

\hypertarget{ref-wickham:2007d}{}
Wickham, Hadley. 2009. ``A Layered Grammar of Graphics.'' \emph{Journal
of Computational and Graphical Statistics}.

\hypertarget{ref-wickham:2008b}{}
Wickham, Hadley, Michael Lawrence, Duncan Temple Lang, and Deborah F
Swayne. 2008. ``An Introduction to Rggobi.'' \emph{R-News} 8 (2): 3--7.
\url{http://CRAN.R-project.org/doc/Rnews/Rnews_2008-2.pdf}.

\hypertarget{ref-wilkinson:2006}{}
Wilkinson, Leland. 2005. \emph{The Grammar of Graphics}. 2nd ed.
Statistics and Computing. Springer.
