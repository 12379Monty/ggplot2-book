\chapter{Toolbox}\label{cha:toolbox}

\section{Introduction}

The layered structure of ggplot encourages you to design and construct
graphics in a structured manner. You have learned what a layer is and
how to add one to your graphic, but not what geoms and statistics are
available to help you build revealing plots. This chapter lists some of
the many geoms and stats included in ggplot, broken down by their
purpose. This chapter will provide a good overview of the available
options, but it does not describe each geom and stat in detail. For more
information about individual geoms, along with many more examples
illustrating their use, see the online and electronic documentation. You
may also want to consult the documentation to learn more about the
datasets used in this chapter.

This chapter is broken up into the following sections, each of which
deals with a particular graphical challenge. This is not an exhaustive
or exclusive categorisation, and there are many other possible ways to
break up graphics into different categories. Each geom can be used for
many different purposes, especially if you are creative. However, this
breakdown should cover many common tasks and help you learn about some
of the possibilities.

\begin{itemize}
\itemsep1pt\parskip0pt\parsep0pt
\item
  Basic plot types to produce common, `named' graphics like scatterplots
  and line charts (\hyperref[sec:basics]{link to section}).
\item
  Displaying distributions, continuous and discrete, 1d and 2d, joint
  and conditional (\hyperref[sec:distributions]{link to section}).
\item
  Dealing with overplotting in scatterplots a challenge with large
  datasets (\hyperref[sec:overplotting]{link to section}).
\item
  Surface plots display 3d surfaces in 2d (\hyperref[sec:surface]{link
  to section}).
\item
  Statistical summaries display informative data summaries
  (\hyperref[sec:summary]{link to section})
\item
  Drawing maps (\hyperref[sec:maps]{link to section}).
\item
  Revealing uncertainty and error, with various 1d and 2d intervals
  (\hyperref[sec:uncertainty]{link to section}).
\item
  Annotating a plot to label, describe and explain with supplemental
  information \hyperref[sec:annotating]{link to section}.
\item
  Weighted data \hyperref[sec:weighting]{link to section}.
\end{itemize}

The examples in this section use a mixture of \texttt{ggplot()} and
\texttt{qplot()} calls, reflecting real-life use. If you need a reminder
on how to translate between the two, see
\hyperref[sec:qplot-ggplot]{translating between qplot and ggplot}. The
examples do not go into much depth, but hopefully if you flick through
this chapter, you'll be able to see a plot that looks like the one
you're trying to create.

\section{Overall layering strategy}\label{sec:strategy}

It is useful to think about the purpose of each layer before it is
added. In general, there are three purposes for a layer:
\index{Layers!strategy}

\begin{itemize}
\itemsep1pt\parskip0pt\parsep0pt
\item
  To display the \textbf{data}. We plot the raw data for many reasons,
  relying on our skills at pattern detection to spot gross structure,
  local structure, and outliers. This layer appears on virtually every
  graphic. In the earliest stages of data exploration, it is often the
  only layer.
\item
  To display a statistical \textbf{summary} of the data. As we develop
  and explore models of the data, it is useful to display model
  predictions in the context of the data. We learn from the data
  summaries and we evaluate the model. Showing the data helps us improve
  the model, and showing the model helps reveal subtleties of the data
  that we might otherwise miss. Summaries are usually drawn on top of
  the data.
\end{itemize}

If you review the examples in the preceding chapter, you'll see many
examples of plots of data with an added layer displaying a statistical
summary.

\begin{itemize}
\itemsep1pt\parskip0pt\parsep0pt
\item
  To add additional \textbf{metadata}, context and annotations. A
  metadata layer displays background context or annotations that help to
  give meaning to the raw data. Metadata can be useful in the background
  and foreground.
\end{itemize}

A map is often used as a background layer with spatial data. Background
metadata should be rendered so that it doesn't interfere with your
perception of the data, so is usually displayed underneath the data and
formatted so that it is minimally perceptible. That is, if you
concentrate on it, you can see it with ease, but it doesn't jump out at
you when you are casually browsing the plot.

Other metadata is used to highlight important features of the data. If
you have added explanatory labels to a couple of inflection points or
outliers, then you want to render them so that they pop out at the
viewer. In that case, you want this to be the very last layer drawn.

\hyperdef{}{sec:basics}{\section{Basic plot types}\label{sec:basics}}

These geoms are the fundamental building blocks of \texttt{ggplot}. They
are useful in their own right, but also to construct more complex geoms.
Most of these geoms are associated with a named plot: when that geom is
used by itself in a plot, that plot has a special name.

Each of these geoms is two dimensional and requires both \texttt{x} and
\texttt{y} aesthetics. All understand \texttt{colour} and \texttt{size}
aesthetics, and the filled geoms (bar, tile and polygon) also understand
\texttt{fill}. The point geom uses \texttt{shape} and line and path
geoms understand \texttt{linetype}. The geoms are used for displaying
data, summaries computed elsewhere, and metadata.

\begin{itemize}
\itemsep1pt\parskip0pt\parsep0pt
\item
  \texttt{geom\_area()} draws an \textbf{area plot}, which is a line
  plot filled to the y-axis (filled lines). Multiple groups will be
  stacked on top of each other. \index{Area plot} \indexf{geom_area}
\item
  \texttt{geom\_bar(stat = "identity")} makes a \textbf{barchart}. We
  need \texttt{stat = "identity"} because the default stat automatically
  counts values (so is essentially a 1d geom, see
  \hyperref[sec:distributions]{displaying distributions}. The identity
  stat leaves the data unchanged. \index{Barchart} \indexf{geom_bar}
\end{itemize}

By default, multiple bars in the same location will be stacked on top of
one another.

\begin{itemize}
\itemsep1pt\parskip0pt\parsep0pt
\item
  \texttt{geom\_line()} makes a \textbf{line plot}. The \texttt{group}
  aesthetic determines which observations are connected; see
  \hyperref[sub:grouping]{grouping} for more details.
  \texttt{geom\_path()} is similar to a \texttt{geom\_line()}, but lines
  are connected in the order they appear in the data, not from left to
  right. \index{Line plot} \indexf{geom_line} \indexf{geom_path}
\item
  \texttt{geom\_point()} produces a \textbf{scatterplot}.
  \indexf{geom_point}
\item
  \texttt{geom\_polygon()} draws polygons, which are filled paths. Each
  vertex of the polygon requires a separate row in the data. It is often
  useful to merge a data frame of polygon coordinates with the data just
  prior to plotting. \hyperref[sec:maps]{Drawing maps} illustrates this
  concept in more detail for map data. \indexf{geom_polygon}
\item
  \texttt{geom\_text()} adds labels at the specified points. This is the
  only geom in this group that requires another aesthetic:
  \texttt{label}. It also has optional aesthetics \texttt{hjust} and
  \texttt{vjust} that control the horizontal and vertical position of
  the text; and \texttt{angle} which controls the rotation of the text.
  See \hyperref[cha:specifications]{specifications} for more details.
  \index{Labels} \indexf{geom_text}
\item
  \texttt{geom\_tile()} makes an image plot or level plot. The tiles
  form a regular tessellation of the plane and typically have the
  \texttt{fill} aesthetic mapped to another variable. \index{Image plot}
  \index{Level plot} \indexf{geom_tile}
\end{itemize}

Each of these geoms is illustrated in Figure \ref{fig:geom-basic},
created with the code below.

\begin{Shaded}
\begin{Highlighting}[]
\NormalTok{df <-}\StringTok{ }\KeywordTok{data.frame}\NormalTok{(}
  \DataTypeTok{x =} \KeywordTok{c}\NormalTok{(}\DecValTok{3}\NormalTok{, }\DecValTok{1}\NormalTok{, }\DecValTok{5}\NormalTok{), }
  \DataTypeTok{y =} \KeywordTok{c}\NormalTok{(}\DecValTok{2}\NormalTok{, }\DecValTok{4}\NormalTok{, }\DecValTok{6}\NormalTok{), }
  \DataTypeTok{label =} \KeywordTok{c}\NormalTok{(}\StringTok{"a"}\NormalTok{,}\StringTok{"b"}\NormalTok{,}\StringTok{"c"}\NormalTok{)}
\NormalTok{)}
\NormalTok{p <-}\StringTok{ }\KeywordTok{ggplot}\NormalTok{(df, }\KeywordTok{aes}\NormalTok{(x, y, }\DataTypeTok{label =} \NormalTok{label)) +}\StringTok{ }
\StringTok{  }\KeywordTok{labs}\NormalTok{(}\DataTypeTok{x =} \StringTok{""}\NormalTok{, }\DataTypeTok{y =} \StringTok{""}\NormalTok{)}
\NormalTok{p +}\StringTok{ }\KeywordTok{geom_point}\NormalTok{() +}\StringTok{ }\KeywordTok{labs}\NormalTok{(}\DataTypeTok{title =} \StringTok{"geom_point"}\NormalTok{)}
\NormalTok{p +}\StringTok{ }\KeywordTok{geom_bar}\NormalTok{(}\DataTypeTok{stat=}\StringTok{"identity"}\NormalTok{) +}\StringTok{ }
\StringTok{  }\KeywordTok{labs}\NormalTok{(}\DataTypeTok{title =} \StringTok{"geom_bar(stat=}\CharTok{\textbackslash{}"}\StringTok{identity}\CharTok{\textbackslash{}"}\StringTok{)"}\NormalTok{)}
\NormalTok{p +}\StringTok{ }\KeywordTok{geom_line}\NormalTok{() +}\StringTok{ }\KeywordTok{labs}\NormalTok{(}\DataTypeTok{title =} \StringTok{"geom_line"}\NormalTok{)}
\NormalTok{p +}\StringTok{ }\KeywordTok{geom_area}\NormalTok{() +}\StringTok{ }\KeywordTok{labs}\NormalTok{(}\DataTypeTok{title =} \StringTok{"geom_area"}\NormalTok{)}
\NormalTok{p +}\StringTok{ }\KeywordTok{geom_path}\NormalTok{() +}\StringTok{ }\KeywordTok{labs}\NormalTok{(}\DataTypeTok{title =} \StringTok{"geom_path"}\NormalTok{)}
\NormalTok{p +}\StringTok{ }\KeywordTok{geom_text}\NormalTok{() +}\StringTok{ }\KeywordTok{labs}\NormalTok{(}\DataTypeTok{title =} \StringTok{"geom_text"}\NormalTok{)}
\NormalTok{p +}\StringTok{ }\KeywordTok{geom_tile}\NormalTok{() +}\StringTok{ }\KeywordTok{labs}\NormalTok{(}\DataTypeTok{title =} \StringTok{"geom_tile"}\NormalTok{)}
\NormalTok{p +}\StringTok{ }\KeywordTok{geom_polygon}\NormalTok{() +}\StringTok{ }\KeywordTok{labs}\NormalTok{(}\DataTypeTok{title =} \StringTok{"geom_polygon"}\NormalTok{)}
\end{Highlighting}
\end{Shaded}

\begin{figure}

{\centering \includegraphics[width=0.24\linewidth]{figures/toolboxgeom-basic-1} \includegraphics[width=0.24\linewidth]{figures/toolboxgeom-basic-2} \includegraphics[width=0.24\linewidth]{figures/toolboxgeom-basic-3} \includegraphics[width=0.24\linewidth]{figures/toolboxgeom-basic-4} \includegraphics[width=0.24\linewidth]{figures/toolboxgeom-basic-5} \includegraphics[width=0.24\linewidth]{figures/toolboxgeom-basic-6} \includegraphics[width=0.24\linewidth]{figures/toolboxgeom-basic-7} \includegraphics[width=0.24\linewidth]{figures/toolboxgeom-basic-8} 

}

\caption{The basic geoms applied to the same data. Many give rise to named plots (from top left to bottom right): scatterplot, bar chart, line chart, area chart, path plot, labelled scatterplot, image/level plot and polygon plot. Observe the different axis ranges for the bar, area and tile plots: these geoms take up space outside the range of the data, and so push the axes out.\label{fig:geom-basic}}
\end{figure}

\hyperdef{}{sec:distributions}{\section{Displaying
distributions}\label{sec:distributions}}

There are a number of geoms that can be used to display distributions,
depending on the dimensionality of the distribution, whether it is
continuous or discrete, and whether you are interested in conditional or
joint distribution. \index{Distributions}

For 1d continuous distributions the most important geom is the
histogram. Figure \ref{fig:geom-1d-con} uses the histogram to display
the distribution of diamond \texttt{depth}. It is important to
experiment with bin placement to find a revealing view. You can change
the \texttt{binwidth}, or specify the exact location of the
\texttt{breaks}. \index{Histogram!choosing bins} \indexf{geom_histogram}

\begin{Shaded}
\begin{Highlighting}[]
\KeywordTok{qplot}\NormalTok{(depth, }\DataTypeTok{data=}\NormalTok{diamonds, }\DataTypeTok{geom=}\StringTok{"histogram"}\NormalTok{)}
\KeywordTok{qplot}\NormalTok{(depth, }\DataTypeTok{data=}\NormalTok{diamonds, }\DataTypeTok{geom=}\StringTok{"histogram"}\NormalTok{, }
      \DataTypeTok{xlim=}\KeywordTok{c}\NormalTok{(}\DecValTok{55}\NormalTok{, }\DecValTok{70}\NormalTok{), }\DataTypeTok{binwidth=}\FloatTok{0.1}\NormalTok{)}
\end{Highlighting}
\end{Shaded}

\begin{figure}

{\centering \includegraphics[width=0.49\linewidth]{figures/toolboxgeom-1d-con-1} \includegraphics[width=0.49\linewidth]{figures/toolboxgeom-1d-con-2} 

}

\caption{(Left) Never rely on the default parameters to get a revealing view of the distribution.  (Right) Zooming in on the x axis, \texttt{xlim = c(55, 70)}, and selecting a smaller bin width, \texttt{binwidth = 0.1}, reveals far more detail. We can see that the distribution is slightly skew-right. Don't forget to include information about important parameters (like bin width) in the caption.\label{fig:geom-1d-con}}
\end{figure}

If you want to compare the distribution between groups, you have a few
options: create small multiples of the histogram,
\texttt{facets = . \textasciitilde{} var}; use a frequency polygon,
\texttt{geom = "freqpoly"}; or create a conditional density plot,
\texttt{position = "fill"}. These options are illustrated in Figure
\ref{fig:compare-dist}, created with the code below.
\index{Frequency polygon} \index{Conditional density plot}
\indexf{geom_freqpoly} \indexf{geom_density}

\begin{Shaded}
\begin{Highlighting}[]
\NormalTok{depth_dist <-}\StringTok{ }\KeywordTok{ggplot}\NormalTok{(diamonds, }\KeywordTok{aes}\NormalTok{(depth)) +}\StringTok{ }\KeywordTok{xlim}\NormalTok{(}\DecValTok{58}\NormalTok{, }\DecValTok{68}\NormalTok{)}
\NormalTok{depth_dist +}\StringTok{ }
\StringTok{  }\KeywordTok{geom_histogram}\NormalTok{(}\KeywordTok{aes}\NormalTok{(}\DataTypeTok{y =} \NormalTok{..density..), }\DataTypeTok{binwidth =} \FloatTok{0.1}\NormalTok{) +}
\StringTok{  }\KeywordTok{facet_grid}\NormalTok{(cut ~}\StringTok{ }\NormalTok{.)}
\NormalTok{depth_dist +}\StringTok{ }\KeywordTok{geom_histogram}\NormalTok{(}\KeywordTok{aes}\NormalTok{(}\DataTypeTok{fill =} \NormalTok{cut), }\DataTypeTok{binwidth =} \FloatTok{0.1}\NormalTok{, }
  \DataTypeTok{position =} \StringTok{"fill"}\NormalTok{)}
\NormalTok{depth_dist +}\StringTok{ }\KeywordTok{geom_freqpoly}\NormalTok{(}\KeywordTok{aes}\NormalTok{(}\DataTypeTok{y =} \NormalTok{..density.., }\DataTypeTok{colour =} \NormalTok{cut), }
  \DataTypeTok{binwidth =} \FloatTok{0.1}\NormalTok{) }
\end{Highlighting}
\end{Shaded}

\begin{figure}

{\centering \includegraphics[width=\linewidth]{figures/toolboxcompare-dist-1} \includegraphics[width=\linewidth]{figures/toolboxcompare-dist-2} \includegraphics[width=\linewidth]{figures/toolboxcompare-dist-3} 

}

\caption{Three views of the distribution of depth and cut. From top to bottom: faceted histogram, a conditional density plot, and frequency polygons. All show an interesting pattern: as quality increases, the distribution shifts to the left and becomes more symmetric.\label{fig:compare-dist}}
\end{figure}

Both the histogram and frequency polygon geom use \texttt{stat\_bin()}.
This statistic produces two output variables \texttt{count} and
\texttt{density}. The count is the default as it is most interpretable.
The density is basically the count divided by the total count, and is
useful when you want to compare the shape of the distributions, not the
overall size. You will often prefer this when comparing the distribution
of subsets that have different sizes. \indexf{stat_bin}

Many of the distribution-related geoms come in geom/stat pairs. Most of
these geoms are aliases: a basic geom is combined with a stat to produce
the desired plot. The boxplot may appear to be an exception to this
rule, but behind the scenes \texttt{geom\_boxplot()} uses a combination
of the basic bars, lines and points.

\begin{itemize}
\itemsep1pt\parskip0pt\parsep0pt
\item
  \texttt{geom\_boxplot()} = \texttt{stat\_boxplot()} +
  \texttt{geom\_boxplot()}: box-and-whisker plot, for a continuous
  variable conditioned by a categorical variable. This is a useful
  display when the categorical variable has many distinct values. When
  there are few values, the techniques described above give a better
  view of the shape of the distribution. This technique can also be used
  for continuous variables, if they are first finely binned. Figure
  \ref{fig:geom-boxplot} shows boxplots conditioned on both categorical
  and continuous variables. \index{Boxplot} \indexf{geom_boxplot}
  \indexf{stat_boxplot}
\end{itemize}

\begin{Shaded}
\begin{Highlighting}[]
\KeywordTok{qplot}\NormalTok{(cut, depth, }\DataTypeTok{data=}\NormalTok{diamonds, }\DataTypeTok{geom=}\StringTok{"boxplot"}\NormalTok{)}
\KeywordTok{qplot}\NormalTok{(carat, depth, }\DataTypeTok{data=}\NormalTok{diamonds, }\DataTypeTok{geom=}\StringTok{"boxplot"}\NormalTok{, }
      \DataTypeTok{group =} \NormalTok{plyr::}\KeywordTok{round_any}\NormalTok{(carat, }\FloatTok{0.1}\NormalTok{, floor), }\DataTypeTok{xlim =} \KeywordTok{c}\NormalTok{(}\DecValTok{0}\NormalTok{, }\DecValTok{3}\NormalTok{))}
\end{Highlighting}
\end{Shaded}

\begin{figure}

{\centering \includegraphics[width=0.49\linewidth]{figures/toolboxgeom-boxplot-1} \includegraphics[width=0.49\linewidth]{figures/toolboxgeom-boxplot-2} 

}

\caption{The boxplot geom can be of use to see the distribution of a continuous variable conditional on a discrete varable like cut (left), or continuous variable like carat (right). For continuous variables, the group aesthetic must be set to get multiple boxplots. Here \texttt{group = plyr::round\_any(carat, 0.1, floor)} is used to get a boxplot for each 0.1 carat bin.\label{fig:geom-boxplot}}
\end{figure}

\begin{itemize}
\itemsep1pt\parskip0pt\parsep0pt
\item
  \texttt{geom\_jitter()} = \texttt{position\_jitter()} +
  \texttt{geom\_point()}: a crude way of looking at discrete
  distributions by adding random noise to the discrete values so that
  they don't overplot. An example is shown in Figure
  \ref{fig:geom-jitter} created with the code below. \index{Jittering}
  \indexf{geom_jitter} \indexf{position_jitter}
\end{itemize}

\begin{Shaded}
\begin{Highlighting}[]
\KeywordTok{qplot}\NormalTok{(class, cty, }\DataTypeTok{data=}\NormalTok{mpg, }\DataTypeTok{geom=}\StringTok{"jitter"}\NormalTok{)}
\KeywordTok{qplot}\NormalTok{(class, drv, }\DataTypeTok{data=}\NormalTok{mpg, }\DataTypeTok{geom=}\StringTok{"jitter"}\NormalTok{)}
\end{Highlighting}
\end{Shaded}

\begin{figure}

{\centering \includegraphics[width=0.8\linewidth]{figures/toolboxgeom-jitter-1} \includegraphics[width=0.8\linewidth]{figures/toolboxgeom-jitter-2} 

}

\caption{The jitter geom can be used to give a crude visualisation of 2d distributions with a discrete component. Generally this works better for smaller datasets. Car class vs. continuous variable city mpg (top) and discrete variable drive train (bottom).\label{fig:geom-jitter}}
\end{figure}

\begin{itemize}
\itemsep1pt\parskip0pt\parsep0pt
\item
  \texttt{geom\_density} = \texttt{stat\_density} + \texttt{geom\_area}:
  a smoothed version of the frequency polygon based on kernel smoothers.
  Also described in \hyperref[sub:distribution]{histogram and density
  plots}. Use a density plot when you know that the underlying density
  is smooth, continuous and unbounded. You can use the \texttt{adjust}
  parameter to make the density more or less smooth. An example is shown
  in Figure \ref{fig:geom-density} created with the code below.
  \index{Density!plot} \indexf{geom_density} \indexf{stat_density}
  \indexf{geom_area}
\end{itemize}

\begin{Shaded}
\begin{Highlighting}[]
\KeywordTok{qplot}\NormalTok{(depth, }\DataTypeTok{data=}\NormalTok{diamonds, }\DataTypeTok{geom=}\StringTok{"density"}\NormalTok{, }\DataTypeTok{xlim =} \KeywordTok{c}\NormalTok{(}\DecValTok{54}\NormalTok{, }\DecValTok{70}\NormalTok{))}
\KeywordTok{qplot}\NormalTok{(depth, }\DataTypeTok{data=}\NormalTok{diamonds, }\DataTypeTok{geom=}\StringTok{"density"}\NormalTok{, }\DataTypeTok{xlim =} \KeywordTok{c}\NormalTok{(}\DecValTok{54}\NormalTok{, }\DecValTok{70}\NormalTok{), }
      \DataTypeTok{fill =} \NormalTok{cut, }\DataTypeTok{alpha =} \KeywordTok{I}\NormalTok{(}\FloatTok{0.2}\NormalTok{))}
\end{Highlighting}
\end{Shaded}

\begin{figure}

{\centering \includegraphics[width=0.49\linewidth]{figures/toolboxgeom-density-1} \includegraphics[width=0.49\linewidth]{figures/toolboxgeom-density-2} 

}

\caption{The density plot is a smoothed version of the histogram. It has desirable theoretical properties, but is more difficult to relate back to the data. A density plot of depth (left), coloured by cut (right).\label{fig:geom-density}}
\end{figure}

Visualising a joint 2d continuous distribution is described in the next
section.

\hyperdef{}{sec:overplotting}{\section{Dealing with
overplotting}\label{sec:overplotting}}

The scatterplot is a very important tool for assessing the relationship
between two continuous variables. However, when the data is large, often
points will be plotted on top of each other, obscuring the true
relationship. In extreme cases, you will only be able to see the extent
of the data, and any conclusions drawn from the graphic will be suspect.
This problem is called overplotting and there are a number of ways to
deal with it: \index{Overplotting}

\begin{itemize}
\itemsep1pt\parskip0pt\parsep0pt
\item
  Small amounts of overplotting can sometimes be alleviated by making
  the points smaller, or using hollow glyphs, as shown in Figure
  \ref{fig:overp-glyph}. The data is 2000 points sampled from two
  independent normal distributions, and the code to produce the graphic
  is shown below. \indexf{geom_point}
\end{itemize}

\begin{Shaded}
\begin{Highlighting}[]
\NormalTok{df <-}\StringTok{ }\KeywordTok{data.frame}\NormalTok{(}\DataTypeTok{x =} \KeywordTok{rnorm}\NormalTok{(}\DecValTok{2000}\NormalTok{), }\DataTypeTok{y =} \KeywordTok{rnorm}\NormalTok{(}\DecValTok{2000}\NormalTok{))}
\NormalTok{norm <-}\StringTok{ }\KeywordTok{ggplot}\NormalTok{(df, }\KeywordTok{aes}\NormalTok{(x, y))}
\NormalTok{norm +}\StringTok{ }\KeywordTok{geom_point}\NormalTok{()}
\NormalTok{norm +}\StringTok{ }\KeywordTok{geom_point}\NormalTok{(}\DataTypeTok{shape =} \DecValTok{1}\NormalTok{)}
\NormalTok{norm +}\StringTok{ }\KeywordTok{geom_point}\NormalTok{(}\DataTypeTok{shape =} \StringTok{"."}\NormalTok{) }\CommentTok{# Pixel sized}
\end{Highlighting}
\end{Shaded}

\begin{figure}

{\centering \includegraphics[width=0.32\linewidth]{figures/toolboxoverp-glyph-1} \includegraphics[width=0.32\linewidth]{figures/toolboxoverp-glyph-2} \includegraphics[width=0.32\linewidth]{figures/toolboxoverp-glyph-3} 

}

\caption{Modifying the glyph used can help with mild to moderate overplotting.  From left to right: the default shape, \texttt{shape = 1} (hollow points), and \texttt{shape= "."} (pixel points).\label{fig:overp-glyph}}
\end{figure}

\begin{itemize}
\itemsep1pt\parskip0pt\parsep0pt
\item
  For larger datasets with more overplotting, you can use alpha blending
  (transparency) to make the points transparent. If you specify alpha as
  a ratio, the denominator gives the number of points that must be
  overplotted to give a solid colour. In R, the lowest amount of
  transparency you can use is 1/256, so it will not be effective for
  heavy overplotting. Figure \ref{fig:overp-alpha} demonstrates some of
  these options with the following code. \indexc{alpha}
  \index{Transparency} \index{Colour!transparency}
\end{itemize}

\begin{Shaded}
\begin{Highlighting}[]
\NormalTok{norm +}\StringTok{ }\KeywordTok{geom_point}\NormalTok{(}\DataTypeTok{alpha =} \DecValTok{1}\NormalTok{/}\DecValTok{3}\NormalTok{)}
\NormalTok{norm +}\StringTok{ }\KeywordTok{geom_point}\NormalTok{(}\DataTypeTok{alpha =} \DecValTok{1}\NormalTok{/}\DecValTok{5}\NormalTok{)}
\NormalTok{norm +}\StringTok{ }\KeywordTok{geom_point}\NormalTok{(}\DataTypeTok{alpha =} \DecValTok{1}\NormalTok{/}\DecValTok{10}\NormalTok{)}
\end{Highlighting}
\end{Shaded}

\begin{figure}

{\centering \includegraphics[width=0.32\linewidth]{figures/toolboxoverp-alpha-1} \includegraphics[width=0.32\linewidth]{figures/toolboxoverp-alpha-2} \includegraphics[width=0.32\linewidth]{figures/toolboxoverp-alpha-3} 

}

\caption{Using alpha blending to alleviate overplotting in sample data from a bivariate normal.  Alpha values from left to right: 1/3, 1/5, 1/10.\label{fig:overp-alpha}}
\end{figure}

\begin{itemize}
\itemsep1pt\parskip0pt\parsep0pt
\item
  If there is some discreteness in the data, you can randomly jitter the
  points to alleviate some overlaps. This is particularly useful in
  conjunction with transparency. By default, the amount of jitter added
  is 40\% of the resolution of the data, which leaves a small gap
  between adjacent regions. In Figure \ref{fig:overp-jitter}, table is
  recorded to the nearest integers, so we set a jitter width of half of
  that. The complete code is shown below. \index{Jittering}
  \indexf{geom_jitter} \indexf{position_jitter}
\end{itemize}

\begin{Shaded}
\begin{Highlighting}[]
\NormalTok{td <-}\StringTok{ }\KeywordTok{ggplot}\NormalTok{(diamonds, }\KeywordTok{aes}\NormalTok{(table, depth)) +}\StringTok{ }
\StringTok{  }\KeywordTok{xlim}\NormalTok{(}\DecValTok{50}\NormalTok{, }\DecValTok{70}\NormalTok{) +}\StringTok{ }\KeywordTok{ylim}\NormalTok{(}\DecValTok{50}\NormalTok{, }\DecValTok{70}\NormalTok{)}
\NormalTok{td +}\StringTok{ }\KeywordTok{geom_point}\NormalTok{()}
\NormalTok{td +}\StringTok{ }\KeywordTok{geom_jitter}\NormalTok{()}
\NormalTok{jit <-}\StringTok{ }\KeywordTok{position_jitter}\NormalTok{(}\DataTypeTok{width =} \FloatTok{0.5}\NormalTok{)}
\NormalTok{td +}\StringTok{ }\KeywordTok{geom_jitter}\NormalTok{(}\DataTypeTok{position =} \NormalTok{jit)}
\NormalTok{td +}\StringTok{ }\KeywordTok{geom_jitter}\NormalTok{(}\DataTypeTok{position =} \NormalTok{jit, }\DataTypeTok{alpha =} \DecValTok{1}\NormalTok{/}\DecValTok{10}\NormalTok{)}
\NormalTok{td +}\StringTok{ }\KeywordTok{geom_jitter}\NormalTok{(}\DataTypeTok{position =} \NormalTok{jit, }\DataTypeTok{alpha =} \DecValTok{1}\NormalTok{/}\DecValTok{50}\NormalTok{)}
\NormalTok{td +}\StringTok{ }\KeywordTok{geom_jitter}\NormalTok{(}\DataTypeTok{position =} \NormalTok{jit, }\DataTypeTok{alpha =} \DecValTok{1}\NormalTok{/}\DecValTok{200}\NormalTok{)}
\end{Highlighting}
\end{Shaded}

\begin{figure}

{\centering \includegraphics[width=0.32\linewidth]{figures/toolboxoverp-jitter-1} \includegraphics[width=0.32\linewidth]{figures/toolboxoverp-jitter-2} \includegraphics[width=0.32\linewidth]{figures/toolboxoverp-jitter-3} \includegraphics[width=0.32\linewidth]{figures/toolboxoverp-jitter-4} \includegraphics[width=0.32\linewidth]{figures/toolboxoverp-jitter-5} \includegraphics[width=0.32\linewidth]{figures/toolboxoverp-jitter-6} 

}

\caption{A plot of table vs. depth from the diamonds data, showing the use of jitter and alpha blending to alleviate overplotting in discrete data. From left to right: geom point, geom jitter with default jitter, geom jitter with horizontal jitter of 0.5 (half the gap between bands), alpha of 1/10, alpha of 1/50, alpha of 1/200.\label{fig:overp-jitter}}
\end{figure}

Alternatively, we can think of overplotting as a 2d density estimation
problem, which gives rise to two more approaches:

\begin{itemize}
\itemsep1pt\parskip0pt\parsep0pt
\item
  Bin the points and count the number in each bin, then visualise that
  count in some way (the 2d generalisation of the histogram). Breaking
  the plot into many small squares can produce distracting visual
  artefacts. (D. B. Carr et al. 1987) suggests using hexagons instead,
  and this is implemented with \texttt{geom\_hexagon()}, using the
  capabilities of the \textbf{hexbin} package (D. Carr, Lewin-Koh, and
  Maechler 2008). Figure \ref{fig:overp-bin} compares square and
  hexagonal bins, using parameters \texttt{bins} and \texttt{binwidth}
  to control the number and size of the bins. The complete code is shown
  below. \index{Binning!1d|see{Histogram}} \index{Binning!2d}
  \indexf{geom_hexagon} \indexf{stat_binhex} \indexf{stat_bin2d}
\end{itemize}

\begin{Shaded}
\begin{Highlighting}[]
\NormalTok{d <-}\StringTok{ }\KeywordTok{ggplot}\NormalTok{(diamonds, }\KeywordTok{aes}\NormalTok{(carat, price)) +}\StringTok{ }\KeywordTok{xlim}\NormalTok{(}\DecValTok{1}\NormalTok{, }\DecValTok{3}\NormalTok{) +}
\StringTok{  }\KeywordTok{theme}\NormalTok{(}\DataTypeTok{legend.position =} \StringTok{"none"}\NormalTok{)}
\NormalTok{d +}\StringTok{ }\KeywordTok{stat_bin2d}\NormalTok{()}
\NormalTok{d +}\StringTok{ }\KeywordTok{stat_bin2d}\NormalTok{(}\DataTypeTok{bins =} \DecValTok{10}\NormalTok{)}
\NormalTok{d +}\StringTok{ }\KeywordTok{stat_bin2d}\NormalTok{(}\DataTypeTok{binwidth=}\KeywordTok{c}\NormalTok{(}\FloatTok{0.02}\NormalTok{, }\DecValTok{200}\NormalTok{))}
\NormalTok{d +}\StringTok{ }\KeywordTok{stat_binhex}\NormalTok{()}
\NormalTok{d +}\StringTok{ }\KeywordTok{stat_binhex}\NormalTok{(}\DataTypeTok{bins =} \DecValTok{10}\NormalTok{)}
\NormalTok{d +}\StringTok{ }\KeywordTok{stat_binhex}\NormalTok{(}\DataTypeTok{binwidth=}\KeywordTok{c}\NormalTok{(}\FloatTok{0.02}\NormalTok{, }\DecValTok{200}\NormalTok{))}
\end{Highlighting}
\end{Shaded}

\begin{figure}

{\centering \includegraphics[width=0.32\linewidth]{figures/toolboxoverp-bin-1} \includegraphics[width=0.32\linewidth]{figures/toolboxoverp-bin-2} \includegraphics[width=0.32\linewidth]{figures/toolboxoverp-bin-3} \includegraphics[width=0.32\linewidth]{figures/toolboxoverp-bin-4} \includegraphics[width=0.32\linewidth]{figures/toolboxoverp-bin-5} \includegraphics[width=0.32\linewidth]{figures/toolboxoverp-bin-6} 

}

\caption{Binning with, top row, square bins, and bottom row, hexagonal bins. Left column uses default parameters, middle column \texttt{bins = 10}, and right column \texttt{binwidth = c(0.02, 200)}. Legends have been omitted to save space.\label{fig:overp-bin}}
\end{figure}

\begin{itemize}
\itemsep1pt\parskip0pt\parsep0pt
\item
  Estimate the 2d density with \texttt{stat\_density2d()}, and overlay
  contours from this distribution on the scatterplot, or display the
  density by itself as coloured tiles, or points with size proportional
  to density. Figure \ref{fig:overp-density} shows a few of these
  options with the code below.\index{Density!2d} \indexf{stat_density2d}
  \indexf{geom_density2d}
\end{itemize}

\begin{Shaded}
\begin{Highlighting}[]
\NormalTok{d <-}\StringTok{ }\KeywordTok{ggplot}\NormalTok{(diamonds, }\KeywordTok{aes}\NormalTok{(carat, price)) +}\StringTok{ }\KeywordTok{xlim}\NormalTok{(}\DecValTok{1}\NormalTok{,}\DecValTok{3}\NormalTok{) +}\StringTok{ }
\StringTok{  }\KeywordTok{theme}\NormalTok{(}\DataTypeTok{legend.position =} \StringTok{"none"}\NormalTok{)}
\NormalTok{d +}\StringTok{ }\KeywordTok{geom_point}\NormalTok{() +}\StringTok{ }\KeywordTok{geom_density2d}\NormalTok{()}
\NormalTok{d +}\StringTok{ }\KeywordTok{stat_density2d}\NormalTok{(}\DataTypeTok{geom =} \StringTok{"point"}\NormalTok{, }\KeywordTok{aes}\NormalTok{(}\DataTypeTok{size =} \NormalTok{..density..), }
                   \DataTypeTok{contour =} \NormalTok{F) +}\StringTok{ }\KeywordTok{scale_size}\NormalTok{(}\DataTypeTok{range =} \KeywordTok{c}\NormalTok{(}\FloatTok{0.2}\NormalTok{, }\FloatTok{1.5}\NormalTok{))}
\NormalTok{d +}\StringTok{ }\KeywordTok{stat_density2d}\NormalTok{(}\DataTypeTok{geom =} \StringTok{"tile"}\NormalTok{, }\KeywordTok{aes}\NormalTok{(}\DataTypeTok{fill =} \NormalTok{..density..), }
                   \DataTypeTok{contour =} \NormalTok{F) }
\KeywordTok{last_plot}\NormalTok{() +}\StringTok{ }\KeywordTok{scale_fill_gradient}\NormalTok{(}\DataTypeTok{limits =} \KeywordTok{c}\NormalTok{(}\FloatTok{1e-5}\NormalTok{,}\FloatTok{8e-4}\NormalTok{))}
\end{Highlighting}
\end{Shaded}

\begin{figure}

{\centering \includegraphics[width=0.4\linewidth]{figures/toolboxoverp-density-1} \includegraphics[width=0.4\linewidth]{figures/toolboxoverp-density-2} \includegraphics[width=0.4\linewidth]{figures/toolboxoverp-density-3} \includegraphics[width=0.4\linewidth]{figures/toolboxoverp-density-4} 

}

\caption{Using density estimation to model and visualise point densities.  (Top) Image displays of the density; (bottom) point and contour based displays.\label{fig:overp-density}}
\end{figure}

\begin{itemize}
\itemsep1pt\parskip0pt\parsep0pt
\item
  If you are interested in the conditional distribution of y given x,
  then the techniques of \hyperref[sub:distribution]{histogram and
  density plots} will also be useful.
\end{itemize}

Another approach to dealing with overplotting is to add data summaries
to help guide the eye to the true shape of the pattern within the data.
For example, you could add a smooth line showing the centre of the data
with \texttt{geom\_smooth}. \hyperref[sec:summary]{Statistical
summaries} has more ideas.

\hyperdef{}{sec:surface}{\section{Surface plots}\label{sec:surface}}

\texttt{ggplot} currently does not support true 3d surfaces. However, it
does support the common tools for representing 3d surfaces in 2d:
contours, coloured tiles and bubble plots. These were used to
illustrated the 2d density surfaces in the previous section. You may
also want to look at RGL,
\url{http://rgl.neoscientists.org/about.shtml}, for interactive 3d
plots, including true 3d surfaces. \index{3d graphics}
\index{Surface plots}

\hyperdef{}{sec:maps}{\section{Drawing maps}\label{sec:maps}}

\texttt{ggplot} provides some tools to make it easy to combine maps from
the \textbf{maps} package with other \texttt{ggplot} graphics. Table
\ref{tbl:maps} lists the available maps, which are unfortunately rather
US centric. There are two basic reasons you might want to use map data:
to add reference outlines to a plot of spatial data, or to construct a
choropleth map by filling regions with colour. \index{Maps!drawing}

\begin{table}
  \begin{center}
  \begin{tabular}{ll}
    \toprule
    Country & Map name \\
    \midrule
    France & france \\
    Italy & italy \\
    New Zealand & nz \\
    USA at county level & county \\
    USA at state level & state \\
    USA borders & usa \\
    Entire world & world \\ 
    \bottomrule
  \end{tabular}
  \end{center}
  \caption{Maps available in the maps package}
  \label{tbl:maps}
\end{table}

Adding map border is performed by the \texttt{borders()} function. The
first two arguments select the \texttt{map} and \texttt{region} within
the map to display. The remaining arguments control the appearance of
the borders: their \texttt{colour} and \texttt{size}. If you'd prefer
filled polygons instead of just borders, you can set the \texttt{fill}
colour. The following code uses \texttt{borders()} to display the
spatial data shown in Figure \ref{fig:borders}. \index{Maps!borders}
\indexf{geom_polygon}

\begin{Shaded}
\begin{Highlighting}[]
\KeywordTok{library}\NormalTok{(}\StringTok{"maps"}\NormalTok{)}
\KeywordTok{data}\NormalTok{(us.cities)}
\NormalTok{big_cities <-}\StringTok{ }\KeywordTok{subset}\NormalTok{(us.cities, pop >}\StringTok{ }\DecValTok{500000}\NormalTok{)}
\KeywordTok{qplot}\NormalTok{(long, lat, }\DataTypeTok{data =} \NormalTok{big_cities) +}\StringTok{ }\KeywordTok{borders}\NormalTok{(}\StringTok{"state"}\NormalTok{, }\DataTypeTok{size =} \FloatTok{0.5}\NormalTok{)}

\NormalTok{tx_cities <-}\StringTok{ }\KeywordTok{subset}\NormalTok{(us.cities, country.etc ==}\StringTok{ "TX"}\NormalTok{)}
\KeywordTok{ggplot}\NormalTok{(tx_cities, }\KeywordTok{aes}\NormalTok{(long, lat)) +}
\StringTok{  }\KeywordTok{borders}\NormalTok{(}\StringTok{"county"}\NormalTok{, }\StringTok{"texas"}\NormalTok{, }\DataTypeTok{colour =} \StringTok{"grey70"}\NormalTok{) +}
\StringTok{  }\KeywordTok{geom_point}\NormalTok{(}\DataTypeTok{alpha =} \FloatTok{0.5}\NormalTok{)}
\end{Highlighting}
\end{Shaded}

\begin{figure}

{\centering \includegraphics[width=0.49\linewidth]{figures/toolboxborders-1} \includegraphics[width=0.49\linewidth]{figures/toolboxborders-2} 

}

\caption{Example using the borders function.  (Left) All cities with population (as of January 2006) of greater than half a million, (right) cities in Texas.\label{fig:borders}}
\end{figure}

Choropleth maps are a little trickier and a lot less automated because
it is challenging to match the identifiers in your data to the
identifiers in the map data. The following example shows how to use
\texttt{map\_data()} to convert a map into a data frame, which can then
be \texttt{merge()}d with your data to produce a choropleth map. The
results are shown in Figure \ref{fig:choropleth}. The details for your
data will probably be different, but the key is to have a column in your
data and a column in the map data that can be matched.
\index{Maps!choropleth}

\begin{Shaded}
\begin{Highlighting}[]
\KeywordTok{library}\NormalTok{(}\StringTok{"maps"}\NormalTok{)}
\NormalTok{states <-}\StringTok{ }\KeywordTok{map_data}\NormalTok{(}\StringTok{"state"}\NormalTok{)}
\NormalTok{arrests <-}\StringTok{ }\NormalTok{USArrests}
\KeywordTok{names}\NormalTok{(arrests) <-}\StringTok{ }\KeywordTok{tolower}\NormalTok{(}\KeywordTok{names}\NormalTok{(arrests))}
\NormalTok{arrests$region <-}\StringTok{ }\KeywordTok{tolower}\NormalTok{(}\KeywordTok{rownames}\NormalTok{(USArrests))}

\NormalTok{choro <-}\StringTok{ }\KeywordTok{merge}\NormalTok{(states, arrests, }\DataTypeTok{by =} \StringTok{"region"}\NormalTok{)}
\CommentTok{# Reorder the rows because order matters when drawing polygons}
\CommentTok{# and merge destroys the original ordering}
\NormalTok{choro <-}\StringTok{ }\NormalTok{choro[}\KeywordTok{order}\NormalTok{(choro$order), ]}
\KeywordTok{qplot}\NormalTok{(long, lat, }\DataTypeTok{data =} \NormalTok{choro, }\DataTypeTok{group =} \NormalTok{group, }
  \DataTypeTok{fill =} \NormalTok{assault, }\DataTypeTok{geom =} \StringTok{"polygon"}\NormalTok{)}
\KeywordTok{qplot}\NormalTok{(long, lat, }\DataTypeTok{data =} \NormalTok{choro, }\DataTypeTok{group =} \NormalTok{group, }
  \DataTypeTok{fill =} \NormalTok{assault /}\StringTok{ }\NormalTok{murder, }\DataTypeTok{geom =} \StringTok{"polygon"}\NormalTok{)}
\end{Highlighting}
\end{Shaded}

\begin{figure}

{\centering \includegraphics[width=0.49\linewidth]{figures/toolboxchoropleth-1} \includegraphics[width=0.49\linewidth]{figures/toolboxchoropleth-2} 

}

\caption{Two choropleth maps showing number of assaults (left) and the ratio of assaults to murders (right).\label{fig:choropleth}}
\end{figure}

The \texttt{map\_data()} function is also useful if you'd like to
process the map data in some way. In the following example we compute
the (approximate) centre of each county in Iowa and then use those
centres to label the map.

\begin{Shaded}
\begin{Highlighting}[]
\NormalTok{>}\StringTok{ }\NormalTok{ia <-}\StringTok{ }\KeywordTok{map_data}\NormalTok{(}\StringTok{"county"}\NormalTok{, }\StringTok{"iowa"}\NormalTok{)}
\NormalTok{>}\StringTok{ }\NormalTok{mid_range <-}\StringTok{ }\NormalTok{function(x) }\KeywordTok{mean}\NormalTok{(}\KeywordTok{range}\NormalTok{(x, }\DataTypeTok{na.rm =} \OtherTok{TRUE}\NormalTok{))}
\NormalTok{>}\StringTok{ }\NormalTok{centres <-}\StringTok{ }\NormalTok{ia %>%}\StringTok{ }\NormalTok{tbl_df %>%}\StringTok{ }\KeywordTok{group_by}\NormalTok{(subregion) %>%}
\NormalTok{+}\StringTok{   }\KeywordTok{summarise}\NormalTok{(}\DataTypeTok{lat =} \KeywordTok{mid_range}\NormalTok{(lat), }\DataTypeTok{long =} \KeywordTok{mid_range}\NormalTok{(long))}
\NormalTok{>}\StringTok{ }\KeywordTok{ggplot}\NormalTok{(ia, }\KeywordTok{aes}\NormalTok{(long, lat)) +}\StringTok{ }
\NormalTok{+}\StringTok{   }\KeywordTok{geom_polygon}\NormalTok{(}\KeywordTok{aes}\NormalTok{(}\DataTypeTok{group =} \NormalTok{group), }
\NormalTok{+}\StringTok{     }\DataTypeTok{fill =} \OtherTok{NA}\NormalTok{, }\DataTypeTok{colour =} \StringTok{"grey60"}\NormalTok{) +}
\NormalTok{+}\StringTok{   }\KeywordTok{geom_text}\NormalTok{(}\KeywordTok{aes}\NormalTok{(}\DataTypeTok{label =} \NormalTok{subregion), }\DataTypeTok{data =} \NormalTok{centres, }
\NormalTok{+}\StringTok{     }\DataTypeTok{size =} \DecValTok{2}\NormalTok{, }\DataTypeTok{angle =} \DecValTok{45}\NormalTok{)}
\end{Highlighting}
\end{Shaded}

\begin{flushleft}\includegraphics[width=0.49\linewidth]{figures/toolboxiowa-1} \end{flushleft}

\hyperdef{}{sec:uncertainty}{\section{Revealing
uncertainty}\label{sec:uncertainty}}

If you have information about the uncertainty present in your data,
whether it be from a model or from distributional assumptions, it is
often important to display it. There are four basic families of geoms
that can be used for this job, depending on whether the x values are
discrete or continuous, and whether or not you want to display the
middle of the interval, or just the extent. These geoms are listed in
Table \ref{tbl:interval}. These geoms assume that you are interested in
the distribution of y conditional on x and use the aesthetics
\texttt{ymin} and \texttt{ymax} to determine the range of the y values.
If you want the opposite, see \hyperref[sub:cartesian]{cartesian
coordinate systems}. \index{Uncertainty!visualising} \index{Error bar}
\index{Error band} \indexf{geom_ribbon} \indexf{geom_smooth}
\indexf{geom_errorbar} \indexf{geom_linerange} \indexf{geom_crossbar}
\indexf{geom_pointrange}

Because there are so many different ways to calculate standard errors,
the calculation is up to you. \index{Standard errors} For very simple
cases, \texttt{ggplot} provides some tools in the form of summary
functions described in \hyperref[sec:summary]{statistical summaries},
otherwise you will have to do it yourself. The \textbf{effects} package
(Fox 2008) is particularly useful for extracting these values from
linear models. The following example fits a two-way model with
interaction, and shows how to extract and visualise marginal and
conditional effects. Figure \ref{fig:model-cat} focusses on the
categorical variable colour, and Figure \ref{fig:model-cont} focusses on
the continuous variable carat. \index{Package!effects}

\begin{Shaded}
\begin{Highlighting}[]
\NormalTok{>}\StringTok{ }\NormalTok{d <-}\StringTok{ }\KeywordTok{subset}\NormalTok{(diamonds, carat <}\StringTok{ }\FloatTok{2.5} \NormalTok{&}\StringTok{ }
\NormalTok{+}\StringTok{   }\KeywordTok{rbinom}\NormalTok{(}\KeywordTok{nrow}\NormalTok{(diamonds), }\DecValTok{1}\NormalTok{, }\FloatTok{0.2}\NormalTok{) ==}\StringTok{ }\DecValTok{1}\NormalTok{)}
\NormalTok{>}\StringTok{ }\NormalTok{d$lcarat <-}\StringTok{ }\KeywordTok{log10}\NormalTok{(d$carat)}
\NormalTok{>}\StringTok{ }\NormalTok{d$lprice <-}\StringTok{ }\KeywordTok{log10}\NormalTok{(d$price)}
\NormalTok{>}\StringTok{ }
\ErrorTok{>}\StringTok{ }\CommentTok{# Remove overall linear trend}
\ErrorTok{>}\StringTok{ }\NormalTok{detrend <-}\StringTok{ }\KeywordTok{lm}\NormalTok{(lprice ~}\StringTok{ }\NormalTok{lcarat, }\DataTypeTok{data =} \NormalTok{d)}
\NormalTok{>}\StringTok{ }\NormalTok{d$lprice2 <-}\StringTok{ }\KeywordTok{resid}\NormalTok{(detrend)}
\NormalTok{>}\StringTok{ }
\ErrorTok{>}\StringTok{ }\NormalTok{mod <-}\StringTok{ }\KeywordTok{lm}\NormalTok{(lprice2 ~}\StringTok{ }\NormalTok{lcarat *}\StringTok{ }\NormalTok{color, }\DataTypeTok{data =} \NormalTok{d)}
\NormalTok{>}\StringTok{ }
\ErrorTok{>}\StringTok{ }\KeywordTok{library}\NormalTok{(}\StringTok{"effects"}\NormalTok{)}
\NormalTok{>}\StringTok{ }\NormalTok{effectdf <-}\StringTok{ }\NormalTok{function(...) \{}
\NormalTok{+}\StringTok{   }\KeywordTok{suppressWarnings}\NormalTok{(}\KeywordTok{as.data.frame}\NormalTok{(}\KeywordTok{effect}\NormalTok{(...)))}
\NormalTok{+}\StringTok{ }\NormalTok{\}}
\NormalTok{>}\StringTok{ }\NormalTok{color <-}\StringTok{ }\KeywordTok{effectdf}\NormalTok{(}\StringTok{"color"}\NormalTok{, mod)}
\NormalTok{>}\StringTok{ }\NormalTok{both1 <-}\StringTok{ }\KeywordTok{effectdf}\NormalTok{(}\StringTok{"lcarat:color"}\NormalTok{, mod)}
\NormalTok{>}\StringTok{ }
\ErrorTok{>}\StringTok{ }\NormalTok{carat <-}\StringTok{ }\KeywordTok{effectdf}\NormalTok{(}\StringTok{"lcarat"}\NormalTok{, mod, }\DataTypeTok{default.levels =} \DecValTok{50}\NormalTok{)}
\NormalTok{>}\StringTok{ }\NormalTok{both2 <-}\StringTok{ }\KeywordTok{effectdf}\NormalTok{(}\StringTok{"lcarat:color"}\NormalTok{, mod, }\DataTypeTok{default.levels =} \DecValTok{3}\NormalTok{)}
\end{Highlighting}
\end{Shaded}

\begin{figure}

{\centering \includegraphics[width=0.49\linewidth]{figures/toolboxldata-1} \includegraphics[width=0.49\linewidth]{figures/toolboxldata-2} 

}

\caption{Data transformed to remove most obvious effects. (Left) Both x and y axes are log10 transformed to remove non-linearity. (Right) The major linear trend is removed.\label{fig:ldata}}
\end{figure}

\begin{figure}

{\centering \includegraphics[width=0.49\linewidth]{figures/toolboxmodel-cat-1} \includegraphics[width=0.49\linewidth]{figures/toolboxmodel-cat-2} 

}

\caption{Displaying uncertainty in model estimates for colour. (Left) Marginal effect of colour. (Right) conditional effects of colour for different levels of carat. Error bars show 95\% pointwise confidence intervals.\label{fig:model-cat}}
\end{figure}

\begin{figure}

{\centering \includegraphics[width=0.49\linewidth]{figures/toolboxmodel-cont-1} \includegraphics[width=0.49\linewidth]{figures/toolboxmodel-cont-2} 

}

\caption{Displaying uncertainty in model estimates for carat. (Left) marginal effect of carat. (Right) conditional effects of carat for different levels of colour. Bands show 95\% point-wise confidence intervals.\label{fig:model-cont}}
\end{figure}

Note, when captioning such figures, you need to carefully describe the
nature of the confidence intervals, and whether or not it is meaningful
to look at the overlap. That is, are the standard errors for the means
or for the differences between means? The packages \textbf{multcomp} and
\textbf{multcompView} are useful for calculating and displaying these
errors while correctly adjusting for multiple comparisons.
\index{Package!multcomp} \index{Package!multcompView}

\hyperdef{}{sec:summary}{\section{Statistical
summaries}\label{sec:summary}}

It's often useful to be able to summarise the y values for each unique x
value. In \texttt{ggplot}, this role is played by
\texttt{stat\_summary()}, which provides a flexible way of summarising
the conditional distribution of y with the aesthetics \texttt{ymin},
\texttt{y} and \texttt{ymax}. Figure \ref{fig:stat-summary} shows some
of the variety of summaries that can be achieved with this tool.
\index{Summary!statistical} \indexf{stat_summary} \index{Stats!summary}

When using \texttt{stat\_summary()} you can either supply these the
summary functions individually or altogether. These alternatives are
described below.

\begin{Shaded}
\begin{Highlighting}[]
\NormalTok{m <-}\StringTok{ }\KeywordTok{ggplot}\NormalTok{(movies, }\KeywordTok{aes}\NormalTok{(year, rating))}
\NormalTok{m +}\StringTok{ }\KeywordTok{stat_summary}\NormalTok{(}\DataTypeTok{fun.y =} \StringTok{"median"}\NormalTok{, }\DataTypeTok{geom =} \StringTok{"line"}\NormalTok{)}
\NormalTok{m +}\StringTok{ }\KeywordTok{stat_summary}\NormalTok{(}\DataTypeTok{fun.data =} \StringTok{"median_hilow"}\NormalTok{, }\DataTypeTok{geom =} \StringTok{"smooth"}\NormalTok{)}
\NormalTok{m +}\StringTok{ }\KeywordTok{stat_summary}\NormalTok{(}\DataTypeTok{fun.y =} \StringTok{"mean"}\NormalTok{, }\DataTypeTok{geom =} \StringTok{"line"}\NormalTok{)}
\NormalTok{m +}\StringTok{ }\KeywordTok{stat_summary}\NormalTok{(}\DataTypeTok{fun.data =} \StringTok{"mean_cl_boot"}\NormalTok{, }\DataTypeTok{geom =} \StringTok{"smooth"}\NormalTok{)}
\NormalTok{m2 <-}\StringTok{ }\KeywordTok{ggplot}\NormalTok{(movies, }\KeywordTok{aes}\NormalTok{(}\KeywordTok{round}\NormalTok{(rating), }\KeywordTok{log10}\NormalTok{(votes)))}
\NormalTok{m2 +}\StringTok{ }\KeywordTok{stat_summary}\NormalTok{(}\DataTypeTok{fun.y =} \StringTok{"mean"}\NormalTok{, }\DataTypeTok{geom =} \StringTok{"point"}\NormalTok{)}
\NormalTok{m2 +}\StringTok{ }\KeywordTok{stat_summary}\NormalTok{(}\DataTypeTok{fun.data =} \StringTok{"mean_cl_normal"}\NormalTok{, }\DataTypeTok{geom =} \StringTok{"errorbar"}\NormalTok{)}
\NormalTok{m2 +}\StringTok{ }\KeywordTok{stat_summary}\NormalTok{(}\DataTypeTok{fun.data =} \StringTok{"median_hilow"}\NormalTok{, }\DataTypeTok{geom =} \StringTok{"pointrange"}\NormalTok{)}
\NormalTok{m2 +}\StringTok{ }\KeywordTok{stat_summary}\NormalTok{(}\DataTypeTok{fun.data =} \StringTok{"median_hilow"}\NormalTok{, }\DataTypeTok{geom =} \StringTok{"crossbar"}\NormalTok{)}
\end{Highlighting}
\end{Shaded}

\begin{figure}

{\centering \includegraphics[width=0.24\linewidth]{figures/toolboxstat-summary-1} \includegraphics[width=0.24\linewidth]{figures/toolboxstat-summary-2} \includegraphics[width=0.24\linewidth]{figures/toolboxstat-summary-3} \includegraphics[width=0.24\linewidth]{figures/toolboxstat-summary-4} \includegraphics[width=0.24\linewidth]{figures/toolboxstat-summary-5} \includegraphics[width=0.24\linewidth]{figures/toolboxstat-summary-6} \includegraphics[width=0.24\linewidth]{figures/toolboxstat-summary-7} \includegraphics[width=0.24\linewidth]{figures/toolboxstat-summary-8} 

}

\caption{Examples of \texttt{stat\_summary()} in use. (Top) Continuous x with, from left to right, median and line, \texttt{median\_hilow()} and smooth, mean and line, and \texttt{mean\_cl\_boot()} and smooth. (Bottom) Discrete x with, from left to right, \texttt{mean()} and point, \texttt{mean\_cl\_normal()} and error bar, \texttt{median\_hilow()} and point range, and \texttt{median\_hilow()} and crossbar. Note that \texttt{ggplot} displays the full range of the data, not just the range of the summary statistics.\label{fig:stat-summary}}
\end{figure}

\subsection{Individual summary functions}

The arguments \texttt{fun.y}, \texttt{fun.ymin} and \texttt{fun.ymax}
accept simple numeric summary functions. You can use any summary
function that takes a vector of numbers and returns a single numeric
value: \texttt{mean()}, \texttt{median()}, \texttt{min()},
\texttt{max()}.

\begin{Shaded}
\begin{Highlighting}[]
\NormalTok{>}\StringTok{ }\NormalTok{midm <-}\StringTok{ }\NormalTok{function(x) }\KeywordTok{mean}\NormalTok{(x, }\DataTypeTok{trim =} \FloatTok{0.5}\NormalTok{)}
\NormalTok{>}\StringTok{ }\NormalTok{m2 +}\StringTok{ }
\NormalTok{+}\StringTok{   }\KeywordTok{stat_summary}\NormalTok{(}\KeywordTok{aes}\NormalTok{(}\DataTypeTok{colour =} \StringTok{"trimmed"}\NormalTok{), }\DataTypeTok{fun.y =} \NormalTok{midm, }
\NormalTok{+}\StringTok{     }\DataTypeTok{geom =} \StringTok{"point"}\NormalTok{) +}
\NormalTok{+}\StringTok{   }\KeywordTok{stat_summary}\NormalTok{(}\KeywordTok{aes}\NormalTok{(}\DataTypeTok{colour =} \StringTok{"raw"}\NormalTok{), }\DataTypeTok{fun.y =} \NormalTok{mean, }
\NormalTok{+}\StringTok{     }\DataTypeTok{geom =} \StringTok{"point"}\NormalTok{) +}\StringTok{ }
\NormalTok{+}\StringTok{   }\KeywordTok{scale_colour_hue}\NormalTok{(}\StringTok{"Mean"}\NormalTok{)}
\end{Highlighting}
\end{Shaded}

\begin{flushleft}\includegraphics[width=0.49\linewidth]{figures/toolboxstat-trim-1} \end{flushleft}

\subsection{Single summary function}

\texttt{fun.data} can be used with more complex summary functions such
as one of the summary functions from the \textbf{Hmisc} package (Harrell
2008) described in Table \ref{tbl:hmisc}. You can also write your own
summary function. This summary function should return a named vector as
output, as shown in the following example.

\begin{Shaded}
\begin{Highlighting}[]
\NormalTok{>}\StringTok{ }\NormalTok{iqr <-}\StringTok{ }\NormalTok{function(x, ...) \{}
\NormalTok{+}\StringTok{   }\NormalTok{qs <-}\StringTok{ }\KeywordTok{quantile}\NormalTok{(}\KeywordTok{as.numeric}\NormalTok{(x), }\KeywordTok{c}\NormalTok{(}\FloatTok{0.25}\NormalTok{, }\FloatTok{0.75}\NormalTok{), }\DataTypeTok{na.rm =} \NormalTok{T)}
\NormalTok{+}\StringTok{   }\KeywordTok{names}\NormalTok{(qs) <-}\StringTok{ }\KeywordTok{c}\NormalTok{(}\StringTok{"ymin"}\NormalTok{, }\StringTok{"ymax"}\NormalTok{)}
\NormalTok{+}\StringTok{   }\NormalTok{qs}
\NormalTok{+}\StringTok{ }\NormalTok{\}}
\NormalTok{>}\StringTok{ }\NormalTok{m +}\StringTok{ }\KeywordTok{stat_summary}\NormalTok{(}\DataTypeTok{fun.data =} \StringTok{"iqr"}\NormalTok{, }\DataTypeTok{geom=}\StringTok{"ribbon"}\NormalTok{)}
\end{Highlighting}
\end{Shaded}

\begin{flushleft}\includegraphics[width=0.49\linewidth]{figures/toolboxiqr-1} \end{flushleft}

\hyperdef{}{sec:annotating}{\section{Annotating a
plot}\label{sec:annotating}}

When annotating your plot with additional labels, the important thing to
remember is that these annotations are just extra data. There are two
basic ways to add annotations: one at a time, or many at once.
\index{Annotation}

Adding one at a time works best for small numbers of annotations with
varying aesthetics. You just set all the values to give the desired
properties. If you have multiple annotations with similar properties, it
may make sense to put them all in a data frame and add them at once. The
example below demonstrates both approaches by adding information about
presidents to economic data.

\begin{Shaded}
\begin{Highlighting}[]
\NormalTok{>}\StringTok{ }\NormalTok{(unemp <-}\StringTok{ }\KeywordTok{qplot}\NormalTok{(date, unemploy, }\DataTypeTok{data=}\NormalTok{economics, }\DataTypeTok{geom=}\StringTok{"line"}\NormalTok{, }
\NormalTok{+}\StringTok{                 }\DataTypeTok{xlab =} \StringTok{""}\NormalTok{, }\DataTypeTok{ylab =} \StringTok{"No. unemployed (1000s)"}\NormalTok{))}
\end{Highlighting}
\end{Shaded}

\begin{flushleft}\includegraphics[width=\linewidth]{figures/toolboxumep-1} \end{flushleft}

\begin{Shaded}
\begin{Highlighting}[]
\NormalTok{>}\StringTok{ }
\ErrorTok{>}\StringTok{ }\NormalTok{presidential <-}\StringTok{ }\NormalTok{presidential[-(}\DecValTok{1}\NormalTok{:}\DecValTok{3}\NormalTok{), ]}
\NormalTok{>}\StringTok{ }
\ErrorTok{>}\StringTok{ }\NormalTok{yrng <-}\StringTok{ }\KeywordTok{range}\NormalTok{(economics$unemploy)}
\NormalTok{>}\StringTok{ }\NormalTok{xrng <-}\StringTok{ }\KeywordTok{range}\NormalTok{(economics$date)}
\NormalTok{>}\StringTok{ }\NormalTok{unemp +}\StringTok{ }\KeywordTok{geom_vline}\NormalTok{(}\KeywordTok{aes}\NormalTok{(}\DataTypeTok{xintercept =} \KeywordTok{as.numeric}\NormalTok{(start)), }\DataTypeTok{data =} \NormalTok{presidential)}
\end{Highlighting}
\end{Shaded}

\begin{flushleft}\includegraphics[width=\linewidth]{figures/toolboxumep-2} \end{flushleft}

\begin{Shaded}
\begin{Highlighting}[]
\NormalTok{>}\StringTok{ }\NormalTok{unemp +}\StringTok{ }\KeywordTok{geom_rect}\NormalTok{(}\KeywordTok{aes}\NormalTok{(}\OtherTok{NULL}\NormalTok{, }\OtherTok{NULL}\NormalTok{, }\DataTypeTok{xmin =} \NormalTok{start, }\DataTypeTok{xmax =} \NormalTok{end, }
\NormalTok{+}\StringTok{                       }\DataTypeTok{fill =} \NormalTok{party), }\DataTypeTok{ymin =} \NormalTok{yrng[}\DecValTok{1}\NormalTok{], }\DataTypeTok{ymax =} \NormalTok{yrng[}\DecValTok{2}\NormalTok{], }
\NormalTok{+}\StringTok{                   }\DataTypeTok{alpha =} \FloatTok{0.2}\NormalTok{, }\DataTypeTok{data =} \NormalTok{presidential) +}\StringTok{ }
\NormalTok{+}\StringTok{   }\KeywordTok{scale_fill_manual}\NormalTok{(}\DataTypeTok{values =} \KeywordTok{c}\NormalTok{(}\StringTok{"blue"}\NormalTok{, }\StringTok{"red"}\NormalTok{))}
\end{Highlighting}
\end{Shaded}

\begin{flushleft}\includegraphics[width=\linewidth]{figures/toolboxumep-3} \end{flushleft}

\begin{Shaded}
\begin{Highlighting}[]
\NormalTok{>}\StringTok{ }\KeywordTok{last_plot}\NormalTok{() +}\StringTok{ }\KeywordTok{geom_text}\NormalTok{(}\KeywordTok{aes}\NormalTok{(}\DataTypeTok{x =} \NormalTok{start, }\DataTypeTok{y =} \NormalTok{yrng[}\DecValTok{1}\NormalTok{], }\DataTypeTok{label =} \NormalTok{name), }
\NormalTok{+}\StringTok{                         }\DataTypeTok{data =} \NormalTok{presidential, }\DataTypeTok{size =} \DecValTok{3}\NormalTok{, }\DataTypeTok{hjust =} \DecValTok{0}\NormalTok{, }\DataTypeTok{vjust =} \DecValTok{0}\NormalTok{)}
\end{Highlighting}
\end{Shaded}

\begin{flushleft}\includegraphics[width=\linewidth]{figures/toolboxumep-4} \end{flushleft}

\begin{Shaded}
\begin{Highlighting}[]
\NormalTok{>}\StringTok{ }
\ErrorTok{>}\StringTok{ }\NormalTok{caption <-}\StringTok{ }\KeywordTok{paste}\NormalTok{(}\KeywordTok{strwrap}\NormalTok{(}\StringTok{"Unemployment rates in the US have }
\StringTok{+   varied a lot over the years"}\NormalTok{, }\DecValTok{40}\NormalTok{), }\DataTypeTok{collapse=}\StringTok{"}\CharTok{\textbackslash{}n}\StringTok{"}\NormalTok{)}
\NormalTok{>}\StringTok{ }\NormalTok{unemp +}\StringTok{ }\KeywordTok{geom_text}\NormalTok{(}\KeywordTok{aes}\NormalTok{(x, y, }\DataTypeTok{label =} \NormalTok{caption), }
\NormalTok{+}\StringTok{                   }\DataTypeTok{data =} \KeywordTok{data.frame}\NormalTok{(}\DataTypeTok{x =} \NormalTok{xrng[}\DecValTok{2}\NormalTok{], }\DataTypeTok{y =} \NormalTok{yrng[}\DecValTok{2}\NormalTok{]), }
\NormalTok{+}\StringTok{                   }\DataTypeTok{hjust =} \DecValTok{1}\NormalTok{, }\DataTypeTok{vjust =} \DecValTok{1}\NormalTok{, }\DataTypeTok{size =} \DecValTok{4}\NormalTok{)}
\end{Highlighting}
\end{Shaded}

\begin{flushleft}\includegraphics[width=\linewidth]{figures/toolboxumep-5} \end{flushleft}

\begin{Shaded}
\begin{Highlighting}[]
\NormalTok{>}\StringTok{ }
\ErrorTok{>}\StringTok{ }\NormalTok{highest <-}\StringTok{ }\KeywordTok{subset}\NormalTok{(economics, unemploy ==}\StringTok{ }\KeywordTok{max}\NormalTok{(unemploy))}
\NormalTok{>}\StringTok{ }\NormalTok{unemp +}\StringTok{ }\KeywordTok{geom_point}\NormalTok{(}\DataTypeTok{data =} \NormalTok{highest, }
\NormalTok{+}\StringTok{                    }\DataTypeTok{size =} \DecValTok{3}\NormalTok{, }\DataTypeTok{colour =} \StringTok{"red"}\NormalTok{, }\DataTypeTok{alpha =} \FloatTok{0.5}\NormalTok{)}
\end{Highlighting}
\end{Shaded}

\begin{flushleft}\includegraphics[width=\linewidth]{figures/toolboxumep-6} \end{flushleft}

\begin{itemize}
\itemsep1pt\parskip0pt\parsep0pt
\item
  \texttt{geom\_text()} for adding text descriptions or labelling
  points. Most plots will not benefit from adding text to every single
  observation on the plot. However, pulling out just a few observations
  (using subset) can be very useful. Typically you will want to label
  outliers or other important points. \index{Labels}
\item
  \texttt{geom\_vline()}, \texttt{geom\_hline()}: add vertical or
  horizontal lines to a plot. \indexf{geom_vline} \indexf{geom_hline}
\item
  \texttt{geom\_abline()}: add lines with arbitrary slope and intercept
  to a plot. \indexf{geom_abline}
\item
  \texttt{geom\_rect()} for highlighting interesting rectangular regions
  of the plot. \texttt{geom\_rect()} has aesthetics \texttt{xmin},
  \texttt{xmax}, \texttt{ymin} and \texttt{ymax}. \indexf{geom_rect}
\item
  \texttt{geom\_line()}, \texttt{geom\_path()} and
  \texttt{geom\_segment()} for adding lines. All these geoms have an
  \texttt{arrow} parameter, which allows you to place an arrowhead on
  the line. You create arrowheads with the \texttt{arrow()} function,
  which has arguments \texttt{angle}, \texttt{length}, \texttt{ends} and
  \texttt{type}. \indexf{geom_line}
\end{itemize}

\hyperdef{}{sec:weighting}{\section{Weighted data}\label{sec:weighting}}

When you have aggregated data where each row in the dataset represents
multiple observations, you need some way to take into account the
weighting variable. We will use some data collected on Midwest states in
the 2000 US census. The data consists mainly of percentages (e.g.,
percent white, percent below poverty line, percent with college degree)
and some information for each county (area, total population, population
density). \index{Weighting}

There are a few different things we might want to weight by:

\begin{itemize}
\itemsep1pt\parskip0pt\parsep0pt
\item
  nothing, to look at numbers of counties
\item
  total population, to work with absolute numbers
\item
  area, to investigate geographic effects
\end{itemize}

The choice of a weighting variable profoundly affects what we are
looking at in the plot and the conclusions that we will draw. There are
two aesthetic attributes that can be used to adjust for weights.
Firstly, for simple geoms like lines and points, you can make the size
of the grob proportional to the number of points, using the
\texttt{size} aesthetic, as with the following code, whose results are
shown in Figure \ref{fig:miss-basic}.

\begin{Shaded}
\begin{Highlighting}[]
\KeywordTok{qplot}\NormalTok{(percwhite, percbelowpoverty, }\DataTypeTok{data =} \NormalTok{midwest)}
\KeywordTok{qplot}\NormalTok{(percwhite, percbelowpoverty, }\DataTypeTok{data =} \NormalTok{midwest, }
  \DataTypeTok{size =} \NormalTok{poptotal /}\StringTok{ }\FloatTok{1e6}\NormalTok{) +}\StringTok{ }\KeywordTok{scale_size_area}\NormalTok{(}\StringTok{"Population}\CharTok{\textbackslash{}n}\StringTok{(millions)"}\NormalTok{, }
  \DataTypeTok{breaks =} \KeywordTok{c}\NormalTok{(}\FloatTok{0.5}\NormalTok{, }\DecValTok{1}\NormalTok{, }\DecValTok{2}\NormalTok{, }\DecValTok{4}\NormalTok{))}
\KeywordTok{qplot}\NormalTok{(percwhite, percbelowpoverty, }\DataTypeTok{data =} \NormalTok{midwest, }\DataTypeTok{size =} \NormalTok{area) +}\StringTok{ }
\StringTok{  }\KeywordTok{scale_size_area}\NormalTok{()}
\end{Highlighting}
\end{Shaded}

\begin{figure}

{\centering \includegraphics[width=0.32\linewidth]{figures/toolboxmiss-basic-1} \includegraphics[width=0.32\linewidth]{figures/toolboxmiss-basic-2} \includegraphics[width=0.32\linewidth]{figures/toolboxmiss-basic-3} 

}

\caption{Using size to display weights. No weighting (left), weighting by population (centre) and by area (right).\label{fig:miss-basic}}
\end{figure}

For more complicated grobs which involve some statistical
transformation, we specify weights with the \texttt{weight} aesthetic.
These weights will be passed on to the statistical summary function.
Weights are supported for every case where it makes sense: smoothers,
quantile regressions, boxplots, histograms, and density plots. You can't
see this weighting variable directly, and it doesn't produce a legend,
but it will change the results of the statistical summary. Figure
\ref{fig:weight-lm} shows how weighting by population density affects
the relationship between percent white and percent below the poverty
line.

\begin{Shaded}
\begin{Highlighting}[]
\NormalTok{lm_smooth <-}\StringTok{ }\KeywordTok{geom_smooth}\NormalTok{(}\DataTypeTok{method =} \NormalTok{lm, }\DataTypeTok{size =} \DecValTok{1}\NormalTok{)}
\KeywordTok{qplot}\NormalTok{(percwhite, percbelowpoverty, }\DataTypeTok{data =} \NormalTok{midwest) +}\StringTok{ }\NormalTok{lm_smooth }
\KeywordTok{qplot}\NormalTok{(percwhite, percbelowpoverty, }\DataTypeTok{data =} \NormalTok{midwest, }
  \DataTypeTok{weight =} \NormalTok{popdensity, }\DataTypeTok{size =} \NormalTok{popdensity) +}\StringTok{ }\NormalTok{lm_smooth}
\end{Highlighting}
\end{Shaded}

\begin{figure}

{\centering \includegraphics[width=0.49\linewidth]{figures/toolboxweight-lm-1} \includegraphics[width=0.49\linewidth]{figures/toolboxweight-lm-2} 

}

\caption{An unweighted line of best fit (left) and weighted by population size (right).\label{fig:weight-lm}}
\end{figure}

When we weight a histogram or density plot by total population, we
change from looking at the distribution of the number of counties, to
the distribution of the number of people. Figure \ref{fig:weight-hist}
shows the difference this makes for a histogram of the percentage below
the poverty line. \index{Histogram!weighted}

\begin{Shaded}
\begin{Highlighting}[]
\KeywordTok{qplot}\NormalTok{(percbelowpoverty, }\DataTypeTok{data =} \NormalTok{midwest, }\DataTypeTok{binwidth =} \DecValTok{1}\NormalTok{)}
\KeywordTok{qplot}\NormalTok{(percbelowpoverty, }\DataTypeTok{data =} \NormalTok{midwest, }\DataTypeTok{weight =} \NormalTok{poptotal, }
  \DataTypeTok{binwidth =} \DecValTok{1}\NormalTok{) +}\StringTok{ }\KeywordTok{ylab}\NormalTok{(}\StringTok{"population"}\NormalTok{)}
\end{Highlighting}
\end{Shaded}

\begin{figure}

{\centering \includegraphics[width=0.49\linewidth]{figures/toolboxweight-hist-1} \includegraphics[width=0.49\linewidth]{figures/toolboxweight-hist-2} 

}

\caption{The difference between an unweighted (left) and weighted (right) histogram. The unweighted histogram shows number of counties, while the weighted histogram shows population. The weighting considerably changes the interpretation!\label{fig:weight-hist}}
\end{figure}

Carr, D. B., R. J. Littlefield, W. L. Nicholson, and J. S. Littlefield.
1987. ``Scatterplot Matrix Techniques for Large N.'' \emph{Journal of
the American Statistical Association} 82 (398): 424--36.

Carr, Dan, Nicholas Lewin-Koh, and Martin Maechler. 2008. \emph{Hexbin:
Hexagonal Binning Routines}.

Fox, John. 2008. \emph{Effects: Effect Displays for Linear and
Generalized Linear Models}.
\url{http://socserv.socsci.mcmaster.ca/jfox/}.

Harrell, Jr, Frank E. 2008. \emph{Hmisc: Harrell Miscellaneous}.
\url{http://biostat.mc.vanderbilt.edu/s/Hmisc}.
