\chapter{Reducing duplication}\label{cha:duplication}

\section{Introduction}

A major requirement of a good data analysis is flexibility. If the data
changes, or you discover something that makes you rethink your basic
assumptions, you need to be able to easily change many plots at once.
The main inhibitor of flexibility is duplication. If you have the same
plotting statement repeated over and over again, you have to make the
same change in many different places. Often just the thought of making
all those changes is exhausting!

This chapter describes three ways of reducing duplication. In
\hyperref[sec:iteration]{iteration}, you will learn how to iteratively
modify the previous plot, allowing you to build on top of your previous
work without having to retype a lot of code.
\hyperref[sec:templates]{Plot templates} will show you how to produce
plot `templates' that encapsulate repeated components that are defined
once and used in many different places. Finally,
\hyperref[sec:functions]{plot functions} talks about how to create
functions that create or modify plots. \index{Duplication!reducing}
\index{Reducing duplication}

\hyperdef{}{sec:iteration}{\section{Iteration}\label{sec:iteration}}

Whenever you create or modify a plot, ggplot saves a copy of the result
so you can refer to it in later expressions. You can access this plot
with \texttt{last\_plot()}. This is useful in interactive work as you
can start with a basic plot and then iteratively add layers and tweak
the scales until you get to the final result. The following code
demonstrates iteratively zooming in on a plot to find a region of
interest, and then adding a layer which highlights something interesting
that we have found: very few diamonds have equal x and y dimensions. The
plots are shown in Figure \ref{fig:iterate-limits}. \index{Iteration}
\indexf{last_plot} \index{Duplication!iteration}

\begin{Shaded}
\begin{Highlighting}[]
\KeywordTok{qplot}\NormalTok{(x, y, }\DataTypeTok{data =} \NormalTok{diamonds, }\DataTypeTok{na.rm =} \OtherTok{TRUE}\NormalTok{)}
\KeywordTok{last_plot}\NormalTok{() +}\StringTok{ }\KeywordTok{xlim}\NormalTok{(}\DecValTok{3}\NormalTok{, }\DecValTok{11}\NormalTok{) +}\StringTok{ }\KeywordTok{ylim}\NormalTok{(}\DecValTok{3}\NormalTok{, }\DecValTok{11}\NormalTok{)}
\KeywordTok{last_plot}\NormalTok{() +}\StringTok{ }\KeywordTok{xlim}\NormalTok{(}\DecValTok{4}\NormalTok{, }\DecValTok{10}\NormalTok{) +}\StringTok{ }\KeywordTok{ylim}\NormalTok{(}\DecValTok{4}\NormalTok{, }\DecValTok{10}\NormalTok{)}
\KeywordTok{last_plot}\NormalTok{() +}\StringTok{ }\KeywordTok{xlim}\NormalTok{(}\DecValTok{4}\NormalTok{, }\DecValTok{5}\NormalTok{) +}\StringTok{ }\KeywordTok{ylim}\NormalTok{(}\DecValTok{4}\NormalTok{, }\DecValTok{5}\NormalTok{)}
\KeywordTok{last_plot}\NormalTok{() +}\StringTok{ }\KeywordTok{xlim}\NormalTok{(}\DecValTok{4}\NormalTok{, }\FloatTok{4.5}\NormalTok{) +}\StringTok{ }\KeywordTok{ylim}\NormalTok{(}\DecValTok{4}\NormalTok{, }\FloatTok{4.5}\NormalTok{)}
\KeywordTok{last_plot}\NormalTok{() +}\StringTok{ }\KeywordTok{geom_abline}\NormalTok{(}\DataTypeTok{colour =} \StringTok{"red"}\NormalTok{)}
\end{Highlighting}
\end{Shaded}

\begin{figure}

{\centering \includegraphics[width=0.32\linewidth]{figures/duplicationiterate-limits-1} \includegraphics[width=0.32\linewidth]{figures/duplicationiterate-limits-2} \includegraphics[width=0.32\linewidth]{figures/duplicationiterate-limits-3} \includegraphics[width=0.32\linewidth]{figures/duplicationiterate-limits-4} \includegraphics[width=0.32\linewidth]{figures/duplicationiterate-limits-5} \includegraphics[width=0.32\linewidth]{figures/duplicationiterate-limits-6} 

}

\caption{When 'zooming' in on the plot, it's useful to use \texttt{last\_plot()} iteratively to quickly find the best view. The final plot adds a line with slope 1 and intercept 0, confirming it is the square diamonds that are missing.\label{fig:iterate-limits}}
\end{figure}

Once you have tweaked the plot to your liking, it's a good idea to go
back and create a single expression that generates your final plot. This
is important as when you come back to the plot, you'll be able to
re-create the plot quickly, without having to step through your original
process. You may want to add a comment to your code to indicate exactly
why you chose that final plot. This is good practice in general for R
code: after experimenting interactively, you always want to create a
source file that re-creates your analysis. The following code shows the
final plot after our interactive modifications above.

\begin{Shaded}
\begin{Highlighting}[]
\KeywordTok{qplot}\NormalTok{(x, y, }\DataTypeTok{data =} \NormalTok{diamonds, }\DataTypeTok{na.rm =} \NormalTok{T) +}\StringTok{ }
\StringTok{  }\KeywordTok{geom_abline}\NormalTok{(}\DataTypeTok{colour =} \StringTok{"red"}\NormalTok{) +}
\StringTok{  }\KeywordTok{xlim}\NormalTok{(}\DecValTok{4}\NormalTok{, }\FloatTok{4.5}\NormalTok{) +}\StringTok{ }\KeywordTok{ylim}\NormalTok{(}\DecValTok{4}\NormalTok{, }\FloatTok{4.5}\NormalTok{)}
\end{Highlighting}
\end{Shaded}

\hyperdef{}{sec:templates}{\section{Plot
templates}\label{sec:templates}}

Each component of a ggplot plot is its own object and can be created,
stored and applied independently to a plot. This makes it possible to
create reusable components that can automate common tasks and helps to
offset the cost of typing the long function names. The following example
creates some colour scales and then applies them to plots. The results
are shown in Figure \ref{fig:gradient-rb}. \index{Templates}
\index{Duplication!templates}

\begin{Shaded}
\begin{Highlighting}[]
\NormalTok{gradient_rb <-}\StringTok{ }\KeywordTok{scale_colour_gradient}\NormalTok{(}\DataTypeTok{low =} \StringTok{"red"}\NormalTok{, }\DataTypeTok{high =} \StringTok{"blue"}\NormalTok{)}
\KeywordTok{qplot}\NormalTok{(cty, hwy, }\DataTypeTok{data =} \NormalTok{mpg, }\DataTypeTok{colour =} \NormalTok{displ) +}\StringTok{ }\NormalTok{gradient_rb}
\KeywordTok{qplot}\NormalTok{(bodywt, brainwt, }\DataTypeTok{data =} \NormalTok{msleep, }\DataTypeTok{colour =} \NormalTok{awake, }\DataTypeTok{log=}\StringTok{"xy"}\NormalTok{) +}
\StringTok{  }\NormalTok{gradient_rb}
\end{Highlighting}
\end{Shaded}

\begin{figure}

{\centering \includegraphics[width=0.49\linewidth]{figures/duplicationgradient-rb-1} \includegraphics[width=0.49\linewidth]{figures/duplicationgradient-rb-2} 

}

\caption{Saving a scale to a variable makes it easy to apply exactly the same scale to multiple plots.  You can do the same thing with layers and facets too.\label{fig:gradient-rb}}
\end{figure}

As well as saving single objects, you can also save vectors of ggplot
components. Adding a vector of components to a plot is equivalent to
adding each component of the vector in turn. The following example
creates two continuous scales that can be used to turn off the display
of axis labels and ticks. You only need to create these objects once and
you can apply them to many different plots, as shown in the code below
and Figure \ref{fig:quiet}. \index{Layers!reusing}

\begin{Shaded}
\begin{Highlighting}[]
\NormalTok{xquiet <-}\StringTok{ }\KeywordTok{scale_x_continuous}\NormalTok{(}\DataTypeTok{breaks =} \OtherTok{NULL}\NormalTok{)}
\NormalTok{yquiet <-}\StringTok{ }\KeywordTok{scale_y_continuous}\NormalTok{(}\DataTypeTok{breaks =} \OtherTok{NULL}\NormalTok{)}

\KeywordTok{qplot}\NormalTok{(mpg, wt, }\DataTypeTok{data =} \NormalTok{mtcars) +}\StringTok{ }\NormalTok{xquiet }
\KeywordTok{qplot}\NormalTok{(displ, cty, }\DataTypeTok{data =} \NormalTok{mpg) +}\StringTok{ }\NormalTok{xquiet +}\StringTok{ }\NormalTok{yquiet }
\end{Highlighting}
\end{Shaded}

\begin{figure}

{\centering \includegraphics[width=0.49\linewidth]{figures/duplicationquiet-1} \includegraphics[width=0.49\linewidth]{figures/duplicationquiet-2} 

}

\caption{Using 'quiet' x and y scales removes the labels and hides ticks and gridlines.\label{fig:quiet}}
\end{figure}

Similarly, it's easy to write simple functions that change the defaults
of a layer. For example, if you wanted to create a function that added
linear models to a plot, you could create a function like the one below.
The results are shown in Figure \ref{fig:geom-lm}.

\begin{Shaded}
\begin{Highlighting}[]
\NormalTok{geom_lm <-}\StringTok{ }\NormalTok{function(}\DataTypeTok{formula =} \NormalTok{y ~}\StringTok{ }\NormalTok{x) \{}
  \KeywordTok{geom_smooth}\NormalTok{(}\DataTypeTok{formula =} \NormalTok{formula, }\DataTypeTok{se =} \OtherTok{FALSE}\NormalTok{, }\DataTypeTok{method =} \StringTok{"lm"}\NormalTok{)}
\NormalTok{\}}
\KeywordTok{qplot}\NormalTok{(mpg, wt, }\DataTypeTok{data =} \NormalTok{mtcars) +}\StringTok{ }\KeywordTok{geom_lm}\NormalTok{()}
\KeywordTok{library}\NormalTok{(}\StringTok{"splines"}\NormalTok{)}
\KeywordTok{qplot}\NormalTok{(mpg, wt, }\DataTypeTok{data =} \NormalTok{mtcars) +}\StringTok{ }\KeywordTok{geom_lm}\NormalTok{(y ~}\StringTok{ }\KeywordTok{ns}\NormalTok{(x, }\DecValTok{3}\NormalTok{))}
\end{Highlighting}
\end{Shaded}

\begin{figure}

{\centering \includegraphics[width=0.49\linewidth]{figures/duplicationgeom-lm-1} \includegraphics[width=0.49\linewidth]{figures/duplicationgeom-lm-2} 

}

\caption{Creating a custom geom function saves typing when creating plots with similar (but not the same) components.\label{fig:geom-lm}}
\end{figure}

Depending on how complicated your function is, it might even return
multiple components in a vector. You can build up arbitrarily complex
plots this way, reducing duplication wherever you find it. If you want
to create a plot that combines together many different components in a
pre-specified way, you might need to write a function that produces the
entire plot. This is described in the next section.

\hyperdef{}{sec:functions}{\section{Plot
functions}\label{sec:functions}}

If you are using the same basic plot again and again with different
datasets or different parameters, it may be worthwhile to wrap up all
the different options into a single function. Maybe you need to perform
some data restructuring or transformation, or need to combine the data
with a predefined model. In that case you will need to write a function
that produces ggplot plots. It's hard to give advice on how to go about
this because there are so many different possible scenarios, but this
section aims to point out some important things to think about.
\index{Duplication!functions} \index{Functions that create plots}

\begin{itemize}
\itemsep1pt\parskip0pt\parsep0pt
\item
  Since you're creating the plot within the environment of a function,
  you need to be extra careful about supplying the data to
  \texttt{ggplot()} as a data frame, and you need to double check that
  you haven't accidentally referred to any function local variables in
  your aesthetic mappings.
\item
  If you want to allow the user to provide their own variables for
  aesthetic mappings, I'd suggest using \texttt{aes\_string()}. This
  function works just like \texttt{aes}, but uses strings rather than
  unevaluated expressions. \texttt{aes\_string("cty", colour = "hwy")}
  is equivalent to \texttt{aes(cty, colour = hwy)}. Strings are much
  easier to work with than expressions.
  \index{Mappings!creating programmatically} \indexf{aes_string}
\item
  As mentioned in \hyperref[cha:data]{data}, you want to separate your
  plotting code into a function that does any data transformations and
  manipulations and a function that creates the plot. Generally, your
  plotting function should do no data manipulation, just create a plot.
  The following example shows one way to create parallel coordinate plot
  function, wrapping up the code used in
  \hyperref[sub:molten-data]{parallel coordinates plot}.
  \index{Parallel coordinates plot}
\end{itemize}

\begin{Shaded}
\begin{Highlighting}[]
\NormalTok{>}\StringTok{ }\NormalTok{rescale01 <-}\StringTok{ }\NormalTok{function(x) \{}
\NormalTok{+}\StringTok{   }\NormalTok{rng <-}\StringTok{ }\KeywordTok{range}\NormalTok{(x, }\DataTypeTok{na.rm =} \OtherTok{TRUE}\NormalTok{)}
\NormalTok{+}\StringTok{   }\NormalTok{(x -}\StringTok{ }\NormalTok{rng[}\DecValTok{1}\NormalTok{]) /}\StringTok{ }\NormalTok{(rng[}\DecValTok{2}\NormalTok{] -}\StringTok{ }\NormalTok{rng[}\DecValTok{1}\NormalTok{])}
\NormalTok{+}\StringTok{ }\NormalTok{\}}
\NormalTok{>}\StringTok{ }\NormalTok{pcp_data <-}\StringTok{ }\NormalTok{function(df) \{}
\NormalTok{+}\StringTok{   }\NormalTok{numeric <-}\StringTok{ }\KeywordTok{sapply}\NormalTok{(df, is.numeric)}
\NormalTok{+}\StringTok{   }\CommentTok{# Rescale numeric columns}
\NormalTok{+}\StringTok{   }\NormalTok{df[numeric] <-}\StringTok{ }\KeywordTok{lapply}\NormalTok{(df[numeric], rescale01)}
\NormalTok{+}\StringTok{   }\CommentTok{# Add row identified}
\NormalTok{+}\StringTok{   }\NormalTok{df$.row <-}\StringTok{ }\KeywordTok{rownames}\NormalTok{(df)}
\NormalTok{+}\StringTok{   }\CommentTok{# Treat numerics as value (aka measure) variables }
\NormalTok{+}\StringTok{   }\NormalTok{dfg <-}\StringTok{ }\NormalTok{tidyr::}\KeywordTok{gather_}\NormalTok{(df, }\StringTok{"variable"}\NormalTok{, }\StringTok{"value"}\NormalTok{, }\KeywordTok{names}\NormalTok{(df)[numeric])}
\NormalTok{+}\StringTok{   }\CommentTok{# Add pcp to class of the data frame}
\NormalTok{+}\StringTok{   }\KeywordTok{class}\NormalTok{(dfg) <-}\StringTok{ }\KeywordTok{c}\NormalTok{(}\StringTok{"pcp"}\NormalTok{, }\KeywordTok{class}\NormalTok{(dfg))}
\NormalTok{+}\StringTok{   }\NormalTok{dfg}
\NormalTok{+}\StringTok{ }\NormalTok{\}}
\NormalTok{>}\StringTok{ }\NormalTok{pcp <-}\StringTok{ }\NormalTok{function(df, ...) \{}
\NormalTok{+}\StringTok{   }\NormalTok{df <-}\StringTok{ }\KeywordTok{pcp_data}\NormalTok{(df)}
\NormalTok{+}\StringTok{   }\KeywordTok{ggplot}\NormalTok{(df, }\KeywordTok{aes}\NormalTok{(variable, value)) +}\StringTok{ }\KeywordTok{geom_line}\NormalTok{(}\KeywordTok{aes}\NormalTok{(}\DataTypeTok{group =} \NormalTok{.row))}
\NormalTok{+}\StringTok{ }\NormalTok{\}}
\NormalTok{>}\StringTok{ }\KeywordTok{pcp}\NormalTok{(mpg)}
\end{Highlighting}
\end{Shaded}

\begin{flushleft}\includegraphics[width=0.75\linewidth]{figures/duplicationpcp_data-1} \end{flushleft}

\begin{Shaded}
\begin{Highlighting}[]
\NormalTok{>}\StringTok{ }\KeywordTok{pcp}\NormalTok{(mpg) +}\StringTok{ }\KeywordTok{aes}\NormalTok{(}\DataTypeTok{colour =} \NormalTok{drv)}
\end{Highlighting}
\end{Shaded}

\begin{flushleft}\includegraphics[width=0.75\linewidth]{figures/duplicationpcp_data-2} \end{flushleft}

The best example of this technique is \texttt{qplot()}, and if you're
interesting in writing your own functions I strongly recommend you have
a look at the source code for this function and step through it line by
line to see how it works. If you've made your way this far through the
book you should have a pretty good grasp of all the ggplot related code:
most of the complexity is R tricks to correctly interpret all of the
possible plot types.
