\chapter{Translating between different syntaxes}\label{cha:translating}

\section{Introduction}

\texttt{ggplot} does not exist in isolation, but is part of a long
history of graphical tools in R and elsewhere. This chapter describes
how to convert between \texttt{ggplot} commands and other plotting
systems:

\begin{itemize}
\itemsep1pt\parskip0pt\parsep0pt
\item
  Within \texttt{ggplot}, \hyperref[sec:qplot-ggplot]{from
  \texttt{qplot()} to \texttt{ggplot()}}.
\item
  \hyperref[sec:translate-base]{From base graphics}.
\item
  \hyperref[sec:translate-lattice]{From lattice graphics}.
\item
  \hyperref[sec:translate-gpl]{From \textbf{gpl}}.
\end{itemize}

Each section gives a general outline on how to convert between the
difference types, followed by a number of examples.

\hyperdef{}{sec:qplot-ggplot}{\section{Translating between qplot and
ggplot}\label{sec:qplot-ggplot}}

Within \texttt{ggplot}, there are two basic methods to create plots,
with \texttt{qplot()} and \texttt{ggplot()}. \texttt{qplot()} is
designed primarily for interactive use: it makes a number of assumptions
that speed most cases, but when designing multi-layered plots with
different data sources it can get in the way. This section describes
what those defaults are, and how they map to the fuller
\texttt{ggplot()} syntax.
\index{qplot!translating to ggplot@translating to \texttt{ggplot()}}
\index{Translating!from qplot} \indexf{qplot}
\index{ggplot!translating from qplot@translating from \texttt{qplot()}}

By default, \texttt{qplot()} assumes that you want a scatterplot, i.e.,
you want to use \texttt{geom\_point()}.

\begin{Shaded}
\begin{Highlighting}[]
\KeywordTok{qplot}\NormalTok{(x, y, }\DataTypeTok{data =} \NormalTok{data)}
\KeywordTok{ggplot}\NormalTok{(data, }\KeywordTok{aes}\NormalTok{(x, y)) +}\StringTok{ }\KeywordTok{geom_point}\NormalTok{()}
\end{Highlighting}
\end{Shaded}

\subsection{Aesthetics}

If you map additional aesthetics, these will be added to the defaults.
With \texttt{qplot()} there is no way to use different aesthetic
mappings (or data) in different layers.
\index{Aesthetics!translating from qplot}

\begin{Shaded}
\begin{Highlighting}[]
\KeywordTok{qplot}\NormalTok{(x, y, }\DataTypeTok{data =} \NormalTok{data, }\DataTypeTok{shape =} \NormalTok{shape, }\DataTypeTok{colour =} \NormalTok{colour)}
\KeywordTok{ggplot}\NormalTok{(data, }\KeywordTok{aes}\NormalTok{(x, y, }\DataTypeTok{shape =} \NormalTok{shape, }\DataTypeTok{colour =} \NormalTok{colour)) +}\StringTok{ }
\StringTok{  }\KeywordTok{geom_point}\NormalTok{()}
\end{Highlighting}
\end{Shaded}

Aesthetic parameters in \texttt{qplot()} always try to map the aesthetic
to a variable. If the argument is not a variable but a value,
effectively a new column is added to the original dataset with that
value. To set an aesthetic to a value and override the default
appearance, you surround the value with \texttt{I()} in
\texttt{qplot()}, or pass it as a parameter to the layer.
\hyperref[sub:setting-mapping]{Setting vs.~mapping} expands on the
differences between setting and mapping.

\begin{Shaded}
\begin{Highlighting}[]
\KeywordTok{qplot}\NormalTok{(x, y, }\DataTypeTok{data =} \NormalTok{data, }\DataTypeTok{colour =} \KeywordTok{I}\NormalTok{(}\StringTok{"red"}\NormalTok{))}
\KeywordTok{ggplot}\NormalTok{(data, }\KeywordTok{aes}\NormalTok{(x, y)) +}\StringTok{ }\KeywordTok{geom_point}\NormalTok{(}\DataTypeTok{colour =} \StringTok{"red"}\NormalTok{)}
\end{Highlighting}
\end{Shaded}

\subsection{Layers}

Changing the geom parameter changes the geom added to the plot:

\begin{Shaded}
\begin{Highlighting}[]
\KeywordTok{qplot}\NormalTok{(x, y, }\DataTypeTok{data =} \NormalTok{data, }\DataTypeTok{geom =} \StringTok{"line"}\NormalTok{)}
\KeywordTok{ggplot}\NormalTok{(data, }\KeywordTok{aes}\NormalTok{(x, y)) +}\StringTok{ }\KeywordTok{geom_line}\NormalTok{()}
\end{Highlighting}
\end{Shaded}

If a vector of multiple geom names is supplied to the geom argument,
each geom will be added in turn:

\begin{Shaded}
\begin{Highlighting}[]
\KeywordTok{qplot}\NormalTok{(x, y, }\DataTypeTok{data =} \NormalTok{data, }\DataTypeTok{geom =} \KeywordTok{c}\NormalTok{(}\StringTok{"point"}\NormalTok{, }\StringTok{"smooth"}\NormalTok{))}
\KeywordTok{ggplot}\NormalTok{(data, }\KeywordTok{aes}\NormalTok{(x, y)) +}\StringTok{ }\KeywordTok{geom_point}\NormalTok{() +}\StringTok{ }\KeywordTok{geom_smooth}\NormalTok{()}
\end{Highlighting}
\end{Shaded}

Unlike the rest of \texttt{ggplot}, stats and geoms are independent:

\begin{Shaded}
\begin{Highlighting}[]
\KeywordTok{qplot}\NormalTok{(x, y, }\DataTypeTok{data =} \NormalTok{data, }\DataTypeTok{stat =} \StringTok{"bin"}\NormalTok{)}
\KeywordTok{ggplot}\NormalTok{(data, }\KeywordTok{aes}\NormalTok{(x, y)) +}\StringTok{ }\KeywordTok{geom_point}\NormalTok{(}\DataTypeTok{stat =} \StringTok{"bin"}\NormalTok{)  }
\end{Highlighting}
\end{Shaded}

Any layer parameters will be passed on to all layers. Most layers will
ignore parameters that they don't need.

\begin{Shaded}
\begin{Highlighting}[]
\KeywordTok{qplot}\NormalTok{(x, y, }\DataTypeTok{data =} \NormalTok{data, }\DataTypeTok{geom =} \KeywordTok{c}\NormalTok{(}\StringTok{"point"}\NormalTok{, }\StringTok{"smooth"}\NormalTok{), }
  \DataTypeTok{method =} \StringTok{"lm"}\NormalTok{)}
\KeywordTok{ggplot}\NormalTok{(data, }\KeywordTok{aes}\NormalTok{(x, y)) +}\StringTok{ }
\StringTok{  }\KeywordTok{geom_point}\NormalTok{(}\DataTypeTok{method =} \StringTok{"lm"}\NormalTok{) +}\StringTok{ }\KeywordTok{geom_smooth}\NormalTok{(}\DataTypeTok{method =} \StringTok{"lm"}\NormalTok{)}
\end{Highlighting}
\end{Shaded}

\subsection{Scales and axes}

You can control basic properties of the x and y scales with the
\texttt{xlim}, \texttt{ylim}, \texttt{xlab} and \texttt{ylab} arguments:

\begin{Shaded}
\begin{Highlighting}[]
\KeywordTok{qplot}\NormalTok{(x, y, }\DataTypeTok{data =} \NormalTok{data, }\DataTypeTok{xlim =} \KeywordTok{c}\NormalTok{(}\DecValTok{1}\NormalTok{, }\DecValTok{5}\NormalTok{), }\DataTypeTok{xlab =} \StringTok{"my label"}\NormalTok{)}
\KeywordTok{ggplot}\NormalTok{(data, }\KeywordTok{aes}\NormalTok{(x, y)) +}\StringTok{ }\KeywordTok{geom_point}\NormalTok{() +}\StringTok{ }
\StringTok{  }\KeywordTok{scale_x_continuous}\NormalTok{(}\StringTok{"my label"}\NormalTok{, }\DataTypeTok{limits =} \KeywordTok{c}\NormalTok{(}\DecValTok{1}\NormalTok{, }\DecValTok{5}\NormalTok{))}

\KeywordTok{qplot}\NormalTok{(x, y, }\DataTypeTok{data =} \NormalTok{data, }\DataTypeTok{xlim =} \KeywordTok{c}\NormalTok{(}\DecValTok{1}\NormalTok{, }\DecValTok{5}\NormalTok{), }\DataTypeTok{ylim =} \KeywordTok{c}\NormalTok{(}\DecValTok{10}\NormalTok{, }\DecValTok{20}\NormalTok{))}
\KeywordTok{ggplot}\NormalTok{(data, }\KeywordTok{aes}\NormalTok{(x, y)) +}\StringTok{ }\KeywordTok{geom_point}\NormalTok{() +}\StringTok{ }
\StringTok{  }\KeywordTok{scale_x_continuous}\NormalTok{(}\DataTypeTok{limits =} \KeywordTok{c}\NormalTok{(}\DecValTok{1}\NormalTok{, }\DecValTok{5}\NormalTok{))}
  \KeywordTok{scale_y_continuous}\NormalTok{(}\DataTypeTok{limits =} \KeywordTok{c}\NormalTok{(}\DecValTok{10}\NormalTok{, }\DecValTok{20}\NormalTok{))}
\end{Highlighting}
\end{Shaded}

Like \texttt{plot()}, \texttt{qplot()} has a convenient way of log
transforming the axes. There are many other possible transformations
that are not accessible from within \texttt{qplot()} see
\hyperref[sub:scale-position]{position scales} for more details.

\begin{Shaded}
\begin{Highlighting}[]
\KeywordTok{qplot}\NormalTok{(x, y, }\DataTypeTok{data =} \NormalTok{data, }\DataTypeTok{log=}\StringTok{"xy"}\NormalTok{)}
\KeywordTok{ggplot}\NormalTok{(data, }\KeywordTok{aes}\NormalTok{(x, y)) +}\StringTok{ }\KeywordTok{geom_point}\NormalTok{() +}\StringTok{ }
\StringTok{  }\KeywordTok{scale_x_log10}\NormalTok{() +}\StringTok{ }\KeywordTok{scale_y_log10}\NormalTok{()}
\end{Highlighting}
\end{Shaded}

\subsection{Plot options}

\texttt{qplot()} recognises the same options as plot does, and converts
them to their \texttt{ggplot} equivalents.
\hyperref[sec:themeux5felements]{Theme elements and element functions}
lists all possible plot options and their effects.

\begin{Shaded}
\begin{Highlighting}[]
\KeywordTok{qplot}\NormalTok{(x, y, }\DataTypeTok{data =} \NormalTok{data, }\DataTypeTok{main=}\StringTok{"title"}\NormalTok{, }\DataTypeTok{asp =} \DecValTok{1}\NormalTok{)}
\KeywordTok{ggplot}\NormalTok{(data, }\KeywordTok{aes}\NormalTok{(x, y)) +}\StringTok{ }\KeywordTok{geom_point}\NormalTok{() +}\StringTok{ }
\StringTok{  }\KeywordTok{opts}\NormalTok{(}\DataTypeTok{title =} \StringTok{"title"}\NormalTok{, }\DataTypeTok{aspect.ratio =} \DecValTok{1}\NormalTok{)}
\end{Highlighting}
\end{Shaded}

\hyperdef{}{sec:translate-base}{\section{Base
graphics}\label{sec:translate-base}}

There are two types of graphics functions in base graphics, those that
draw complete graphics and those that add to existing graphics.
\index{Base graphics!translating from}
\index{Translating!from base graphics}

\subsection{High-level plotting commands}

\texttt{qplot()} has been designed to mimic \texttt{plot()}, and can do
the job of all other high-level plotting commands. There are only two
graph types from base graphics that cannot be replicated with
\texttt{ggplot}: \texttt{filled.contour()} and \texttt{persp()}

\begin{Shaded}
\begin{Highlighting}[]
\KeywordTok{plot}\NormalTok{(x, y);  }\KeywordTok{dotchart}\NormalTok{(x, y); }\KeywordTok{stripchart}\NormalTok{(x, y)}
\KeywordTok{qplot}\NormalTok{(x, y)}

\KeywordTok{plot}\NormalTok{(x, y, }\DataTypeTok{type =} \StringTok{"l"}\NormalTok{)}
\KeywordTok{qplot}\NormalTok{(x, y, }\DataTypeTok{geom =} \StringTok{"line"}\NormalTok{)}

\KeywordTok{plot}\NormalTok{(x, y, }\DataTypeTok{type =} \StringTok{"s"}\NormalTok{)}
\KeywordTok{qplot}\NormalTok{(x, y, }\DataTypeTok{geom =} \StringTok{"step"}\NormalTok{)}

\KeywordTok{plot}\NormalTok{(x, y, }\DataTypeTok{type =} \StringTok{"b"}\NormalTok{)}
\KeywordTok{qplot}\NormalTok{(x, y, }\DataTypeTok{geom =} \KeywordTok{c}\NormalTok{(}\StringTok{"point"}\NormalTok{, }\StringTok{"line"}\NormalTok{))}

\KeywordTok{boxplot}\NormalTok{(x, y)}
\KeywordTok{qplot}\NormalTok{(x, y, }\DataTypeTok{geom =} \StringTok{"boxplot"}\NormalTok{)}

\KeywordTok{hist}\NormalTok{(x)}
\KeywordTok{qplot}\NormalTok{(x, }\DataTypeTok{geom =} \StringTok{"histogram"}\NormalTok{)}

\KeywordTok{cdplot}\NormalTok{(x, y)}
\KeywordTok{qplot}\NormalTok{(x, }\DataTypeTok{fill =} \NormalTok{y, }\DataTypeTok{geom =} \StringTok{"density"}\NormalTok{, }\DataTypeTok{position =} \StringTok{"fill"}\NormalTok{)}

\KeywordTok{coplot}\NormalTok{(y ~}\StringTok{ }\NormalTok{x |}\StringTok{ }\NormalTok{a +}\StringTok{ }\NormalTok{b)}
\KeywordTok{qplot}\NormalTok{(x, y, }\DataTypeTok{facets =} \NormalTok{a ~}\StringTok{ }\NormalTok{b)}
\end{Highlighting}
\end{Shaded}

Many of the geoms are parameterised differently than base graphics. For
example, \texttt{hist()} is parameterised in terms of the number of
bins, while \texttt{geom\_histogram()} is parameterised in terms of the
width of each bin.

\begin{Shaded}
\begin{Highlighting}[]
\KeywordTok{hist}\NormalTok{(x, }\DataTypeTok{bins =} \DecValTok{100}\NormalTok{)}
\KeywordTok{qplot}\NormalTok{(x, }\DataTypeTok{geom =} \StringTok{"histogram"}\NormalTok{, }\DataTypeTok{binwidth =} \DecValTok{1}\NormalTok{)}
\end{Highlighting}
\end{Shaded}

\texttt{qplot()} often requires data in a slightly different format to
the base graphics functions. For example, the bar geom works with
untabulated data, not tabulated data like \texttt{barplot()}; the tile
and contour geoms expect data in a data frame, not a matrix like
\texttt{image()} and \texttt{contour()}.

\begin{Shaded}
\begin{Highlighting}[]
\KeywordTok{barplot}\NormalTok{(}\KeywordTok{table}\NormalTok{(x))}
\KeywordTok{qplot}\NormalTok{(x, }\DataTypeTok{geom =} \StringTok{"bar"}\NormalTok{)}

\KeywordTok{barplot}\NormalTok{(x)}
\KeywordTok{qplot}\NormalTok{(}\KeywordTok{names}\NormalTok{(x), x, }\DataTypeTok{geom =} \StringTok{"bar"}\NormalTok{, }\DataTypeTok{stat =} \StringTok{"identity"}\NormalTok{)}

\KeywordTok{image}\NormalTok{(x)}
\KeywordTok{qplot}\NormalTok{(X1, X2, }\DataTypeTok{data =} \KeywordTok{melt}\NormalTok{(x), }\DataTypeTok{geom =} \StringTok{"tile"}\NormalTok{, }\DataTypeTok{fill =} \NormalTok{value)}

\KeywordTok{contour}\NormalTok{(x)}
\KeywordTok{qplot}\NormalTok{(X1, X2, }\DataTypeTok{data =} \KeywordTok{melt}\NormalTok{(x), }\DataTypeTok{geom =} \StringTok{"contour"}\NormalTok{, }\DataTypeTok{fill =} \NormalTok{value)}
\end{Highlighting}
\end{Shaded}

Generally, the base graphics functions work with individual vectors, not
data frames like \texttt{ggplot}. \texttt{qplot()} will try to construct
a data frame if one is not specified, but it is not always possible. If
you get strange errors, you may need to create the data frame yourself.

\begin{Shaded}
\begin{Highlighting}[]
\KeywordTok{with}\NormalTok{(df, }\KeywordTok{plot}\NormalTok{(x, y))}
\KeywordTok{qplot}\NormalTok{(x, y, }\DataTypeTok{data =} \NormalTok{df)}
\end{Highlighting}
\end{Shaded}

By default, \texttt{qplot()} maps values to aesthetics with a scale. To
override this behaviour and set aesthetics, overriding the defaults, you
need to use \texttt{I()}.

\begin{Shaded}
\begin{Highlighting}[]
\KeywordTok{plot}\NormalTok{(x, y, }\DataTypeTok{col =} \StringTok{"red"}\NormalTok{, }\DataTypeTok{cex =} \DecValTok{1}\NormalTok{)}
\KeywordTok{qplot}\NormalTok{(x, y, }\DataTypeTok{colour =} \KeywordTok{I}\NormalTok{(}\StringTok{"red"}\NormalTok{), }\DataTypeTok{size =} \KeywordTok{I}\NormalTok{(}\DecValTok{1}\NormalTok{))}
\end{Highlighting}
\end{Shaded}

\subsection{Low-level drawing}

The low-level drawing functions which add to an existing plot are
equivalent to adding a new layer in \texttt{ggplot}, described in
Table\textasciitilde{}\ref{tbl:base-equiv}.

\begin{Shaded}
\begin{Highlighting}[]
\KeywordTok{plot}\NormalTok{(x, y)}
\KeywordTok{lines}\NormalTok{(x, y)}

\KeywordTok{qplot}\NormalTok{(x, y) +}\StringTok{ }\KeywordTok{geom_line}\NormalTok{()}

\CommentTok{# Or, building up piece-meal}
\KeywordTok{qplot}\NormalTok{(x, y)}
\KeywordTok{last_plot}\NormalTok{() +}\StringTok{ }\KeywordTok{geom_line}\NormalTok{()}
\end{Highlighting}
\end{Shaded}

\subsection{Legends, axes and grid lines}

In \texttt{ggplot}, the appearance of legends and axes is controlled by
the scales. Axes are produced by the x and y scales, while all other
scales produce legends. See \hyperref[sec:themes]{plot themes}, to
change the appearance of axes and legends, and,
\hyperref[sec:guides]{scales}, to change their contents. The appearance
of grid lines is controlled by the \texttt{grid.major} and
\texttt{grid.minor} theme options, and their position by the breaks of
the x and y scales.

\subsection{Colour palettes}

Instead of global colour palettes, \texttt{ggplot} has scales for
individual plots. Much of the time you can rely on the default colour
scale (which has somewhat better perceptual properties), but if you want
to reuse an existing colour palette, you can use
\texttt{scale\_colour\_manual()}. You will need to make sure that the
colour is a factor for this to work. \index{Colour!palettes}

\begin{Shaded}
\begin{Highlighting}[]
\KeywordTok{palette}\NormalTok{(}\KeywordTok{rainbow}\NormalTok{(}\DecValTok{5}\NormalTok{))}
\KeywordTok{plot}\NormalTok{(}\DecValTok{1}\NormalTok{:}\DecValTok{5}\NormalTok{, }\DecValTok{1}\NormalTok{:}\DecValTok{5}\NormalTok{, }\DataTypeTok{col =} \DecValTok{1}\NormalTok{:}\DecValTok{5}\NormalTok{, }\DataTypeTok{pch =} \DecValTok{19}\NormalTok{, }\DataTypeTok{cex =} \DecValTok{4}\NormalTok{)}

\KeywordTok{qplot}\NormalTok{(}\DecValTok{1}\NormalTok{:}\DecValTok{5}\NormalTok{, }\DecValTok{1}\NormalTok{:}\DecValTok{5}\NormalTok{, }\DataTypeTok{col =} \KeywordTok{factor}\NormalTok{(}\DecValTok{1}\NormalTok{:}\DecValTok{5}\NormalTok{), }\DataTypeTok{size =} \KeywordTok{I}\NormalTok{(}\DecValTok{4}\NormalTok{))}
\KeywordTok{last_plot}\NormalTok{() +}\StringTok{ }\KeywordTok{scale_colour_manual}\NormalTok{(}\DataTypeTok{values =} \KeywordTok{rainbow}\NormalTok{(}\DecValTok{5}\NormalTok{))}
\end{Highlighting}
\end{Shaded}

In \texttt{ggplot}, you can also use palettes with continuous values,
with intermediate values being linearly interpolated.

\begin{Shaded}
\begin{Highlighting}[]
\KeywordTok{qplot}\NormalTok{(}\DecValTok{0}\NormalTok{:}\DecValTok{100}\NormalTok{, }\DecValTok{0}\NormalTok{:}\DecValTok{100}\NormalTok{, }\DataTypeTok{col =} \DecValTok{0}\NormalTok{:}\DecValTok{100}\NormalTok{, }\DataTypeTok{size =} \KeywordTok{I}\NormalTok{(}\DecValTok{4}\NormalTok{)) +}
\StringTok{  }\KeywordTok{scale_colour_gradientn}\NormalTok{(}\DataTypeTok{colours =} \KeywordTok{rainbow}\NormalTok{(}\DecValTok{7}\NormalTok{))}
\KeywordTok{last_plot}\NormalTok{() +}
\StringTok{  }\KeywordTok{scale_colour_gradientn}\NormalTok{(}\DataTypeTok{colours =} \KeywordTok{terrain.colors}\NormalTok{(}\DecValTok{7}\NormalTok{))}
\end{Highlighting}
\end{Shaded}

\subsection{Graphical parameters}

The majority of \texttt{par} settings have some analogue within the
theme system, or in the defaults of the geoms and scales. The appearance
plot border drawn by \texttt{box()} can be controlled in a similar way
by the \texttt{panel.background} and \texttt{plot.background} theme
elements. Instead of using \texttt{title()}, the plot title is set with
the \texttt{title} option.

\hyperdef{}{sec:translate-lattice}{\section{Lattice
graphics}\label{sec:translate-lattice}}

The major difference between lattice and \texttt{ggplot} is that lattice
uses a formula-based interface. \texttt{ggplot} does not because the
formula does not generalise well to more complicated situations.
\index{Lattice graphics!translating from}
\index{Translating!from lattice}

\begin{Shaded}
\begin{Highlighting}[]
\KeywordTok{xyplot}\NormalTok{(rating ~}\StringTok{ }\NormalTok{year, }\DataTypeTok{data=}\NormalTok{movies)}
\KeywordTok{qplot}\NormalTok{(year, rating, }\DataTypeTok{data=}\NormalTok{movies)}

\KeywordTok{xyplot}\NormalTok{(rating ~}\StringTok{ }\NormalTok{year |}\StringTok{ }\NormalTok{Comedy +}\StringTok{ }\NormalTok{Action, }\DataTypeTok{data =} \NormalTok{movies)}
\KeywordTok{qplot}\NormalTok{(year, rating, }\DataTypeTok{data =} \NormalTok{movies, }\DataTypeTok{facets =} \NormalTok{~}\StringTok{ }\NormalTok{Comedy +}\StringTok{ }\NormalTok{Action)}
\CommentTok{# Or maybe}
\KeywordTok{qplot}\NormalTok{(year, rating, }\DataTypeTok{data =} \NormalTok{movies, }\DataTypeTok{facets =} \NormalTok{Comedy ~}\StringTok{ }\NormalTok{Action)}
\end{Highlighting}
\end{Shaded}

While lattice has many different functions to produce different types of
graphics (which are all basically equivalent to setting the panel
argument), \texttt{ggplot} has \texttt{qplot()}.

\begin{Shaded}
\begin{Highlighting}[]
\KeywordTok{stripplot}\NormalTok{(~}\StringTok{ }\NormalTok{rating, }\DataTypeTok{data =} \NormalTok{movies, }\DataTypeTok{jitter.data =} \OtherTok{TRUE}\NormalTok{)}
\KeywordTok{qplot}\NormalTok{(rating, }\DecValTok{1}\NormalTok{, }\DataTypeTok{data =} \NormalTok{movies, }\DataTypeTok{geom =} \StringTok{"jitter"}\NormalTok{)}

\KeywordTok{histogram}\NormalTok{(~}\StringTok{ }\NormalTok{rating, }\DataTypeTok{data =} \NormalTok{movies)}
\KeywordTok{qplot}\NormalTok{(rating, }\DataTypeTok{data =} \NormalTok{movies, }\DataTypeTok{geom =} \StringTok{"histogram"}\NormalTok{)}

\KeywordTok{bwplot}\NormalTok{(Comedy ~}\StringTok{ }\NormalTok{rating ,}\DataTypeTok{data =} \NormalTok{movies)}
\KeywordTok{qplot}\NormalTok{(}\KeywordTok{factor}\NormalTok{(Comedy), rating, }\DataTypeTok{data =} \NormalTok{movies, }\DataTypeTok{type =} \StringTok{"boxplot"}\NormalTok{)}

\KeywordTok{xyplot}\NormalTok{(wt ~}\StringTok{ }\NormalTok{mpg, mtcars, }\DataTypeTok{type =} \KeywordTok{c}\NormalTok{(}\StringTok{"p"}\NormalTok{,}\StringTok{"smooth"}\NormalTok{))}
\KeywordTok{qplot}\NormalTok{(mpg, wt, }\DataTypeTok{data =} \NormalTok{mtcars, }\DataTypeTok{geom =} \KeywordTok{c}\NormalTok{(}\StringTok{"point"}\NormalTok{,}\StringTok{"smooth"}\NormalTok{))}

\KeywordTok{xyplot}\NormalTok{(wt ~}\StringTok{ }\NormalTok{mpg, mtcars, }\DataTypeTok{type =} \KeywordTok{c}\NormalTok{(}\StringTok{"p"}\NormalTok{,}\StringTok{"r"}\NormalTok{))}
\KeywordTok{qplot}\NormalTok{(mpg, wt, }\DataTypeTok{data =} \NormalTok{mtcars, }\DataTypeTok{geom =} \KeywordTok{c}\NormalTok{(}\StringTok{"point"}\NormalTok{,}\StringTok{"smooth"}\NormalTok{),}
  \DataTypeTok{method =} \StringTok{"lm"}\NormalTok{)}
\end{Highlighting}
\end{Shaded}

The capabilities for scale manipulations are similar in both
\texttt{ggplot} and lattice, although the syntax is a little different.

\begin{Shaded}
\begin{Highlighting}[]
\KeywordTok{xyplot}\NormalTok{(wt ~}\StringTok{ }\NormalTok{mpg |}\StringTok{ }\NormalTok{cyl, mtcars, }\DataTypeTok{scales =} \KeywordTok{list}\NormalTok{(}\DataTypeTok{y =} \KeywordTok{list}\NormalTok{(}\DataTypeTok{relation =} \StringTok{"free"}\NormalTok{)))}
\KeywordTok{qplot}\NormalTok{(mpg, wt, }\DataTypeTok{data =} \NormalTok{mtcars) +}\StringTok{ }\KeywordTok{facet_wrap}\NormalTok{(~}\StringTok{ }\NormalTok{cyl, }\DataTypeTok{scales =} \StringTok{"free"}\NormalTok{)}

\KeywordTok{xyplot}\NormalTok{(wt ~}\StringTok{ }\NormalTok{mpg |}\StringTok{ }\NormalTok{cyl, mtcars, }\DataTypeTok{scales =} \KeywordTok{list}\NormalTok{(}\DataTypeTok{log =} \DecValTok{10}\NormalTok{))}
\KeywordTok{qplot}\NormalTok{(mpg, wt, }\DataTypeTok{data =} \NormalTok{mtcars, }\DataTypeTok{log =} \StringTok{"xy"}\NormalTok{) }

\KeywordTok{xyplot}\NormalTok{(wt ~}\StringTok{ }\NormalTok{mpg |}\StringTok{ }\NormalTok{cyl, mtcars, }\DataTypeTok{scales =} \KeywordTok{list}\NormalTok{(}\DataTypeTok{log =} \DecValTok{2}\NormalTok{))}
\KeywordTok{qplot}\NormalTok{(mpg, wt, }\DataTypeTok{data =} \NormalTok{mtcars) +}\StringTok{ }
\StringTok{  }\KeywordTok{scale_x_log2}\NormalTok{() +}\StringTok{ }\KeywordTok{scale_y_log2}\NormalTok{()}

\KeywordTok{xyplot}\NormalTok{(wt ~}\StringTok{ }\NormalTok{mpg, mtcars, }\DataTypeTok{group =} \NormalTok{cyl, }\DataTypeTok{auto.key =} \OtherTok{TRUE}\NormalTok{)}
\CommentTok{# Map directly to an aesthetic like colour, size, or shape.}
\KeywordTok{qplot}\NormalTok{(mpg, wt, }\DataTypeTok{data =} \NormalTok{mtcars, }\DataTypeTok{colour =} \NormalTok{cyl)}

\KeywordTok{xyplot}\NormalTok{(wt ~}\StringTok{ }\NormalTok{mpg, mtcars, }\DataTypeTok{xlim =} \KeywordTok{c}\NormalTok{(}\DecValTok{20}\NormalTok{,}\DecValTok{30}\NormalTok{))}
\CommentTok{# Works like lattice, except you can't specify a different limit }
\CommentTok{# for each panel/facet}
\KeywordTok{qplot}\NormalTok{(mpg, wt, }\DataTypeTok{data =} \NormalTok{mtcars, }\DataTypeTok{xlim =} \KeywordTok{c}\NormalTok{(}\DecValTok{20}\NormalTok{,}\DecValTok{30}\NormalTok{))}
\end{Highlighting}
\end{Shaded}

Both lattice and \texttt{ggplot} have similar options for controlling
labels on the plot.

\begin{Shaded}
\begin{Highlighting}[]
\KeywordTok{xyplot}\NormalTok{(wt ~}\StringTok{ }\NormalTok{mpg, mtcars, }
  \DataTypeTok{xlab =} \StringTok{"Miles per gallon"}\NormalTok{, }\DataTypeTok{ylab =} \StringTok{"Weight"}\NormalTok{, }
  \DataTypeTok{main =} \StringTok{"Weight-efficiency tradeoff"}\NormalTok{)}
\KeywordTok{qplot}\NormalTok{(mpg, wt, }\DataTypeTok{data =} \NormalTok{mtcars, }
  \DataTypeTok{xlab =} \StringTok{"Miles per gallon"}\NormalTok{, }\DataTypeTok{ylab =} \StringTok{"Weight"}\NormalTok{, }
  \DataTypeTok{main =} \StringTok{"Weight-efficiency tradeoff"}\NormalTok{)}

\KeywordTok{xyplot}\NormalTok{(wt ~}\StringTok{ }\NormalTok{mpg, mtcars, }\DataTypeTok{aspect =} \DecValTok{1}\NormalTok{) }
\KeywordTok{qplot}\NormalTok{(mpg, wt, }\DataTypeTok{data =} \NormalTok{mtcars, }\DataTypeTok{asp =} \DecValTok{1}\NormalTok{)}
\end{Highlighting}
\end{Shaded}

\texttt{par.settings()} is equivalent to \texttt{+ theme()} and
\texttt{trellis.options.set()} and \texttt{trellis.par.get()} to
\texttt{theme\_set()} and \texttt{theme\_get()}.

More complicated lattice formulas are equivalent to rearranging the data
before using \texttt{ggplot}.

\hyperdef{}{sec:translate-gpl}{\section{GPL}\label{sec:translate-gpl}}

The Grammar of Graphics uses two specifications. A concise format is
used to caption figures, and a more detailed xml format stored on disk.
The following example of the concise format is adapted from (Figure 1.5,
page 13 in Wilkinson 2005). \index{GPL!translating from}
\index{Translating!from GPL}

\begin{verbatim}
DATA: source("demographics")
DATA: longitude, latitude = map(source("World"))
TRANS: bd = max(birth - death, 0)
COORD: project.mercator()
ELEMENT: point(position(lon * lat), size(bd), color(color.red))
ELEMENT: polygon(position(longitude * latitude))
\end{verbatim}

This is relatively simple to adapt to the syntax of \texttt{ggplot}:

\begin{itemize}
\itemsep1pt\parskip0pt\parsep0pt
\item
  \texttt{ggplot()} is used to specify the default data and default
  aesthetic mappings.
\item
  Data is provided as standard R data.frames existing in the global
  environment; it does not need to be explicitly loaded. We also use a
  slightly different world dataset, with columns lat and long. This lets
  us use the same aesthetic mappings for both datasets. Layers can
  override the default data and aesthetic mappings provided by the plot.
\item
  We replace \texttt{TRANS} with an explicit transformation by R code.
\item
  \texttt{ELEMENT}s are replaced with layers, which explicitly specify
  the data source. Each geom has a default statistic which is used to
  transform the data prior to plotting. For the geoms in this example,
  the default statistic is the identity function. Fixed aesthetics (the
  colour red in this example) are supplied as additional arguments to
  the layer, rather than as special constants.
\item
  \texttt{SCALE} component has been omitted from this example (so that
  the defaults are used). In both the \texttt{ggplot} and GoG examples,
  scales are defined by default. In ggplot you can override the defaults
  by adding a scale object, e.g., \texttt{scale\_colour} or
  \texttt{scale\_size}.
\item
  \texttt{COORD} uses a slightly different format. In general, most of
  the components specifications in ggplot are slightly different to
  those in GoG, in order to be more familiar to R users.
\item
  Each component is added together with \(+\) to create the final plot.
\end{itemize}

All up the equivalent \texttt{ggplot} code is:

\begin{Shaded}
\begin{Highlighting}[]
\NormalTok{demographics <-}\StringTok{ }\KeywordTok{transform}\NormalTok{(demographics, }
  \DataTypeTok{bd =} \KeywordTok{pmax}\NormalTok{(birth -}\StringTok{ }\NormalTok{death, }\DecValTok{0}\NormalTok{))}

\KeywordTok{ggplot}\NormalTok{(demographic, }\KeywordTok{aes}\NormalTok{(lon, lat)) +}\StringTok{ }
\StringTok{  }\KeywordTok{geom_polyogon}\NormalTok{(}\DataTypeTok{data =} \NormalTok{world) +}
\StringTok{  }\KeywordTok{geom_point}\NormalTok{(}\KeywordTok{aes}\NormalTok{(}\DataTypeTok{size =} \NormalTok{bd), }\DataTypeTok{colour =} \StringTok{"red"}\NormalTok{) +}
\StringTok{  }\KeywordTok{coord_map}\NormalTok{(}\DataTypeTok{projection =} \StringTok{"mercator"}\NormalTok{)}
\end{Highlighting}
\end{Shaded}

Wilkinson, Leland. 2005. \emph{The Grammar of Graphics}. 2nd ed.
Statistics and Computing. Springer.
