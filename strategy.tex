\providecommand{\setflag}{\newif \ifwhole \wholefalse}
\setflag
\ifwhole\else

% Typography and geometry ----------------------------------------------------
\documentclass[letterpaper]{scrbook}
\usepackage[inner=3cm,top=2.5cm,outer=3.5cm]{geometry}

\renewcommand\familydefault{bch}
\usepackage[utf8]{inputenc}
\usepackage{microtype}
\usepackage[small]{caption}
\usepackage[small]{titlesec}
\raggedbottom

% Graphics -------------------------------------------------------------------
\usepackage[pdftex]{graphicx}
\graphicspath{{_include/}}
\DeclareGraphicsExtensions{.png,.pdf}

% Code formatting ------------------------------------------------------------
\usepackage{fancyvrb}
\usepackage{courier}
\usepackage{listings}
\usepackage{color}
\usepackage{alltt}


\definecolor{comment}{rgb}{0.60, 0.60, 0.53}
\definecolor{background}{rgb}{0.97, 0.97, 1.00}
\definecolor{string}{rgb}{0.863, 0.066, 0.266}
\definecolor{number}{rgb}{0.0, 0.6, 0.6}
\definecolor{variable}{rgb}{0.00, 0.52, 0.70}
\lstset{
  basicstyle=\ttfamily,
  keywordstyle=\bfseries, 
  identifierstyle=,  
  commentstyle=\color{comment} \emph,
  stringstyle=\color{string},
  showstringspaces=false,
  columns = fullflexible,
  backgroundcolor=\color{background},
  mathescape = true,
  escapeinside=&&,
  fancyvrb
}
\newcommand{\code}[1]{\lstinline!#1!}
\newcommand{\f}[1]{\lstinline!#1()!}



% Links ----------------------------------------------------------------------

\usepackage{hyperref}
\definecolor{slateblue}{rgb}{0.07,0.07,0.488}
\hypersetup{colorlinks=true,linkcolor=slateblue,anchorcolor=slateblue,citecolor=slateblue,filecolor=slateblue,urlcolor=slateblue,bookmarksnumbered=true,pdfview=FitB}
\usepackage{url}

% Tables ---------------------------------------------------------------------
\usepackage{longtable}
\usepackage{booktabs}

% Miscellaneous --------------------------------------------------------------
\usepackage{pdfsync}
\usepackage{appendix}

\usepackage[round,sort&compress,sectionbib]{natbib}
\bibliographystyle{plainnat}


\title{ggplot2}
\author{Hadley Wickham}

\begin{document}
\fi


\chapter{Strategies for using ggplot2 effectively}
\section{Introduction}


How to structure plot objects so that you can change them as flexibly as possible.

\section{Plot templates}
\label{sec:templates}

An important component of doing a data analysis is flexibility.  If the data changes, or you discover something that makes you rethink your basic assumptions, you need to be able to easily change all of the plots that you have produced.  ggplot2 has been designed with this flexibility in mind, and this section discusses some of the ways you can use ggplot2 to be as flexible as possible.

One of the main components that inhibits flexibility is duplication. If you have the same basic plot components repeated over and over again, you have to make the same change in many different places.  Just the thought of having to do that can be prohibitive!

Do no repeat yourself.

This section discusses three strategies for avoiding duplication:

\begin{itemize}
  \item Modifying the ``last'' plot

  \item Creating ``templates'' with lists of ggplot objects

  \item Writing functions that build plots (or modifying existing plots)

  \item Writing aesthetic mappings programmatically with {\tt aes\_string()}.
\end{itemize}


\section{Split-apply-combine: Mapping numeric summaries to plots}
\label{sec:split_apply_combine}

With facetting, ggplot2 makes it very easy to identical plots for different subsets of your data.  It may not be quite so easy to do the same with your numerical analysis tool, so this sections discusses one strategy to make this possible.  


\ifwhole
\else
  \nobibliography{/Users/hadley/documents/phd/references}
  \end{document}
\fi
