\providecommand{\setflag}{\newif \ifwhole \wholefalse}
\setflag
\ifwhole\else

% Typography and geometry ----------------------------------------------------
\documentclass[letterpaper]{scrbook}
\usepackage[inner=3cm,top=2.5cm,outer=3.5cm]{geometry}

\renewcommand\familydefault{bch}
\usepackage[utf8]{inputenc}
\usepackage{microtype}
\usepackage[small]{caption}
\usepackage[small]{titlesec}
\raggedbottom

% Graphics -------------------------------------------------------------------
\usepackage[pdftex]{graphicx}
\graphicspath{{_include/}}
\DeclareGraphicsExtensions{.png,.pdf}

% Code formatting ------------------------------------------------------------
\usepackage{fancyvrb}
\usepackage{courier}
\usepackage{listings}
\usepackage{color}
\usepackage{alltt}


\definecolor{comment}{rgb}{0.60, 0.60, 0.53}
\definecolor{background}{rgb}{0.97, 0.97, 1.00}
\definecolor{string}{rgb}{0.863, 0.066, 0.266}
\definecolor{number}{rgb}{0.0, 0.6, 0.6}
\definecolor{variable}{rgb}{0.00, 0.52, 0.70}
\lstset{
  basicstyle=\ttfamily,
  keywordstyle=\bfseries, 
  identifierstyle=,  
  commentstyle=\color{comment} \emph,
  stringstyle=\color{string},
  showstringspaces=false,
  columns = fullflexible,
  backgroundcolor=\color{background},
  mathescape = true,
  escapeinside=&&,
  fancyvrb
}
\newcommand{\code}[1]{\lstinline!#1!}
\newcommand{\f}[1]{\lstinline!#1()!}



% Links ----------------------------------------------------------------------

\usepackage{hyperref}
\definecolor{slateblue}{rgb}{0.07,0.07,0.488}
\hypersetup{colorlinks=true,linkcolor=slateblue,anchorcolor=slateblue,citecolor=slateblue,filecolor=slateblue,urlcolor=slateblue,bookmarksnumbered=true,pdfview=FitB}
\usepackage{url}

% Tables ---------------------------------------------------------------------
\usepackage{longtable}
\usepackage{booktabs}

% Miscellaneous --------------------------------------------------------------
\usepackage{pdfsync}
\usepackage{appendix}

\usepackage[round,sort&compress,sectionbib]{natbib}
\bibliographystyle{plainnat}


\title{ggplot2}
\author{Hadley Wickham}

\begin{document}
\fi


% SET_DEFAULTS
%   GG-WIDTH: 4  GG-HEIGHT: 4
%   TEX-WIDTH: 0.5\textwidth
%   INLINE: TRUE
%   CACHE: TRUE
% 

% END

\chapter{Build a plot layer by layer}
\label{cha:layers}

\section{Introduction}

% This is not really true: layering was barely mentioned in the 
% previous chapter, so it needs to be defined here.
% As described in the previous chapter, layers can use different data sets and aesthetic mappings.  When creating a plot with \f{qplot}, you can't access these differences - all layers use the same default dataset and aesthetic mappings.  This means that you can't create certain more complex plots with \f{qplot}.  This chapter outlines an alternative method: building up the plot layer by layer.  

Layering is the mechanism by which additional elements are added to a plot.  Each added element can come from a different dataset and have a different aesthetic mapping, allowing us to create plots that could not be generated using \f{qplot}, which permits only a single dataset and a single aesthetic mapping.

Section~\ref{sec:ggplot} tells you how to initialize a plot object. The plot object is not yet ready to be displayed until at least one layer is added, as as described in Section~\ref{sec:layers}. Sections~\ref{sec:data} and \ref{sec:aes} describe the data and aesthetic mappings in more detail, including more information about how layer settings override the plot defaults, the difference between setting values and mapping aesthetics, and the important group aesthetic.  Sections~\ref{sec:geom} and \ref{sec:stat} list the geoms and stats available in \ggplot.  Section~\ref{sec:pull-together} concludes by introducing you to some plotting techniques that take advantage of what you have learned in this chapter.

This chapter is mainly a technical description of how layers, geoms and statistics work - how you call and customise them.  The next chapter, the \ggplot ``toolbox'', describes how you can use different geoms and stats to do data analysis. 
% In practice, I'd recommend flicking between these two chapters so you can learn both how layers work, and how you can use them to achieve your goals.
% dfs: This isn't a very appealing instruction, because it doesn't
% give you user a workable reading strategy.  I'll make a note here and 
% plan to revisit this later.
These two chapters are companions, with this one explaining how layers work and the next one how you can use layers to achieve your graphical goals.

\section{Creating a plot}
\label{sec:ggplot}

To build up a plot layer by layer, we start by creating a plot object. When discussing \f{qplot}, we didn't note that it creates a plot object, but it does. It creates a plot object, adds layers, and shows the result, applying a lot of default values along the way. To initialize a plot object without any shortcuts, simply call \f{ggplot} with no arguments; assign the result to a variable in order to add to it afterward.  ggplot accepts two optional arguments: {\bf data} and aesthetic {\bf mapping}.  The data argument needs little explanation: It's the data frame that you want to visualise.  You are already familiar with aesthetic mappings from \f{qplot}, and the syntax here is quite similar, although you need to wrap the pairs of aesthetic attribute and variable name in an \f{aes}:

% LISTING
% 
% p <- ggplot(mtcars, aes(x = wt, y = mpg, colour = cyl))
\input{_include/95610ac07e2c3dc8f4174924b1accbf4.tex}
% END

These arguments set up defaults for the plot and can be omitted if you specify data and aesthetics when adding each layer. 

This initial plot object can not be rendered until we add at least one layer.

\section{Layers}
\label{sec:layers}

A minimal layer may do nothing more than specify a {\bf geom}, a way of visually representing the data.  The plot object we just initialized can be rendered after this addition, specifying a scatterplot:

% LISTING
% 
% p <- p + layer(geom = ``point'')  
\input{_include/7fbdecf9d60a89c8e1d65cea679364e3.tex}
% p
% END
\noindent Note the use of \code{+} to add the layer to the plot.  This layer uses the plot defaults for data and aesthetic mapping and it uses default values for two optional arguments: the statistical transformation (the stat) and the position adjustment.  (Position adjustments are described in detail in Chapter~\ref{cha:position}; in brief, they shift objects on the plot to avoid overplotting.)  A more fully specified layer can take any or all of these arguments:

% LISTING
% 
% layer(geom, geom_params, stat, stat_params, data, mapping, position)
\input{_include/96dbc1af8d31b91d69bba7717e65c4c5.tex}
% END

\noindent This more complicated layer calls for a histogram in ``steelblue'' and a bin width of 2:

% LISTING
% 
% p <- ggplot(mtcars, aes(x = mpg))
% p <- p + layer(
%   geom = ``histogram'', 
%   geom_params = list(fill = "steelblue"),
%   stat_params = list(binwidth = 2)
% )
% p
\input{_include/ac6fc8efec4aa770704443e5f90940d0.tex}
% END

% dfs: I don't know what a legal value of stat might be in the above command.  
% identity doesn't work, and neither does histogram.

\noindent  This layer specification is precise but verbose. We can simplify it by using shortcuts that rely on the fact that every geom is associated with a default statistic and position, and every statistic with a default geom.  This means that you only need to specify one of \code{stat} or \code{geom} to get a completely specified layer, with parameters passed on to the geom or stat as appropriate.  This expression generates the same layer as the full layer command above:

% LISTING
% 
% geom_histogram(binwidth = 2, fill = "steelblue")
\input{_include/c2d6deb696a3dfc1ea30ea9f72acc181.tex}
% END

\noindent All the shortcut functions have the same basic form, beginning with {\bf geom\_} or {\bf stat\_}:

% LISTING
% 
% geom_XXX(mapping, data, ..., geom, position)
% stat_XXX(mapping, data, ..., stat, position)
\input{_include/721310811b324a4322cf025af4c1d8e8.tex}
% END
\noindent Their common parameters define the components of the layer:

\begin{itemize}
	\item {\bf mapping} (optional): A set of aesthetic mappings, specified using the {\tt aes()} function and combined with the plot defaults as described in Section~\ref{sec:aes}.

	\item {\bf data} (optional): A data set which overrides the default plot data set.  It is most commonly omitted, in which case the layer will use the default plot data.
	
	\item {\bf ...}: Parameters for the geom or stat, such as bin width in the histogram or bandwidth for a loess smoother.  Any aesthetic that the geom recognises can also be specified as a parameter to the layer.  See Section~\ref{sub:setting-mapping}.
	
	\item {\bf geom} or {\bf stat} (optional):  You can override the default {\tt stat} for a {\tt geom}, or the default {\tt geom} for a {\tt stat}.  This is a text string containing the name of the geom to use.  Using the default will give you a standard plot; overriding the defaults allows you to achieve something more exotic, as described in Section~\ref{sub:new_plot_types}.
  
\end{itemize}

Simple layers can be used with \f{qplot}, without the option of changing the data or aesthetic mapping:

% LISTING
% 
% qplot(mtcars, aes(mpg, wt)) + geom_point()
% qplot(mtcars, aes(mpg, wt, colour = factor(cyl))) + geom_smooth()
\input{_include/04f14b0830ab649c4d86cbc25d0f7873.tex}
% END

You've seen that plot objects can be stored as variables. The summary function can be helpful for inspecting the structure of a plot without plotting it:

% INTERWEAVE
%
% p <- ggplot(data=mtcars, aes(mpg, wt))
% summary(p)
%
% p <- p + geom_point()
% summary(p)
\input{_include/97457577a2b4d03eb5bc5323d4186f45.tex}
% END

Layers can also be stored as variables, so that it is easy to write clean code that generates a family of related plots.  For example, a set of plots can be initialised using different data then enhanced identically with carefully constructed layers.

Note that the order of \code{data} and \code{mapping} arguments is switched between {\tt ggplot()} and the layer functions.  This should improve ease of use, because you almost always specify data for the plot, and almost always specify aesthetics ~--~ but {\em not} data ~--~ for the layers.  However, I suggest explicitly naming other arguments rather than relying on positional matching.  This makes the code more readable and it is the style followed in this book.

The following sections describe the data and aesthetic mappings in more detail, then go on to describe the available geoms and stats.

\section{Data}
\label{sec:data}

The restriction on the data is simple: It must be a data frame.  It is not necessary to specify a default dataset except when using faceting; faceting is a global operation (i.e., it works on all layers) and it needs to have some a dataset to add in any missing columns.  See Section~\ref{sub:missing_faceting_columns} for more details.  If the dataset is omitted, every layer must supply its own data.

It is important to remember that the data is stored in the plot object as a copy, not a reference.  If you want to update (or change) the data later on, you can replace the default dataset with \code{\%+\%}:

% LISTING
% 
% p <- ggplot(mtcars, aes(mpg, wt, colour = cyl)) + geom_point()
% p
% mtcars$cyl <- factor(mtcars$cyl)
% p %+% mtcars
\input{_include/54658d1c846136617768d3c96ab528df.tex}
% END

\noindent This can be useful if you need to produce the same plot for different datasets.  It's also an easy way to experiment with influential points or imputation schemes.

% HERE dfs

\section{Aesthetic mapping}
\label{sec:aes}

To describe the way that variables in the data are mapped to things that we can perceive on the plot (the ``aesthetics''), we use the {\tt aes} function.  The {\tt aes} function takes a list of aesthetic-variable pairs as follows:

% LISTING
% 
% aes(x = weight, y = height, colour = age)
\input{_include/eb4989dafa003f510f392a47c1965fad.tex}
% END

Here we are mapping x-position to weight, y-position to height and colour to age.  Alternatively, the first two arguments can be left without names, in which case they are assumed to be for the x and y variables.  (This matches the way that \f{qplot} is normally used.)

% LISTING
% 
% aes(weight, height, colour = age)
\input{_include/07c1baa7ae460b23eac9c318c27c52ea.tex}
% END

Things inside \f{aes} can be functions, but all variables must be contained inside the plot (or layer) data.  This is important because ggplot2 objects are entirely self-contained.  You can save one to disk and later plot it without needing anything else from that session.

\subsection{Plots and layers}
\label{sub:plots_and_layers}

Mappings are stored in two places: in the plot defaults and in individual layers.  The plot defaults can be set when the plot is created, or added on afterwards, with addition.

% INTERWEAVE
% 
% p <- ggplot()
% summary(p)
% summary(p + aes(wt, hp))
\input{_include/3b40dd4be601fd761990321914a6fe69.tex}
% END

When you add a layer to a plot, you can just use the default mapping: 

% INTERWEAVE
% 
% p <- ggplot(mtcars, aes(x = mpg, y = wt))
% p + geom_point()
\input{_include/d6c2b4f5c1243e1f710b36393c8a600d.tex}
% END

Or you can override the defaults:

% INTERWEAVE
% 
% p + geom_point(aes(colour = factor(cyl)))
% p + geom_point(aes(y = disp))
\input{_include/2aa96cf95624773a83ea59e29db05a12.tex}
% END

\noindent The rules are summarised in Table~\ref{tbl:aes-override}.  Note that you are overriding the aesthetics only in that layer and unless you specify otherwise the axis and legend names will come from the default assignments. 

\begin{table}
  \begin{center}
  \begin{tabular}{lll}
    \toprule
    Operation & Layer aesthetics  & Result \\
    \midrule
    Add       & colour = cyl & x = mpg, y = wt, colour = cyl \\
    Override  & y = disp     & x = mpg, y = disp \\
    Delete    & y = NULL     & x = mpg \\
    \bottomrule
  \end{tabular}
  \end{center}
  \caption{Rules for combining layer aesthetic mapping with default mapping of \code{aes(x = mpg, y= wt)}:  additional aesthetics are added, existing overridden, and {\tt NULL} removes.}
  \label{tbl:aes-override}
\end{table}


\subsection{Setting vs. mapping}
\label{sub:setting-mapping}

For every aesthetic the geom function understands, you can also set that aesthetic as an parameter to the function.  Aesthetics can vary for each observation being plotted, while parameters can not.  For example:

% LISTING
% 
% p <- ggplot()
% p + geom_point(colour="darkblue")  
\input{_include/04edc955ceda9562289e6cb0ba8ca099.tex}
% END

\noindent will set the line colour to be dark blue instead of black.  This is quite different to:

% LISTING
% 
% p + geom_point(aes(colour="darkblue"))
\input{_include/7039aa7d66af82abd606f93467febfce.tex}
% END

\noindent which effectively creates a new column with the value ``red'' repeated according to R's vector recycling rules and then uses the default colour scale to map that value to a colour. Chapter~\ref{cha:specifications} describes how values should be specified for the various aesthetics.  With qplot, you can do the same thing by putting the value inside of \f{I}, e.g.\ \code{colour = I("darkblue")}.  The difference between setting and mapping is illustrated in Figure~\ref{fig:qplot-setting}.  

% FIGURE
%   LABEL: qplot-setting
%   CAPTION: The difference between \leftc setting colour to \code{"darkblue"}
%   and \rightc mapping colour to \code{"darkblue"}.  When \code{"darkblue"}
%   is mapped to colour, it is treated as a regular value and scaled with
%   the default colour scale.  This results in pinkish points and a legend.
% 
% qplot(mpg, wt, data=mtcars, colour = I("darkblue"))
% qplot(mpg, wt, data=mtcars, colour = "darkblue")
\input{_include/084af3bdd6bf65c66906b38eeb7ff755.tex}
% END

\subsection{Grouping}
\label{sub:grouping}

The {\tt group} aesthetic divides the the data set into discrete components.   This is used by the line geom to determine which observations to connect; by the boxplot geom to determine which points to summarise in one box; and by the smooth geom to determine which group of points should be included in the smooth. By default, the group aesthetic is set to the combination (interaction) of all discrete variables used in the plot.  Generally, this will create the correct separation of the data, but sometimes you will need to override it.  There are three common cases where this occurs:

\begin{itemize}
  \item You want to separate your data into groups, but not distinguish them with different aesthetics.  This is common for longitudinal studies with many subjects.  When looking at the data in aggregate you want to be able to distinguish individual subjects, but you don't need to be able to identify them.
  
  \item You want add summaries to different levels of the data. Here, different layers might have different grouping aesthetics, so that some display the full data, while others display summaries of larger groups.

  \item You have a discrete x scale, but want lines to connect horizontally across groups.  This sort of plot includes interaction plots, profile plots and parallel coordinate plots.  
  
\end{itemize}

To illustrate these ideas, we will look at a simple longitudinal data set, \code{Oxboys} from the \pkg{nlme} package.  This records the heights (\var{height}) and ages (\var{age}) of 26 boys (\var{Subject}) measured at nine different occasions (\var{Occasion}). Figure~\ref{fig:group-lines} shows a time series plot for each boy.  This requires us to identify the observations that should be connected with {\tt aes(group = Subject)}.  Figure~\ref{fig:group-smoothes} adds a smooth line to this plot.  If we use the same grouping as the line geom, we will get a smooth for each boy.  This is probably not what we want, so we override the group aesthetic for this layer to get a single line: {\tt aes(group = 1)}.  Figure~\ref{fig:group-boxplots} explores the distribution of heights at each occasion of measurement.  The default grouping works here because occasion is a discrete variable.  If we overlay individual trajectories we again need to override the default grouping for that layer with {\tt aes(group = Subject)}.

% FIGURE
%   LABEL: group-lines
%   CAPTION: \leftc Correctly specifying {\tt group = Subject} produces
%   one line per subject.  \rightc This pattern is characteristic of an 
%   incorrect grouping aesthetic.
% 
% data(Oxboys, package="nlme")
% qplot(age, height, data=Oxboys, group = Subject, geom="line")
% qplot(age, height, data=Oxboys, geom="line")
\input{_include/f8049cf110638c127875c39c76d31b3c.tex}
% END

% FIGURE
%   LABEL: group-smoothes
%   CAPTION: Adding smoothes to the Oxboys data.  \leftc Using the 
%   same grouping as the lines results in a line of best fit for 
%   each boy.  \rightc Using {\tt aes(group = 1)} in the smooth layer
%   fits a single line of best fit across all boys.
% 
% qplot(age, height, data=Oxboys, group = Subject, geom="line") + geom_smooth(method="lm")
% qplot(age, height, data=Oxboys, group = Subject, geom="line") + geom_smooth(aes(group = 1), method="lm")
\input{_include/881ecc26d3a6162106bded6eff182d9b.tex}
% END
 
% FIGURE
%   LABEL: group-boxplots
%   CAPTION: \leftc If boxplots are used to look at the distribution of
%   heights at each occasion (a discrete variable), the default grouping 
%   works correctly.  \rightc If trajectories of individual boys are overlayed
%   with {\tt geom\_line()} then {\tt aes(group = Subject)} must be set 
%   for that layer.
% 
% qplot(Occasion, height, data=Oxboys, geom="boxplot")
% qplot(Occasion, height, data=Oxboys, geom="boxplot", colour=I("#3366FF"), size= I(1)) + geom_line(aes(group = Subject))
\input{_include/3558728378a305acefcdd9d27470d8e6.tex}
% END

The {\tt interaction()} function is particularly useful if there isn't a pre-existing variable that separates the groups you are interested in, but a combination of variables does.  

\section{Geoms}
\label{sec:geom}

Geoms, or geometric elements, perform the actual rendering of the plot. Geoms are also responsible for drawing legends, as explained in Section~\ref{sec:legends_and_axes}.  Table~\ref{tbl:geoms} lists all of the geoms available in ggplot.  

\begin{table}
  \begin{center}
  \begin{tabular}{lp{3in}}
      \toprule
      Name & Description \\
      \midrule
      abline       & Line, specified by slope and intercept                                       \\
      area         & Area plots                                                                   \\
      bar          & Bars, rectangles with bases on y-axis                                        \\
      blank        & Blank, draws nothing                                                         \\
      boxplot      & Box and whiskers plot                                                        \\
      contour      & Display contours of a 3d surface in 2d                                       \\
      crossbar     & Hollow bar with middle indicated by horizontal line                          \\
      density      & Display a smooth density estimate                                            \\
      density\_2d & Contours from a 2d density estimate                                          \\
      errorbar     & Error bars                                                                   \\
      histogram    & Histogram                                                                    \\
      hline        & Line, horizontal                                                             \\
      interval     & Base for all interval (range) geoms                                          \\
      jitter       & Points, jittered to reduce overplotting                                      \\
      line         & Connect observations, in ordered by x value                                  \\
      linerange    & An interval represented by a vertical line                                   \\
      path         & Connect observations, in original order                                      \\
      point        & Points, as for a scatterplot                                                 \\
      pointrange   & An interval represented by a vertical line, with a point in the middle       \\
      polygon      & Polygon, a filled path                                                       \\
      quantile     & Add quantile lines from a quantile regression                                \\
      ribbon       & Ribbons, y range with continuous x values                                    \\
      rug          & Marginal rug plots                                                           \\
      segment      & Single line segments                                                         \\
      smooth       & Add a smoothed condition mean.                                               \\
      step         & Connect observations by stairs                                               \\
      text         & Textual annotations                                                          \\
      tile         & Tile plot as densely as possible, assuming that every tile is the same size. \\
      vline        & Line, vertical                                                               \\
  
      \bottomrule
  \end{tabular}
  \end{center}
  \caption{Geoms in \ggplot}
  \label{tbl:geoms}
\end{table}


\section{Stat}
\label{sec:stat}

Statistical transformations, stats, summarise the data in some way.  All available stats are listed in Table~\ref{tbl:stats}. 

\begin{table}
  \begin{center}
  \begin{tabular}{lp{3in}}
    \toprule
    Name & Description \\
    \midrule
    bin          & Bin data                                                   \\
    boxplot      & Calculate components of box and whisker plot               \\
    contour      & Contours of 3d data                                        \\
    density      & Density estimation, 1D                                     \\
    density\_2d & Density estimation, 2D                                     \\
    function     & Superimpose a function                                     \\
    identity     & Don't transform data                                       \\
    qq           & Calculation for quantile-quantile plot                     \\
    quantile     & Continuous quantiles                                       \\
    smooth       & Add a smoother                                             \\
    spoke        & Convert angle and radius to xend and yend                  \\
    step         & Create stair steps                                         \\
    sum          & Sum unique values.  Useful for overplotting on scatterplots\\
    summary      & Summarise y values at every unique x                       \\
    unique       & Remove duplicates                                          \\
    \bottomrule
  \end{tabular}
  \end{center}
  \caption{Stats in \ggplot}
  \label{tbl:stats}
\end{table}

Each stat generate a number of output variables that can be used in aesthetic mappings.  For example, \code{stat_bin}, the statistic used to make histogram produces the following variables:

\begin{itemize}
  \item \var{count}, the number of observations in that bin
  \item \var{density}, the density of observations in that bin (percent of total /  bar width)
  \item \var{x}, the centre of the bin
\end{itemize}

These generated variables can be used instead of the variables present in the original data set.  For example, by the default histogram geom assigns the height of the bars to the number of observations (\var{count}), but if you'd prefer a more traditional histogram you can use the density instead (\var{density}).  To distinguish between generated and variables, generate variables names must be surround with {\tt ..} when used.  The following example shows a density histogram of \var{carat} from the diamonds dataset.

% INTERWEAVE
%   GG-WIDTH: 8
%   TEX-WIDTH: \textwidth
% 
% qplot(carat, ..density.., data = diamonds, geom="histogram", binwidth = .1)
\input{_include/af356e585ad6aaa7244c432197d37d2d.tex}
% END

\section{Pulling it all together}
\label{sec:pull-together}



By creatively combining different layers, you can create intricate and useful graphics.  The following examples demonstrate some of the ways to use the capabilities of layers that you have learned about in this chapter.  These are just to get you started - you are only limited by your imagination!

\subsection{Combining geoms and stats}
\label{sub:new_plot_types}

By connecting geoms with different statistics, you can easily create new types of graphics.  Figure~\ref{fig:hist-variations} shows three variations on a histogram.  They all use the same statistical transformation underlying a histogram, but use different geoms and aesthetics to display the results: the area geom, the point geom and the tile geom.  

% FIGURE
%   LABEL: hist-variations
%   TEX-WIDTH: 0.33\textwidth
%   CAPTION: Three variations on the histogram. \leftc A frequency polygon,
%   \middlec a scatterplot with both size and height mapped to frequency, 
%   \rightc an heatmap representing frequency with colour.
% 
% d <- ggplot(diamonds, aes(carat)) + xlim(0, 3)
% d + stat_bin(binwidth = 0.1, geom = "area")
% d + stat_bin(aes(size = ..density..), binwidth = 0.1, geom = "point")  + scale_area()
% d + stat_bin(aes(y=1, fill = ..count..), binwidth = 0.1, 
%  geom = "tile") + scale_y_continuous("")
\input{_include/7b01d524eb417b55fb82e2a555ae6d2a.tex}
% END

In fact, a number of geoms are created this way.  They are aren't really geoms in their own right, but are created from another geom by changing a few differences.  Table~\ref{tbl:aliased-geoms} lists these ``aliased'' geoms that have been created so that you can easily refer to common plot types that are combination other geoms and stats.  Many of these geoms are used in a different manner to the base geom, and so the examples may be quite different, another reason for creating a separate geom.  

\begin{table}
  \begin{center}
  \begin{tabular}{lll}
    \toprule
    Aliased geom & Base geom & Changes in default \\
    \midrule
    area      & ribbon & \verb!aes(min = 0, max = y), position = "stack"!  \\
    density   & area   & \verb!stat = "density"!    \\
    histogram & bar    & \verb!stat = "bin"!        \\
    jitter    & point  & \verb!position = "jitter"! \\
    quantile  & line   & \verb!stat = "quantile"!   \\
    smooth    & ribbon & \verb!stat = "smooth"!     \\
    \bottomrule
  \end{tabular}
  \end{center}
  \caption{Geoms that modify the defaults of another geom.}
  \label{tbl:aliased-geoms}
\end{table}

\subsection{Varying aesthetics and data}
\label{sub:different_aesthetics}

One of the more powerful capabilities of \ggplot is the ability to plot different data sets on different layers.  This may seem strange: why would you want to plot different data on the same plot?  In practice, you often have related data sets that should be shown together.  A very common example is supplementing the data with predictions from a model.  While the smooth geom can add a wide range of different smoothes to your plot, it is no substitute for a external quantitative model that summarises your understanding of the data.

Let's look again at the \code{oxboys} dataset which we displayed in Section~\ref{sub:grouping}.  There we showed linear fits to individual boys, and to the whole group.  Neither model is particularly appropriate: the group model ignores the within subject correlation, and the individual model doesn't use information about the typical growth pattern to more accurate predict individuals.  In practice we might use a linear mixed effects model to do better.  This section explores how we can combine the output from this more sophisticated model with the original data to gain more insight into both the data and the model.

First we'll load the \pkg{nlme} package, and fit a model with varying intercepts and slopes.  (Exploring the fit of individual models shows that this is a reasonable first pass).  We'll also create a plot that we'll use as a template.  When rendered this will produce the first plot in Figure~\ref{fig:qplot-setting}, but we're not going to use in its default state.

% INTERWEAVE
% 
% library(nlme)
% model <- lme(height ~ age, data = Oxboys, random = ~ 1 + age | Subject)
% oplot <- qplot(age, height, data=Oxboys, group = Subject, geom="line")
\input{_include/b0535fe22fae9720461cd7c3c3962c06.tex}
% END

Next we'll compare the predicted trajectories to the actual trajectories.  We do this by building up a grid that contains all combinations of ages and subjects that we're interested in.  This is probably overkill for this simple linear case (we were only really need two values of age to draw the predicted straight line), but is important for more complex models.  Next we add the predictions from the model back into this dataset, as a variable called \var{height}.  

% INTERWEAVE
% 
% age_grid <- seq(-1, 1, length = 10)
% subjects <- unique(Oxboys$Subject)
% 
% preds <- expand.grid(age = age_grid, Subject = subjects)
% preds$height <- predict(model, preds)
\input{_include/a031713a61cfb27bad192c7267f63db4.tex}
% END

Once we have the predictions we can then displays them with the original data.  Because we have used the same variable names as the original \code{Oxboys} dataset, we don't need to specify any aesthetics, we can just override the default dataset.  We also set two aesthetic parameters to make it a bit easier to compare the predictions to the actual values.

% INTERWEAVE
% 
% oplot + geom_line(data = preds, colour = "#3366FF", size= 0.4)
\input{_include/1862dbb5797260c68eb6b0842280bc95.tex}
% END

It looks like the model does a good job of capturing the high-level structure of the data, but it's hard to see the details (these plots of longitudinal data are often called spaghetti plots for good reason).  Another way to compare the model to the data is to look at residuals, so we'll add the predictions from the model back on to the original data (\var{fitted}), and then calculate residuals (\var{resid}).  The plot is a little more complicated: we update the plot dataset, then change the default y aesthetic to \var{resid}, and then add a smooth line for all observations.

% INTERWEAVE
% 
% Oxboys$fitted <- predict(model)
% Oxboys$resid <- with(Oxboys, fitted - height)
% 
% oplot %+% Oxboys + aes(y = resid) + geom_smooth(aes(group=1))
\input{_include/30c822351d7624c395ef4d286ceabe47.tex}
% END

This shows a clear deficiency in our model, so we refit it with a quadratic term, recalculate predictions and residuals, and replot.  There now much less evidence of model inadequacy.

% INTERWEAVE
% 
% model2 <- update(model, height ~ age + I(age ^ 2))
% Oxboys$fitted2 <- predict(model2)
% Oxboys$resid2 <- with(Oxboys, fitted2 - height)
% 
% oplot %+% Oxboys + aes(y = resid2) + geom_smooth(aes(group=1))
\input{_include/845d84406f7447cd0c76ee4d74c9885b.tex}
% END


\ifwhole
\else
  \nobibliography{/Users/hadley/documents/phd/references}
  \end{document}
\fi
