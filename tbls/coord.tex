\begin{table}[ht]
\centering
\begin{tabular}{ll}
  \hline
Name & Description \\ 
  \hline
cartesian & The Cartesian coordinate system is the most familiar, and common, type of coordinate system. Setting limits on the coordinate system will zoom the plot (like you're looking at it with a magnifying glass), and will not change the underlying data like setting limits on a scale will. \\ 
  fixed & A fixed scale coordinate system forces a specified ratio between the physical representation of data units on the axes. The ratio represents the number of units on the y-axis equivalent to one unit on the x-axis. The default, ratio = 1, ensures that one unit on the x-axis is the same length as one unit on the y-axis. Ratios higher than one make units on the y axis longer than units on the x-axis, and vice versa. This is similar to eqscplot, but it works for all types of graphics. \\ 
  flip & Flipped cartesian coordinates so that horizontal becomes vertical, and vertical, horizontal. This is primarily useful for converting geoms and statistics which display y conditional on x, to x conditional on y. \\ 
  map & This coordinate system provides the full range of map projections available in the mapproj package. \\ 
  polar & The polar coordinate system is most commonly used for pie charts, which are a stacked bar chart in polar coordinates. \\ 
  trans & coord\_trans is different to scale transformations in that it occurs after statistical transformation and will affect the visual appearance of geoms - there is no guarantee that straight lines will continue to be straight. \\ 
   \hline
\end{tabular}
\caption{Coordinate systems available in ggplot. 
\texttt{coord_equal()}, \texttt{coord_flip()} and \texttt{coord_trans()} 
are all basically Cartesian in nature (i.e., the dimensions combine orthogonally), 
while \texttt{coord_map()} and \texttt{coord_polar()} are more complex.} 
\label{coord}
\end{table}
