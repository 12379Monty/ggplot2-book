\documentclass[oneside,letterpaper]{scrartcl}
\usepackage{fullpage}
\usepackage[utf8]{inputenc}
\usepackage[pdftex]{graphicx}
\usepackage{hyperref}
\usepackage{minitoc}
\usepackage{pdfsync}
\usepackage{alltt}
\usepackage{url}
\usepackage[round,sort&compress,sectionbib]{natbib}
\bibliographystyle{plainnat}

%\setcounter{secnumdepth}{-1}

\title{useR series book proposoal: ggplot}
\author{Hadley Wickham \\
\normalsize{2519 Chamberlain Street, Apt 314} \\
\normalsize{Ames IA 50014}
}

\begin{document}
\maketitle

Possible titles (depending on the marketing of the book):

\begin{itemize}
  \item ggplot: powerful, user-friendly graphics with R
  \item ggplot: an implementation of the grammar of graphics
\end{itemize}

\section{Description}\label{sec:description}

% Description: What is the subject of your book? What are your goals in
% writing the book? For whom is it intended? Please be specific about the
% prerequisites; can you name one or two books that provide the necessary
% background?

The primary goals of the book are to help the reader:

\begin{itemize}
  \item learn how to create graphics in R using ggplot, and
  \item learn about The Grammar of Graphics \citep{wilkinson:2006}, the theoretical framework that underlies ggplot
\end{itemize}

\noindent Secondary goals:

\begin{itemize}
  \item learn a little a bit about what graphs to produce
\end{itemize}

The primary audience is people who are dissatisfied with the current plotting frameworks available in R.  The current frameworks are fundamentally limited because they lack a firm underlying model, which also makes them hard to learn: everything is a special case.  ggplot should be much easier to learn as it has been designed to be consistent from the ground up.  I also see a second audience who will be interested in the book: those who have read the Grammar of Graphics, but were frustrated that they could not try it out.  

The book teaches a new plotting system, using a framework that people are aware of but not very familiar with.  For this reason, the book is aimed at an introductory level and is largely self-contained.  However, some basic R skills, to the level described in \citet{dalgaard:2002}, will be necessary to get the most out of it.

ggplot is currently used in two courses  at Iowa State, Stat480 and Stat503.  Di Cook and I will be offering a GGobi and ggplot course in Salt Lake City, July 2007, and I have been invited to give a three day course at Institute for Environmental Sciences, University of Zürich in Switzerland.
 
ggplot has received very positive attention from the statistics community. ggplot (along with the reshape package) won the 2006 John Chambers award and I have also received a number of very positive emails about ggplot.  A selection of these are listed below, along with the affiliations of the author, if known.

\begin{itemize}

	\item ``The combination of ggplot and grid is really the best graphics stuff I have worked with. Amazing.'' Chair, University of Potsdam

	\item ``Wow''.  Chair, Vanderbilt University

	\item ``I am a long time R user and have become very interested in ggplot and rggobi (and reshaping looks good too!).''.  Assistant Professor, Miami University.

	\item ``Great. Thanks very much. I've been using ggplot a bit--it's a good tool to have.'' Research Professor, University of Maryland

	\item ``I'm a huge fan of both reshape and ggplot.  I think they're very well designed and extremely useful.''

	\item ``This is so good!! And I am really grateful. I can now tweak this in many different ways to get all that I want. Thank you so much.''

	\item ``I have recently started using your ggplot package, and I like it very much.''

	\item ``I've just started using ggplot and love the way the graphics look.''

	\item ``I've been playing around with ggplot() for a little more than a week now.  I really like the ease of use, like being able to create a plot with points, colored by some grouping variable, and fitting a curve with SE bars.  I also really like being able to add to the graph after it's been drawn!  Nice features.''.  Assistant Director, GlaxoSmithKline

	\item ``It's extremely handy and I am very appreciative of the presence of ggplot in R.  I have to say that I have often struggled to do this sort of thing with lattice and it is much quicker with ggplot.'' Research Associate, University of Lyon

	\item ``Thanks a gain for your help so far - I must say I like ggplot more and more.''

	\item ``Thank you and congratulations for ggplot, it's a most impressive software.''

	\item ``I am trying to get a hold on the use of ggplot because this really the best plotting devise on R and better than any other out of R.''
\end{itemize}

The book is structured to lead the reader from being a new user of ggplot to a developer creating new components for specialised plots:

\begin{itemize}
	\item Chapter One describes how you can quickly get started using {\tt qplot} to make graphics, just as you can using {\tt plot}.  This chapter introduces several important ggplot concepts: grob functions, aesthetic mappings and facetting.
	
	\item While {\tt qplot} is a quick way to get started, you are not using the full power of the grammar.  Chapter Two describes the grammar of graphics underlying ggplot, and it differs from Wilkinson's grammar.  The theory will be illustrated with examples showing how to build up a plot piece by piece, exercising full control over the available options.  You will learn about the different components of a plot, laying the ground for the following chapters which describe these components in detail and teach you how to build your own.  You will also learn some techniques using the reshape package to get data into a convenient form for ggplot.

	\item The most crucial components of a plot are the {\em geometric} and {\em statistic} objects, and Chapters Three and Four describes what they do, how they work, and list the most commonly used.  Mastery of this chapter will give you the ability to pick and choose the most appropriate tool for your visual display needs.  The chapter concludes by showing you how to build your own geom and stat objects so that you can extend ggplot for your needs.

	\item Understanding how scales works is crucial for fine tuning the perceptual properties of your plot.  Customising scales gives fine control over the exact appearance of the plot, and helps to support the story that you are telling.  Chapter Five will show you what scales are available, how to adjust their parameters, and how to create your own.

	\item Non-cartesian coordinate systems are somewhat rare, but when you need them, it's hard to go without.  In Chapter Six, the different coordinate systems are described and illustrated, and you will learn how to create you own.
	
	\item Sometimes you need more control over the output than ggplot provides.  In this case, you will need to modify the low level grid output used to draw the graphics.  In Chapter Seven, you will learn how this output is constructed, how to control and modify it, and how to add additional annotations to the plot.

\end{itemize}

\section{Marketing}\label{sec:marketing} 

% Marketing: Is your book primarily a monograph, textbook, or lecture note
% volume? (Lecture notes are less polished, more specialized, and/or less
% self-contained than monographs.) If there will be professional/reference
% sales (monograph or lecture note), what are the research areas of the
% people who will be interested and the most important associations to
% which it should be marketed? If it is a textbook, what are typical
% course titles and at what level are they taught? Will you include
% exercise sets?

This book would primarily be a reference book and guide to producing statistical graphics in R.  It would be appropriate as a supplemental textbook for an introductory--intermediate R course, and will provide exercise sets.  It will have considerable appeal to anyone using R to produce high-quality graphics for communication, and also for those using R to produce exploratory graphics.  

The accompanying book site, \url{http://had.co.nz/ggplot} will provide supplemental material to aid the transition from other plotting systems.  Currently, I have a brief conversion guide from lattice, another R plotting package, and I have plans to provide a guide that recreates the graphics of \citet{wilkinson:2006} using ggplot.

\section{Comparisons}\label{sec:comparisons} 

% Comparisons: What is the best book available in this area? Why is your
% book needed (new topics, different approach/level/types of
% applications)? If it is a reference book, why will someone want to buy
% it? How would you explain in non-technical language why a bookstore
% buyer or librarian should order your book? (Why is this area important,
% what problems does it try to solve, what is unique about your book, and
% who will buy it, in a paragraph or so that we and a bookstore buyer will
% understand.) If your book is primarily a text, why will someone want to
% adopt it? In both cases: Are there software issues and if so, how will
% you deal with them?

This book is needed because there is no book that specifically deals with creating statistical graphics with R, particularly using the grammar of graphics.  Some books which cover similar topics to this book are:

\begin{itemize}
  \item \citet{wilkinson:2006}.  Very theoretical, and highly respected.  My book would supplement the Grammar of Graphics by providing a practical, usable guide.  ggplot is the only open-source implementation of the grammar of graphics, and so will also be interest to anyone who has read the book and would like to experiment with the ideas.  My feeling is that within the statistics community the book is highly respected, but has had little impact because until now there has been no way to actually use it.
  
  \item \citet{murrell:2005}.  A very low-level introduction to R graphics, covering base graphics and grid extensively, and touching on lattice graphics.  Mainly focussed on creating custom graphics from scratch (I used the book extensively while building the ggplot package), but not a lot about statistical graphics.

  \item \citet{cook:2007}.  Excellent guide to data analysis and exploration through interactive graphics, but little guidance on producing polished graphics for publication or communication.  Uses ggplot extensively to draw graphics, but does not discuss how to tune them yourself.

  \item Lattice graphics, Deepayan Sarkar (speculative). I don't know if this book exists, but if it does, my book would complement it, by describing an alternative way to produce graphics in R.  Lattice is more established, but is based on older framework for graphics \citep{cleveland:1994} and development has slowed in recent years.  Obviously I'm rather biased, but I think that ggplot is significantly more powerful and easier to use.
  
  \item \citet{chambers:1983,cleveland:1993,cleveland:1994,robbins:2004} each provide some advice on choosing what to plot, not just how to plot it.  Similarly, my book will provide some advice on how to find the right graph for your problem, but it is not the main focus of the book. 
  
\end{itemize}

Software accompanying the book will be provided as an R package, available from CRAN.  

% Would like to make first two chapters available on the website.  Good guide for new users, but only teaches basics.  Shows what you can do with the package, and will hopefully encourage users to by the book.  Difficult to estimate number of current users.  Websites receives 75 visits /month (roughly corresponds to number of visitors) and 125 pages / month.  No way to judge number of package downloads.  
% 
% Taught in two courses: Stat480 (by me) and Stat503 (by Di Cook). Students seem to like it.

\section{Reviewers}\label{sec:reviewers}

% Reviewers:  Are there any people whom you would recommend as reviewers
% (experts in this area or teaching a suitable course)?

\begin{itemize}

	\item Lee Wilkinson, SPSS.

	\item Paul Murrell, University of Auckland.

	\item Ross Ihaka, University of Auckland.

	\item Antony Unwin, Augsburg Universitat.

\end{itemize}

\section{Schedule}\label{sec:schedule}

% Schedule: What is the status of your book? Is it at the idea stage only,
% do you have lecture notes, or do you have reasonably polished chapters
% ready for review? When can you send sample chapters? (Sample chapters
% should be typical of the level and style of the book.) When do you
% expect to finish the book?

The first three chapters are reasonably complete, and the outline of remaining chapters sketched out.  The first three chapters are 38 letter-size pages, and there are another 22 in rough draft form.  At the latest, I want to be finished before the end of August 2008.

\section{References}\label{sec:references} 

% References: If your book is research-level, a brief bibliography will
% help us evaluate your proposal. Be sure to include any important recent
% books.

Bibliography included in attached sample chapters.

\section{Production}\label{sec:production}

% Production: What is the estimated length? What word processing system
% will you use? (We will supply a style file for LaTeX and Word.) Are
% color figures essential to your book? If so, about how many would have
% to be in color? Color printing is still very expensive and color figures
% will increase the price so black and white should be used unless color
% is essential.

Estimated length: 100-150 pages.  I think colour is essential for a graphics book of this nature, and would probably be used on 20-30 pages.  The book is produced using latex and R, with a custom system to allow embedding of R code and graphics inside the latex document.

\bibliography{bibliography}
\end{document}

\end{document}
