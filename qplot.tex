\providecommand{\setflag}{\newif \ifwhole \wholefalse}
\setflag
\ifwhole\else
\documentclass[oneside,letterpaper]{scrbook}
\usepackage{fullpage}
\usepackage[utf8]{inputenc}
\usepackage[pdftex]{graphicx}
\usepackage{hyperref}
\usepackage{minitoc}
\usepackage{pdfsync}
\usepackage{alltt}
\usepackage[round,sort&compress,sectionbib]{natbib}
\bibliographystyle{plainnat}

%\setcounter{secnumdepth}{-1}

\title{ggplot}
\author{Hadley Wickham}

\renewcommand{\topfraction}{0.9}	% max fraction of floats at top
\renewcommand{\bottomfraction}{0.8}	% max fraction of floats at bottom
%   Parameters for TEXT pages (not float pages):
\setcounter{topnumber}{2}
\setcounter{bottomnumber}{2}
\renewcommand{\dbltopfraction}{0.9}	% fit big float above 2-col. text
\renewcommand{\textfraction}{0.07}	% allow minimal text w. figs
%   Parameters for FLOAT pages (not text pages):
\renewcommand{\floatpagefraction}{0.7}	% require fuller float pages
% N.B.: floatpagefraction MUST be less than topfraction !!
\renewcommand{\dblfloatpagefraction}{0.7}	% require fuller float pages

\newcommand{\grobref}[1]{{\tt #1} (page \pageref{sub:#1})}
\newcommand{\secref}[1]{\ref{#1} (\pageref{#1})}


\raggedbottom

\begin{document}
\fi

\chapter{Getting started with ggplot: qplot}

\section{Introduction}

Plotting data is the most important part of a graphics package and this chapter gets you started with you first ggplot function, {\tt qplot} (short for {\bf q}uick plot).  If you have used the {\tt plot} function in base R before, {\tt qplot} should be easy to use as most of the arguments have the same use.  


\section{Examples}

\section{Other arguments}

\section{Adding output}

You can also use qplot to add more elements to an existing plot.  This is useful when building up a plot from multiple data sources.  Note, however, that there are some limitations on legends which means that you may need to arrange your data into one data frame first.  XXX describes this in detail.


\ifwhole
\else
	\bibliography{bibliography}
  \end{document}
\fi
