\providecommand{\setflag}{\newif \ifwhole \wholefalse}
\setflag
\ifwhole\else
\documentclass[oneside,letterpaper]{scrbook}
\usepackage{fullpage}
\usepackage[utf8]{inputenc}
\usepackage[pdftex]{graphicx}
\usepackage{hyperref}
\usepackage{minitoc}
\usepackage{pdfsync}
\usepackage{alltt}
\usepackage[round,sort&compress,sectionbib]{natbib}
\bibliographystyle{plainnat}

%\setcounter{secnumdepth}{-1}

\title{ggplot}
\author{Hadley Wickham}

\renewcommand{\topfraction}{0.9}	% max fraction of floats at top
\renewcommand{\bottomfraction}{0.8}	% max fraction of floats at bottom
%   Parameters for TEXT pages (not float pages):
\setcounter{topnumber}{2}
\setcounter{bottomnumber}{2}
\renewcommand{\dbltopfraction}{0.9}	% fit big float above 2-col. text
\renewcommand{\textfraction}{0.07}	% allow minimal text w. figs
%   Parameters for FLOAT pages (not text pages):
\renewcommand{\floatpagefraction}{0.7}	% require fuller float pages
% N.B.: floatpagefraction MUST be less than topfraction !!
\renewcommand{\dblfloatpagefraction}{0.7}	% require fuller float pages

\newcommand{\grobref}[1]{{\tt #1} (page \pageref{sub:#1})}
\newcommand{\secref}[1]{\ref{#1} (\pageref{#1})}


\raggedbottom

\begin{document}
\fi

\chapter{Introduction}

ggplot is an R package for producing statistical graphics.  It builds on top of the grid graphics system \citep{grid}, and uses the philosophy outlined in the Grammar of Graphics \citep{wilkinson:2006}.  Because ggplot is based on a formal model of graphics, it provides a powerful and flexible plotting system that is still easy to use.  ggplot also provides many features that take the hassle out of making graphics. For example, it produces legends automatically, makes it easy to combine data from multiple sources, to produce the same plot for different subsets of a data set.

This book provides a practical introduction to ggplot with lots of example code and graphics. It also explains the grammar on which ggplot is based. Like other formal systems, ggplot is useful even when you don't understand the underlying model. However, the more you learn about the grammar, the more effectively you'll be able to use ggplot. This book will introduce you to ggplot as a novice, unfamiliar with the grammar, and turn you into an expert who can build new components to extend the grammar.

This book assumes some familiarity with R, to the level described in Dalgaard’s Introductory Statistics with R.

\section{What is the grammar of graphics?}

\citet{wilkinson:2006} created the grammar of graphics to describe the deep features that underlie all statistical graphics.  The grammar of graphics is an answer to a question: what is a statistical graphic?  My take on the grammar is that a graphic is a mapping from data to  aesthetic attributes (colour, shape, size) of geometric objects (points, lines, bars).  The plot may also contain statistical transformations of the data, and is drawn on a specific  coordinate system.  Facetting can be used to generate the same plot for different subsets of the dataset.  It is the combination of these independent components that make up a graphic.  

The grammar, my interpretation of the grammar, and how it is used in ggplot is described in detail in chapter 3.  Mastery of the grammar will go hand in hand with mastery of ggplot.

The components of the grammar are described in more detail below.

\begin{itemize}
	\item Data is the most important thing, and the thing that you bring to the table.
	\item Geometric objects (or geoms for short) represent what you actually see on the plot: points, lines, polygons, etc.
	\item Statistics transform the data in many useful ways.  For example, binning and counting to create a histogram.  They are optional, but very useful.
	\item Scales map values in the data space to values in an aesthetic space, whether it be colour, or size, or shape.  Scales also provide an inverse mapping, a legend, to make it possible to read the original data values off the graph.
	\item A coordinate system describes how data coordinates are mapped to the plane of the graphic.  It also provides axes and gridlines to make it possible to read the graph.  We normally use a cartesian coordinate system, but many others are available, including polar, cartographic projections, and hierarchical coordinate systems for categorical data.
	\item A facetting, or conditioning, specification.  It is often useful to be able to reproduce the same plot for different subsets of the data.  The facetting specification describes those subsets and how the facets should be arranged in to a plot.
\end{itemize}

It is also important to talk about what the grammar doesn't do:

\begin{itemize}
	\item It doesn't advise what graphics you should use to answer the questions you are interested in.  While this book endeavours to promote a sensible process for producing plots of data, the focus of the book is on how to produce the plots you want, not knowing what plots to produce. For more advice on this topic, you may want to consult these sources: \citet{robbins:2004,cleveland:1993,chambers:1983,tukey:1977}.

	\item Ironically, the grammar doesn't specify what a graphic should look like.  The finer points of display, for example, font size or background colour, are not specified by the grammar.  In practice, a useful plotting system will need to describe these, as ggplot doese. Similarly, the grammar does not specify how to make an attractive graphic, and while the defaults in ggplot have been chosen with care, you may need to consult other references to create an attractive plot: \citep{tufte:1990,tufte:1997,tufte:2001,tufte:2006}.

	\item It does not describe interaction: the grammar of graphics describes only static graphics, and there is essentially no benefit to displaying on a computer screen as opposed to on a piece of paper.  ggplot can only create static graphics, so for dynamic and interactive graphics you will have to look elsewhere.  [Mondrian book] and [GGobi book] provide excellent introductions to two different interactive graphics packages: Mondrian and GGobi.

\end{itemize}

\section{How does ggplot fit in with other R graphics?}

There are a number of other graphics systems available in R: base graphics, grid graphics and trellis/lattice graphics.  How does ggplot differ from them?

\begin{itemize} 
	\item Base graphics were ``hacked'' together by Ross Ihaka based on experience implementing S graphics driver and partly looking at \cite{chambers:1983} \citetext{priv.\ comm.}.  Base graphics has basically a pen on paper model: you can only draw on top, not modify or delete existing content or change the axes etc.  There is no (user accessible) representation of the graphics, apart from the appearance on the plot. Base graphics includes both tools for drawing primitives and entire plots. No other plotting system in R does this.  Base graphics functions are generally fast, but have limited scope. When you create a single scatterplot, or histogram, or a set of boxplots, you're probably using base graphics.

	\item The development of grid graphics, a much richer system of graphical primitives, started in 2000.  Grid was, and is, developed by Paul Murrell, growing out of his PhD work, with the initial aim of building a solid foundation for ``something trellis-like''.  Deepayan Sarkar soon took on the development of the trellis-like stage, leaving Paul to build an excellent foundation \citetext{priv.\ comm.}.  
	
	Grid grobs (graphical objects) can be represented independently of the plot and modified later. A system of viewports (each containing its own coordinate system) makes it easier to layout complex graphics. Grid provides drawing primitives, but no tools for producing statistical graphics.  

	\item The lattice package \citep{sarkar:2005}, developed by Deepayan Sarkar, uses grid graphics to implement the trellis graphics system of \citet{cleveland:1993,cleveland:1994}, and is a considerable improvement over base graphics.  You can easily produce conditioned plots, and some plotting details (eg.\ legends) are taken care of automatically.  However, lattice graphics lacks a formal model, which can make it hard to extend.

	\item ggplot, started in 2005, is an attempt to take the good things about base and lattice graphics and improve on them with a strong underlying model which supports the production of any kind of statistical graphic, based on principles outlined above.  The solid underlying model of ggplot makes it easy to describe a wide range of graphics with a compact syntax, and independent components make extension easy.  Like lattice, ggplot uses grid to draw the graphics, which means you can exercise much low level control over the appearance of the plot

\end{itemize}

Many other R packages implement specialist graphics but no others provide a framework for producing statistical graphics. While this is fine if you just want to produce a one-off graphic, it can be hard to seamlessly combine these with other graphics you are using.

\section{About this book}\label{sec:about_this_book}

% This book is available online for free at \url{http://had.co.nz/ggplot/book}.  However, if you want to support the development of ggplot (and save yourself the hassle of printing and binding a large pdf) you can also buy a printed version for \$US 40 (price may change).  

The book is structured to lead you from being a new user of ggplot to a developer creating new components for specialised plots:

\begin{itemize}
	\item Chapter One describes how you can quickly get started using {\tt qplot} to make graphics, just like you can using {\tt plot}.  This chapter introduces several important ggplot concepts: grob functions, aesthetic mappings and facetting.
	
	\item While {\tt qplot} is a quick way to get started, you are not using the full power of the grammar.  Chapter Two describes how the grammar of graphics and how the grammar of ggplot differs to the original.  The theory will be illustrated with examples showing how to build up a plot piece by piece, exercising full control over the available options.  You will learn about the different components of a plot, laying the ground for the following chapters which describe these components in detail and teach you how to build your own.  You will also learn some techniques using the reshape package to get data into a convenient form for ggplot.

	\item The most crucial components of a plot are the geometric and statistic objects, and Chapters Three and Four describes what they do, how they work, and list the most commonly used.  Mastery of this chapter will give you the ability to pick and choose the most appropriate tool for your visual display needs.  The chapter concludes by showing you how to build your own geom and stat objects so that you can extend ggplot for your needs.

	\item Understanding how scales works is crucial for fine tuning the perceptual properties of your plot.  Customising scales gives fine control over the exact appearance of the plot, and helps to support the story that you are telling.  Chapter Five will show you what scales are available, how to adjust their parameters, and how to create your own.

	\item Non-cartesian coordinate systems are somewhat rare, but when you need them, its hard to go without.  In Chapter Six, the different coordinate systems are described and illustrated, and you will learn how to create you own.
	
	% \item Chapter Seven introduces my philosophy of data.  This chapter isn't crucial, but will help you understand the type of data {\tt ggplot} expects, and how to transform your data into that format.  It also discusses facetting in more detail, and discusses some ideas for combining modelling and graphics.
	
	\item Sometimes you need more control over the output than ggplot provides.  In this case, you will need to modify the low level grid output used to draw the graphics.  In Chapter Seven, you will how this output is constructed, how to control and modify it, and how to add additional annotations to the plot.

\end{itemize}

The ggplot website, \url{http://had.co.nz/ggplot}, provides updates to this book, information about features in the latest version of ggplot, and talks and papers related to ggplot.  All graphics used on the book are displayed on the site, along with the code and data needed to reproduce them.  There is also a gallery of ggplot graphics used in real life.  If you would like your graphics to be included in the gallery, please send me reproducible code and a paragraph or two describing your plot.

\section{Installation}\label{sub:installation}

To get started using ggplot, the first thing you need to do is install it.  Make sure you have a recent version of R from \url{http://r-project.org}, and then follow the instructions below to download and install the ggplot package.  

There are usually two versions of ggplot available, one stable version and one development version. The stable version is well-tested and well-documented before release.  It is available on CRAN, and can be installed with the following R code:

\begin{verbatim}
	install.packages("ggplot", dep=TRUE)
\end{verbatim}

The development version is not so well tested or documented, but includes new features that I'm working on.  It may also contain bug fixes for recently discovered bugs.  It can be installed as follows:

\begin{verbatim}
	install.packages("ggplot", repos="http://www.ggobi.org/r", dep=TRUE)
\end{verbatim}

A changelog listing changes between versions is available on the ggplot website.  I will do my best to make sure that changes are backward compatible, so you shouldn't have to rewrite your old code.  However, from time to time, I may need to make bigger changes that do affect your code.  If you need to ensure that your old code will continue to run, I would recommend using use R's versioned installs:

\begin{verbatim}
	install.packages("ggplot", dep=TRUE, installWithVers=TRUE)
\end{verbatim}

Now installed packages will have a version number associated with them, and you can load a specific version like so:

\begin{verbatim}
	library(ggplot, version="0.5")
\end{verbatim}

{\tt ggplot} isn't perfect, so from time-to-time you may encounter something that doesn't way the way you think it should.  If this happens, please email me \href{mailto:h.wickham@gmail.com}{h.wickham@gmail.com} a reproducible example of your problem, as well as a description of what you think should happen.  The more information you provide, the more likely I am going to be able to help you.

\section{Acknowledgements}\label{sec:acknolwedgements}

Many people have contributed to this book with high-level structural insights, and bug reports.  In particular, I would to thank: Lee Wilkinson, for discussions that cemented my understand of the grammar; Gabor Grothendienk for early helpful comments; Heike Hofmann and Di Cook for being great major professors; Charlotte Wickham; the students of stat480 and stat503 at ISU, for using it; Debby Swayne, for a masses helpful feedback and advice.

\ifwhole
\else
	\bibliography{bibliography}
  \end{document}
\fi
