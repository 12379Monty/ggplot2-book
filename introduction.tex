\providecommand{\setflag}{\newif \ifwhole \wholefalse}
\setflag
\ifwhole\else

% Typography and geometry ----------------------------------------------------
\documentclass[letterpaper]{scrbook}
\usepackage[inner=3cm,top=2.5cm,outer=3.5cm]{geometry}

\renewcommand\familydefault{bch}
\usepackage[utf8]{inputenc}
\usepackage{microtype}
\usepackage[small]{caption}
\usepackage[small]{titlesec}
\raggedbottom

% Graphics -------------------------------------------------------------------
\usepackage[pdftex]{graphicx}
\graphicspath{{_include/}}
\DeclareGraphicsExtensions{.png,.pdf}

% Code formatting ------------------------------------------------------------
\usepackage{fancyvrb}
\usepackage{courier}
\usepackage{listings}
\usepackage{color}
\usepackage{alltt}


\definecolor{comment}{rgb}{0.60, 0.60, 0.53}
\definecolor{background}{rgb}{0.97, 0.97, 1.00}
\definecolor{string}{rgb}{0.863, 0.066, 0.266}
\definecolor{number}{rgb}{0.0, 0.6, 0.6}
\definecolor{variable}{rgb}{0.00, 0.52, 0.70}
\lstset{
  basicstyle=\ttfamily,
  keywordstyle=\bfseries, 
  identifierstyle=,  
  commentstyle=\color{comment} \emph,
  stringstyle=\color{string},
  showstringspaces=false,
  columns = fullflexible,
  backgroundcolor=\color{background},
  mathescape = true,
  escapeinside=&&,
  fancyvrb
}
\newcommand{\code}[1]{\lstinline!#1!}
\newcommand{\f}[1]{\lstinline!#1()!}



% Links ----------------------------------------------------------------------

\usepackage{hyperref}
\definecolor{slateblue}{rgb}{0.07,0.07,0.488}
\hypersetup{colorlinks=true,linkcolor=slateblue,anchorcolor=slateblue,citecolor=slateblue,filecolor=slateblue,urlcolor=slateblue,bookmarksnumbered=true,pdfview=FitB}
\usepackage{url}

% Tables ---------------------------------------------------------------------
\usepackage{longtable}
\usepackage{booktabs}

% Miscellaneous --------------------------------------------------------------
\usepackage{pdfsync}
\usepackage{appendix}

\usepackage[round,sort&compress,sectionbib]{natbib}
\bibliographystyle{plainnat}


\title{ggplot2}
\author{Hadley Wickham}

\begin{document}
\fi


\chapter{Preface}

\section{Introduction}

What is \ggplot? 

\begin{itemize}
  \item An R package for producing statistical graphics.  

  \item A language, based on the Grammar of Graphics \citep{wilkinson:2006},  for describing and creating plots. 

  \item A set of independent components that minimise the code needed to produce complex graphics

  \item Plots that can be built up iteratively and edited later

\end{itemize} 

It builds on top of the grid graphics system \citep{grid}, and also provides many features that take the hassle out of making graphics. For example, it produces legends automatically, makes it easy to combine data from multiple sources, to produce the same plot for different subsets of a data set.

This book provides a practical introduction to \ggplot with lots of example code and graphics. It also explains the grammar on which \ggplot is based. Like other formal systems, \ggplot is useful even when you don't understand the underlying model. However, the more you learn about the it, the more effectively you'll be able to use \ggplot. 

This book assumes basic some familiarity with R, to the level described in the first chapter of Dalgaard’s Introductory Statistics with R. This book will introduce you to \ggplot as a novice, unfamiliar with the grammar, and turn you into an expert who can build new components to extend the grammar.

\section{Other resources}
\label{sec:other_resources}

This book does not exhaustively cover every possible use of \ggplot.  It does not document every function in detail and it does not describe every possible plot you could make.  What it does do, however, is teach you what all the pieces are and how they fit together.  By the end of reading this book, you should be able to build new and unique plots specifically tailored to your needs.  

However, you still need to be able to get precise details of individual components.  The best resource for this will always be the built in documentation.  This is accessible online, \url{http://had.co.nz/ggplot}, and from within R, using the usual help syntax. This website also lists talks and papers related to \ggplot.  The {\sc cran} website, \url{http://cran.r-project.org/web/packages/ggplot2/}, is another useful resource.  This provides convenient links to what's new and different in each release.

The book website, \url{http://had.co.nz/ggplot2/book}, provides updates to this book,   All graphics used on the book are displayed on the site, along with the code and data needed to reproduce them.  There is also a gallery of \ggplot graphics used in real life.  If you would like your graphics to be included in the gallery, please send me reproducible code and a paragraph or two describing your plot.


\section{What is the grammar of graphics?}

\citet{wilkinson:2006} created the grammar of graphics to describe the deep features that underlie all statistical graphics.  The grammar of graphics is an answer to a question: what is a statistical graphic?  My take on the grammar is that a graphic is a mapping from data to  aesthetic attributes (colour, shape, size) of geometric objects (points, lines, bars).  The plot may also contain statistical transformations of the data, and is drawn on a specific  coordinate system.  Facetting can be used to generate the same plot for different subsets of the dataset.  It is the combination of these independent components that make up a graphic.  A detailed description of the formal grammar of \ggplot and how it differs from Wilkinson's can be found in \citet{wickham:2007d}.

As the book progresses, the formal grammar will be explained in increasing detail.  The first description of the components follows below.  It introduces some of the terminology that will be used throughout the book, and outlines the basic responsibilities of each component.  

\begin{itemize}
  \item The \textbf{data} that you want to visualise, and a set of aesthetic \textbf{mapping}s describing how variables in the data are mapped to aesthetic attributes that you can perceive.

  \item Geometric objects (\textbf{geom}s for short) represent what you actually see on the plot: points, lines, polygons, etc.

  \item Statistics transformations (\textbf{stat}s for short) summarise data in many useful ways.  For example, binning and counting to create a histogram.  They are optional, but very useful.

  \item The \textbf{scale}s map values in the data space to values in an aesthetic space, whether it be colour, or size, or shape.  Scales also provide an inverse mapping, a legend or axis, to make it possible to read the original data values off the graph.

  \item A coordinate system (\textbf{coord} for short) describes how data coordinates are mapped to the plane of the graphic.  It also provides axes and gridlines to make it possible to read the graph.  We normally use a cartesian coordinate system, but many others are available, including polar, cartographic projections, and hierarchical coordinate systems for categorical data.

  \item A \textbf{facet}ing specification describes how to break up the data into subsets and how to display those subsets as small multiples.  This is also known as conditioning or latticing/trellising.

\end{itemize}

It is also important to talk about what the grammar doesn't do:

\begin{itemize}
  \item It doesn't suggest what graphics you should use to answer the questions you are interested in.  While this book endeavours to promote a sensible process for producing plots of data, the focus of the book is on how to produce the plots you want, not knowing what plots to produce. For more advice on this topic, you may want to consult  \citet{robbins:2004,cleveland:1993,chambers:1983,tukey:1977}.

  \item Ironically, the grammar doesn't specify what a graphic should look like.  The finer points of display, for example, font size or background colour, are not specified by the grammar.  In practice, a useful plotting system will need to describe these, as \ggplot does. Similarly, the grammar does not specify how to make an attractive graphic, and while the defaults in \ggplot have been chosen with care, you may need to consult other references to create an attractive plot: \citet{tufte:1990,tufte:1997,tufte:2001,tufte:2006}.

  \item It does not describe interaction: the grammar of graphics describes only static graphics, and there is essentially no benefit to displaying on a computer screen as opposed to on a piece of paper.  \ggplot can only create static graphics, so for dynamic and interactive graphics you will have to look elsewhere.  \citet{cook:2007} provides an excellent introductions the interactive graphics package GGobi.  GGobi can be connected to R with the \pkg{rggobi} package \citep{wickham:2007i}.

\end{itemize}

\section{How does \ggplot fit in with other R graphics?}

There are a number of other graphics systems available in R: base graphics, grid graphics and trellis/lattice graphics.  How does \ggplot differ from them?

\begin{itemize} 
  \item Base graphics were ``hacked'' together by Ross Ihaka based on experience implementing S graphics driver and partly looking at \cite{chambers:1983}.  Base graphics basically has a pen on paper model: you can only draw on top of the plot, you can not modify or delete existing content.  There is no (user accessible) representation of the graphics, apart from their appearance on the screen. Base graphics includes both tools for drawing primitives and entire plots.  Base graphics functions are generally fast, but have limited scope. When you create a single scatterplot, or histogram, or a set of boxplots, you're probably using base graphics.

  \item The development of \pkg{grid} graphics, a much richer system of graphical primitives, started in 2000.  Grid is developed by Paul Murrell, growing out of his PhD work \citep{murrell:1998}. Grid grobs (graphical objects) can be represented independently of the plot and modified later. A system of viewports (each containing its own coordinate system) makes it easier to layout complex graphics. Grid provides drawing primitives, but no tools for producing statistical graphics.  

  \item The \pkg{lattice} package \citep{lattice}, developed by Deepayan Sarkar, uses grid graphics to implement the trellis graphics system of \citet{cleveland:1993,cleveland:1994}, and is a considerable improvement over base graphics.  You can easily produce conditioned plots, and some plotting details (e.g.\ legends) are taken care of automatically.  However, lattice graphics lacks a formal model, which can make it hard to extend.  Lattice graphics are explained in depth in \citep{sarkar:2008}.

  \item \ggplot, started in 2005, is an attempt to take the good things about base and lattice graphics and improve on them with a strong underlying model which supports the production of any kind of statistical graphic, based on principles outlined above.  The solid underlying model of \ggplot makes it easy to describe a wide range of graphics with a compact syntax, and independent components make extension easy.  Like \pkg{lattice}, \ggplot uses grid to draw the graphics, which means you can exercise much low level control over the appearance of the plot

\end{itemize}

Many other R packages, such as \pkg{vcd} \citep{meyer:2006}, \pkg{plotrix} \citep{plotrix} and \pkg{gplots} \citep{gplots}, implement specialist graphics, but no others provide a framework for producing statistical graphics.  A comprehensive resource listing all graphics functionality available in other contributed packages is the graphics task view at \url{http://cran.r-project.org/web/views/Graphics.html}.  

\section{About this book}\label{sec:about_this_book}

% The book is divided into four sections:
% 
% \begin{itemize}
%   \item Introduction to the grammar and to \ggplot
% 
%   \item More depth about the grammar, describing each piece and how they fit together.
% 
%   \item Tools and strategies for data analysis (i.e. how do I actually use \ggplot to better understand my data)
% 
%   \item Advanced techniques - these require understanding of other aspects of R.  Polishing plots requires knowledge of grid, and writing your own requires some knowledge of the proto OO system
% \end{itemize}


\begin{itemize}
  \item Chapter~\ref{cha:qplot} describes how to quickly get started using {\tt qplot} to make graphics, just like you can using {\tt plot}.  This chapter introduces several important \ggplot concepts: geoms, aesthetic mappings and facetting.
  
  \item While {\tt qplot} is a quick way to get started, you are not using the full power of the grammar.  Chapter~\ref{cha:mastery} describes the layered grammar of graphics which underlies \ggplot.  The theory is illustrated in Chapter~\ref{cha:layers} with examples showing how to build up a plot piece by piece, exercising full control over the available options.  You will learn about the different components of a plot, laying the ground for the following chapters which describe these components in detail and teach you how to build your own. 

  \item Chapter~\ref{cha:toolbox} describes how assemble the components of \ggplot to solve particular plotting problems.

  \item Understanding how scales works is crucial for fine tuning the perceptual properties of your plot.  Customising scales gives fine control over the exact appearance of the plot, and helps to support the story that you are telling.  Chapter~\ref{cha:scales} will show you what scales are available, how to adjust their parameters, and how to create your own.

  \item There are three different ways to tweak the position of plots: coordinate systems, facetting and position adjustments.  These are described in Chapter~\ref{cha:position}
  
  % \item Chapter Seven introduces my philosophy of data.  This chapter isn't crucial, but will help you understand the type of data \ggplot expects, and how to transform your data into that format.  It also discusses facetting in more detail, and discusses some ideas for combining modelling and graphics.
  
  \item Sometimes you need more control over the output than \ggplot provides.  In this case, you will need to modify the low level grid output used to draw the graphics.  In Chapter~\ref{cha:grid}, you will learn how this output is constructed, how to control and modify it, and how to add additional annotations to the plot.

  \item Two appendices provide additional useful information.  Appendix~\ref{cha:specification} describes how colours, shapes, line types and sizes can be specified by hand, and Appendix~\ref{cha:translating} shows how to translate from base graphics, lattice graphics, and Wilkinson's {\sc gpl} to \ggplot syntax.

\end{itemize}

\section{Installation}\label{sub:installation}

To use \ggplot, you must first install it. Make sure you have a recent version of R (at least version 2.7) from \url{http://r-project.org}, and then run the following line of code to download and install the \ggplot package.  

\begin{verbatim}
  install.packages("ggplot2")
\end{verbatim}

% A changelog listing changes between versions is available on the \ggplot website.  I will do my best to make sure that changes are backward compatible, so you shouldn't have to rewrite your old code.  However, from time to time, I may need to make bigger changes that do affect your code.  If you need to ensure that your old code will continue to run, I would recommend using use R's versioned installs:
% 
% \begin{verbatim}
%   install.packages("ggplot2", installWithVers=TRUE)
% \end{verbatim}
% 
% Now installed packages will have a version number associated with them, and you can load a specific version like so:
% 
% \begin{verbatim}
%   library(ggplot2, version="0.5")
% \end{verbatim}

\ggplot isn't perfect, so from time to time you may encounter something that doesn't work the way it should.  If this happens, please email me at \href{mailto:h.wickham@gmail.com}{h.wickham@gmail.com} with a reproducible example of your problem, as well as a description of what you think should have happened.  The more information you provide, the easier it is for me to help you.

\section{Acknowledgements}\label{sec:acknolwedgements}

Many people have contributed to this book with high-level structural insights, spelling and grammar corrections and bug reports.  In particular, I would to thank: Lee Wilkinson, for discussions that cemented my understanding of the grammar; Gabor Grothendieck, for early helpful comments; Heike Hofmann and Di Cook, for being great major professors; Charlotte Wickham; the students of stat480 and stat503 at ISU, for using it; Debby Swayne, for masses of helpful feedback and advice; Bob Muenchen; Reinhold Kleigl; Philipp Pagel...

\ifwhole
\else
  \nobibliography{/Users/hadley/documents/phd/references}
  \end{document}
\fi
