\providecommand{\setflag}{\newif \ifwhole \wholefalse}
\setflag
\ifwhole\else
\documentclass[oneside,letterpaper]{scrbook}
\usepackage{fullpage}
\usepackage[utf8]{inputenc}
\usepackage[pdftex]{graphicx}
\usepackage{hyperref}
\usepackage{minitoc}
\usepackage{pdfsync}
\usepackage{alltt}
\usepackage[round,sort&compress,sectionbib]{natbib}
\bibliographystyle{plainnat}

%\setcounter{secnumdepth}{-1}

\title{ggplot}
\author{Hadley Wickham}

\renewcommand{\topfraction}{0.9}	% max fraction of floats at top
\renewcommand{\bottomfraction}{0.8}	% max fraction of floats at bottom
%   Parameters for TEXT pages (not float pages):
\setcounter{topnumber}{2}
\setcounter{bottomnumber}{2}
\renewcommand{\dbltopfraction}{0.9}	% fit big float above 2-col. text
\renewcommand{\textfraction}{0.07}	% allow minimal text w. figs
%   Parameters for FLOAT pages (not text pages):
\renewcommand{\floatpagefraction}{0.7}	% require fuller float pages
% N.B.: floatpagefraction MUST be less than topfraction !!
\renewcommand{\dblfloatpagefraction}{0.7}	% require fuller float pages

\newcommand{\grobref}[1]{{\tt #1} (page \pageref{sub:#1})}
\newcommand{\secref}[1]{\ref{#1} (\pageref{#1})}


\raggedbottom

\begin{document}
\fi

\chapter{Introduction}

ggplot is a new R package for producing statistical graphics.  It builds on the grid graphics system, and uses the philosophy outlined in the Grammar of Graphics to produce a powerful and flexible plotting system that is still easy to use.  This book provides a practical introduction to ggplot with lots of example code and graphics.  


This book is available online for free (\url{http://had.co.nz/ggplot/book}), or if you would like to support the development of ggplot (and have a handy desktop reference) you can also buy a hardcopy version for a very reasonable price.

You can find online updates to this book, information about features in the latest version of ggplot, as well as talks and papers at \url{http://had.co.nz/ggplot}.  There is also a gallery of example graphics.  If you would like your graphics to be included in the gallery, please send me reproducible code and a paragraph or two describing your plot.

\section{Installation}\label{sub:installation}

There are usually two versions of ggplot available, one stable version and one development version.

The stable version is well-tested and well-documented before release.  It is available on CRAN, and can be installed with the following R code:

\begin{verbatim}
	install.packages("ggplot")
\end{verbatim}

The development version is not so well tested or documented, but includes new features that I'm working on.  It may also contain bug fixes for recently discovered bugs.  It can be installed by:

\begin{verbatim}
	install.packages("ggplot", repos="http://www.ggobi.org/r")
\end{verbatim}

\section{R graphics overview}

To give you some idea of how ggplot fits into the broader landscape of R graphics, this section describes some of the other graphics packages available in R.

\begin{itemize}
	\item  the problem with base graphics: is that it's basically a pen on
	paper model - you can only draw on top, not modify or delete existing
	content or change the axes etc...  Base graphics includes both tools for drawing primitives and entire plots.  No other plotting system does this.

	\item Grid graphics are a big step up from the drawing capabilities of base graphics.  The plot objects can be represented independently of the plot and modified later on.  A system of viewports (each containing its own coordinate system) makes it easier to layout complex graphics.  

	\item trellis/lattice builds on grid graphics, and improves on base graphics a bit with a rudimentary object
	model independent of the plot on screen.  The chief advantage of
	lattice over base graphics is that you can easily produce
	trellised/conditioned plots - basically where you reproduce the same
	plot for different subsets of the data

	\item ggplot is an attempt to substantially improve on this by having a
	model which supports the production of an kind of statistical graphic,
	build on principles outlined in the Grammar of Graphics (by Lee
	Wilkinson)

	\item Many other packages implement specialist graphics but no others provide a framework for producing statistical graphics.  While this is useful for producing a one-off graphic, it is generally harder to combine these with other graphics you may be using.  

\end{itemize}

\wholefalse
\ifwhole
\else
	\bibliography{bibliography}
  \end{document}
\fi
