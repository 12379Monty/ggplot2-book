\providecommand{\setflag}{\newif \ifwhole \wholefalse}
\setflag
\ifwhole\else
\documentclass[oneside,letterpaper]{scrbook}
\usepackage{fullpage}
\usepackage[utf8]{inputenc}
\usepackage[pdftex]{graphicx}
\usepackage{hyperref}
\usepackage{minitoc}
\usepackage{pdfsync}
\usepackage{alltt}
\usepackage[round,sort&compress,sectionbib]{natbib}
\bibliographystyle{plainnat}

%\setcounter{secnumdepth}{-1}

\title{ggplot}
\author{Hadley Wickham}

\renewcommand{\topfraction}{0.9}	% max fraction of floats at top
\renewcommand{\bottomfraction}{0.8}	% max fraction of floats at bottom
%   Parameters for TEXT pages (not float pages):
\setcounter{topnumber}{2}
\setcounter{bottomnumber}{2}
\renewcommand{\dbltopfraction}{0.9}	% fit big float above 2-col. text
\renewcommand{\textfraction}{0.07}	% allow minimal text w. figs
%   Parameters for FLOAT pages (not text pages):
\renewcommand{\floatpagefraction}{0.7}	% require fuller float pages
% N.B.: floatpagefraction MUST be less than topfraction !!
\renewcommand{\dblfloatpagefraction}{0.7}	% require fuller float pages

\newcommand{\grobref}[1]{{\tt #1} (page \pageref{sub:#1})}
\newcommand{\secref}[1]{\ref{#1} (\pageref{#1})}


\raggedbottom

\begin{document}
\fi

\chapter{Data wrangling}

\section{Introduction}\label{sec:introduction}

To get the most out of ggplot, you need to be able to get your data into the form that {\tt ggplot} wants.  This chapter discusses what the best format is, and some ways to get your data into that form.  It includes an introduction to the {\tt reshape} package, which is a very useful companion to {\tt ggplot}.  I also discuss facetting and its uses in more detail, as well as providing many examples so that you can get more of an idea how it works.

Unlike other graphics packages in R, {\tt ggplot} provides very few ways to get your data in.  For example, lattice functions can take an optional data frame or use vectors direct from the environment.  {\tt ggplot} only works with data in the form of a data frame (although qplot provides a convenient way of creating a data frame from vectors that exist in your data space).  This means that you have to explicitly arrange your data in the format that ggplot uses.  This makes it a little bit more verbose, but you can be more sure about exactly what is going on with your data.  It also allows a cleaner separation of concerns so that the graphics package deals only with plotting data, not wrangling it into different forms as well.

The package that I recommend for getting data into the right form (because I wrote it!), is {\tt reshape}.

This corresponds to my general philosophy on data.  

This makes it very easy to produce the same plot for different data sets.  A single data.frame is also easier to save than a multitude of vectors, which means it's easier to reproduce your results later or send them to someone else.  

The most important thing is that everything should be explicit.  Your data is the most important thing, and if it gets corrupted or arranged in an inappropriate manner everything based on that data will be compromised.  For this reason, you need to do everything yourself.

Use lower case column names (a bit easier to type, but main thing is to be consistent)

\section{Using reshape}\label{sec:using_reshape}

A useful tool for getting data into the right shape to plot.  See the reshape documentation for more details, especially the introductory pdf.  The ``molten'' form can be useful in conjunction with {\tt ggplot} if your variables are all on the same scale, as you can plot many of the variables at once using line grobs (\grobref{ggline}) and the {\tt group} attribute.  This is similar to what the parallel coordinates plot does.

Here is a basic introduction and some examples particularly relevant for {\tt ggplot}.

% decumar<<< 
% interweave({
% str(smiths)
% })
% |||
%>>>

df <- data.frame(time=1:10, a=1:10, b=rnorm(10), c=(1:10)^2/10, d=sin(1:10 * pi/2))

dfm <- melt(df, id="time")
head(dfm)
qplot(time, value, data=dfm, . ~ variable, type="line")

# or if you _really_ want a 2x2 matrix

dfm$a <- ifelse(dfm$variable %in% c("a","b"), 1, 2)
dfm$b <- ifelse(dfm$variable %in% c("a","c"), 1, 2)

qplot(time, value, data=dfm, a ~ b, type="line")

\section{Facetting}\label{sec:facetting}

Facetting is discussed previous in XXX and XXX.  Here I will go into more detail, and provide more examples.  When specifying a facetting formula, you specify a grid of row and column variables.  Variables appearing on a row or column together, is like nesting, only combinations that appear in the data will appear in the plot.  If variables that are specified on rows and columns are crossed: all combinations will be shown, including those that didn't appear in the original data set.  

You can set up facetting when you create the plot, {\tt ggplot(df, a ~ b)}, or later using \texttt{setfacets(p, formula = . ~ . , margins = FALSE)}.

You can also specify margins to display, by giving the names of the variables you want margins for.  

% decumar<<< 
% interweave({
% str(smiths)
% })
% |||
%>>>

Currently, {\tt ggplot} doesn't make any distinction between structural and non-structural missings in the facets - you can't tell whether a combination did not appear in the original dataset, or if it had no data.  Dealing with this properly may require some extensions to R's missing value system.  

\subsection{Continuous variables}\label{sub:continuous_variables}

To use continuous variables as facetting variables, you will first need to convert them to categorical.  Also, as the facetting formula currently does not support calculated variables, so you will need to save them in the data frame first.

{\tt ggplot} provides a convenient function for converting a continuous variable to a categorical variable: {\tt chop}.  Chop takes the following arguments:

\begin{itemize}
	\item {\tt x}: vector of numbers to chop up into categorical variable
	\item {\tt n}: number of bins to cut the variable into
	\item {\tt method}: method to use, equal ranges, equal numbers in each range or pretty ranges
\end{itemize}

The following examples shows the use of chop to explore the earthquakes data set.

% decumar<<< 
% ggopt(aspect.ratio = 1)
% interweave({
% quakes$depthc <- chop(quakes$depth)
% qplot(long, lat, . ~ depthc, data=quakes)
% quakes$magc <- chop(quakes$mag, n=6, method="pretty")
% qplot(long, lat, . ~ magc, data=quakes)
% })
% |||
% >>>

Breaking by quantiles, so that each facet has approximately the same number of points, is usually the best choice as it is easier to compare the facets because you know the same number of points are in each, even in the presences of overplotting.

\subsection{Missing facetting columns}\label{sub:missing_facetting_columsn}

If you facet on the original data set and add grobs with a new dataset, {\tt ggplot} follows some simple heuristics to try and do what you want.  

If the new dataset contains all of the facetting columns then each facet will have a specialised grob. If the new dataset contains none of the facetting columns, then each facet will use the same grob.  For anything in between, i.e. the new dataset contains some of the existing facetting columns, then the grobs will be duplicated over the facetting columns that are not present.  This sounds complicated, but most of the time it should do what you want.  If you need precise control, just make sure that every new dataset contains all the facetting columns used in the original dataset.

% decumar<<< 
% interweave({
% auckland <- data.frame(lat=-20 , lat=-25, long=175, label="Auck" )
% p <- ggpoint(ggplot(quakes, formula=. ~ depth, aes=list(x=lat, y=long)))
% ggtext(p, data=auckland, aes=list(label=label))
% })
% |||
% >>>  

\ifwhole
\else
	\bibliography{bibliography}
  \end{document}
\fi
